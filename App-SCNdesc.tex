% !TEX root = master.tex
\newpage
\section{Subalpine Conifer (SCN)}

\subsection*{General Information}

\subsubsection{Cover Type Overview}

\textbf{Subalpine Conifer (SCN)}
\newline
Crosswalks
\begin{itemize}
	\item EVeg: Regional Dominance Type 1
	\begin{itemize}
		\item Alpine Mixed Scrub
		\item Mountain Hemlock
		\item Subalpine Conifers
		\item Whitebark Pine
	\end{itemize}

	\item LandFire BpS Model
	\begin{itemize}
		\item Subalpine Conifer
	\end{itemize}

	\item Presettlement Fire Regime Type
	\begin{itemize}
		\item 0610330 Mediterranean California Subalpine Woodland
		\item 0610440 Northern California Mesic Subalpine Woodland
		\item 0610710 Sierra Nevada Alpine Dwarf-Shrubland
	\end{itemize}
\end{itemize}

\noindent \textbf{Subalpine Conifer with Aspen (SCN-ASP)}
This type is created by overlaying the NRIS TERRA Inventory of Aspen on top of the EVeg layer. Where it intersects with SCN it is assigned to SCN-ASP.
\newline

\noindent Reviewed by Marc Meyer, Southern Sierra Province Ecologist, USDA Forest Service

\subsubsection{Vegetation Description}
\textbf{Subalpine Conifer (SCN)} The SCN landscape is comprised of a mosaic of subalpine forests/woodlands, meadows, rock outcrops, and scrub vegetation types. These forests are open stands of conifers occurring on generally sandy soils or rocky slopes at elevations above the upper montane forest stands of \emph{Abies magnifica}. Stand densities are low. Many, but not all, species form shrubby krummholz forms of growth near their upper elevational limits (Fites-Kaufman 2007). 

\emph{Tsuga mertensiana} is often the most common tree species and mixes with \emph{P. contorta} ssp. \emph{murrayana, A. magnifica, Pinus monticola}, and \emph{Pinus albicaulis}. In some areas, \emph{P. contorta} ssp. \emph{murrayana} dominates post-disturbances stands. \emph{T. mertensiana} seedlings are relatively shade tolerant compared to other subalpine conifers and do well under closed canopy conditions. \emph{P. albicaulis} presence increases in the southern portion of the project area (Fites-Kaufman 2007, LandFire 2007a).

Treeline growth of multistemmed trees and shrubby krummholz growth of conifers varies with latitude in the Sierra Nevada. Treeline in the northern Sierra Nevada is dominated by \emph{P. albicaulis}, which frequently occurs with a krummholz form of growth near its upper limit. Several other species may also form krummholz growth forms, including \emph{Juniperus occidentalis}, \emph{Tsuga mertensiana}, \emph{P. contorta} ssp. \emph{murrayana}, and rarely \emph{Pinus jeffreyi} (Fites-Kaufman 2007). 

Although typically of minor importance, a shrub understory may include \emph{Arctostaphylos, Ribes, Phyllodoce, Vaccinium}, and \emph{Kalmia} can occur on moist sites. Herbs present may include \emph{Lupinus, Hieracium, Arabis, Aster}, and \emph{Erigeron}. \emph{Carex} and various grasses are also common (Verner and Purcell 1988, LandFire 2007a).

\medskip
\noindent \textbf{Subalpine Conifer with Aspen (SCN-ASP)} These are upland forests and woodlands dominated by \emph{Populus tremuloides} without a significant conifer component. Conifers may be present in these systems; however, these patches of \emph{P. tremuloides} are not typically successional to conifers. The understory structure may be complex with multiple shrub and herbaceous layers, or simple with just an herbaceous layer. The herbaceous layer may be dense or sparse, dominated by graminoids or forbs. Common shrubs include \emph{Acer, Amelanchier, Artemisia, Juniperus, Prunus, Rosa, Shepherdia, Symphoricarpos}, and the dwarf-shrubs \emph{Mahonia} and \emph{Vaccinium}. Common graminoids may include \emph{Bromus, Calamagrostis, Carex, Elymus, Festuca}, and \emph{Hesperostipa}. Associated forbs may include \emph{Achillea, Eucephalus, Delphinium, Geranium, Heracleum, Ligusticum, Lupinus, Osmorhiza, Pteridium, Rudbeckia, Thalictrum, Valeriana, Wyethia}, and many
others (LandFire 2007b).


\subsubsection{Distribution}
\textbf{Subalpine Conifer (SCN)} The elevational distribution of subalpine forest communities varies with latitude. In the northern Sierra Nevada, such stands begin around 2,450 m and extend up to treeline at 2,750 m to 3,100 m (9,000 ft to 11,000 ft). Both upper and lower limits of subalpine species distributions are driven by a variety of factors, including soil resources, water availability, and climatic limiting factors (Fites-Kaufman 2007).

These forests are characterized by a relatively short growing season with cool temperatures. With the exception of occasional summer thunderstorms, most precipitation falls as snow. Wet years with abundant snowfall can limit growth as these may produce late-lying snowfields that reduce the length of the growing season. Winds can be severe, particularly around exposed ridges. Such wind conditions may produce snow-free winter areas that lower soil temperatures and increase plant water stress (Fites-Kaufman 2007).

Because of the solid granite parent material, areas with deeper soil accumulation can become waterlogged for much of the year. For these reasons, the length of the growing season is a function of not only early season limitation due to low temperatures and snowfields, but also late season limitations due to drought. Studies of the dynamics of alterations of treeline elevation over the past several millennia have reinforced the significance of complex interactions of both temperature and seasonal water availability in determining such changes (Fites-Kaufman 2007). 


\textbf{Subalpine Conifer with Aspen (SCN-ASP)} Sites supporting \emph{P. tremuloides} are associated with added soil moisture, i.e., azonal wet sites. These sites are often close to streams, lakes, and meadows. Other sites include rock reservoirs, springs and seeps. Terrain can be simple to complex. At lower elevations, topographic conditions for this type tends toward positions resulting in relatively colder, wetter conditions within the prevailing climate, e.g., ravines, north slopes, wet depressions, etc. (LandFire 2007b). \emph{P. tremuloides} stands may also be associated with lateral or terminal moraine boulder material, talus-colluvium, rock falls, or lava flows. In addition, pure stands may be found in topographic positions where snow accumulates, mostly at higher north facing elevations, where snow presence means the growing season is too short to support conifers (Shepperd et al. 2006).

\subsection*{Disturbances}

\subsubsection{Wildfire}
\textbf{Subalpine Conifer (SCN)} Most of the subalpine areas of the Sierra Nevada were subjected to repeated glaciation during the Pleistocene, and thus have thin and poorly developed soils with little organic matter. The small amounts of litter accumulation and open stand structure of subalpine forests mean that fire is rare (Fites-Kaufman 2007). It is, however, the major disturbance event of this type (LandFire 2007a). Meyer’s 2013 review suggests that historic and current fire regimes in subalpine forests are normally climate-limited and dominated by surface fires with crown fires occurring occasionally.  

\textbf{Subalpine Conifer with Aspen (SCN-ASP)} Sites supporting \emph{P. tremuloides} are maintained by stand-replacing disturbances that allow regeneration from below-ground suckers. Replacement fire and ground fire are thought to have been common in stable \emph{P. tremuloides} stands historically. Because \emph{P. tremuloides} is associated with mesic conditions, it rarely burns during the normal lightning season. However, during years with little precipitation stands may be more susceptible to burning. Evidence from fire scars and historical studies show that past fires occurred mostly during the spring and fall. These are typically self-perpetuating stands (LandFire)

Estimates of fire rotations for these variants are available from the LandFire project and a few review papers. The LandFire project’s published fire return intervals are based on a series of associated models created using the Vegetation Dynamics Development Tool (VDDT). In VDDT, fires are specified concurrently with the transition that follows them. For example, a replacement fire causes a transition to the early development stage. In the RMLands model, such fires are classified as high mortality. However, in VDDT mixed severity fires may cause a transition to early development, a transition to a more open seral stage, or no transition at all. In this case, we categorize the first example as a high mortality fire, and the second and third examples as a low mortality fire. Based on this approach, we calculated fire rotations and the probability of high mortality fire for each of the SCN and SCN-ASP seral stage (Tables~\ref{tab:scndesc_fire} and \ref{tab:rfr-aspdesc_fire}). We computed overall target fire rotations based on values from Mallek et al. (2013) and Van de Water and Safford (2011) as well as consultations with Meyer, Safford, and Estes (personal communication). 





\begin{table}[]
\small
\centering
\caption{Fire rotation (years) and proportion of high (versus low) mortality fires for Subalpine Conifer. Values were derived from VDDT model 0610440 (LandFire 2007), Mallek et al. (2013), and Estes, Safford, and Meyer (personal communication). }
\label{tab:scndesc_fire}
\begin{tabular}{@{}lcc@{}}
\toprule
\textbf{Condition}         & \multicolumn{1}{l}{\textbf{Fire Rotation}} & \multicolumn{1}{l}{\textbf{\begin{tabular}[c]{@{}l@{}}Proportion \\ High Mortality\end{tabular}}} \\ \midrule
Target                      & 296           & n/a                           \\
Early Development - All     & 500           & 1                             \\
Mid Development - Closed    & 333           & 0.67                          \\
Mid Development - Moderate  & 317           & 0.63                          \\
Mid Development - Open      & 303           & 0.61                          \\
Late Development - Closed   & 333           & 0.67                          \\
Late Development - Moderate & 317           & 0.63                          \\
Late Development - Open     & 303           & 0.61 						      \\ \bottomrule
\end{tabular}
\end{table}

\begin{table}[]
\small
\centering
\caption{Fire rotation (years) and proportion of high (versus low) mortality fires for Subalpine Conifer - Aspen type. Values were derived from VDDT model 0610110 (LandFire 2007), Van de Water and Safford (pers. comm. 2013), Safford, and Estes (personal communication).}
\label{tab:scnasp-desc_fire}
\begin{tabular}{@{}lcc@{}}
\toprule
\textbf{Condition}         & \multicolumn{1}{l}{\textbf{Fire Rotation}} & \multicolumn{1}{l}{\textbf{\begin{tabular}[c]{@{}l@{}}Proportion \\ High Mortality\end{tabular}}} \\ \midrule
Target                           & 296           & n/a                           \\
Early Development - Aspen        & 200           & 1                             \\
Mid Development - Aspen          & 333           & 0.67                          \\
Late Development - Conifer-Aspen & 317           & 0.63						      \\ \bottomrule
\end{tabular}
\end{table}

\subsubsection{Other Disturbance}
Other disturbances are not currently modeled, but may, depending on the seral stage affected and mortality levels, reset patches to early development, maintain existing seral stage, or shift/accelerate succession to a more open seral stage. 

\subsection*{Vegetation Seral Stages}
We recognize seven separate seral stages for SCN: Early Development (ED), Mid Development - Open Canopy Cover (MDO), Mid Development - Moderate Canopy Cover, Mid Development - Closed Canopy Cover (MDC), Late Development - Open Canopy Cover (LDO), Late Development - Moderate Canopy Cover (LDM), and Late Development - Closed Canopy Cover (LDC) (Figure~\ref{transmodel_scn}). The SCN-ASP variant is assigned to three seral stages: Early Development - Aspen (ED-A), Mid Development - Aspen (MD-A), and Late Development - Conifer with Aspen (LD-CA) (Figure~\ref{transmodel_scn-asp}).

Our seral stages are an alternative to ``successional'' classes that imply a linear progression of states and tend not to incorporate disturbance. The seral stages identified here are derived from a combination of successional processes and anthropogenic and natural disturbance, and are intended to represent a composition and structural condition that can be arrived at from multiple other conditions described for that landcover type. Thus our seral stages incorporate age, size, canopy cover, and vegetation composition. In general, the delineation of stages has originated from the LandFire biophysical setting model descriptive of a given landcover type; however, seral stages are not necessarily identical to the classes identified in those models.


\begin{figure}[htbp]
\centering
\includegraphics[width=0.8\textwidth]{/Users/mmallek/Documents/Thesis/statetransmodel/StateTransitionModel/7class.png}
\caption{State and Transition Model for Subalpine Conifer Forest (not inclusive of the aspen variant). Each dark grey box represents one of the seven seral stage for this landcover type. Each column of boxes represents a stage of development: early, middle, and late. Each row of boxes represents a different level of canopy cover: closed (70-100\%), moderate (40-70\%), and open (0-40\%). Transitions between states/seral stage may occur as a result of high mortality fire, low mortality fire, or succession. Specific pathways for each are denoted by the appropriate color line and arrow: red lines relate to high mortality fire, orange lines relate to low mortality fire, and green lines relate to natural succession.} 
\label{transmodel_scn}
\end{figure}

\begin{figure}[htbp]
\centering
\includegraphics[width=0.8\textwidth]{/Users/mmallek/Documents/Thesis/statetransmodel/StateTransitionModel/3class-asp.png}
\caption{State and Transition Model for Subalpine Conifer Forest, Aspen variant. Each dark grey box represents one of the three seral stages for this landcover type. Three seral stages of development are represented: early, middle, and late. Transitions between states/seral stages may occur as a result of high mortality fire, low mortality fire, or succession. Specific pathways for each are denoted by the appropriate color line and arrow: red lines relate to high mortality fire, orange lines relate to low mortality fire, and green lines relate to natural succession.} 
\label{transmodel_scn-asp}
\end{figure}

\paragraph{Early Development (ED)}

\paragraph{Description} The first few years following stand-replacing wildfire are characterized by bare ground, herbs, shrubs, and varying densities of tree seedlings (presumably dependent on seed sources). Dominant species include coniferous tree seedlings, resprouting grasses and shrubs, and invading herbs. Shrubs include \emph{Ribes} spp. Herbs and grasses include \emph{Aster, Pedicularis, Hieracium, Arabis, Erigeron, Carex, Luzula}, and \emph{Poa} (LandFire 2007a).

\paragraph{Succession Transition} In the absence of disturbance, patches in this seral stage will begin transitioning to mid development after 20 years at a rate of 0.4 per time step. Transition to either MDC or MDO can occur, although transition to MDC occurs 90\% of the time. At 80 years, all patches will succeed. On average, patches remain in ED for 33 years.

\paragraph{Wildfire Transition} High mortality wildfire (100\% of fires) recycles the patch through the Early Development seral stage. Low mortality wildfire is not modeled for this seral stage.

\noindent\hrulefill


\paragraph{Mid Development - Open Canopy Cover (MDO)} 

\paragraph{Description} This seral stage represents delayed tree regeneration and long-term domination by shrubs and herbs. Shrubs include \emph{Ribes} spp. Herbs and grasses include \emph{Aster, Pedicularis, Hieracium, Arabis, Erigeron, Carex, Luzula}, and \emph{Poa}. Trees are represented by seedlings and saplings of \emph{T. mertensiana}, \emph{P. contorta} ssp. \emph{murrayana}, and other species (LandFire 2007a).

\paragraph{Succession Transition} Patches in this seral stage will maintain under low mortality disturbance. In the absence of low mortality disturbance, patches in the MDO seral stage will begin transitioning to MDM after 40 years at a rate of 0.3 per timestep. Succession to LDO takes place variably after 60 years since entering a middle development seral stage, at a rate of 0.45 per timestep. All patches succeed by 130 years in mid development.  On average (across all canopy cover seral stages), patches remain in mid development for 71 years.

\paragraph{Wildfire Transition} High mortality wildfire (61\% of fires) recycles the patch through the Early Development seral stage. Low mortality wildfire (39\%) maintains the patch in MDO.

\noindent\hrulefill

\paragraph{Mid Development - Moderate Canopy Cover (MDM)}

\paragraph{Description} This seral stage represents rapid regeneration by \emph{P. contorta} ssp. \emph{murrayana}, with additional conifers coming in, including \emph{T. mertensiana}, \emph{A. magnifica}, and \emph{P. monticola}. Shrubs include \emph{Ribes} spp. Herbs and grasses include \emph{Aster, Pedicularis, Hieracium, Arabis, Erigeron, Carex, Luzula}, and \emph{Poa}. (LandFire 2007a).

\paragraph{Succession Transition} In the absence of low mortality disturbance, patches in the MDM seral stage will begin transitioning to MDC after 40 years at a rate of 0.3 per timestep. Succession to LDM takes place variably after 60 years since entering a middle development seral stage, at a rate of 0.45 per timestep. All patches succeed by 130 years in mid development.  On average (across all canopy cover seral stages), patches remain in mid development for 71 years.
 
\paragraph{Wildfire Transition} High mortality wildfire (63\% of fires) recycles the patch through the Early Development seral stage. Low mortality wildfire (37\%) triggers a transition to MDO.

\noindent\hrulefill

\paragraph{Mid Development - Closed Canopy Cover (MDC)}

\paragraph{Description} This seral stage represents rapid regeneration by \emph{P. contorta} ssp. \emph{murrayana}, with additional conifers coming in, including \emph{T. mertensiana}, \emph{A. magnifica}, and \emph{P. monticola}. Shrubs include \emph{Ribes} spp. Herbs and grasses include \emph{Aster, Pedicularis, Hieracium, Arabis, Erigeron, Carex, Luzula}, and \emph{Poa}. (LandFire 2007a).

\paragraph{Succession Transition} After 60 years without a wildfire-triggered transition, patches in this seral stage will begin transitioning to LDC at a rate of 0.45 per time step. Succession to LDC may occur once the patch age since transition to the mid development stage is at least 60 years. After 130 years, all patches will succeed.

\paragraph{Wildfire Transition} High mortality wildfire (67\% of fires) recycles the patch through the Early Development seral stage. Low mortality wildfire (33\%) triggers a transition to MDM.

\noindent\hrulefill


\paragraph{Late Development - Open Canopy Cover (LDO)}

\paragraph{Description} This seral stage represents late-successional stands with large individuals (DBH greater than 20 in) of \emph{T. mertensiana} and other species. The open stand structure is maintained by mixed severity fire and insect-caused tree mortality (the latter not modeled at this time). Shrubs include \emph{Ribes} spp. Herbs and grasses include \emph{Aster, Pedicularis, Hieracium, Arabis, Erigeron, Carex, Luzula}, and \emph{Poa}. (LandFire 2007a).

\paragraph{Succession Transition} In the absence of any fire, succession to LDM begins at 40 years at a rate of 0.3 per timestep.

\paragraph{Wildfire Transition} High mortality wildfire (61\% of fires) recycles the patch through the Early Development seral stage. Low mortality wildfire (39\%) maintains the patch in LDO.

\noindent\hrulefill

\paragraph{Late Development - Moderate Canopy Cover (LDM)}

\paragraph{Description} This seral stage represents late-successional stands with large individuals (DBH greater than 20 in) of \emph{T. mertensiana} and other species, and advanced regeneration of \emph{T. mertensiana} and other shade tolerant species. The moderately open stand structure is generated by recent low mortality fire and insect-caused tree mortality (the latter not modeled at this time). Shrubs include \emph{Ribes} spp. Herbs and grasses include \emph{Aster, Pedicularis, Hieracium, Arabis, Erigeron, Carex, Luzula}, and \emph{Poa}. (LandFire 2007a).

\paragraph{Succession Transition} In the absence of any fire, succession to LDC begins at 40 years at a rate of 0.3 per timestep.

\paragraph{Wildfire Transition} High mortality wildfire (63\% of fires) recycles the patch through the Early Development seral stage. Low mortality wildfire (37\%) triggers a transition to LDO. 

\noindent\hrulefill

\paragraph{Late Development - Closed Canopy Cover (LDC)}

\paragraph{Description} This seral stage represents late-successional stands with large individuals (DBH greater than 20 in) of \emph{T. mertensiana} and other species, and advanced regeneration of \emph{T. mertensiana} and other shade tolerant species. Shrubs include \emph{Ribes} spp. Herbs and grasses include \emph{Aster, Pedicularis, Hieracium, Arabis, Erigeron, Carex, Luzula}, and \emph{Poa}. (LandFire 2007a).

\paragraph{Succession Transition} Patches in this seral stage will maintain in the absence of disturbance.

\paragraph{Wildfire Transition} High mortality wildfire (67\% of fires) recycles the patch through the Early Development seral stage. Low mortality wildfire (33\%) triggers a transition to LDM. 

\noindent\hrulefill
\noindent\hrulefill

\subsubsection{Aspen Variant}

\paragraph{Early Development - Aspen (ED-A)}

\paragraph{Description} Grasses, forbs, low shrubs, and sparse to moderate cover of tree seedlings/saplings (primarily \emph{P. tremuloides}) with an open canopy. This seral stage is characterized by the recruitment of a new cohort of early successional, shade-intolerant tree species into an open area created by a stand-replacing disturbance. Following disturbance, succession proceeds rapidly from an herbaceous layer to shrubs and trees, which invade together (Verner 1988). \emph{P. tremuloides} suckers over 6ft tall develop within about 10 years (LandFire 2007b). 


\paragraph{Succession Transition} Unless it burns, a patch in the early seral stage persists for 10 years, at which point it transitions to MD-A.

\paragraph{Wildfire Transition} High mortality wildfire (100\% of fires) recycles the patch through the ED-A seral stage. Low mortality wildfire is not modeled for this seral stage.

\noindent\hrulefill


\paragraph{Mid Development - Aspen (MD-A)}

\paragraph{Description} \emph{P. tremuloides} trees 5-16'' DBH. Canopy cover is highly variable, and can range from 40-100\%. These patches range in age from 10 to 110 years (LandFire 2007b).

\paragraph{Succession Transition} Patches in the MD-A seral stage persist for at least 80 years in the absence of fire, at which point they begin transitioning to LD-CA at a rate of 0.3 per timestep. After 200 years since entering MD-A, any remaining patches transition to LD-CA. 

\paragraph{Wildfire Transition} High mortality wildfire (67\% of fires in this seral stage) recycles the patch through the ED-A seral stage. No transition occurs as a result of low mortality fire (33\%).

\noindent\hrulefill



\paragraph{Late Development - Aspen with Conifer (LD-AC)}

\paragraph{Description} These stands have been protected from fire since the last stand-replacing disturbance. \emph{P. tremuloides} trees are predominantly 16'' DBH and greater. Conifers are encroaching and can eventually overtop the aspen (LandFire 2007).

\paragraph{Succession Transition} Patches in this seral stage will maintain in the absence of disturbance.

\paragraph{Wildfire Transition} High mortality wildfire (63\% of fires in this seral stage) returns the patch to ED-A. Low mortality wildfire (33\%) maintains the patch in LD-CA.

\noindent\hrulefill




\subsection*{Seral Stage Classification}
\begin{table}[hbp]
\small
\centering
\caption{Classification of cover seral stage for SCN. Diameter at Breast Height (DBH) and Cover From Above (CFA) values taken from EVeg polygons. DBH categories are: null, 0-0.9'', 1-4.9'', 5-9.9'', 10-19.9'', 20-29.9'', 30''+. CFA categories are null, 0-10\%, 10-20\%, \dots , 90-100\%. Each row in the table below should be read with a boolean AND across each column.}
\label{scn_classification}
\begin{tabular}{@{}lrrrrr@{}}
\toprule
\textbf{\begin{tabular}[l]{@{}l@{}}Cover \\ Condition\end{tabular}} & \textbf{\begin{tabular}[r]{@{}r@{}}Overstory Tree \\ Diameter 1 \\ (DBH)\end{tabular}} & \textbf{\begin{tabular}[r]{@{}r@{}}Overstory Tree \\ Diameter 2 \\ (DBH)\end{tabular}} & \textbf{\begin{tabular}[r]{@{}r@{}}Total Tree\\ CFA (\%)\end{tabular}} & \textbf{\begin{tabular}[r]{@{}r@{}}Conifer \\ CFA (\%)\end{tabular}} & \textbf{\begin{tabular}[r]{@{}r@{}}Hardwood \\ CFA (\%)\end{tabular}} \\ \midrule
Early All        & null           & any & any    & any    & any  \\
Early All        & 0-4.9''         & any & any    & any    & any  \\
Mid Open         & 5-19.9''        & any & null   & null   & null \\
Mid Open         & 5-19.9''        & any & 0-40   & any    & any  \\
Mid Open         & 5-19.9''        & any & null   & 0-40   & null \\
Mid Moderate     & 5-19.9''        & any & 40-70  & any    & any  \\
Mid Moderate     & 5-19.9''        & any & null   & 40-70  & null \\
Mid Closed       & 5-19.9''        & any & 70-100 & any    & any  \\
Mid Closed       & 5-19.9''        & any & null   & 70-100 & any  \\
Late Closed      & 20''+           & any & 70-100 & any    & any  \\
Late Closed      & 20''+           & any & null   & 70-100 & any  \\
Late Moderate    & 20''+           & any & 40-70  & any    & any  \\
Late Moderate    & 20''+           & any & null   & 40-70  & any  \\
Late Open        & 20''+           & any & null   & null   & null \\
Late Open        & 20''+           & any & 0-40   & any    & any  \\
Late Open        & 20''+           & any & null   & 0-40   & null \\ \bottomrule
\end{tabular}
\end{table}

SCN-ASP seral stages were assigned manually using NAIP 2010 Color IR imagery to assess seral stage.


\clearpage

\subsection*{References}
\begin{hangparas}{.25in}{1} 
Estes, Becky. Personal communication, 15 August 2013.

Fites-Kaufman, Jo Ann, Phil Rundel, Nathan Stephenson, and Dave A. Wixelman. ``Montane and Subalpine Vegetation of the Sierra Nevada and Cascade Ranges.'' In \emph{Terrestrial Vegetation of California, 3rd Edition}, edited by Michael Barbour, Todd Keeler-Wolf, and Allan A. Schoenherr, 456-501. Berkeley and Los Angeles: University of California Press, 2007. 

LandFire. ``Biophysical Setting Models.'' Biophysical Setting 0610440: Northern California Mesic Subalpine Woodland. 2007a. LANDFIRE Project, U.S. Department of Agriculture, Forest Service; U.S. Department of the Interior. \burl{http://www.landfire.gov/national_veg_models_op2.php}. Accessed 9 November 2012.

LandFire. ``Biophysical Setting Models.'' Biophysical Setting 0610610: Inter-Mountain Basins Aspen-Mixed Conifer Forest and Woodland. 2007b. LANDFIRE Project, U.S. Department of Agriculture, Forest Service; U.S. Department of the Interior. \burl{http://www.landfire.gov/national_veg_models_op2.php}. Accessed 7 January 2013.

Meyer, Marc D. ``Natural Range of Variation of Red Fir Forests in the Bioregional Assessment Area'' (unpublished paper, Ecology Group, Pacific Southwest Research Station, 2013).

Safford, Hugh S. Personal communication, 5 May 2013, 15 August 2013.

Shepperd, Wayn De, Paul C. Rogers, David Burton, and Dale L. Bartos. ``Ecology, Biodiversity, Management, and Restoration of Aspen in the Sierra Nevada.'' General Technical Report RMRS-GTR-178. Rocky Mountain Research Station, Forest Service, U.S. Department of Agriculture, 2006.

Van de Water, Kip M. and Safford, Hugh D. ``A Summary of Fire Frequency Estimates for California Vegetation Before Euro-American Settlement.'' \emph{Fire Ecology} 7.3 (2011): 26-57. doi: 10.4996/fireecology.0703026.

Verner, Jared. ``Aspen (ASP).'' \emph{A Guide to Wildlife Habitats of California}. 1988. Mayer, Kenneth E. and Laudenslayer, William F., eds. California Department of Fish and Game. \burl{http://www.dfg.ca.gov/biogeodata/cwhr/pdfs/ASP.pdf}. Accessed 4 December 2012.

\end{hangparas}

