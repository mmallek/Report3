% !TEX root = master.tex
\newpage
\section{Sierran Mixed Conifer (SMC)}

\subsection*{General Information}

\subsubsection{Cover Type Overview}

\textbf{Sierran Mixed Conifer (SMC)}
\newline
Crosswalks
\begin{itemize}
	\item EVeg: Regional Dominance Type 1
	\begin{itemize}
		\item Mixed Conifer - Fir
		\item Mixed Conifer - Pine
	\end{itemize}

	\begin{adjustwidth}{2cm}{}
	\textbf{Mesic Modifer }
	\begin{itemize}
		\item LandFire BpS Model: 0610280 Mediterranean California Mesic Mixed Conifer Forest and Woodland
		\item Presettlement Fire Regime Type: Moist Mixed Conifer
		\item This type is created by intersecting a binary xeric/mesic layer with the existing vegetation layer. SMC cells that intersect with mesic cells are assigned to the mesic modifier.
	\end{itemize}
	\textbf{Xeric Modifer} 
	\item LandFire BpS Model
	\begin{itemize}
		\item LandFire BpS Model: 0610270 Mediterranean California Dry-Mesic Mixed Conifer Forest and Woodland
		\item Presettlement Fire Regime Type: Dry Mixed Conifer
		\item This type is created by intersecting a binary xeric/mesic layer with the existing vegetation layer. SMC cells that intersect with xeric cells are assigned to the xeric modifier.
	\end{itemize}
	\textbf{Ultramafic Modifer} 
	\item LandFire BpS Model
	\begin{itemize}
		\item LandFire BpS Model: 0710220 Klamath-Siskiyou Upper Montane Serpentine Mixed Conifer Woodland
		\item Presettlement Fire Regime Type
		\item This type is created by intersecting an ultramafic soils/geology layer with the existing vegetation layer. Where ultramafic cells intersect with SMC they are assigned to the ultramafic modifier.
	\end{itemize}
	\textbf{Ultramafic Modifer} 
	\item LandFire BpS Model
	\begin{itemize}
		\item 0710220 Klamath-Siskiyou Upper Montane Serpentine Mixed Conifer Woodland
		\item This type is created by intersecting an ultramafic soils/geology layer with the existing vegetation layer. Where ultramafic cells intersect with RFR they are assigned to the ultramafic modifier.
	\end{itemize}
	\textbf{Sierran Mixed Conifer  with Aspen (RFR\_ASP)} 
	\begin{itemize}
		\item This type is created by overlaying the NRIS TERRA Inventory of Aspen on top of the EVeg layer. Where it intersects with SMC it is assigned to SMC\_ASP.
	\end{itemize}
\end{adjustwidth}
\end{itemize}

\noindent Reviewed by Hugh Safford, Regional Ecologist, USDA Forest Service; Becky Estes, Central Sierra Province Ecologist, USDA Forest Service


\subsubsection{Vegetation Description}

\textbf{Sierran Mixed Conifer (SMC)} The Sierran Mixed Conifer landcover type is typically composed of three or more conifers, sometimes mixed with hardwoods. In forests experiencing the natural fire regime, stand and landscape structure are both highly heterogeneous, and age structure is usually uneven. Past management (e.g. logging and fire suppression) and its effects on forest succession have resulted in greater structural homogeneity and a dramatic increase in the presence of shade tolerant/fire intolerant tree species. Old-growth stands where fire has been excluded are often multi-storied, with the overstory comprised of various species (often dominated by pines) and the understory dominated by \emph{Abies concolor} and \emph{Calocedrus decurrens}. In the absence of fire, forested stands can form closed, multilayered canopies with over 100\% overlapping cover. Such dense stands were probably relatively uncommon before settlement, and found in moist microsites, on north slopes, and at higher elevations. When openings occur, shrubs are common in the understory. Before Euroamerican settlement, this landcover type was dominated by open stand conditions and old forest, but today closed canopy conditions dominated by middle aged trees are more common. Even aged stands are also widespread (Allen 2005). 

Five conifers and one hardwood typify this landcover type: \emph{A. concolor, Pseudotsuga menziesii, Pinus ponderosa, Pinus lambertiana, C. decurrens}, and \emph{Quercus kelloggii}. \emph{A. concolor} tends to be the most ubiquitous species because it is the competitive dominant in this landcover type. It tolerates shade, reproduces prolifically in the absence of fire, and has the ability to survive long periods of overtopping in brush fields. \emph{P. menziesii} replaces white fir as the competitive dominant at lower elevations. \emph{P. ponderosa}, which was historically the dominant species in SMC forest, still dominates at lower elevations and on south slopes. Like \emph{P. lambertiana}, its densities have been much reduced by logging. \emph{Pinus jeffreyi} commonly replaces P. ponderosa at high elevations, on cold sites, or on ultramafic soils. \emph{Abies magnifica} is a minor associate at the highest elevations, as are \emph{Pinus monticola} and \emph{Pinus contorta} ssp. \emph{murrayana}. \emph{P. lambertiana} is found throughout the landcover type, but its densities have been much reduced by selective logging and white pine blister rust. \emph{Q. kelloggii} is a common component in stands on warm, dry sites. It sprouts prolifically after fire, and although it does best on open sites, it is maintained under adverse conditions such as overtopping by conifers and thin soils (Allen 2005). In some locations, \emph{Populus tremuloides} is also a component of the stand and, when present, typically dominates during the early seral stages following disturbance.

\emph{Ceanothus, Arctostaphylos, Chrysolepis, Prunus, Ribes, Rosa}, and \emph{Chamaebatia} are common shrub genera in the understory (Allen 2005). Grasses and forbs are diverse but rarely contribute much cover, except where stand structure is open. 


\begin{adjustwidth}{2cm}{}
\textbf{Mesic Modifer } The primary species associated with mesic sites are \emph{A. concolor, P. menziesii, C. decurrens}, and \emph{P. lambertiana}. \emph{P. contorta} ssp. \emph{murrayana} may also be associated with mesic forests at higher elevations. As elevations begin to increase, \emph{A. magnifica} becomes more prominent. \emph{Lithocarpus densiflora} is an indicator of lower elevation sites with high water availability, either from meteoric or surface water. Understory diversity is often low in these sites, as high canopy cover and tree density reduce solar incidence at the soil surface. Very often the ground is covered in thick litter and duff. Some shade tolerant shrub and herb species occur.

\textbf{Xeric Modifer}  Xeric sites are characterized by the presence of shade intolerant/fire tolerant conifer species such as \emph{P. ponderosa}, \emph{P. jeffreyi}, and \emph{P. lambertiana}, as well as the occurrence of varying amounts of more shade tolerant species like \emph{A. concolor} and \emph{C. decurrens}  \emph{Q. kelloggii} is locally common. The pines normally are prominent on south and west facing slopes, \emph{A. concolor} and sometimes \emph{P. menziesii}  on north and east slopes, and \emph{C. decurrens} as a secondary component on all slopes. At lower elevations, \emph{Pinus sabiniana}, and \emph{Quercus chrysolepis} may become common associates. Understory shrubs include \emph{Ceanothus, Arctostaphylos, Chamaebatia}, and \emph{Artemisia} and \emph{Purshia} in dry, eastern sites.

\textbf{Ultramafic Modifer} Ultramafic soils support a number of endemic plant species. Slowly growing and often stunted \emph{P. contorta} ssp. \emph{murrayana} and \emph{P. jeffreyi} occur in combinations or in nearly pure open stands. Other tree associates on ultramafics include \emph{P. menziesii}, \emph{C. decurrens}, and \emph{Pinus attenuata}. Hardwoods are usually sparse, but shrubs such as \emph{Arctostaphylos, Quercus, Rhamnus, Lithocarpus, Rhododendron}, and \emph{Ceanothus} may occur on these sites. Often, a dramatic landscape shift occurs across abrupt discontinuities between ultramafics and other rock types. For example, regional stands of dense conifer forests are replaced by stunted and open stands of other conifers, by chaparral or even by barrens on which woody vegetation is absent (``CalVeg Zone 1'' 2011).

\end{adjustwidth}

\textbf{Sierran Mixed Conifer  with Aspen} When \emph{P. tremuloides} co-occurs with SMC on the west side of the crest, it is typically found in smaller patches, often less than 2 ha (5 acres) in size. This variant is not subject to the modifiers described above because it is only found on mesic sites. Mature stands in which \emph{P. tremuloides} are still dominant are usually relatively open. Average canopy closures of stands in eastern California range from 60 to 100 percent in young and intermediate-aged stands and from 25 to 60 percent in mature stands. The open nature of the stands results in substantial light penetration to the ground (Verner 1988).



\subsubsection{Distribution}

\textbf{Sierran Mixed Conifer } SMC generally forms a vegetation band ranging from 500 to 2000 m (1500 to 6500 ft). It dominates the western middle elevation slopes of the Sierra Nevada. Soils supporting SMC are varied in depth and composition, and are derived primarily from Mesozoic granitic, Paleozoic metamorphic rocks, and Cenozoic volcanic rocks (Allen 2005). 

A xeric-mesic gradient was developed based on four variables: 1) aspect, 2) potential evapotranspiration, 3) topographic wetness index, and 4) soil water storage. The variables were standardized by z-score such that higher values correspond to more mesic environments. Thus, potential evapotranspiration was inverted to maintain this balance. The four variables were combined with equal weights. This final variables was split into xeric vs. mesic, with xeric occupying the negative end of the range up to $\frac{1}{4}$ standard deviation below the mean (zero) and mesic occupying the remaining portion of the spectrum.


\begin{adjustwidth}{2cm}{}
\textbf{Mesic Modifer } Generally found on favorable slopes, primarily north and east aspects throughout the geographic range, as well as along streams in drier areas. It is more common at higher elevations as compared to the xeric type (``CalVeg Zone 1'' 2011).

\textbf{Xeric Modifer} Occurs on south and west-facing aspects (LandFire 2007b). At lower elevations patches may be found on north slopes. At higher elevations this landcover type most typically occurs on south, east and west aspects. 

\textbf{Ultramafic Modifer} Ultramafics have been mapped at various spatial densities throughout the elevational range of the SMC landcover type. Low to moderate elevations in ultramafic and serpentinized areas often produce soils low in essential minerals like calcium potassium, and nitrogen, and have excessive accumulations of heavy metals such as nickel and chromium. These sites vary widely in the degree of serpentinization and effects on their overlying plant communities (``CalVeg Zone 1'' 2011). Note, the terms ``ultramafic rock'' and ``serpentine'' are broad terms used to describe a number of different but related rock types, including serpentinite, peridotite, dunite, pyroxenite, talc and soapstone, among others (O’Geen et al. 2007). 

\end{adjustwidth}

\textbf{Sierran Mixed Conifer with Aspen} Sites supporting \emph{P. tremuloides} are usually associated with added soil moisture, i.e., azonal wet sites. These sites are found throughout the SMC zone, often close to streams and lakes. Other sites include meadow edges, rock reservoirs, springs and seeps. Terrain can be simple to complex. At lower elevations, topographic conditions for this type tends toward positions resulting in relatively colder, wetter conditions within the prevailing climate, e.g., ravines, north slopes, wet depressions, etc. (LandFire 2007c).

\subsection*{Disturbances}

\subsubsection{Wildfire}

\textbf{Sierran Mixed Conifer } Wildfires are common and frequent; mortality depends on vegetation vulnerability and wildfire intensity. Low mortality fires kill small trees and may consume above-ground portions of small oaks, shrubs and herbs, but do not kill large trees or below-ground organs of most oaks, shrubs and herbs which promptly resprout. High mortality fires kill trees of all sizes and may kill many of the shrubs and herbs as well. However, high mortality fires typically kill only the above ground portions of the oaks, shrubs and herbs; consequently, most oaks, shrubs and herbs promptly resprout from surviving below ground organs.

Data on fire return intervals (FRIs) are available from a few review papers. Mallek et al. (2013) calculated presettlement fire rotation for 7 major forest types in the Sierra Nevada. Skinner and Chang (1996) aggregated FRIs from the Sierra Nevada and separated pre-1850 data from overall data. Van de Water and Safford’s 2011 review paper aggregates hundreds of articles, conference proceedings, and LandFire data on fire return intervals, with an emphasis on Californian sources. We also include here data from the pertinent individual LandFire BpS models (2007a, 2007b, 2007c, 2007d).

Estimates of fire rotations for these variants are available from the LandFire project and a few review papers. The LandFire project’s published fire return intervals are based on a series of associated models created using the Vegetation Dynamics Development Tool (VDDT). In VDDT, fires are specified concurrently with the transition that follows them. For example, a replacement fire causes a transition to the early development stage. In the RMLands model, such fires are classified as high mortality. However, in VDDT mixed severity fires may cause a transition to early development, a transition to a more open seral stage, or no transition at all. In this case, we categorize the first example as a high mortality fire, and the second and third examples as a low mortality fire. Based on this approach, we calculated fire rotations and the probability of high mortality fire for each of the SMC seral stages across the three variants, as well as for the SMC\_ASP variant (Tables 1–4). We computed overall target fire rotations based on expert input from Safford and Estes, values from Mallek et al. (2013), and Van de Water and Safford (2011). 


\begin{adjustwidth}{2cm}{}
\textbf{Mesic Modifer } Low mortality fire is fairly frequent. Fire severity is typically positively correlated with slope position. 

\textbf{Xeric Modifer} Fire of all severity levels is fairly common. This landcover type has one of the shortest fire rotations. 

\textbf{Ultramafic Modifer} This type has a very limited distribution and consequently limited information for fire occurrence history. Low mortality fire is more common than high mortality fire. Most medium and high severity fire occurs on middle and upper slope positions.

\end{adjustwidth}

\textbf{Sierran Mixed Conifer  with Aspen} Sites supporting \emph{P. tremuloides} are maintained by stand-replacing disturbances that allow regeneration from below-ground suckers. Upland clones are impaired or suppressed by conifer ingrowth and overtopping and intensive grazing that inhibits growth. In a reference condition scenario, a few stands will advance toward conifer dominance, but in the current landscape scenario where fire has been reduced from reference conditions there are many more conifer-dominated mixed aspen stands (LandFire 2007c, Verner 1988).


\begin{table}[]
\small
\centering
\caption{Fire rotation (years) and proportion of high (versus low) mortality fires for Sierran Mixed Conifer – Mesic. Values were derived from Mallek et al. (2013) and VDDT model 0610280 (LandFire 2007a). }
\label{tab:smcmdesc_fire}
\begin{tabular}{@{}lcc@{}}
\toprule
\textbf{Condition}         & \multicolumn{1}{l}{\textbf{Fire Rotation}} & \multicolumn{1}{l}{\textbf{\begin{tabular}[c]{@{}l@{}}Probability of \\ High Mortality\end{tabular}}} \\ \midrule
Target                      & 29            & n/a                           \\
Early Development – All     & 44            & 1                             \\
Mid Development – Closed    & 19            & 0.23                          \\
Mid Development – Moderate  & 13            & 0.17                          \\
Mid Development – Open      & 10            & 0.14                          \\
Late Development – Closed   & 34            & 0.37                          \\
Late Development – Moderate & 13            & 0.14                          \\
Late Development – Open     & 8             & 0.08                     \\ \bottomrule
\end{tabular}
\end{table}

\begin{table}[]
\small
\centering
\caption{Fire rotation (years) and proportion of high (versus low) mortality fires for Sierran Mixed Conifer – Xeric. Values were derived from Mallek et al. (2013) and VDDT model 0610280 (LandFire 2007b).}
\label{tab:smcxdesc_fire}
\begin{tabular}{@{}lcc@{}}
\toprule
\textbf{Condition}         & \multicolumn{1}{l}{\textbf{Fire Rotation}} & \multicolumn{1}{l}{\textbf{\begin{tabular}[c]{@{}l@{}}Probability of \\ High Mortality\end{tabular}}} \\ \midrule
Target                      & 22            & n/a                           \\
Early Development – All     & 32            & 1                             \\
Mid Development – Closed    & 11            & 0.48                          \\
Mid Development – Moderate  & 10            & 0.26                          \\
Mid Development – Open      & 9             & 0.09                          \\
Late Development – Closed   & 16            & 0.25                          \\
Late Development – Moderate & 10            & 0.11                          \\
Late Development – Open     & 8             & 0.05                  \\ \bottomrule
\end{tabular}
\end{table}

\begin{table}[]
\small
\centering
\caption{Fire rotation (years) and proportion of high (versus low) mortality fires for Sierran Mixed Conifer – Ultramafic. Values were derived from Van de Water and Safford (2011), and Mallek et al. (2013) and VDDT model 071220 (LandFire 2007d). }
\label{tab:smcudesc_fire}
\begin{tabular}{@{}lcc@{}}
\toprule
\textbf{Condition}         & \multicolumn{1}{l}{\textbf{Fire Rotation}} & \multicolumn{1}{l}{\textbf{\begin{tabular}[c]{@{}l@{}}Probability of \\ High Mortality\end{tabular}}} \\ \midrule
Target                      & 60            & n/a                           \\
Early Development – All     & 89            & 1                             \\
Mid Development – Closed    & 39            & 0.23                          \\
Mid Development – Moderate  & 27            & 0.17                          \\
Mid Development – Open      & 21            & 0.14                          \\
Late Development – Closed   & 69            & 0.37                          \\
Late Development – Moderate & 27            & 0.14                          \\
Late Development – Open     & 16            & 0.08                  \\ \bottomrule
\end{tabular}
\end{table}

\begin{table}[]
\small
\centering
\caption{Fire rotation (years) and proportion of high (versus low) mortality fires for Sierran Mixed Conifer – Aspen type. Values were derived from VDDT models 0610280 and 0610610 (LandFire 2007a, LandFire 2007c) and Van de Water and Safford (2011). }
\label{tab:smc-aspdesc_fire}
\begin{tabular}{@{}lcc@{}}
\toprule
\textbf{Condition}         & \multicolumn{1}{l}{\textbf{Fire Rotation}} & \multicolumn{1}{l}{\textbf{\begin{tabular}[c]{@{}l@{}}Probability of \\ High Mortality\end{tabular}}} \\ \midrule
Target                           & 29            & n/a                           \\
Early Development – Aspen        & 44            & 1                             \\
Mid Development – Aspen          & 19            & 0.26                          \\
Mid Development – Aspen-Conifer  & 13            & 0.18                          \\
Late Development – Conifer-Aspen & 13            & 0.14                          \\
Late Development – Closed        & 34            & 0.37                  \\ \bottomrule
\end{tabular}
\end{table}

\subsubsection{Other Disturbance}
Other disturbances are not currently modeled, but may, depending on the seral stage affected and mortality levels, reset patches to early development, maintain existing seral stages, or shift/accelerate succession to a more open seral stage. All of the tree species associated with this vegetation type are susceptible to a wide variety of pathogens and insects. 

\subsection*{Vegetation Seral Stages}
We recognize seven separate seral stages for SMC: Early Development (ED), Mid Development – Open Canopy Cover (MDO), Mid Development – Moderate Canopy Cover, Mid Development – Closed Canopy Cover (MDC), Late Development – Open Canopy Cover (LDO), Late Development – Moderate Canopy Cover (LDM), and Late Development – Closed Canopy Cover (LDC). The SMC-ASP variant is also assigned to five seral stages: Early Development – Aspen (ED-A), Mid Development – Aspen (MD-A), Mid Development – Aspen with Conifer (MD-AC), Late Development Closed (LDC), and Late Development – Conifer with Aspen (LD-CA).

Our seral stages are an alternative to ``successional'' classes that imply a linear progression of states and tend not to incorporate disturbance. The seral stages identified here are derived from a combination of successional processes and anthropogenic and natural disturbance, and are intended to represent a composition and structural condition that can be arrived at from multiple other conditions described for that landcover type. Thus our seral stages incorporate age, size, canopy cover, and vegetation composition. In general, the delineation of stages has originated from the LandFire biophysical setting model descriptive of a given landcover type; however, seral stages are not necessarily identical to the classes identified in those models.


\subsection*{Sierran Mixed Conifer }

\paragraph{Description}
\paragraph{Early Development (ED)} This seral stage is characterized by the recruitment of a new cohort of early successional tree species into an open area created by a stand-replacing disturbance. After disturbance, succession proceeds from an ephemeral herb to perennial grass-herb community. This seral stage generally lasts only a few years before shifting to a shrub-seedling-sapling seral stage dominated by any of the following genera: \emph{Arctostaphylos, Ceanothus, Prunus, Ribes, and Chamaebatia}, as well as \emph{Q. vaccinifolia}. Tree seedlings/saplings typical of the cover type can be either high or low density depending on local environmental conditions and climate conditions following the disturbance. In some cases (e.g., favorable climate conditions develop following the stand-replacing disturbance and a good seed source), tree seedlings may develop a nearly continuous canopy and succeed relatively quickly to mid-development seral stages. In other cases, and more commonly on xeric or ultramafic sites, chaparral conditions may dominate and persist for long periods of time (LandFire 2007a, LandFire 2007b).

\paragraph{Succession Transition}

\begin{adjustwidth}{2cm}{}
\textbf{Mesic Modifer } In the absence of disturbance, patches in this seral stage will begin transitioning to MDC or MDO after 20 years at a rate of 0.8 per timestep. The transition to MDC is twice as likely as transition to MDO.  At 40 years, all remaining patches will succeed to either MDC or MDO. On average, patches remain in ED for 26 years.

\textbf{Xeric Modifer}  Transition to the MD seral stages may be substantially delayed. Thus, in the absence of disturbance, patches in this seral stage will begin transitioning to MDO after 40 years and may be delayed in ED for as long as 80 years. During this period, succession occurs at a rate of 0.4 per timestep. On average, patches remain in ED for 53 years.

\textbf{Ultramafic Modifer}  Transition to the MD seral stage may be substantially delayed. Thus, in the absence of disturbance, patches in this seral stage will begin transitioning to MDO after 80 years and may be delayed in ED for as long as 150 years. During this period, succession occurs at a rate of 0.2 per timestep. On average, patches remain in ED for 105 years.

\end{adjustwidth}



\paragraph{Wildfire Transition} High mortality wildfire (100\% of fires in this seral stage) recycles the patch through the Early Development seral stage, regardless of soil type. Low mortality wildfire is not modeled for this seral stage. 

\noindent\hrulefill


\paragraph{Mid Development – Open Canopy Cover (MDO)} 

\paragraph{Description} Heterogeneous ground cover of grasses, forbs, and shrubs. Trees present are pole to medium sized conifers with canopy cover less than 40\% (LandFire 2007a). Conifer species likely present include \emph{A. concolor, C. decurrens P. ponderosa, P. menziesii}, and \emph{P. lambertiana}. Pines predominate on xeric sites while firs predominate on mesic sites. \emph{Q. kelloggi} may occur as well, mostly on warmer slopes and where soils are less productive (LandFire 2007a). Ultramafic sites will have similar species composition, especially at edges, but \emph{P. jeffreyi}, and \emph{C. decurrens} are relatively more common (O’Geen et al. 2007).

\paragraph{Succession Transition}
\begin{adjustwidth}{2cm}{}
\textbf{Mesic Modifer } In the absence of low mortality disturbance, patches in the MDO seral stage will begin transitioning to MDM after 15 years at a rate of 0.9 per timestep. Succession to LDO takes place variably after 100 years since entering a middle development seral stage, at a rate of 0.4 per timestep. All patches succeed by 150 years in MD.  On average (across all canopy cover seral stages), patches remain in mid development for 113 years. 

\textbf{Xeric Modifer} In the absence of low mortality disturbance, patches in the MDO seral stage will begin transitioning to MDC after 84 years at a rate of 0.3 per timestep. Succession to LDO takes place variably beginning at 160 years since transition to middle development, at a rate of 0.4 per timestep. All patches succeed by 200 years. On average (across all canopy cover seral stages), patches remain in mid development for 173 years.

\textbf{Ultramafic Modifer} In the absence of low mortality disturbance, patches in the MDO seral stage will begin transitioning to MDC after 40 years in MDO at a rate of 0.1 per timestep. Succession to LDO takes place variably beginning at 200 years since transition to middle development at a rate of 0.4 per timestep. All patches succeed by 260 years. On average (across all canopy cover seral stages), patches remain in mid development for 213 years.

\end{adjustwidth}

\paragraph{Wildfire Transition}
\begin{adjustwidth}{2cm}{}
\textbf{Mesic Modifer } High mortality wildfire (14\% of fires in this seral stage) returns the patch to Early Development. Low mortality fire (86\%) maintains the MDO seral stage and allows for succession to LDO. 

\textbf{Xeric Modifer}  High mortality wildfire (9\% of fires in this seral stage) returns the patch to Early Development. Low mortality fire (91\%) maintains the MDO seral stage and allows for succession to LDO. 

\textbf{Ultramafic Modifer}  High mortality wildfire (14\% of fires in this seral stage) returns the patch to Early Development. Low mortality fire (86\%) maintains the MDO seral stage and allows for succession to LDO.

\end{adjustwidth}

\noindent\hrulefill

\paragraph{Mid Development – Moderate Canopy Cover (MDM)}

\paragraph{Description} Sparse ground cover of grasses, forbs, and shrubs; moderate to dense cover of trees. Conifers are pole to medium-sized, with canopy cover from 40-70\%. Conifer species likely present include \emph{A. concolor, C. decurrens, P. ponderosa, P. menziesii}, and \emph{P. lambertiana}. \emph{Q. kelloggi} may occur as well, mostly on warmer slopes and where soils are less productive (LandFire 2007a, LandFire 2007b). Ultramafic sites will have similar species composition, especially at edges, but \emph{P. jeffreyi}, and \emph{C. decurrens} are relatively more common (O’Geen et al. 2007).

\paragraph{Succession Transition}
\begin{adjustwidth}{2cm}{}
\textbf{Mesic Modifer } In the absence of low mortality disturbance, patches in the MDM seral stage will begin transitioning to MDC after 15 years at a rate of 0.9 per timestep. Patches in the MDM seral stage begin transitioning to LDM once the time since transition to a mid development seral stage is at least 100 years at a rate of 0.6 per timestep. All patches succeed by 150 years in mid development. On average (across all canopy cover seral stages), patches remain in mid development for 113 years.

\textbf{Xeric Modifer}  Transition to late seral seral stages may be delayed. In the absence of low mortality disturbance, patches in the MDM seral stage will begin transitioning to MDC after 40 years at a rate of 0.3 per timestep. Patches in this seral stage will begin transitioning to LDC after 160 years in an MD seral stage at a rate of 0.4 per time step and may be delayed in the MDC seral stage for up to 200 years. On average (across all canopy cover seral stages), patches remain in mid development for 173 years. 

\textbf{Ultramafic Modifer} Transition to late seral seral stages may be substantially delayed. Thus, in the absence of disturbance, patches in this seral stage will begin transitioning to MDC after 40 years at a rate of 0.1 per timestep. Patches in the MDM seral stage begin transitioning to LDM once the time since transition to a mid development seral stage is at least 200 years at a rate of 0.4 per timestep. All patches succeed by 260 years in mid development. On average (across all canopy cover seral stages), patches remain in mid development for 213 years.

\end{adjustwidth}

\paragraph{Wildfire Transition}
\begin{adjustwidth}{2cm}{}
\textbf{Mesic Modifer } High mortality wildfire (17\% of fires in this seral stage) returns the patch to ED. Low mortality wildfire (83\%) opens the stand up to MDO 36\% of the time; otherwise, the patch remains in MDM. 

\textbf{Xeric Modifer}  High mortality wildfire (26\% of fires in this seral stage) returns the patch to ED. Low mortality wildfire (74\%) opens the stand up to MDO 32\% of the time; otherwise, the patch remains in MDM.

\textbf{Ultramafic Modifer} High mortality wildfire (17\% of fires in this seral stage) returns the patch to ED. Low mortality wildfire (83\%) opens the stand up to MDO 36\% of the time; otherwise, the patch remains in MDM.

\end{adjustwidth}

\noindent\hrulefill

\paragraph{Mid Development – Closed Canopy Cover (MDC)}

\paragraph{Description} Sparse ground cover of grasses, forbs, and shrubs; moderate to dense cover of trees. Conifers are pole to medium-sized, with canopy cover from 70-100\%. Conifer species likely present include \emph{A. concolor, C. decurrens, P. ponderosa, P. menziesii}, and \emph{P. lambertiana}. \emph{Q. kelloggi} may occur as well, mostly on warmer slopes and where soils are less productive (LandFire 2007a, LandFire 2007b). Ultramafic sites will have similar species composition, especially at edges, but \emph{P. jeffreyi}, and \emph{C. decurrens} are relatively more common (O’Geen et al. 2007).

\paragraph{Succession Transition}
\begin{adjustwidth}{2cm}{}
\textbf{Mesic Modifer }  Patches in the MDM seral stage begin transitioning to LDM once the time since transition to a mid development seral stage is at least 100 years in the absence of fire, at which point stands succeed to LDC at a rate of 0.4 per timestep. All patches succeed by 150 years in mid development. On average (across all canopy cover seral stages), patches remain in mid development for 113 years.

\textbf{Xeric Modifer}  Transition to late seral seral stages may be delayed. Thus, in the absence of disturbance, patches in this seral stage will begin transitioning to LDC after 160 years in an mid development seral stage at a rate of 0.4 per time step and may be delayed in the mid development stage for up to 200 years. 

\textbf{Ultramafic Modifer} Transition to late seral seral stages may be substantially delayed. Thus, in the absence of disturbance, patches in this seral stage will begin transitioning to LDC after 200 years in the mid development stage at a rate of 0.4 per time step and may be delayed in a mid development seral stage for up to 260 years.

\end{adjustwidth}

\paragraph{Wildfire Transition}
\begin{adjustwidth}{2cm}{}
\textbf{Mesic Modifer } High mortality wildfire (23\% of fires in this seral stage) returns the patch to ED. Low mortality wildfire (77\%) opens the stand up to MDM 53\% of the time; otherwise, the patch remains in MDC. 

\textbf{Xeric Modifer} High mortality wildfire (48\% of fires in this seral stage) returns the patch to ED. Low mortality wildfire (52\%) opens the stand up to MDM 42\% of the time; otherwise, the patch remains in MDC.

\textbf{Ultramafic Modifer} High mortality wildfire (23\% of fires in this seral stage) returns the patch to ED. Low mortality wildfire (77\%) opens the stand up to MDM 53\% of the time; otherwise, the patch remains in MDC.

\end{adjustwidth}

\noindent\hrulefill


\paragraph{Late Development – Open Canopy Cover (LDO)}

\paragraph{Description} Heterogenous ground cover of grasses, forbs, and low shrubs; low density (less than 40\% canopy cover) of large trees. Occurring in small to moderately-sized patches on southerly aspects and ridge tops. Upper canopy trees may be very large, but overall size classes vary with a patchy distribution and open canopy. This seral stage develops when low-mortality disturbance is fairly frequent; it persists as long as low-mortality fires continue to occur periodically. Conifer species likely present include \emph{A. concolor, C. decurrens, P. ponderosa, P. menziesii}, and \emph{P. lambertiana}. \emph{Q. kelloggi} may occur as well, mostly on warmer slopes and where soils are less productive (LandFire 2007a, LandFire 2007b). Ultramafic sites will have similar species composition, especially at edges, but \emph{P. jeffreyi}, and \emph{C. decurrens} are relatively more common (O’Geen et al. 2007).


\paragraph{Succession Transition}
\begin{adjustwidth}{2cm}{}
\textbf{Mesic Modifer } In the presence of low mortality disturbance, patches in this seral stage can self-perpetuate, but after 15 years with no fire, these patches will begin transitioning to LDM at a rate of 0.9 per timestep.

\textbf{Xeric Modifer}  Succession to LDM may occur after 20 years with no fire at a rate of 0.6 per timestep. 

\textbf{Ultramafic Modifer} Patches occurring on ultramafic soils may succeed to LDC after 25 years with no fire at a rate of 0.2 per timestep.

\end{adjustwidth}

\paragraph{Wildfire Transition}
\begin{adjustwidth}{2cm}{}
\textbf{Mesic Modifer } High mortality wildfire (8\% of fires in this seral stage) returns the patch to early development. Low mortality wildfire (92\%) maintains LDO.

\textbf{Xeric Modifer} High mortality wildfire (5\% of fires in this seral stage) returns the patch to early development. Low mortality wildfire (95\%) maintains LDO. 

\textbf{Ultramafic Modifer} High mortality wildfire (8\% of fires in this seral stage) returns the patch to early development. Low mortality wildfire (92\%) maintains LDO.

\end{adjustwidth}

\noindent\hrulefill

\paragraph{Late Development – Moderate Canopy Cover (LDM)}

\paragraph{Description} Overstory of large and very large trees with canopy cover 40-70\%. Understory characterized by medium and smaller-sized shade-tolerant conifers (LandFire 2007a). Conifer species likely present include \emph{A. concolor, C. decurrens, P. ponderosa, P. menziesii}, and \emph{P. lambertiana}. \emph{Q. kelloggi} may occur as well, mostly on warmer slopes and where soils are less productive (LandFire 2007a, LandFire 2007b). Ultramafic sites will have similar species composition, especially at edges, but \emph{P. jeffreyi}, and \emph{C. decurrens} are relatively more common (O’Geen et al. 2007).


\paragraph{Succession Transition} 
\begin{adjustwidth}{2cm}{}
\textbf{Mesic Modifer } In the presence of low mortality disturbance, patches in this seral stage can self-perpetuate, but after 15 years with no fire, these patches will begin transitioning to LDC at a rate of 0.9 per timestep.

\textbf{Xeric Modifer} Succession to LDC may occur after 20 years with no fire at a rate of 0.6 per timestep. 

\textbf{Ultramafic Modifer} Patches occurring on ultramafic soils may succeed to LDC after 25 years with no fire at a rate of 0.2 per timestep.

\end{adjustwidth}
\paragraph{Wildfire Transition}
\begin{adjustwidth}{2cm}{}
\textbf{Mesic Modifer } High mortality wildfire (14\% of fires in this seral stage) will return the patch to Early Development. Low mortality wildfire (86\%) usually has little effect, although 24\% of the time it opens the stand up to LDO. 

\textbf{Xeric Modifer} High mortality wildfire (11\% of fires in this seral stage) will return the patch to Early Development. Low mortality wildfire (99\%) usually has little effect, although 30\% of the time it opens the stand up to LDO. 

\textbf{Ultramafic Modifer} High mortality wildfire (14\% of fires in this seral stage) will return the patch to Early Development. Low mortality wildfire (86\%) usually has little effect, although 24\% of the time it opens the stand up to LDO. 

\end{adjustwidth}

\noindent\hrulefill

\paragraph{Late Development – Closed Canopy Cover (LDC)}

\paragraph{Description} Overstory of large and very large trees with canopy cover over 70\%. Understory characterized by medium and smaller-sized shade-tolerant conifers (LandFire 2007a). Conifer species likely present include \emph{A. concolor, C. decurrens, P. ponderosa, P. menziesii}, and \emph{P. lambertiana}. \emph{Q. kelloggi} may occur as well, mostly on warmer slopes and where soils are less productive (LandFire 2007a, LandFire 2007b). Ultramafic sites will have similar species composition, especially at edges, but \emph{P. jeffreyi}, and \emph{C. decurrens} are relatively more common (O’Geen et al. 2007).

\paragraph{Succession Transition} In the absence of disturbance, patches in this seral stage will maintain, regardless of soil characteristics.

\paragraph{Wildfire Transition}

\begin{adjustwidth}{2cm}{}
\textbf{Mesic Modifer } High mortality wildfire (37\% of fires in this seral stage) will return the patch to Early Development. Low mortality wildfire (63\%) may have little effect, but 54\% of the time it opens the stand up to LDM. 

\textbf{Xeric Modifer} High mortality wildfire (25\% of fires in this seral stage) will return the patch to Early Development. Low mortality wildfire (65.9\%) may have little effect, but 57\% of the time it opens the stand up to LDM. 

\textbf{Ultramafic Modifer} High mortality wildfire (37\% of fires in this seral stage) will return the patch to Early Development. Low mortality wildfire (63\%) may have little effect, but 54\% of the time it opens the stand up to LDM.

\end{adjustwidth}

\noindent\hrulefill
\noindent\hrulefill

\subsubsection{Aspen Variant}

\paragraph{Early Development – Aspen (ED–A)}

\paragraph{Description} Grasses, forbs, low shrubs, and sparse to moderate cover of tree seedlings/saplings (primarily \emph{P. tremuloides}) with an open canopy. This seral stage is characterized by the recruitment of a new cohort of early successional, shade-intolerant tree species into an open area created by a stand-replacing disturbance.

Following disturbance, succession proceeds rapidly from an herbaceous layer to shrubs and trees, which invade together (Verner 1988). \emph{P. tremuloides} suckers over 6ft tall develop within about 10 years (LandFire 2007c). 



\paragraph{Succession Transition} Unless it burns, a patch in ED–A persists for 10 years, at which point it transitions to MD-A.

\paragraph{Wildfire Transition} High mortality wildfire (100\% of fires in this seral stage) recycles the patch through the ED–A seral stage. Low mortality wildfire is not modeled for this seral stage.

\noindent\hrulefill


\paragraph{Mid Development – Aspen (MD–A)}

\paragraph{Description} \emph{P. tremuloides} trees 5-16 in DBH. Canopy cover is highly variable, and can range from 40-100\%. These patches range in age from 10 to 110 years. Some understory conifers, including \emph{P. ponderosa}, \emph{P. lambertiana}, and \emph{A. concolor} are encroaching, but \emph{P. tremuloides} is still the dominant component of the stand (LandFire 2007c).

\paragraph{Succession Transition} Patches in the MD-A seral stage persist for at least 50 years in the absence of fire, after which stands begin transitioning to MD-AC at a rate of 0.6 per timestep. After 100 years since entering MD-A, any remaining patches transition to MD-AC. 

\paragraph{Wildfire Transition} High mortality wildfire (26\% of fires in this seral stage) recycles the patch through the ED–A seral stage. No transition occurs as a result of low mortality fire.

\noindent\hrulefill

\paragraph{Mid Development – Aspen with Conifer (MD–AC)}

\paragraph{Description} These stands have been protected from fire since the last stand-replacing disturbance. \emph{P. tremuloides} trees are predominantly 16in DBH and greater. Conifers are present and overtopping the \emph{P. tremuloides}. \emph{A. concolor} is a typical conifer that is successional to \emph{P. tremuloides}, and is depicted here, but other conifers including \emph{P. ponderosa} and \emph{P. lambertiana} are also possible. Conifers are pole to medium-sized, and conifer cover is at least 40\% (LandFire 2007c).

\paragraph{Succession Transition} Patches in the MD-AC seral stage persist for 100 years in the absence of high mortality fire, at which point which patches transition to LDC. 

\paragraph{Wildfire Transition} High mortality wildfire (18\% of fires in this seral stage) returns the patch to ED-A. Low mortality wildfire (82\%) maintains the patch in MD–AC.

\noindent\hrulefill

\paragraph{Late Development – Closed (LDC)}

\paragraph{Description} Some \emph{P. tremuloides} continue to be present in the understory, but large conifers are now the dominant tree species, having overtopped the \emph{P. tremuloides}. Smaller conifers are present in the midstory as well. Conifer species likely present include \emph{A. concolor, C. decurrens, P. ponderosa, P. menziesii}, and \emph{P. lambertiana}. (LandFire 2007a, LandFire 2007b, LandFire 2007c). This seral stage is analogous to the LDC seral stage for the SMC variant.

\paragraph{Succession Transition} In the absence of disturbance, patches in this seral stage will maintain, regardless of soil characteristics.

\paragraph{Wildfire Transition} High mortality wildfire (37\% of fires in this seral stage) will return the patch to ED–A. Low mortality wildfire (63\%) opens the stand up to LD-CA 54\% of the time.

\noindent\hrulefill


\paragraph{Late Development – Conifer with Aspen (LD–CA)}

\paragraph{Description} If stands are sufficiently protected from fire such that conifer species overtop \emph{P. tremuloides} and become large, they may be able to withstand some fire that more sensitive \emph{P. tremuloides} cannot. When this occurs, it creates a patch characterized by late development conifers, such as \emph{A. concolor, P. ponderosa}, or \emph{P. lambertiana}, and early seral \emph{P. tremuloides}. 

\paragraph{Succession Transition} Patches in the LD-CA seral stage persist for 70 years, at which time patches transition to LDC. 

\paragraph{Wildfire Transition} High mortality wildfire (14\% of fires in this seral stage) returns the patch to ED-A. Low mortality wildfire (86\%) maintains the stand in LD-CA. 

\noindent\hrulefill




\subsection*{Seral Stage Classification}
\begin{table}[]
\small
\centering
\caption{Classification of cover seral stage for SMC. Diameter at Breast Height (DBH) and Cover From Above (CFA) values taken from EVeg polygons. DBH categories are: null, 0-0.9'', 1-4.9'', 5-9.9'', 10-19.9'', 20-29.9'', 30''+. CFA categories are null, 0-10\%, 10-20\%, … , 90-100\%. Each row in the table below should be read with a boolean AND across each column of a row.}
\label{rfr_classification}
\begin{tabular}{@{}lrrrrr@{}}
\toprule
\textbf{\begin{tabular}[l]{@{}l@{}}Cover \\ Condition\end{tabular}} & \textbf{\begin{tabular}[r]{@{}r@{}}Overstory Tree \\ Diameter 1 \\ (DBH)\end{tabular}} & \textbf{\begin{tabular}[r]{@{}r@{}}Overstory Tree \\ Diameter 2 \\ (DBH)\end{tabular}} & \textbf{\begin{tabular}[r]{@{}r@{}}Total Tree\\ CFA (\%)\end{tabular}} & \textbf{\begin{tabular}[r]{@{}r@{}}Conifer \\ CFA (\%)\end{tabular}} & \textbf{\begin{tabular}[r]{@{}r@{}}Hardwood \\ CFA (\%)\end{tabular}} \\ \midrule
Early All        & null           & any & any    & any    & any  \\
Early All        & 0-4.9''         & any & any    & any    & any  \\
Mid Open         & 5-19.9''        & any & null   & null   & null \\
Mid Open         & 5-19.9''        & any & 0-40   & any    & any  \\
Mid Open         & 5-19.9''        & any & null   & 0-40   & null \\
Mid Moderate     & 5-19.9''        & any & 40-70  & any    & any  \\
Mid Moderate     & 5-19.9''        & any & null   & 40-70  & null \\
Mid Closed       & 5-19.9''        & any & 70-100 & any    & any  \\
Mid Closed       & 5-19.9''        & any & null   & 70-100 & any  \\
Late Closed      & 20''+           & any & 70-100 & any    & any  \\
Late Closed      & 20''+           & any & null   & 70-100 & any  \\
Late Moderate    & 20''+           & any & 40-70  & any    & any  \\
Late Moderate    & 20''+           & any & null   & 40-70  & any  \\
Late Open        & 20''+           & any & null   & null   & null \\
Late Open        & 20''+           & any & 0-40   & any    & any  \\
Late Open        & 20''+           & any & null   & 0-40   & null  \\ \bottomrule
\end{tabular}
\end{table}

SMC-ASP seral stages were assigned manually using NAIP 2010 Color IR imagery to assess seral stage.

\subsection*{Draft Model}
\begin{figure}[htbp]
\centering
\includegraphics[width=0.8\textwidth]{/Users/mmallek/Tahoe/Report3/images/state_trans_model.pdf}
\caption{State and Transition Model for Sierran Mixed Conifer Forest. Each dark grey box represents one of the seven seral stages for this landcover type. Each column of boxes represents a stage of development: early, middle, and late. Each row of boxes represents a different level of canopy cover: closed (70-100\%), moderate (40-70\%), and open (0-40\%). Transitions between states/seral stages may occur as a result of high mortality fire, low mortality fire, or succession. Specific pathways for each are denoted by the appropriate color line and arrow: red lines relate to high mortality fire, orange lines relate to low mortality fire, and green lines relate to natural succession.} 
\label{transmodel}
\end{figure}

\subsection*{References}
\begin{hangparas}{.25in}{1} 
Allen, Barbara H. ``Sierran Mixed Conifer (SMC).'' \emph{A Guide to Wildlife Habitats of California}, edited by Kenneth E. Mayer and William F. Laudenslayer. California Deparment of Fish and Game, 1988, updated 2005. \burl{http://www.dfg.ca.gov/biogeodata/cwhr/pdfs/SMC.pdf}. Accessed 4 December 2012.

``CalVeg Zone 1.'' Vegetation Descriptions. Vegetation Classification and Mapping.  11 December 2008. U.S. Forest Service. \burl{http://www.fs.usda.gov/Internet/FSE_DOCUMENTS/fsbdev3_046448.pdf}. Accessed 2 April 2013.
Estes, Becky. Personal communication, 15 August 2013.

LandFire. ``Biophysical Setting Models.'' Biophysical Setting 0610280: Mediterranean California Mesic Mixed Conifer Forest and Woodland. 2007a. LANDFIRE Project, U.S. Department of Agriculture, Forest Service; U.S. Department of the Interior. \burl{http://www.landfire.gov/national_veg_models_op2.php}. Accessed 9 November 2012.

LandFire. ``Biophysical Setting Models.'' Biophysical Setting 0610270: Mediterranean California Dry-Mesic Mixed Conifer Forest and Woodland. 2007b. LANDFIRE Project, U.S. Department of Agriculture, Forest Service; U.S. Department of the Interior. \burl{http://www.landfire.gov/national_veg_models_op2.php}. Accessed 9 November 2012.

LandFire. ``Biophysical Setting Models.'' Biophysical Setting 0610610: Inter-Mountain Basins Aspen-Mixed Conifer Forest and Woodland. 2007c. LANDFIRE Project, U.S. Department of Agriculture, Forest Service; U.S. Department of the Interior. \burl{http://www.landfire.gov/national_veg_models_op2.php}. Accessed 7 January 2013.

LandFire. ``Biophysical Setting Models.'' Biophysical Setting 0710220: Klamath-Siskiyou Upper Montane Serpentine Mixed Conifer Woodland. 2007d. LANDFIRE Project, U.S. Department of Agriculture, Forest Service; U.S. Department of the Interior. \burl{http://www.landfire.gov/national_veg_models_op2.php}. Accessed 30 November 2012.

O’Geen, Anthony T., Randy A. Dahlgren, and Daniel Sanchez-Mata. ``California Soils and Examples of Ultramafic Vegetation.'' In \emph{Terrestrial Vegetation of California, 3rd Edition}, edited by Michael Barbour, Todd Keeler-Wolf, and Allan A. Schoenherr, 71-106. Berkeley and Los Angeles: University of California Press, 2007. 

Safford, Hugh S. Personal communication.

Skinner, Carl N. and Chi-Ru Chang. ``Fire Regimes, Past and Present.'' \emph{Sierra Nevada Ecosystem Project: Final report to Congress, vol. II, Assessments and scientific basis for management options}. Davis: University of California, Centers for Water and Wildland Resources, 1996.

Van de Water, Kip M. and Hugh D. Safford. ``A Summary of Fire Frequency Estimates for California Vegetation Before Euro-American Settlement.'' \emph{Fire Ecology} 7.3 (2011): 26-57. doi: 10.4996/fireecology.0703026.

Verner, Jared. ``Aspen (ASP).'' ).'' \emph{A Guide to Wildlife Habitats of California}, edited by Kenneth E. Mayer and William F. Laudenslayer. California Deparment of Fish and Game, 1988. \burl{http://www.dfg.ca.gov/biogeodata/cwhr/pdfs/ASP.pdf}. Accessed 4 December 2012.


\end{hangparas}

