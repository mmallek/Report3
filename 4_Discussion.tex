\chapter{Results}

\section{Historic Range of Variability}

\subsection{Disturbance Regime}

This report focuses on the effects of wildfire as a natural disturbance; the impacts of other natural disturbances during the reference period were likely localized in time or space and therefore probably had far less impact on vegetation patterns over broad spatial and temporal scales than did fire.\todo{is this true?} In the sections below, we briefly describe the simulated disturbance regime (i.e., spatial extent and distribution, frequency and temporal variability). In this subsection, we refrain from describing variations among vegetation types - this will be accomplished in Subsection~\ref{subsec:HRVvegresponse}\todo{verify this}. Finally, it is important to realize that while the information below is presented as ``results,'' it could have easily been presented in the methods section as ``model calibration.'' Key spatial and temporal aspects of the disturbance regime were evaluated during preliminary calibration runs, and we subsequently adjusted model parameters to effect desired changes. Thus, while the information presented below does in fact represent results (output) of the simulation, it also represents a set of targets used to calibrate the model (i.e., adjust model parameters to achieve desired results). While this may seem a bit circular, it was a necessary process for a complex model such as \textsc{RMLands}. Moreover, our real emphasis was on quantifying the vegetation patterns and dynamics resulting from these disturbance processes.

\subsubsection{Disturbed Area}

% 174830 eligible hectares
% 181550 hectares in core
% check math using Wildfire_darea_trajectory.csv
Approximately 96\% of the landscape was eligible for wildfire disturbance (all cover types except Barren and Water). As expected, the frequency and extent of simulated wildfires varied across timesteps (Figures~\ref{fig:darea} and~\ref{fig:darea_hist}). Remarkably, given the rotation interval and percent mortality expected over time on this landscape, large proportions of the project area burned each (5-year) timestep. On average, at least 10\% of the landscape burned at some combination of low and high mortality every ten years. Timesteps with area burned of 30\% or less were the most frequently observed. Fires covering at least 25\% of the landscape burned approximately every 25 years, and half or more of the landscape burned at a 200 year interval\todo{how did I calculate this?}. The smallest area disturbed over the course of the simulation was 0.26\%, while the largest was 66.2\% (of which 22.91\% was high mortality). In general, within a given timestep about a third of the disturbed area burned as high mortality (Table~\ref{tab:darea}. High mortality fires do include the burning of early development vegetation, including chaparral.

\begin{table}[!htbp]
\centering
\caption{Summary statistics for area disturbed by wildfire during the simulation. Values are expressed as percentage of the landscape eligible for disturbance that was actually disturbed.}
\label{tab:darea}
\begin{tabular}{@{}llll@{}}
\toprule
\textbf{\begin{tabular}[c]{@{}l@{}}Summary Statistic \\ (disturbed area/timestep)\end{tabular}}    & \textbf{Low Mortality}   & \textbf{High Mortality}    & \textbf{Any Mortality}   \\
\midrule
Minimum        &    0.22             & 0.03                  &     0.26            \\
Maximum        &   43.27             & 22.91                 &     66.2            \\
 Median        &    9.24             & 3.98                  &     13.12           \\
   Mean        &   11.58             & 5.17                  &     16.75           \\
\bottomrule
\end{tabular}
\end{table}

%\begin{verbatim}
%     summary statistic mort.low mort.high mort.any
%minimum darea/timestep     0.22      0.03     0.26
%maximum darea/timestep    43.27     22.91     66.2
% median darea/timestep     9.24      3.98    13.12
%   mean darea/timestep    11.58      5.17    16.75
%\end{verbatim}

\begin{figure}[!htbp]
\centering
\includegraphics[width=0.9\textwidth]{/Users/mmallek/Tahoe/R/Rplots/November2014/darea.png}
\caption{Disturbance trajectory for wildfire during the simulation. The first timestep is 40 because we excluded earlier timesteps as equilibration. Dark blue values represent high mortality fire, while light blue values represent low mortality fire and are stacked on top of high mortality.}
\label{fig:darea}
\end{figure}

\begin{figure}[!htbp]
\centering
\includegraphics[width=0.6\textwidth]{/Users/mmallek/Tahoe/Report2/images/darea_hist.png}
\caption{Histogram of percent of landscape disturbed by wildfire during the simulation. The distribution is substantially right-skewed, and most fires burn less than 20\% of the eligible landscape.}
\label{fig:darea_hist}
\end{figure}



\subsubsection{Disturbance Size}
As described in Section \ref{subsubsec:distparams}, we specified a target set of disturbance sizes. Because wildfire has many stochastic components, we do not expect the model results to match these targets exactly. Figure~\ref{fig:dsize} compares the observed and target disturbance size distribution.

%\begin{verbatim}
%$`Wildfire disturbance size`$`run number 1`
%   bin obs.freq obs.proportion target.freq target.proportion
%     5   135327   0.7797220526        8464      0.8801996672
%    25    21339   0.1229502529         378      0.0393094842
%   125     5366   0.0309176183         316      0.0328618968
%   625     5011   0.0288721926         293      0.0304700499
%  3125     4803   0.0276737460         111      0.0115432612
% 15625     1503   0.0086599292          48      0.0049916805
% 78125      130   0.0007490291           6      0.0006239601
%133828       79   0.0004551792           0      0.0000000000
%\end{verbatim}

\begin{figure}[!htbp]
\centering
\includegraphics[width=0.7\textwidth]{/Users/mmallek/Tahoe/R/Rplots/November2014/dsize.png}
\caption{Side by side barplot of the observed and target wildfire size distribution for our 500-timestep long run of the model.}
\label{fig:dsize}
\end{figure}



\subsubsection{Effect of Climate} \todo{Do I say I'll do this in methods?}Climate does have a positive relationship with disturbed area, as expected (Figure~\ref{fig:climate_darea}. We show here a fitted line, but note the heteroskedastic variance about the mean. The relationship is fairly weak. During wetter-than-average years, we see less disturbed area. Over 20\% of the landscape burned only in timesteps during which the climate parameter was at least 0.75. However, over 50\% of the landscape burned in a few timesteps less than the average value, 1. Overall we observe that as climate shifts from wet to drought, the disturbed area increases. Climate also has a weak effect on the size of individual fires (Figure~\ref{fig:climate_dsize}). Fire size is also influenced by vegetation susceptibility and the specified disturbance size distribution. Figure~\ref{fig:compare_clim_darea} illustrates the climate parameter values and disturbed area proportion of the landscape for a subset of timesteps during the simulation.

\begin{figure}[!htbp]
  \centering
  \subfloat[][]{
    \centering
    \includegraphics[width=0.5\textwidth]{/Users/mmallek/Tahoe/Report2/images/climate_darea.png}
    \label{fig:climate_darea}
  }%
  %\qquad
  \subfloat[][]{
    \includegraphics[width=0.5\textwidth]{/Users/mmallek/Tahoe/Report2/images/climate_dsize.png}
    \label{fig:climate_dsize}
  }
  \caption{(a) Plot of the climate parameter and disturbed area value for each timestep of the simulation (excluding the  equilibration period). A linear model has been fit to the data and is shown as a blue line; the grey shaded area represents  the 95\% confidence interval around the mean. (b) Plot showing the size of each individual wildfire and the climate parameter value in effect at the time of disturbance for each disturbance during the simulation (excluding the equilibration period).}
  \label{fig:climate_disturbance}
\end{figure}

\begin{figure}[!htbp]
\centering
\includegraphics[width=0.8\textwidth]{/Users/mmallek/Tahoe/Report2/images/climate_darea_vert.png}
\caption{Climate parameter and proportion of eligible landscape disturbed by wildfire for timesteps 250 to 310 of the simulation.}
\label{fig:compare_clim_darea}
\end{figure}

\clearpage
\subsubsection{Effect of Topographic Position} \todo{not sure if I said I'd do this one either}

The topographic position index value for a given cell acts as an input into the susceptibility and mortality values otherwise defined for that cover type and condition class combination. In general, cells with smaller TPI values had reduced susceptibility and mortality. Early development and open canopy conditions tend to result from fire, and we predicted that an increase in fires and in the likelihood of high mortality fire would lead to a decrease in the average canopy cover values for cells with large TPI values. Table~\ref{tab:tpi_cc} displays the results for this simulation for the nine focal cover types. All but one (\textsc{ocfw\_u}) show decreased average canopy cover as TPI increases, with the decrease ranging from 3.3\% in Mixed Evergreen - Mesic to 36.4\% in Sierran Mixed Conifer - Ultramafic. Figure~\ref{fig:tpi_cc} shows the plotted data and fitted linear regression line for each of the nine focal types. \todo{also have a faceted figure that shows all types on the same y axis/scale - for appendix?}

\begin{table}[!htbp]
\caption{For each cover type on the landscape, the percent change in canopy cover from the minimum TPI value for that cover type to the maximum TPI value.}
\label{tab:tpi_cc}
\begin{tabular}{@{}llllll@{}}
\toprule
\small \textbf{\begin{tabular}[c]{@{}l@{}}Cover \\ Name\end{tabular}} & \small \textbf{\begin{tabular}[c]{@{}l@{}}Minimum \\ TPI\end{tabular}} & \small \textbf{\begin{tabular}[c]{@{}l@{}}Maximum \\ TPI\end{tabular}} & \small \textbf{\begin{tabular}[c]{@{}l@{}}Average Canopy \\Cover at \\ Minimum TPI\end{tabular}} & \small \textbf{\begin{tabular}[c]{@{}l@{}}Average Canopy \\ Cover at \\ Maximum TPI\end{tabular}}  & \small \textbf{\begin{tabular}[c]{@{}l@{}}Percent \\ Change in \\ Canopy \\ Cover\end{tabular}} \\ \midrule
MEG\_M       & -300                 & 300                  & 77.4         & 74.9              &  -3.3      \\
MEG\_X       & -299                 & 300                  & 77.8         & 75.0              &  -3.6      \\
OCFW         & -300                 & 300                  & 57.7         & 53.2              &  -7.8      \\
OCFW\_U      & -300                 & 300                  & 21.3         & 22.0              &   2.9       \\
RFR\_M       & -300                 & 300                  & 72.0         & 64.0              & -11.2     \\
RFR\_X       & -259                 & 300                  & 40.9         & 29.1              & -28.8     \\
SMC\_M       & -300                 & 300                  & 58.8         & 53.5              &  -9.1      \\
SMC\_U       & -300                 & 300                  & 39.2         & 25.0              & -36.4     \\
SMC\_X       & -300                 & 300                  & 30.8         & 24.4              & -20.9     \\ \bottomrule
\end{tabular}
\end{table}


\begin{figure}[!htbp]
  \centering
  \subfloat[][]{
    \centering
    \includegraphics[width=0.33\textwidth]{/Users/mmallek/Tahoe/Report2/images/TPI_cc_megm.png}
    \label{fig:tpi_cc_megm}
  }%
  \subfloat[][]{
    \includegraphics[width=0.33\textwidth]{/Users/mmallek/Tahoe/Report2/images/TPI_cc_megx.png}
    \label{fig:tpi_cc_megx}
  }%
  \subfloat[][]{
    \includegraphics[width=0.33\textwidth]{/Users/mmallek/Tahoe/Report2/images/TPI_cc_ocfw.png}
    \label{fig:ocfw}
    }

  \subfloat[][]{
    \centering
    \includegraphics[width=0.33\textwidth]{/Users/mmallek/Tahoe/Report2/images/TPI_cc_ocfwu.png}
    \label{fig:tpi_cc_ocfwu}
  }%
  \subfloat[][]{
    \includegraphics[width=0.33\textwidth]{/Users/mmallek/Tahoe/Report2/images/TPI_cc_rfrm.png}
    \label{fig:tpi_cc_rfrm}
  }%
  \subfloat[][]{
    \includegraphics[width=0.33\textwidth]{/Users/mmallek/Tahoe/Report2/images/TPI_cc_rfrx.png}
    \label{fig:tpi_cc_rfrx}
    }

  \subfloat[][]{
    \centering
    \includegraphics[width=0.33\textwidth]{/Users/mmallek/Tahoe/Report2/images/TPI_cc_smcx.png}
    \label{fig:tpi_cc_smcm}
  }%
  \subfloat[][]{
    \includegraphics[width=0.33\textwidth]{/Users/mmallek/Tahoe/Report2/images/TPI_cc_smcu.png}
    \label{fig:tpi_cc_smcu}
  }%
  \subfloat[][]{
    \includegraphics[width=0.33\textwidth]{/Users/mmallek/Tahoe/Report2/images/TPI_cc_smcx.png}
    \label{fig:tpi_cc_smcx}
    }
  \caption{Average canopy cover for the nine focal cover types during the simulated. Each blue point represents one pixel of an individual cover type on the landscape grid. The black line is the result of a linear regression fit to the data. Note, the $y$ axis is scaled differently across cover types in order to more easily focus on the data for each type. Table~\ref{tab:tpi_cc} provides the numerical representation of the shift from minimum to maximum TPI values for each cover type. (a) Mixed Evergreen - Mesic; (b) Mixed Evergreen - Xeric; (c) Oak-Conifer Forest and Woodland; (d) Oak-Conifer Forest and Woodland - Ultramafic; (e) Red Fir - Mesic; (f) Red Fir - Xeric; (g) Sierran Mixed Conifer - Mesic; (h) Sierran Mixed Conifer - Ultramafic; (i) Sierran Mixed Conifer - Xeric.}
  \label{fig:tpi_cc}
\end{figure}



\newpage
\subsubsection{Fire Rotation}
We present here the results for non-static cover types whose extent is at least 1000 ha. Full results are presented in \todo{Appendix?}. As previously discussed, these results could have been presented in the methods section. Each of the nine cover types shown here were calibrated to within 10\% of their target values, which were based on empirical published values and local expert opinion.

\begin{table}[!htbp]
\centering
\caption{Fire rotation for the nine cover types whose extent cover at least 1000 ha.}
\begin{tabular}{@{}llll@{}}
\toprule
Land Cover Type                              & \begin{tabular}[c]{@{}l@{}}Low Mortality \\ Fire Rotation\end{tabular} & \begin{tabular}[c]{@{}l@{}}High Mortality \\ Fire Rotation\end{tabular} & \begin{tabular}[c]{@{}l@{}}All Fires \\ Rotation\end{tabular} \\ \midrule
Mixed Evergreen - Mesic                      & 63                          & 534                          & 57                 \\
Mixed Evergreen - Xeric                      & 51                          & 472                          & 46                 \\
Oak-Conifer Forest and Woodland              & 33                          & 100                          & 25                 \\
\begin{tabular}[c]{@{}l@{}}Oak-Conifer Forest and \\ Woodland -  Ultramafic\end{tabular} & 48                          & 1192                         & 46                 \\
Red Fir - Mesic                              & 101                         & 164                          & 62                 \\
Red Fir - Xeric                              & 59                          & 117                          & 39                 \\
Sierran Mixed Conifer - Mesic                & 39                          & 115                          & 29                 \\
Sierran Mixed Conifer - Ultramafic           & 106                         & 196                          & 69                 \\
Sierran Mixed Conifer - Xeric                & 40                          & 62                           & 24                 \\
Total                                        & 45                          & 100                          & 31                 \\ \bottomrule
\end{tabular}
\end{table}


\subsubsection{Population Return Interval}
Overall, the point-specific return interval (grand mean) for all eligible cells ranged from 17 years to \textgreater 2500 years (cells that never burned during the simulation) for both classes of wildfire mortality (Figure~\ref{fig:preturn}. The median return interval across all cover types was 42 years for low mortality fire, 111 year for high mortality fire, and 29 years for any fire. The population return interval plots and maps specific to each of the nine focal cover types follow (Figures~\ref{fig:preturn_megm} through \ref{fig:preturn_smcx}.). We compare the current landscape's seral stage distribution to the simulated distribution and compute the HRV departure index in Tables~\ref{tab:covcond1} and \ref{tab:covcond2}

\begin{figure}[!htbp]
  \centering
  \subfloat[][]{
    \centering
    \includegraphics[height=0.4\textheight]{/Users/mmallek/Tahoe/R/Rplots/November2014/preturn_all.png}
    \label{fig:preturn_plot}
  }%
  \qquad
  \subfloat[][]{
    \includegraphics[height=0.4\textheight]{/Users/mmallek/Tahoe/Report2/images/fri_all.png}
    \label{fig:preturn_map}
  }
  \caption{(a) Population return interval (average number of years between fires) distribution for the full landscape under study. The population return interval is the point-specific interval, sometimes described as the ``grand mean'' for a given point. (b) Spatial depiction of fire return intervals across the landscape, for all cover types, in terms of fire return interval. The value at any given cell is the point-specific return interval.}
  \label{fig:preturn}
\end{figure}


\begin{figure}[!htbp]
  \centering
  \subfloat[][]{
    \centering
    \includegraphics[width=0.5\textwidth]{/Users/mmallek/Tahoe/R/Rplots/November2014/preturn_megm.png}
    }%
  \subfloat[][]{
    \includegraphics[width=0.5\textwidth]{/Users/mmallek/Tahoe/Report2/images/fri_megm.png}
    }
  \caption{(a) Population return interval (average number of years between fires) distribution for Mixed Evergreen - Mesic. (b) Spatial depiction of fire return intervals across the landscape. Cover types other than Mixed Evergreen - Mesic are partially obscured in grey. The value at any given cell is the point-specific return interval, which ranges from 19 years to \textgreater 500 years.}
    \label{fig:preturn_megm}
\end{figure}

\begin{figure}[!htbp]
  \centering
  \subfloat[][]{
    \centering
    \includegraphics[width=0.5\textwidth]{/Users/mmallek/Tahoe/R/Rplots/November2014/preturn_megx.png}
    }%
  \subfloat[][]{
    \includegraphics[width=0.5\textwidth]{/Users/mmallek/Tahoe/Report2/images/fri_megx.png}
    }
  \caption{(a) Population return interval (average number of years between fires) distribution for Mixed Evergreen - Xeric.  (b) Spatial depiction of fire return intervals across the landscape. Cover types other than Mixed Evergreen - Xeric are partially obscured in grey. The value at any given cell is the point-specific return interval, which ranges from 20 years to \textgreater 500 years.}
\label{fig:preturn_megx}
\end{figure}

\begin{figure}[!htbp]
  \centering
  \subfloat[][]{
    \centering
    \includegraphics[width=0.5\textwidth]{/Users/mmallek/Tahoe/R/Rplots/November2014/preturn_ocfw.png}
    }%
  \subfloat[][]{
    \includegraphics[width=0.5\textwidth]{/Users/mmallek/Tahoe/Report2/images/fri_ocfw.png}
    }
  \caption{(a) Population return interval (average number of years between fires) distribution for Oak-Conifer Forest and Woodland.  (b) Spatial depiction of fire return intervals across the landscape. Cover types other than Oak-Conifer Forest and Woodland are partially obscured in grey. The value at any given cell is the point-specific return interval, which ranges from 17 years to \textgreater 500 years.}
\label{fig:preturn_ocfw}
\end{figure}

\begin{figure}[!htbp]
  \centering
  \subfloat[][]{
    \centering
    \includegraphics[width=0.5\textwidth]{/Users/mmallek/Tahoe/R/Rplots/November2014/preturn_ocfwu.png}
    }%
  \subfloat[][]{
    \includegraphics[width=0.5\textwidth]{/Users/mmallek/Tahoe/Report2/images/fri_ocfwu.png}
    }
  \caption{(a) Population return interval (average number of years between fires) distribution for Oak-Conifer Forest and Woodland - Ultramafic.  (b) Spatial depiction of fire return intervals across the landscape. Cover types other than Oak-Conifer Forest and Woodland - Ultramafic are partially obscured in grey. The value at any given cell is the point-specific return interval, which ranges from 21 years to \textgreater 500 years.}
\label{fig:preturn_ocfwu}
\end{figure}

\begin{figure}[!htbp]
  \centering
  \subfloat[][]{
    \centering
    \includegraphics[width=0.5\textwidth]{/Users/mmallek/Tahoe/R/Rplots/November2014/preturn_rfrm.png}
    }%
  \subfloat[][]{
    \includegraphics[width=0.5\textwidth]{/Users/mmallek/Tahoe/Report2/images/fri_rfrm.png}
    }
  \caption{(a) Population return interval (average number of years between fires) distribution for Red Fir - Mesic.  (b) Spatial depiction of fire return intervals across the landscape. Cover types other than Red Fir - Mesic are partially obscured in grey. The value at any given cell is the point-specific return interval, which ranges from 21 years to \textgreater 500 years.}
\label{fig:preturn_rfrm}
\end{figure}

\begin{figure}[!htbp]
  \centering
  \subfloat[][]{
    \centering
    \includegraphics[width=0.5\textwidth]{/Users/mmallek/Tahoe/R/Rplots/November2014/preturn_rfrx.png}
    }%
  \subfloat[][]{
    \includegraphics[width=0.5\textwidth]{/Users/mmallek/Tahoe/Report2/images/fri_rfrx.png}
    }
  \caption{(a) Population return interval (average number of years between fires) distribution for Red Fir - Xeric.  (b) Spatial depiction of fire return intervals across the landscape. Cover types other than Red Fir - Xeric are partially obscured in grey. The value at any given cell is the point-specific return interval, which ranges from 20 years to \textgreater 500 years.}
\label{fig:preturn_rfrx}
\end{figure}

\begin{figure}[!htbp]
  \centering
  \subfloat[][]{
    \centering
    \includegraphics[width=0.5\textwidth]{/Users/mmallek/Tahoe/R/Rplots/November2014/preturn_smcm.png}
    }%
  \subfloat[][]{
    \includegraphics[width=0.5\textwidth]{/Users/mmallek/Tahoe/Report2/images/fri_smcm.png}
    }
  \caption{(a) Population return interval (average number of years between fires) distribution for Sierran Mixed Conifer - Mesic.  (b) Spatial depiction of fire return intervals across the landscape. Cover types other than Sierran Mixed Conifer - Mesic are partially obscured in grey. The value at any given cell is the point-specific return interval, which ranges from 18 years to \textgreater 500 years.}
\label{fig:preturn_smcm}
\end{figure}

\begin{figure}[!htbp]
  \centering
  \subfloat[][]{
    \centering
    \includegraphics[width=0.5\textwidth]{/Users/mmallek/Tahoe/R/Rplots/November2014/preturn_smcu.png}
    }%
  \subfloat[][]{
    \includegraphics[width=0.5\textwidth]{/Users/mmallek/Tahoe/Report2/images/fri_smcu.png}
    }
  \caption{(a) Population return interval (average number of years between fires) distribution for Sierran Mixed Conifer - Ultramafic.  (b) Spatial depiction of fire return intervals across the landscape. Cover types other than Sierran Mixed Conifer - Ultramafic are partially obscured in grey. The value at any given cell is the point-specific return interval, which ranges from 20 years to \textgreater 500 years.}
\label{fig:preturn_smcu}
\end{figure}

\begin{figure}[!htbp]
  \centering
  \subfloat[][]{
    \centering
    \includegraphics[width=0.5\textwidth]{/Users/mmallek/Tahoe/R/Rplots/November2014/preturn_smcx.png}
    }%
  \subfloat[][]{
    \includegraphics[width=0.5\textwidth]{/Users/mmallek/Tahoe/Report2/images/fri_smcx.png}
    }
  \caption{(a) Population return interval (average number of years between fires) distribution for Sierran Mixed Conifer - Xeric.  (b) Spatial depiction of fire return intervals across the landscape. Cover types other than Sierran Mixed Conifer - Xeric are partially obscured in grey. The value at any given cell is the point-specific return interval, which ranges from 18 years to \textgreater 500 years.}
\label{fig:preturn_smcx}
\end{figure}

\clearpage

%%%%%%%%%%%%%%%%%%%%%%%%%%%%%%%%
%%%%%%%%%%%%%%%%%%%%%%%%%%%%%%%%
%%%%%%%%%%%%%%%%%%%%%%%%%%%%%%%%
\pagebreak[4]
\subsection{Vegetation Response}
\label{subsec:HRVvegresponse}

\subsubsection{Landscape Composition}

The distribution of area among condition classes within all cover types fluctuated over time, as expected. The relative proportion of each condition class also varied across cover types. In general, the condition class distribution appeared to be in dynamic equilibrium, despite considerable variability from timestep to timestep. The exception is the Oak-Conifer Forest and Woodland cover type, which did not appear to reach equilibrium until around timestep 220. The cover-condition dynamics and current seral stage distribution plots specific to each of the nine focal cover types follow (Figures~\ref{fig:covcond_megm} through \ref{fig:covcond_smcx}.).

A few patterns emerge across the nine focal cover types. A numerical representation of these dynamics is available in Tables~\ref{tab:covcond1} to \ref{tab:covcond3}. In general, early seral conditions were more common during the simulated historic period than on the current landscape. In mesic red fir and sierran mixed conifer forests, closed canopy conditions occupied a greater proportion of the total landscape during the simulated historic period than on the current landscape. In xeric mixed conifer forests, closed canopies were uncommon throughout the simulated period, but occupy 36.68\% of the current landscape.

\begin{figure}[!htbp]
  \centering
  \subfloat[][]{
    \centering
    \includegraphics[width=0.6\textwidth]{/Users/mmallek/Tahoe/R/Rplots/November2014/covcond_megm.png}
    }%
  \subfloat[][]{
  \centering
  \includegraphics[height=2.65in]{/Users/mmallek/Tahoe/R/Rplots/November2014/covcond_current_megm.png}
    }
  \caption{(a) Cover-Condition dynamics for Mixed Evergreen - Mesic. The black vertical line at 40 timesteps marks the end of the equilibration period used in this study. (b) Current seral stage distribution for Mixed Evergreen - Mesic.}
\label{fig:covcond_megm}
\end{figure}

%\clearpage

\begin{figure}[!htbp]
  \centering
  \subfloat[][]{
    \centering
    \includegraphics[width=0.6\textwidth]{/Users/mmallek/Tahoe/R/Rplots/November2014/covcond_megx.png}
    }%
  \subfloat[][]{
    \includegraphics[height=2.65in]{/Users/mmallek/Tahoe/R/Rplots/November2014/covcond_current_megx.png}
    }
  \caption{(a) Cover-Condition dynamics for Mixed Evergreen - Xeric. The black vertical line at 40 timesteps marks the end of the equilibration period used in this study. (b) Current seral stage distribution for Mixed Evergreen - Xeric.} 
  \label{fig:covcond_megx}
\end{figure}

\begin{figure}[!htbp]
  \centering
  \subfloat[][]{
    \centering
    \includegraphics[width=0.6\textwidth]{/Users/mmallek/Tahoe/R/Rplots/November2014/covcond_ocfw.png}
    }%
  \subfloat[][]{
    \includegraphics[height=2.65in]{/Users/mmallek/Tahoe/R/Rplots/November2014/covcond_current_ocfw.png}
    }
  \caption{(a) Cover-Condition dynamics for Oak-Conifer Forest and Woodland. The black vertical line at 40 timesteps marks the end of the equilibration period used in this study. (b) Current seral stage distribution for Oak-Conifer Forest and Woodland.} 
  \label{fig:covcond_ocfw}
\end{figure}

\begin{figure}[!htbp]
  \centering
  \subfloat[][]{
    \centering
    \includegraphics[width=0.6\textwidth]{/Users/mmallek/Tahoe/R/Rplots/November2014/covcond_ocfwu.png}
    }%
  \subfloat[][]{
    \includegraphics[height=2.65in]{/Users/mmallek/Tahoe/R/Rplots/November2014/covcond_current_ocfwu.png}
    }
  \caption{(a) Cover-Condition dynamics for Oak-Conifer Forest and Woodland - Ultramafic. The black vertical line at 40 timesteps marks the end of the equilibration period used in this study. (b) Current seral stage distribution for Oak-Conifer Forest and Woodland - Ultramafic.} 
  \label{fig:covcond_ocfwu}
\end{figure}

\begin{figure}[!htbp]
  \centering
  \subfloat[][]{
    \centering
    \includegraphics[width=0.6\textwidth]{/Users/mmallek/Tahoe/R/Rplots/November2014/covcond_rfrm.png}
    }%
  \subfloat[][]{
    \includegraphics[height=2.65in]{/Users/mmallek/Tahoe/R/Rplots/November2014/covcond_current_rfrm.png}
    }
  \caption{(a) Cover-Condition dynamics for Red Fir - Mesic. The black vertical line at 40 timesteps marks the end of the equilibration period used in this study. (b) Current seral stage distribution for Red Fir - Mesic.} 
  \label{fig:covcond_rfrm}
\end{figure}

\begin{figure}[!htbp]
  \centering
  \subfloat[][]{
    \centering
    \includegraphics[width=0.6\textwidth]{/Users/mmallek/Tahoe/R/Rplots/November2014/covcond_rfrx.png}
    }%
  \subfloat[][]{
    \includegraphics[height=2.65in]{/Users/mmallek/Tahoe/R/Rplots/November2014/covcond_current_rfrx.png}
    }
  \caption{(a) Cover-Condition dynamics for Red Fir - Xeric. The black vertical line at 40 timesteps marks the end of the equilibration period used in this study. (b) Current seral stage distribution for Red Fir - Xeric.} 
  \label{fig:covcond_rfrx}
\end{figure}

\begin{figure}[!htbp]
  \centering
  \subfloat[][]{
    \centering
    \includegraphics[width=0.6\textwidth]{/Users/mmallek/Tahoe/R/Rplots/November2014/covcond_smcm.png}
    }%
  \subfloat[][]{
    \includegraphics[height=2.65in]{/Users/mmallek/Tahoe/R/Rplots/November2014/covcond_current_smcm.png}
    }
  \caption{(a) Cover-Condition dynamics for Sierran Mixed Conifer - Mesic. The black vertical line at 40 timesteps marks the end of the equilibration period used in this study. (b) Current seral stage distribution for Sierran Mixed Conifer - Mesic.} 
  \label{fig:covcond_smcm}
\end{figure}

\begin{figure}[!htbp]
  \centering
  \subfloat[][]{
    \centering
    \includegraphics[width=0.6\textwidth]{/Users/mmallek/Tahoe/R/Rplots/November2014/covcond_smcu.png}
    }%
  \subfloat[][]{
    \includegraphics[height=2.65in]{/Users/mmallek/Tahoe/R/Rplots/November2014/covcond_current_smcu.png}
    }
  \caption{(a) Cover-Condition dynamics for Sierran Mixed Conifer - Ultramafic. The black vertical line at 40 timesteps marks the end of the equilibration period used in this study. (b) Current seral stage distribution for Sierran Mixed Conifer - Ultramafic.} 
  \label{fig:covcond_smcu}
\end{figure}

\begin{figure}[!htbp]
  \centering
  \subfloat[][]{
    \centering
    \includegraphics[width=0.6\textwidth]{/Users/mmallek/Tahoe/R/Rplots/November2014/covcond_smcx.png}
    }%
  \subfloat[][]{
    \includegraphics[height=2.65in]{/Users/mmallek/Tahoe/Report2/images/covcond_current_smcx.png}
    }
  \caption{(a) Cover-Condition dynamics for Sierran Mixed Conifer - Xeric. The black vertical line at 40 timesteps marks the end of the equilibration period used in this study. (b) Current seral stage distribution for Sierran Mixed Conifer - Xeric.} 
  \label{fig:covcond_smcx}
\end{figure}

\begin{sidewaystable}[!htbp]
\caption{Range of variation in landscape structure, illustrating the cover-condition class dynamics for Mixed Evergreen - Mesic (\textsc{meg\_m}), Mixed Evergreen - Xeric (\textsc{meg\_x}), Oak-Conifer Forest and Woodland (\textsc{ocfw}), and Oak-Conifer Forest and Woodland - Ultramafic (\textsc{ocfw\_u}). For condition class abbreviations, see Table~\ref{condtable}.}
\label{tab:covcond1}
\begin{tabular}{@{}lllllllllllll@{}}
\toprule
\footnotesize \textbf{\begin{tabular}[c]{@{}l@{}}Land \\ Cover\\ Type\end{tabular}} & \footnotesize \textbf{\begin{tabular}[c]{@{}l@{}}Condition \\ Class\end{tabular}} & \footnotesize \textbf{srv0\%} & \footnotesize \textbf{srv5\%} & \footnotesize \textbf{srv25\%} & \footnotesize \textbf{srv50\%} & \footnotesize \textbf{srv75\%} & \footnotesize \textbf{srv95\%} & \footnotesize \textbf{srv100\%}   & \footnotesize \textbf{\begin{tabular}[c]{@{}l@{}}Current\\ \%cover\end{tabular}} & \textbf{\begin{tabular}[c]{@{}l@{}}Current\\ \%srv\end{tabular}} & \textbf{\begin{tabular}[c]{@{}l@{}}Departure\\ Index\end{tabular}} \\ \midrule
\footnotesize \textsc{meg\_m}      & \footnotesize \textsc{early\_all}                & \footnotesize 0.67            & \footnotesize 1.33            & \footnotesize 2.3              & \footnotesize 3.29             & \footnotesize 4.93             & \footnotesize 7.59             & \footnotesize 9.89       & \footnotesize 8.21     & \footnotesize 97     & \footnotesize 94       \\
\footnotesize \textsc{meg\_m}      & \footnotesize \textsc{mid\_cl   }                & \footnotesize 0.01            & \footnotesize 0.06            & \footnotesize 0.18             & \footnotesize 0.38             & \footnotesize 0.72             & \footnotesize 2.18             & \footnotesize 5.09       & \footnotesize 36.53    & \footnotesize 100    & \footnotesize 100      \\
\footnotesize \textsc{meg\_m}      & \footnotesize \textsc{mid\_mod  }                & \footnotesize 0.47            & \footnotesize 0.87            & \footnotesize 1.53             & \footnotesize 2.27             & \footnotesize 3.67             & \footnotesize 5.67             & \footnotesize 9.06       & \footnotesize 9.76     & \footnotesize 100    & \footnotesize 100      \\
\footnotesize \textsc{meg\_m}      & \footnotesize \textsc{mid\_op   }                & \footnotesize 0               & \footnotesize 0.03            & \footnotesize 0.06             & \footnotesize 0.11             & \footnotesize 0.2              & \footnotesize 0.37             & \footnotesize 0.76       & \footnotesize 6.37     & \footnotesize 100    & \footnotesize 100      \\
\footnotesize \textsc{meg\_m}      & \footnotesize \textsc{late\_cl  }                & \footnotesize 63.9            & \footnotesize 71.24           & \footnotesize 77.22            & \footnotesize 81.07            & \footnotesize 84.41            & \footnotesize 88.12            & \footnotesize 91.68      & \footnotesize 29.31    & \footnotesize 0      & \footnotesize -100     \\
\footnotesize \textsc{meg\_m}      & \footnotesize \textsc{late\_mod }                & \footnotesize 3.75            & \footnotesize 5.58            & \footnotesize 7.37             & \footnotesize 9.16             & \footnotesize 11.33            & \footnotesize 14.59            & \footnotesize 18.04      & \footnotesize 7.31     & \footnotesize 25     & \footnotesize -50      \\
\footnotesize \textsc{meg\_m}      & \footnotesize \textsc{late\_op  }                & \footnotesize 0.77            & \footnotesize 1.3             & \footnotesize 1.81             & \footnotesize 2.56             & \footnotesize 3.51             & \footnotesize 4.98             & \footnotesize 7.01       & \footnotesize 2.5      & \footnotesize 49     & \footnotesize -2       \\
\footnotesize \textsc{meg\_x}      & \footnotesize \textsc{early\_all}                & \footnotesize 0.71            & \footnotesize 1.47            & \footnotesize 2.87             & \footnotesize 3.92             & \footnotesize 5.3              & \footnotesize 8.01             & \footnotesize 10.86      & \footnotesize 10.88    & \footnotesize 100    & \footnotesize 100      \\
\footnotesize \textsc{meg\_x}      & \footnotesize \textsc{mid\_cl   }                & \footnotesize 0.02            & \footnotesize 0.05            & \footnotesize 0.19             & \footnotesize 0.4              & \footnotesize 0.85             & \footnotesize 1.89             & \footnotesize 5.3        & \footnotesize 48.8     & \footnotesize 100    & \footnotesize 100      \\
\footnotesize \textsc{meg\_x}      & \footnotesize \textsc{mid\_mod  }                & \footnotesize 0.5             & \footnotesize 0.99            & \footnotesize 1.98             & \footnotesize 2.85             & \footnotesize 3.82             & \footnotesize 5.98             & \footnotesize 9.03       & \footnotesize 9.39     & \footnotesize 100    & \footnotesize 100      \\
\footnotesize \textsc{meg\_x}      & \footnotesize \textsc{mid\_op   }                & \footnotesize 0.01            & \footnotesize 0.03            & \footnotesize 0.08             & \footnotesize 0.13             & \footnotesize 0.23             & \footnotesize 0.41             & \footnotesize 0.71       & \footnotesize 12.87    & \footnotesize 100    & \footnotesize 100      \\
\footnotesize \textsc{meg\_x}      & \footnotesize \textsc{late\_cl  }                & \footnotesize 64.45           & \footnotesize 70.72           & \footnotesize 76.78            & \footnotesize 79.98            & \footnotesize 82.89            & \footnotesize 87.05            & \footnotesize 89.89      & \footnotesize 12.84    & \footnotesize 0      & \footnotesize -100     \\
\footnotesize \textsc{meg\_x}      & \footnotesize \textsc{late\_mod }                & \footnotesize 4.44            & \footnotesize 6.3             & \footnotesize 8.03             & \footnotesize 9.71             & \footnotesize 11.76            & \footnotesize 14.77            & \footnotesize 19.2       & \footnotesize 3.84     & \footnotesize 0      & \footnotesize -100     \\
\footnotesize \textsc{meg\_x}      & \footnotesize \textsc{late\_op  }                & \footnotesize 0.72            & \footnotesize 1.15            & \footnotesize 1.68             & \footnotesize 2.16             & \footnotesize 2.93             & \footnotesize 4.07             & \footnotesize 5.67       & \footnotesize 1.38     & \footnotesize 13     & \footnotesize -74      \\
\footnotesize \textsc{ocfw}        & \footnotesize \textsc{early\_all}                & \footnotesize 6.36            & \footnotesize 8.99            & \footnotesize 12.92            & \footnotesize 16.16            & \footnotesize 19.91            & \footnotesize 25.51            & \footnotesize 30.07      & \footnotesize 19.97    & \footnotesize 76     & \footnotesize 52       \\
\footnotesize \textsc{ocfw}        & \footnotesize \textsc{mid\_cl   }                & \footnotesize 5.33            & \footnotesize 9.36            & \footnotesize 13.38            & \footnotesize 17.03            & \footnotesize 20.61            & \footnotesize 24.83            & \footnotesize 29.87      & \footnotesize 37.36    & \footnotesize 100    & \footnotesize 100      \\
\footnotesize \textsc{ocfw}        & \footnotesize \textsc{mid\_mod  }                & \footnotesize 7.53            & \footnotesize 8.87            & \footnotesize 10.33            & \footnotesize 11.41            & \footnotesize 12.73            & \footnotesize 14.95            & \footnotesize 20.25      & \footnotesize 14.61    & \footnotesize 94     & \footnotesize 88       \\
\footnotesize \textsc{ocfw}        & \footnotesize \textsc{mid\_op   }                & \footnotesize 3.94            & \footnotesize 6.26            & \footnotesize 8.46             & \footnotesize 9.9              & \footnotesize 12.18            & \footnotesize 14.99            & \footnotesize 22.05      & \footnotesize 24.34    & \footnotesize 100    & \footnotesize 100      \\
\footnotesize \textsc{ocfw}        & \footnotesize \textsc{late\_cl  }                & \footnotesize 9.21            & \footnotesize 15.5            & \footnotesize 21.56            & \footnotesize 26.14            & \footnotesize 30.91            & \footnotesize 37.54            & \footnotesize 43.09      & \footnotesize 1.58     & \footnotesize 0      & \footnotesize -100     \\
\footnotesize \textsc{ocfw}        & \footnotesize \textsc{late\_mod }                & \footnotesize 7.65            & \footnotesize 9.48            & \footnotesize 11.67            & \footnotesize 13.62            & \footnotesize 15.82            & \footnotesize 19.19            & \footnotesize 21.06      & \footnotesize 1.02     & \footnotesize 0      & \footnotesize -100     \\
\footnotesize \textsc{ocfw}        & \footnotesize \textsc{late\_op  }                & \footnotesize 1.25            & \footnotesize 2               & \footnotesize 2.88             & \footnotesize 3.9              & \footnotesize 5.25             & \footnotesize 7.01             & \footnotesize 9.25       & \footnotesize 1.12     & \footnotesize 0      & \footnotesize -100     \\
\footnotesize \textsc{ocfw\_u}     & \footnotesize \textsc{early\_all}                & \footnotesize 1.36            & \footnotesize 1.87            & \footnotesize 2.68             & \footnotesize 3.53             & \footnotesize 4.95             & \footnotesize 8.12             & \footnotesize 14.9       & \footnotesize 17.76    & \footnotesize 100    & \footnotesize 100      \\
\footnotesize \textsc{ocfw\_u}     & \footnotesize \textsc{mid\_cl   }                & \footnotesize 0.01            & \footnotesize 0.01            & \footnotesize 0.02             & \footnotesize 0.06             & \footnotesize 0.14             & \footnotesize 0.36             & \footnotesize 3.12       & \footnotesize 29.32    & \footnotesize 100    & \footnotesize 100      \\
\footnotesize \textsc{ocfw\_u}     & \footnotesize \textsc{mid\_mod  }                & \footnotesize 0.12            & \footnotesize 0.22            & \footnotesize 0.38             & \footnotesize 0.75             & \footnotesize 1.65             & \footnotesize 4.58             & \footnotesize 8.25       & \footnotesize 11.54    & \footnotesize 100    & \footnotesize 100      \\
\footnotesize \textsc{ocfw\_u}     & \footnotesize \textsc{mid\_op   }                & \footnotesize 1.64            & \footnotesize 2.02            & \footnotesize 2.62             & \footnotesize 3.55             & \footnotesize 5.2              & \footnotesize 11.02            & \footnotesize 22.62      & \footnotesize 33.49    & \footnotesize 100    & \footnotesize 100      \\
\footnotesize \textsc{ocfw\_u}     & \footnotesize \textsc{late\_cl  }                & \footnotesize 0.01            & \footnotesize 0.01            & \footnotesize 0.03             & \footnotesize 0.08             & \footnotesize 0.22             & \footnotesize 1.82             & \footnotesize 3.69       & \footnotesize 5.35     & \footnotesize 100    & \footnotesize 100      \\
\footnotesize \textsc{ocfw\_u}     & \footnotesize \textsc{late\_mod }                & \footnotesize 0.84            & \footnotesize 1.12            & \footnotesize 1.44             & \footnotesize 2.4              & \footnotesize 7.86             & \footnotesize 24.61            & \footnotesize 26.31      & \footnotesize 2.2      & \footnotesize 46     & \footnotesize -8       \\
\footnotesize \textsc{ocfw\_u}     & \footnotesize \textsc{late\_op  }                & \footnotesize 32.9            & \footnotesize 46.29           & \footnotesize 81.2             & \footnotesize 89.44            & \footnotesize 91.76            & \footnotesize 93.26            & \footnotesize 93.85      & \footnotesize 0.34     & \footnotesize 0      & \footnotesize -100     \\
\end{tabular}
\end{sidewaystable}


\begin{sidewaystable}[!htbp]
\caption{Range of variation in landscape structure, illustrating the cover-condition class dynamics for Red Fir - Mesic (\textsc{rfr\_m}), Red Fir - Xeric (\textsc{rfr\_x}), Sierran Mixed Conifer - Mesic (\textsc{smc\_m}), and Sierran Mixed Conifer - Ultramafic (\textsc{smc\_u}). For condition class abbreviations, see Table~\ref{condtable}.}
\label{tab:covcond2}
\begin{tabular}{@{}lllllllllllll@{}}
\toprule
\footnotesize \textbf{\begin{tabular}[c]{@{}l@{}}Land \\ Cover\\ Type\end{tabular}} & \footnotesize \textbf{\begin{tabular}[c]{@{}l@{}}Condition \\ Class\end{tabular}} & \footnotesize \textbf{srv0\%} & \footnotesize \textbf{srv5\%} & \footnotesize \textbf{srv25\%} & \footnotesize \textbf{srv50\%} & \footnotesize \textbf{srv75\%} & \footnotesize \textbf{srv95\%} & \footnotesize \textbf{srv100\%}   & \footnotesize \textbf{\begin{tabular}[c]{@{}l@{}}Current\\ \%cover\end{tabular}} & \textbf{\begin{tabular}[c]{@{}l@{}}Current\\ \%srv\end{tabular}} & \textbf{\begin{tabular}[c]{@{}l@{}}Departure\\ Index\end{tabular}} \\ \midrule
\footnotesize \textsc{rfr\_m}      & \footnotesize \textsc{early\_all}               & \footnotesize 4.93            & \footnotesize 6.9             & \footnotesize 12.25            & \footnotesize 17.75            & \footnotesize 23.57            & \footnotesize 32.15            & \footnotesize 39.4        & \footnotesize 24.21    & \footnotesize 79     & \footnotesize 58       \\
\footnotesize \textsc{rfr\_m}      & \footnotesize \textsc{mid\_cl   }               & \footnotesize 17.05           & \footnotesize 21.21           & \footnotesize 28.09            & \footnotesize 32.61            & \footnotesize 38.31            & \footnotesize 48.76            & \footnotesize 55.34       & \footnotesize 3.63     & \footnotesize 0      & \footnotesize -100     \\
\footnotesize \textsc{rfr\_m}      & \footnotesize \textsc{mid\_mod  }               & \footnotesize 0.42            & \footnotesize 0.64            & \footnotesize 0.92             & \footnotesize 1.25             & \footnotesize 1.58             & \footnotesize 2.35             & \footnotesize 3.53        & \footnotesize 18.67    & \footnotesize 100    & \footnotesize 100      \\
\footnotesize \textsc{rfr\_m}      & \footnotesize \textsc{mid\_op   }               & \footnotesize 0.11            & \footnotesize 0.21            & \footnotesize 0.33             & \footnotesize 0.48             & \footnotesize 0.65             & \footnotesize 0.97             & \footnotesize 1.39        & \footnotesize 16.7     & \footnotesize 100    & \footnotesize 100      \\
\footnotesize \textsc{rfr\_m}      & \footnotesize \textsc{late\_cl  }               & \footnotesize 20.74           & \footnotesize 27.97           & \footnotesize 33.53            & \footnotesize 39.97            & \footnotesize 46.88            & \footnotesize 53.52            & \footnotesize 59.29       & \footnotesize 10.7     & \footnotesize 0      & \footnotesize -100     \\
\footnotesize \textsc{rfr\_m}      & \footnotesize \textsc{late\_mod }               & \footnotesize 1.77            & \footnotesize 2.14            & \footnotesize 2.63             & \footnotesize 3.02             & \footnotesize 3.43             & \footnotesize 4.33             & \footnotesize 5.41        & \footnotesize 21.96    & \footnotesize 100    & \footnotesize 100      \\
\footnotesize \textsc{rfr\_m}      & \footnotesize \textsc{late\_op  }               & \footnotesize 1.15            & \footnotesize 1.57            & \footnotesize 2.13             & \footnotesize 2.89             & \footnotesize 3.64             & \footnotesize 5.15             & \footnotesize 7.7         & \footnotesize 4.13     & \footnotesize 87     & \footnotesize 74       \\
\footnotesize \textsc{rfr\_x}      & \footnotesize \textsc{early\_all}               & \footnotesize 19.92           & \footnotesize 26.96           & \footnotesize 31.05            & \footnotesize 36.71            & \footnotesize 41.13            & \footnotesize 47.81            & \footnotesize 56.55       & \footnotesize 32.39    & \footnotesize 32     & \footnotesize -36      \\
\footnotesize \textsc{rfr\_x}      & \footnotesize \textsc{mid\_cl   }               & \footnotesize 0.04            & \footnotesize 0.09            & \footnotesize 0.21             & \footnotesize 0.37             & \footnotesize 0.66             & \footnotesize 1.44             & \footnotesize 4.14        & \footnotesize 8.26     & \footnotesize 100    & \footnotesize 100      \\
\footnotesize \textsc{rfr\_x}      & \footnotesize \textsc{mid\_mod  }               & \footnotesize 1.14            & \footnotesize 2.49            & \footnotesize 4.19             & \footnotesize 5.86             & \footnotesize 7.7              & \footnotesize 10.55            & \footnotesize 13.59       & \footnotesize 18.66    & \footnotesize 100    & \footnotesize 100      \\
\footnotesize \textsc{rfr\_x}      & \footnotesize \textsc{mid\_op   }               & \footnotesize 8.99            & \footnotesize 12.72           & \footnotesize 17.61            & \footnotesize 19.95            & \footnotesize 22.93            & \footnotesize 28.6             & \footnotesize 33.41       & \footnotesize 12.58    & \footnotesize 5      & \footnotesize -90      \\
\footnotesize \textsc{rfr\_x}      & \footnotesize \textsc{late\_cl  }               & \footnotesize 4.51            & \footnotesize 6.42            & \footnotesize 9.57             & \footnotesize 11.99            & \footnotesize 15.38            & \footnotesize 20.37            & \footnotesize 25.44       & \footnotesize 10.45    & \footnotesize 35     & \footnotesize -30      \\
\footnotesize \textsc{rfr\_x}      & \footnotesize \textsc{late\_mod }               & \footnotesize 7.28            & \footnotesize 10.45           & \footnotesize 12.42            & \footnotesize 14.16            & \footnotesize 15.44            & \footnotesize 17.61            & \footnotesize 20.37       & \footnotesize 14.57    & \footnotesize 59     & \footnotesize 18       \\
\footnotesize \textsc{rfr\_x}      & \footnotesize \textsc{late\_op  }               & \footnotesize 5.38            & \footnotesize 6.55            & \footnotesize 7.88             & \footnotesize 9.65             & \footnotesize 11.03            & \footnotesize 15.23            & \footnotesize 19.34       & \footnotesize 3.1      & \footnotesize 0      & \footnotesize -100     \\
\footnotesize \textsc{smc\_m}      & \footnotesize \textsc{early\_all}               & \footnotesize 3.83            & \footnotesize 7.87            & \footnotesize 11.88            & \footnotesize 15.32            & \footnotesize 19.39            & \footnotesize 25.02            & \footnotesize 32.65       & \footnotesize 14.98    & \footnotesize 47     & \footnotesize -6       \\
\footnotesize \textsc{smc\_m}      & \footnotesize \textsc{mid\_cl   }               & \footnotesize 13.57           & \footnotesize 19.79           & \footnotesize 24.42            & \footnotesize 28.28            & \footnotesize 32.5             & \footnotesize 37.31            & \footnotesize 42.25       & \footnotesize 9.74     & \footnotesize 0      & \footnotesize -100     \\
\footnotesize \textsc{smc\_m}      & \footnotesize \textsc{mid\_mod  }               & \footnotesize 7.45            & \footnotesize 9.02            & \footnotesize 10.8             & \footnotesize 12.23            & \footnotesize 13.48            & \footnotesize 15.31            & \footnotesize 18.95       & \footnotesize 17.97    & \footnotesize 100    & \footnotesize 100      \\
\footnotesize \textsc{smc\_m}      & \footnotesize \textsc{mid\_op   }               & \footnotesize 5.07            & \footnotesize 6.86            & \footnotesize 9.31             & \footnotesize 11.56            & \footnotesize 13.69            & \footnotesize 16.73            & \footnotesize 22.7        & \footnotesize 16.29    & \footnotesize 94     & \footnotesize 88       \\
\footnotesize \textsc{smc\_m}      & \footnotesize \textsc{late\_cl  }               & \footnotesize 7.49            & \footnotesize 11.23           & \footnotesize 15.34            & \footnotesize 18.71            & \footnotesize 23.15            & \footnotesize 27.98            & \footnotesize 32.66       & \footnotesize 23.23    & \footnotesize 76     & \footnotesize 52       \\
\footnotesize \textsc{smc\_m}      & \footnotesize \textsc{late\_mod }               & \footnotesize 5.26            & \footnotesize 6.4             & \footnotesize 7.72             & \footnotesize 8.79             & \footnotesize 9.74             & \footnotesize 11.24            & \footnotesize 12.67       & \footnotesize 14.18    & \footnotesize 100    & \footnotesize 100      \\
\footnotesize \textsc{smc\_m}      & \footnotesize \textsc{late\_op  }               & \footnotesize 1.65            & \footnotesize 2.21            & \footnotesize 3.08             & \footnotesize 3.97             & \footnotesize 4.87             & \footnotesize 6.25             & \footnotesize 9.34        & \footnotesize 3.6      & \footnotesize 41     & \footnotesize -18      \\
\footnotesize \textsc{smc\_u}      & \footnotesize \textsc{early\_all}               & \footnotesize 23.05           & \footnotesize 27.37           & \footnotesize 32.07            & \footnotesize 34.75            & \footnotesize 37.52            & \footnotesize 41.24            & \footnotesize 44.39       & \footnotesize 48.7     & \footnotesize 100    & \footnotesize 100      \\
\footnotesize \textsc{smc\_u}      & \footnotesize \textsc{mid\_cl   }               & \footnotesize 0.34            & \footnotesize 0.49            & \footnotesize 0.7              & \footnotesize 0.86             & \footnotesize 1.22             & \footnotesize 1.96             & \footnotesize 2.78        & \footnotesize 2.99     & \footnotesize 100    & \footnotesize 100      \\
\footnotesize \textsc{smc\_u}      & \footnotesize \textsc{mid\_mod  }               & \footnotesize 2.97            & \footnotesize 4.06            & \footnotesize 5.25             & \footnotesize 6.11             & \footnotesize 7.61             & \footnotesize 9.53             & \footnotesize 17.62       & \footnotesize 6.77     & \footnotesize 64     & \footnotesize 28       \\
\footnotesize \textsc{smc\_u}      & \footnotesize \textsc{mid\_op   }               & \footnotesize 15.82           & \footnotesize 18.71           & \footnotesize 21.85            & \footnotesize 23.59            & \footnotesize 25.95            & \footnotesize 29.58            & \footnotesize 32.88       & \footnotesize 5.33     & \footnotesize 0      & \footnotesize -100     \\
\footnotesize \textsc{smc\_u}      & \footnotesize \textsc{late\_cl  }               & \footnotesize 9.06            & \footnotesize 11.96           & \footnotesize 13.87            & \footnotesize 15.31            & \footnotesize 16.94            & \footnotesize 19.17            & \footnotesize 22.47       & \footnotesize 24.43    & \footnotesize 100    & \footnotesize 100      \\
\footnotesize \textsc{smc\_u}      & \footnotesize \textsc{late\_mod }               & \footnotesize 7.4             & \footnotesize 9.3             & \footnotesize 10.32            & \footnotesize 10.78            & \footnotesize 11.51            & \footnotesize 12.78            & \footnotesize 14.71       & \footnotesize 8.51     & \footnotesize 1      & \footnotesize -98      \\
\footnotesize \textsc{smc\_u}      & \footnotesize \textsc{late\_op  }               & \footnotesize 3.41            & \footnotesize 5.43            & \footnotesize 6.61             & \footnotesize 7.63             & \footnotesize 8.43             & \footnotesize 9.81             & \footnotesize 14.37       & \footnotesize 3.27     & \footnotesize 0      & \footnotesize -100     \\
\end{tabular}
\end{sidewaystable}


\begin{sidewaystable}[!htbp]
\caption{Range of variation in landscape structure, illustrating the cover-condition class dynamics for Sierran Mixed Conifer -~Xeric (\textsc{smc\_x}). For condition class abbreviations, see Table~\ref{condtable}.}
\label{tab:covcond3}
\begin{tabular}{@{}lllllllllllll@{}}
\toprule
\footnotesize \textbf{\begin{tabular}[c]{@{}l@{}}Land \\ Cover\\ Type\end{tabular}} & \footnotesize \textbf{\begin{tabular}[c]{@{}l@{}}Condition \\ Class\end{tabular}} & \footnotesize \textbf{srv0\%} & \footnotesize \textbf{srv5\%} & \footnotesize \textbf{srv25\%} & \footnotesize \textbf{srv50\%} & \footnotesize \textbf{srv75\%} & \footnotesize \textbf{srv95\%} & \footnotesize \textbf{srv100\%}   & \footnotesize \textbf{\begin{tabular}[c]{@{}l@{}}Current\\ \%cover\end{tabular}} & \textbf{\begin{tabular}[c]{@{}l@{}}Current\\ \%srv\end{tabular}} & \textbf{\begin{tabular}[c]{@{}l@{}}Departure\\ Index\end{tabular}} \\ \midrule
\footnotesize \textsc{smc\_x}      & \footnotesize \textsc{early\_all}     & \footnotesize 24.05           & \footnotesize 28.51           & \footnotesize 32.95            & \footnotesize 39.96            & \footnotesize 43.9             & \footnotesize 48.81            & \footnotesize 52.69         & \footnotesize 19.48    & \footnotesize 0      & \footnotesize -100     \\
\footnotesize \textsc{smc\_x}      & \footnotesize \textsc{mid\_cl   }     & \footnotesize 0.04            & \footnotesize 0.12            & \footnotesize 0.26             & \footnotesize 0.46             & \footnotesize 0.74             & \footnotesize 1.45             & \footnotesize 2.99          & \footnotesize 11.96    & \footnotesize 100    & \footnotesize 100      \\
\footnotesize \textsc{smc\_x}      & \footnotesize \textsc{mid\_mod  }     & \footnotesize 1.61            & \footnotesize 3.33            & \footnotesize 4.91             & \footnotesize 6.68             & \footnotesize 8.75             & \footnotesize 12.16            & \footnotesize 15.42         & \footnotesize 14.92    & \footnotesize 100    & \footnotesize 100      \\
\footnotesize \textsc{smc\_x}      & \footnotesize \textsc{mid\_op   }     & \footnotesize 20.36           & \footnotesize 23.22           & \footnotesize 26.93            & \footnotesize 29.48            & \footnotesize 33.15            & \footnotesize 37.32            & \footnotesize 40.63         & \footnotesize 11.48    & \footnotesize 0      & \footnotesize -100     \\
\footnotesize \textsc{smc\_x}      & \footnotesize \textsc{late\_cl  }     & \footnotesize 2.53            & \footnotesize 4.05            & \footnotesize 5.68             & \footnotesize 7.18             & \footnotesize 9                & \footnotesize 11.07            & \footnotesize 12.47         & \footnotesize 24.72    & \footnotesize 100    & \footnotesize 100      \\
\footnotesize \textsc{smc\_x}      & \footnotesize \textsc{late\_mod }     & \footnotesize 6.2             & \footnotesize 6.89            & \footnotesize 7.69             & \footnotesize 8.38             & \footnotesize 9.36             & \footnotesize 10.33            & \footnotesize 12.32         & \footnotesize 13.31    & \footnotesize 100    & \footnotesize 100      \\
\footnotesize \textsc{smc\_x}      & \footnotesize \textsc{late\_op  }     & \footnotesize 3.84            & \footnotesize 4.62            & \footnotesize 6.06             & \footnotesize 7.51             & \footnotesize 8.77             & \footnotesize 10.89            & \footnotesize 14.07         & \footnotesize 4.13     & \footnotesize 1      & \footnotesize -98      \\ \bottomrule 
\end{tabular}
\end{sidewaystable}



\clearpage
\pagebreak[4]
\subsubsection{Landscape Configuration}
We summarize the structure and patterns in the landscape using a suite of statistical measures calculated using \textsc{Fragstats}. Table~\ref{tab:fragland} shows the range of variability for the simulation period as well as the current value and depature index. Patch density and patch richness are within the simulated HRV, but the other fifteen metrics are outside the 5$^{\text{th}}$ to 95$^{\text{th}}$ percentile range of variability. See Section \ref{subsec:dataanalysis} for a detailed description of \textsc{Fragstats} metrics.

The departure index indicates the distance from the 50$^{\text{th}}$ percentile value for a given metric. It is computed by subtracting 50 from the current value's percentile (if that value is between 0 and 50) under the simulated range of variability (SRV) then dividing by 50 and multiplying by 100 (to ensure the departure index scales from 0 to 100). Thus, for the landscape metric \emph{Patch Density}, 19.507 is equivalent to the 39$^{\text{th}}$ percentile of observations during the HRV simulation, and the departure index is $(39-50)/50*100 = -22$). This value is within the HRV for the landscape. However, the landscape metric \emph{Edge Density} is 100, because $128.875 > 123.872$, the largest value observed during the HRV simulation. Edge density at the landscape level is outside the HRV.

\todo{check this paragraph}One of the principal purposes of gaining a better quantitative understanding of the historic reference period is to know whether recent human activities have caused landscapes to move outside their historic range of variability (Landres et al.1999; Swetnam et al. 1999). As described above, we summarized the distribution of each metric calculated over the length of the simulation, minus the equilibration period. We computed the 0$^{\text{th}}$, 5$^{\text{th}}$, 25$^{\text{th}}$, 50$^{\text{th}}$, 75$^{\text{th}}$, 95$^{\text{th}}$ and 100$^{\text{th}}$ percentiles of the distribution of observed values. The current percentile for the statistical range of variability refers to the place within the 0--100$^{\text{th}}$ percentile of the observed, simulated HRV. If the current value is outside the HRV, it is given the appropriate maximum (100) or minimum (0) score. The index of departure from HRV measures the relative distance from the median HRV value to the 0$^{\text{th}}$ or 100$^{\text{th}}$ percentile. At the landscape-level, most computed metrics have values outside the HRV. 

Several of the individual landscape metrics are redundant with one another. For example, \emph{Contagion} and \emph{Edge Density} are inversely related, so it is perhaps helpful, but not necessary, to examine both metrics. In Figures~\ref{fig:fragland_areashape}, \ref{fig:fragland_contagsiei}, and \ref{fig:fragland_core} we highlight a subset of the metrics from Table~\ref{tab:fragland} for the purposes of discussing the landscape under the simulated historic period as compared to the present day. Figures~for all metrics are included in \todo{the appendix?}.


\begin{sidewaystable}[!htbp]
\caption{Range of variability during the simulation for a selected suite of landscape configuration metrics calculated using \textsc{Fragstats}. The landscape metrics listed here are described in detail in the \textsc{Fragstats} methods section. PD = patch density; ED = edge density; AREA\_AM = area-weighted mean patch size; GYRATE\_AM = area-weighted mean patch radius of gyration (correlation length); SHAPE\_AM = area-weighted mean patch shape index; CORE\_AM = area-weighted mean patch core area; CAI\_AM = area-weighted mean patch core area index; SIMI\_MN = mean similarity; CWED = contrast-weighted edge density; TECI = total edge contrast index; ECON\_AM = area-weighted mean edge contrast; CONTAG = contagion; IJI = interspersion and juxtaposition index; PR = patch richness; SIDI = Simpson's diversity index; SIEI = Simpson's evenness index; AI = aggregation index.}
\label{tab:fragland}
\begin{tabular}{@{}lllllllllll@{}}
\toprule
\textbf{\begin{tabular}[c]{@{}l@{}}Landscape\\ Metric\end{tabular}} & \textbf{srv0\%} & \textbf{srv5\%} & \textbf{srv25\%} & \textbf{srv50\%} & \textbf{srv75\%} & \textbf{srv95\%} & \textbf{srv100\%}  & \textbf{\begin{tabular}[c]{@{}l@{}}Current\\ Value\end{tabular}} & \textbf{\begin{tabular}[c]{@{}l@{}}Current\\ \%SRV\end{tabular}} & \textbf{\begin{tabular}[c]{@{}l@{}}Departure\\ Index\end{tabular}} \\ \midrule
\small \textsc{pd}              & \small 17.866          & 18.647          & 19.211           & 19.679           & 20.229           & 20.872           & 21.372     & 19.507        & 39                                                               & -22                                                                \\
\small \textsc{ed}              & 114.606         & 116.425         & 118.376          & 119.488          & 120.683          & 122.519          & 123.872           & 128.875       & 100                                                              & 100                                                                \\
\small \textsc{area\_am}        & 158.59          & 169.503         & 183.793          & 194.062          & 209.983          & 243.753          & 362.032           & 119.985       & 0                                                                & -100                                                               \\
\small \textsc{gyrate\_am}      & 693.818         & 714.212         & 737.307          & 752.733          & 770.93           & 806.692          & 914.181           & 620.951       & 0                                                                & -100                                                               \\
\small \textsc{shape\_am}       & 3.547           & 3.604           & 3.712            & 3.776            & 3.86             & 3.988            & 4.272             & 3.243         & 0                                                                & -100                                                               \\
\small \textsc{core\_am}        & 137.672         & 146.748         & 157.365          & 167.446          & 181.671          & 211.455          & 329.638           & 106.71        & 0                                                                & -100                                                               \\
\small \textsc{cai\_am}         & 59.851          & 60.638          & 61.606           & 62.467           & 63.422           & 64.833           & 66.444            & 65.295        & 98                                                               & 96                                                                 \\
\small \textsc{simi\_mn}        & 2316.796        & 2433.125        & 2575.45          & 2683.733         & 2801.744         & 3029.04          & 3918.372          & 2095.764      & 0                                                                & -100                                                               \\
\small \textsc{cwed}            & 39.586          & 40.169          & 40.889           & 41.449           & 41.936           & 42.453           & 43.513            & 36.092        & 0                                                                & -100                                                               \\
\small \textsc{teci}            & 32.977          & 33.446          & 33.828           & 34.149           & 34.501           & 34.943           & 35.621            & 27.654        & 0                                                                & -100                                                               \\
\small \textsc{econ\_am}        & 33.142          & 33.518          & 33.979           & 34.343           & 34.699           & 35.166           & 36.054            & 27.756        & 0                                                                & -100                                                               \\
\small \textsc{contag}          & 54.205          & 54.481          & 54.874           & 55.222           & 55.605           & 55.973           & 56.647            & 51.172        & 0                                                                & -100                                                               \\
\small \textsc{iji }            & 60.367          & 61.568          & 62.217           & 62.697           & 63.199           & 63.876           & 64.361            & 65.868        & 100                                                              & 100                                                                \\
\small \textsc{pr}              & 111             & 114             & 116              & 118              & 120              & 122              & 124               & 117           & 28                                                               & -44                                                                \\
\small \textsc{sidi}            & 0.932           & 0.936           & 0.939            & 0.942            & 0.944            & 0.948            & 0.951             & 0.962         & 100                                                              & 100                                                                \\
\small \textsc{siei}            & 0.94            & 0.944           & 0.947            & 0.95             & 0.952            & 0.956            & 0.959             & 0.971         & 100                                                              & 100                                                                \\
\small \textsc{ai}              & 81.664          & 81.871          & 82.142           & 82.324           & 82.493           & 82.791           & 83.061            & 80.963        & 0                                                                & -100                                                               \\ \bottomrule
\end{tabular}
\end{sidewaystable}


\clearpage

\begin{figure}[!htbp]
  \centering
  \subfloat[][]{
    \centering
\includegraphics[width=0.5\textwidth]{/Users/mmallek/Tahoe/R/Rplots/November2014/AREA_AM1.png}
    }%
  \subfloat[][]{
\includegraphics[width=0.5\textwidth]{/Users/mmallek/Tahoe/R/Rplots/November2014/SHAPE_AM1.png}
  }
\caption{Landscape \textsc{Fragstats} Metrics. Left, Area-weighted Mean Patch Area. Right, Area-weighted Mean Shape. We use the area-weighted metrics to reduce the influence of the many extremely small patches. The average patch size is larger, and the average patch shape more complex, than the current landscape.} 
\label{fig:fragland_areashape}
\end{figure}

\begin{figure}[!htbp]
  \centering
  \subfloat[][]{
    \centering
\includegraphics[width=0.5\textwidth]{/Users/mmallek/Tahoe/R/Rplots/November2014/CONTAG1.png}
    }%
  \subfloat[][]{
\includegraphics[width=0.5\textwidth]{/Users/mmallek/Tahoe/R/Rplots/November2014/SIEI1.png}
  }
\caption{Landscape \textsc{Fragstats} Metrics. (a) Contagion, a metric describing patch dispersion and interspersion. The landscape during the HRV is much more contagious than the current landscape. (b) Simpson's Evenness Index, which indicates the distance from maximum diversity, or evenness, in the landscape patches. Values for Simpson's Evenness are near 1 during the HRV and in the present landscape, but the HRV values are well below the current conditions.} 
\label{fig:fragland_contagsiei}
\end{figure}

\begin{figure}[!htbp]
  \centering
  \includegraphics[width=0.5\textwidth]{/Users/mmallek/Tahoe/R/Rplots/November2014/CORE_AM1.png}
\caption{Landscape \textsc{Fragstats} Metrics. Results for the Area-weighted Mean Core Area, a measure of interior habitat available at the patch level. During the HRV, the \textsc{core\_am} was much greater than in the current landscape.} 
\label{fig:fragland_core}
\end{figure}



\paragraph{Class-level Results} We compiled the results for several class-level metrics for a few classes of particular interest. We were particularly interested in the Early Development condition for Sierran Mixed Conifer - Mesic, Sierran Mixed Conifer - Xeric, and Oak-Conifer Forest and Woodland, cover types that are very extensive across our landscape, are predicted to be highly affected by past fire suppression, and are characterized by chaparral vegetation present during the early successional stage. The results differ across the three cover types. 

\begin{table}[!htbp]
\begin{tabular}{@{}lllllll@{}}
\toprule
\textbf{Cover Type}    & \textbf{Class Metric} & \textbf{srv5\%} & \textbf{srv50\%} & \textbf{srv95\%} & \textbf{\begin{tabular}[c]{@{}l@{}}Current \\ Value\end{tabular}} & \textbf{\begin{tabular}[c]{@{}l@{}}Current \\ \%SRV\end{tabular}} \\ \midrule
\multirow{4}{*}{\begin{tabular}[c]{@{}l@{}}Oak-Conifer Forest \\ and Woodland\end{tabular}}  & \textsc{area\_am }       & 31.898          & 75.523           & 257.481          & 45.113      & 17    \\
                                                                                          & \textsc{shape\_am}       & 1.894           & 2.281            & 3.202            & 2.069       & 22    \\
                                                                                          & \textsc{core\_am }       & 30.715          & 71.899           & 248.712          & 43.871      & 17    \\
                                                                                          & \textsc{clumpy   }       & 0.83            & 0.853            & 0.875            & 0.837       & 12    \\
\multirow{4}{*}{\begin{tabular}[c]{@{}l@{}}Sierran Mixed \\ Conifer - Mesic\end{tabular}} & \textsc{area\_am }       & 36.582          & 107.019          & 322.34           & 28.587      & 2     \\
                                                                                          & \textsc{shape\_am}       & 2.254           & 2.85             & 3.806            & 2.295       & 8     \\
                                                                                          & \textsc{core\_am }       & 34.674          & 101.666          & 306.569          & 27.758      & 2     \\
                                                                                          & \textsc{clumpy   }       & 0.789           & 0.812            & 0.837            & 0.786       & 5     \\
\multirow{4}{*}{\begin{tabular}[c]{@{}l@{}}Sierran Mixed \\ Conifer - Xeric\end{tabular}} & \textsc{area\_am }       & 36.582          & 107.019          & 322.34           & 28.587      & 2     \\
                                                                                          & \textsc{shape\_am}       & 2.254           & 2.85             & 3.806            & 2.295       & 8     \\
                                                                                          & \textsc{core\_am }       & 34.674          & 101.666          & 306.569          & 27.758      & 2     \\
                                                                                          & \textsc{clumpy   }       & 0.789           & 0.812            & 0.837            & 0.786       & 5     \\ \bottomrule %cmidrule(l){2-7} 
\end{tabular}
\end{table}


%%%%%%%%%%%%%%%%%%%%%%%%%%%%%%%%%%%%%%
%%%%%%%%%%%%%%%%%%%%%%%%%%%%%%%%%%%%%%
%%%%%%%%%%%%%%%%%%%%%%%%%%%%%%%%%%%%%%
%%%%%%%%%%%%%%%%%%%%%%%%%%%%%%%%%%%%%%
%%%%%%%%%%%%%%%%%%%%%%%%%%%%%%%%%%%%%%
%%%%%%%%%%%%%%%%%%%%%%%%%%%%%%%%%%%%%%
%%%%%%%%%%%%%%%%%%%%%%%%%%%%%%%%%%%%%%
%%%%%%%%%%%%%%%%%%%%%%%%%%%%%%%%%%%%%%
%%%%%%%%%%%%%%%%%%%%%%%%%%%%%%%%%%%%%%
%%%%%%%%%%%%%%%%%%%%%%%%%%%%%%%%%%%%%%
%%%%%%%%%%%%%%%%%%%%%%%%%%%%%%%%%%%%%%
%%%%%%%%%%%%%%%%%%%%%%%%%%%%%%%%%%%%%%
%%%%%%%%%%%%%%%%%%%%%%%%%%%%%%%%%%%%%%
%%%%%%%%%%%%%%%%%%%%%%%%%%%%%%%%%%%%%%
%%%%%%%%%%%%%%%%%%%%%%%%%%%%%%%%%%%%%%
%%%%%%%%%%%%%%%%%%%%%%%%%%%%%%%%%%%%%%
%%%%%%%%%%%%%%%%%%%%%%%%%%%%%%%%%%%%%%
%%%%%%%%%%%%%%%%%%%%%%%%%%%%%%%%%%%%%%
%%%%%%%%%%%%%%%%%%%%%%%%%%%%%%%%%%%%%%
%%%%%%%%%%%%%%%%%%%%%%%%%%%%%%%%%%%%%%
%%%%%%%%%%%%%%%%%%%%%%%%%%%%%%%%%%%%%%
%%%%%%%%%%%%%%%%%%%%%%%%%%%%%%%%%%%%%%
%%%%%%%%%%%%%%%%%%%%%%%%%%%%%%%%%%%%%%
%%%%%%%%%%%%%%%%%%%%%%%%%%%%%%%%%%%%%%
%%%%%%%%%%%%%%%%%%%%%%%%%%%%%%%%%%%%%%
%%%%%%%%%%%%%%%%%%%%%%%%%%%%%%%%%%%%%%
%%%%%%%%%%%%%%%%%%%%%%%%%%%%%%%%%%%%%%
%%%%%%%%%%%%%%%%%%%%%%%%%%%%%%%%%%%%%%
%%%%%%%%%%%%%%%%%%%%%%%%%%%%%%%%%%%%%%
%%%%%%%%%%%%%%%%%%%%%%%%%%%%%%%%%%%%%%
%%%%%%%%%%%%%%%%%%%%%%%%%%%%%%%%%%%%%%
%%%%%%%%%%%%%%%%%%%%%%%%%%%%%%%%%%%%%%
%%%%%%%%%%%%%%%%%%%%%%%%%%%%%%%%%%%%%%
%%%%%%%%%%%%%%%%%%%%%%%%%%%%%%%%%%%%%%
%%%%%%%%%%%%%%%%%%%%%%%%%%%%%%%%%%%%%%
%%%%%%%%%%%%%%%%%%%%%%%%%%%%%%%%%%%%%%
%%%%%%%%%%%%%%%%%%%%%%%%%%%%%%%%%%%%%%
%%%%%%%%%%%%%%%%%%%%%%%%%%%%%%%%%%%%%%
%%%%%%%%%%%%%%%%%%%%%%%%%%%%%%%%%%%%%%
%%%%%%%%%%%%%%%%%%%%%%%%%%%%%%%%%%%%%%
%%%%%%%%%%%%%%%%%%%%%%%%%%%%%%%%%%%%%%
%%%%%%%%%%%%%%%%%%%%%%%%%%%%%%%%%%%%%%


\chapter{Discussion}
\section{Scope}
The results of our HRV analyses must be interpreted within the scope and limitations of this study. Most importantly, our analyses were designed to simulate vegetation dynamics under a historic reference period. We chose the period from 1550 to 1850, representing the 300 years prior to European settlement, based on expert opinion (Safford, pers. comm. 20 September 2013). The arrival of European settlers to the Sierra Nevada was spurred primarily but not exclusively by the Gold Rush, and led to sweeping ecological changes that now have greatly altered many Sierran landscapes -- through fire suppression, grazing, road-building, timber cutting, recreation, and other activities (Meyer 2013, Safford 2013, \todo{OTHERS}). Climatically, this time frame does fall during the ``Little Ice Age.'' However, Safford (2013) argues that vegetation change did not change substantially during the time. The period prior to European settlement, then, is a suitable reference condition against which we can compare current landscape structure and dynamics. Moreover, it is frequently used in the western United States as the historical reference period for restoration planning (Safford 2013). The period is also up to several times the length of rotation periods identified for well-understood cover types within the project area. Finally, it is a timeframe for which we have a reasonable amount of specific information to enable us to model the system.

We do not argue here that the chosen reference period was a time of stasis, climatically, ecologically, or culturally. In Figure~\ref{pdsi} we illustrate how the Palmer Drought Severity index, a measure of climate variability, oscillates around an average value throughout the reference period. Multi-year droughts and El Niño/La Niña events also occurred over this time frame (Meyer 2013). Ecologically, our historical period occurred during a very long-term (on the scale of thousands of years) shift to a warmer and drier climate, with an associated shift toward species more tolerant of such conditions, such as yellow pine species, and away from species like white fir, which prefer more mesic conditions. A slow shift toward more frequent fire was also taking place (Safford 2013). Culturally, several Native American tribes were living throughout the project area during the reference period. Debate continues among scientists and researchers as to the extent to which those peoples managed vegetation through setting fires (Safford 2013, \todo{OTHERS}). Because we lack empirical evidence to distinguish between lightning-caused and human-caused fires during the reference period, we decided not to exclude any fire frequency or rotation data on the basis of not being reflective of ``natural'' conditions.

We emphasize that our choice of reference periods does not suggest that it should be our goal in management to recreate all of the ecological conditions and dynamics of this period. Complete achievement of such a goal would be impossible, given the climatic, cultural, and ecological changes that have occurred in the last century. It also would be unacceptable socially, economically, and politically. Nor do we suggest that the reference period was completely ``natural'' or preferable in all ways to today’s landscape. 

\begin{wrapfigure}{r}{0.5\textwidth}
\includegraphics[width=0.48\textwidth]{images/CALVEGmappingzones.png}
\caption{\small CALVEG Mapping Zones. These zones meet U.S. Forest Service standard at national and regional levels. These ecological provinces are associated with dozens of vegetation alliances, which are used to classify vegetation in spatial data products.} 
\label{calveg}
\end{wrapfigure}

However, the reference period proposed will allow us to compare current conditions to a baseline set of data on ecosystem conditions (composition, configuration, and disturbance processes) ``to develop an idea of trend over time and idea of the level of depature of altered ecosystems from their ``natural'' state'' (Safford 2013). The results presented here will complement the Natural Range of Variability assessments compiled by the Forest Service's Pacific Southwest Region Ecology group \todo{cite}. An understanding of natural landscape structures and variability during this reference period also provides a basis for forest management policies and associated actions that seek to mimic natural disturbance patterns  (Romme et al. 2000, Buse and Perera 2002).

The spatial scope of our project extends generally to the the northern Sierra Nevada. When deciding on land cover types, including determining xeric and mesic subtypes, our focus was to best represent the project area and the surrounding landscape. We used the CALVEG Mapping Zone boundary for the ``North Sierra'' (Figure~\ref{calveg}) as our focus for defining vegetation and disturbance, including susceptibility, response to fire, and fire size and distribution \todo{cite?}. The model could be applied, with some revision, to the east-side of the Sierra Nevada, or to the southern mountains.

\section{Limitations}

Because our study relied on the use of computer models, it is imperative that the limitations of these models be understood before applying the results in a management context. Here, we discuss several important limitations, some general to the modeling approach employed here and some specific to how we parameterized these models for application in the northern Sierra Nevada.

First, our approach relies heavily on the use of computer models, and while it is important to recognize the many advantages of models, it is critical to understand that models are abstract and simplified representations of reality. \textsc{RMLands}, in particular, simulates wildfires, but does not simulate all of the disturbance processes or all of the complex interactions among them that characterize real landscapes. Ultimately, the results of a model are constrained by the quality of input data. While \textsc{RMLands} utilizes a rich database, the data layers themselves are not perfect. For example, the vegetation cover layer is subject to human interpretation errors and objective classification errors, and is further limited by the spatial resolution of the grid. Thus, our results should not be interpreted as ``golden''. Rather, they should be used to help identify the most influential factors driving landscape change, critical empirical information needs, interesting system behavior, the limits of our understanding, a basis for exploring “what if” scenarios.

Second, it is important to realize that \textsc{RMLands} requires substantial parameterization before it can be applied to a particular landscape. To the extent possible, we have utilized local empirical data. However, we also drew on relevant scientific studies, often from other geographic locations, and relied heavily on expert opinion when scientific studies and local empirical data were not available. The source of information used to parameterize the models is fully documented and subject to review. Thus, our results should not be viewed as definitive, but rather as an informed estimate of the HRV based on our current scientific understanding. It is important to understand that our estimate of the HRV is subject to change as new scientific understanding or better data become available.

Third, this report (and \textsc{RMLands}) devotes more attention to upland vegetation types than to riparian or aquatic types; indeed, riparian and aquatic vegetation are covered only briefly. There are two reasons for this emphasis on upland vegetation in \textsc{RMLands}: (1) riparian and aquatic vegetation cover only a small (but ecologically critical!) portion of the total landscape, and (2) vegetation patterns and dynamics of riparian and aquatic vegetation are more complex, more variable, and more difficult to model in a straightforward fashion than are patterns and dynamics of upland vegetation. Additional research is needed to fully characterize the range of variability in riparian and aquatic ecosystems in this landscape. 

Fourth, this report (and \textsc{RMLands}) focuses on the effects of one major natural disturbance: fire. Other kinds of natural disturbances also occur, including insects and disease, wind-throw, ungulate and beaver herbivory, avalanches, and other forms of soil movement, but the impacts of these other disturbances tend to be localized in time or space and have far less impact on vegetation patterns over broad spatial and temporal scales than does fire. \todo{can we say this here?}

\todo{Class metrics results} Oak-Conifer Forest and Woodland's early seral patches are, on average, smaller, less complex, and less aggregated today than during the HRV, although this is in relation to the median values rather than the HRV. Current values are within HRV for all but one metric, Mean Similarity. For Sierran Mixed Conifer - Mesic, early seral patches are smaller, less complex, and less aggregated than during HRV, but their departure from the median is greater. Four metrics are outside of the $5^{th}-95^{th}$ percentile range, including Area-weighted Mean Area and Area-weighte Mean Core Area. Sierran Mixed Conifer - Xeric patches in the early seral stage show the greatest departure among the three: virtually all metrics are outside the HRV. Each metric is also less than the HRV, indicating that patches today are smaller, simpler and more dispersed and disjunct from one another than during the HRV.

\section{Analysis by Cover Type}
We defined 31 distinct land cover types in the Yuba River watershed and surrounding area for the purposes of \textsc{RMLands} simulations (Table~\ref{covertable}). A few of these were located in the buffer, but not the project area. Several others were treated as \emph{static} in the simulation: they did not undergo vegetation transitions over time or in response to fire. However, four of the \emph{static} types were allowed to experience wildfires: Agriculture, Grassland, Meadow, and Urban. Grasslands may experience fire, but because they are expected to recover from fire in less than five years (the length of one timestep in our simulation), we assume they remain constant in composition and structure. The discussion that follows focuses on the nine cover types found within the core project area that were treated as dynamic in the model and that occurred over an extent of at least 1000 ha in the project area. For each of these cover types, we briefly describe the simulated disturbance regime (i.e., spatial extent and distribution, frequency and temporal variability) associated with each relevant disturbance process, the vegetation dynamics resulting from the interplay between these disturbance processes and succession, and an examination of the cover type’s current departure from the simulated HRV. 

\todo{Calculate initiations/timestep, or extent/timestep?}
%\subsection{Curl-leaf Mountain Mahogany}
%\subsection{Lodgepole Pine} 
%\subsection{Lodgepole Pine with Aspen}
%\subsection{Mixed Evergreen - Ultramafic} 
%\subsection{Montane Riparian} 
%\subsection{Oak Woodland} 
%\subsection{Red Fir - Ultramafic} 
%\subsection{Red Fir with Aspen} 
%\subsection{Sierran Mixed Conifer with Aspen} 
%\subsection{Subalpine Conifer} 
%\subsection{Western White Pine} 


%topics
% CHECK disturbed area
% climate? tpi? - Looked at climate for OCFW...basically the same as for landscape. not sure what it adds
% fire rotation
% preturn
% covcond 
% class metrics? early?

%%%%%%%%%%%%%%%%%%%%%%%%%%%%%%%%%%%%%%%%%%%%%%%%%%%%%%%%%%%%%%%%%%%%%%%%%%%%%
%%%%%%%%%%%%%%%%%%%%%%%%%%%%%%%%%%%%%%%%%%%%%%%%%%%%%%%%%%%%%%%%%%%%%%%%%%%%%
%%%%%%%%%%%%%%%%%%%%%%%%%%%%%%%%%%%%%%%%%%%%%%%%%%%%%%%%%%%%%%%%%%%%%%%%%%%%%
%%%%%%%%%%%%%%%%%%%%%%%%%%%%%%%%%%%%%%%%%%%%%%%%%%%%%%%%%%%%%%%%%%%%%%%%%%%%%
%%%%%%%%%%%%%%%%%%%%%%%%%%%%%%%%%%%%%%%%%%%%%%%%%%%%%%%%%%%%%%%%%%%%%%%%%%%%%

\subsection{Mixed Evergreen - Mesic} 
\todo{Need to fix all disturbed area proportion tables!!!}
\begin{wrapfigure}{l}{0.5\textwidth}
\centering
    \includegraphics[width=0.48\textwidth]{/Users/mmallek/Tahoe/Report2/images/darea_megm.png}
    \caption{\small Disturbance trajectory for Mixed Evergreen - Mesic. High mortality fire in dark blue; low mortality fire in light blue.}
	\label{fig:darea_megm}
\end{wrapfigure}

Mixed Evergreen - Mesic (\textsc{meg\_m})is a somewhat common cover type within the core project area, encompassing 7,273 ha and comprising roughly 4\% of the project area. The frequency and extent of simulated wildfires in mesic mixed evergreen forest varied markedly among decades (Figure~\ref{fig:darea_megm} and Table~\ref{tab:darea_megm}). 

While mesic mixed evergreen forests escaped fire completely five times during the simulation, during a typical five year period a small portion of the cover type burned, mostly with a low mortality effect. Seldom did large extents burn, at any mortality level, although roughly once every 60 years, \textgreater 25\% of the cover type burned. 

Under this wildfire regime, the return interval between fires (of any mortality level) varied widely from 19 years to \textgreater 500 years, with a median of 62 years, respectively (Figure~\ref{fig:preturn_megm}). The median return interval and rotation values were influenced primarily by the dominant fire type, low mortality. Mesic mixed evergreen forests had a low mortality fire rotation of 63 years and a high mortality fire rotation of 534 years (Table~\ref{tab:darea_megm}). 

In general, return intervals and canopy cover varied spatially across the forest and decreased with increasing TPI, reflecting our parameterization, which was based on the theory that higher, more southerly aspects are drier and more susceptible to fires. Canopy cover decreased by about 3\% when comparing minimum to maximum TPI (Table~\ref{tab:tpi_cc}). 

\todo{include?} Finally, \textsc{meg\_m} stands embedded in a neighborhood containing cover types with shorter return intervals exhibited shorter return intervals, reflecting the importance of landscape context on fire regimes.

%%%
The age structure and dynamics of mesic mixed evergreen forest reflected the interplay between disturbance and succession processes. \todo{Do we want anything on survivorship?} We focus here on simulation results following the equilibration period, and the 5$^{\text{th}}$ to 95$^{\text{th}}$ percentile simulated range of variability. 

The distribution of area among stand conditions within mesic mixed evergreen forest fluctuated over time, though not dramatically so (Figure~\ref{fig:covcond_megm}). Because high mortality fire is very rare in this cover type, and the time to reaching a Late Development stage is relatively short \todo{see cover type document?}, the vast majority of the landscape, varying from 71\% to 88\%, was in the Late Development - Closed condition during the simulation (Table~\ref{tab:covcond1}). 

The seral-stage distribution appeared to be in dynamic equilibrium (i.e., the percentage in each stand condition varied about a stable mean). Our calculated current seral-stage distribution was never observed under the simulated HRV (Table~\ref{tab:covcond1}). The most notable departure was the shift from Mid to Late Development. The current landscape contains 52\% of the mesic mixed evergreen forest in mid development conditions, but the late development conditions were always dominant under the simulated HRV. The current proportions of all mid development canopy cover levels are higher than at any point during the HRV. As described above, much of the shift can be attributed to the dominance of Late Development - Closed, although Late Development - Moderate was also well represented, with an HRV from 6\% to 15\%. The Early Development condition was also more common during the simulation than on the current landscape.

The spatial configuration of stand conditions fluctuated markedly over time as well, although there was considerable variation in the magnitude of variability among configuration metrics \todo{make tables for all the class metrics?}. Area-weighted patch and core area, edge density, and patch density exhibited the greatest variability over time. Because the landscape is so dominated by the late development closed and moderate canopy cover conditions, we can focus on the configuration metrics for these classes. In general, the current landscape contains fewer, smaller, and more clumped patches than existed under the simulated HRV. Patches in late development closed are less geometrically complex but have less area in cores in the current landscape than during the simulated HRV.


\begin{table}[!htbp]
\centering
\caption{Disturbed area summary statistics for Mixed Evergreen - Mesic. Proportions shown are relative to the total area of Mixed Evergreen - Mesic.}
\label{tab:darea_megm}
	\begin{tabular}{@{}llll@{}} 
	\toprule
	\textbf{\begin{tabular}[c]{@{}l@{}}Summary Statistic \\ (disturbed area/timestep)\end{tabular}} & \textbf{\begin{tabular}[c]{@{}l@{}}Low  Mortality\end{tabular}} & \textbf{\begin{tabular}[c]{@{}l@{}}High  Mortality\end{tabular}} & \textbf{\begin{tabular}[c]{@{}l@{}}Any  Mortality\end{tabular}} \\ \midrule
	Minimum       & 0.00        &  0.00         & 0.00         \\
	Maximum       & 46.18       & 7.24          & 50.17         \\
	Median        & 4.99        &  0.50         & 5.99         \\
	Mean          & 8.24        &  1.00         & 9.24         \\ 
	\textbf{Fire Rotation}  &  63  &  534  &  57  \\ \bottomrule
	\end{tabular}
\end{table}

%%%%%%%%%%%%%%%%%%%%%%%%%%%%%%%%%%%%%%%%%%%%%%%%%%%%%%%%%%%%%%%%%%%%%%%%%%%%%
%%%%%%%%%%%%%%%%%%%%%%%%%%%%%%%%%%%%%%%%%%%%%%%%%%%%%%%%%%%%%%%%%%%%%%%%%%%%%
%%%%%%%%%%%%%%%%%%%%%%%%%%%%%%%%%%%%%%%%%%%%%%%%%%%%%%%%%%%%%%%%%%%%%%%%%%%%%
%%%%%%%%%%%%%%%%%%%%%%%%%%%%%%%%%%%%%%%%%%%%%%%%%%%%%%%%%%%%%%%%%%%%%%%%%%%%%
%%%%%%%%%%%%%%%%%%%%%%%%%%%%%%%%%%%%%%%%%%%%%%%%%%%%%%%%%%%%%%%%%%%%%%%%%%%%%

\clearpage
\subsection{Mixed Evergreen - Xeric} 

\begin{wrapfigure}{l}{0.5\textwidth}
\centering
    \includegraphics[width=0.48\textwidth]{/Users/mmallek/Tahoe/Report2/images/darea_megx.png}
    \caption{\small Disturbance trajectory for Mixed Evergreen - Xeric. High mortality fmire in dark blue; low mortality fire in light blue.}
	\label{fig:darea_megx}
\end{wrapfigure}

Mixed Evergreen - Xeric (\textsc{meg\_x})is a somewhat common cover type within the core project area, encompassing 6,768 ha and comprising roughly 4\% of the project area. The frequency and extent of simulated wildfires in xeric mixed evergreen forest varied markedly among decades (Figure~\ref{fig:darea_megx} and Table~\ref{tab:darea_megx}). 

While xeric mixed evergreen forests escaped fire completely four times during the simulation, during a typical five year period a small portion of the cover type burned, across a larger extent on average than the mesic mixed evergreen forest, but still with a mostly low mortality effect. Seldom did large extents burn, at any mortality level, although roughly once every 45 years, \textgreater 25\% of the cover type burned. Over 50\% of the landscape burned at the same interval as 0\% burned, that is, very rarely (576 year interval).

Under this wildfire regime, the return interval between fires (of any mortality level) varied widely from 20 years to \textgreater 500 years, with a median of 48 years (Figure~\ref{fig:preturn_megm}). The median return interval and rotation values were influenced primarily by the dominant fire type, low mortality. Xeric mixed evergreen forests had a low mortality fire rotation of 51 years and a high mortality fire rotation of 472 years (Table~\ref{tab:darea_megm}). 

In general, return intervals and canopy cover varied spatially across the forest and decreased with increasing TPI, reflecting our parameterization, which was based on the theory that higher, more southerly aspects are drier and more susceptible to fires. Canopy cover decreased by about 4\% when comparing minimum to maximum TPI (Table~\ref{tab:tpi_cc}). 

\todo{include?} Finally, \textsc{meg\_x} stands embedded in a neighborhood containing cover types with shorter return intervals exhibited shorter return intervals, reflecting the importance of landscape context on fire regimes.

%%%
The age structure and dynamics of xeric mixed evergreen forest reflected the interplay between disturbance and succession processes. \todo{Do we want anything on survivorship?} We focus here on simulation results following the equilibration period, and the 5$^{\text{th}}$ to 95$^{\text{th}}$ percentile simulated range of variability. 

The distribution of area among stand conditions within xeric mixed evergreen forest fluctuated over time, though not dramatically so (Figure~\ref{fig:covcond_megx}). Because high mortality fire is very rare in this cover type, and the time to reaching a Late Development stage is relatively short \todo{see cover type document?}, the vast majority of the landscape, varying from 71\% to 87\%, was in the Late Development - Closed condition during the simulation (Table~\ref{tab:covcond1}). 

The seral-stage distribution appeared to be in dynamic equilibrium (i.e., the percentage in each stand condition varied about a stable mean). Our calculated current seral-stage distribution was never observed under the simulated HRV (Table~\ref{tab:covcond1}). The most notable departure was the shift from Early and Mid Development to Late Development conditions. The current landscape contains 71\% of the xeric mixed evergreen forest in mid development conditions, but the late development conditions were always dominant under the simulated HRV. The current proportions of all mid development canopy cover levels are higher than at any point during the HRV. As described above, much of the shift can be attributed to the dominance of Late Development - Closed, although Late Development - Moderate was also well represented, with an HRV from 6\% to 15\%. The Early Development condition is more common on the current landscape than during the simulation.

The spatial configuration of stand conditions fluctuated markedly over time as well, although there was considerable variation in the magnitude of variability among configuration metrics \todo{make tables for all the class metrics?}. Area-weighted patch and core area, patch density, and radius of gyration exhibited the greatest variability over time. Because the landscape is so dominated by the late development closed and moderate canopy cover conditions, we can use their configuration metrics as a proxy for the landscape. In general, the current landscape contains fewer, smaller, and more isolated patches than existed under the simulated HRV. Patches in late development closed are less geometrically complex and have less area in cores in the current landscape than during the simulated HRV.


\begin{table}[!htbp]
\centering
\caption{Disturbed area summary statistics for Mixed Evergreen - Xeric. Proportions shown are relative to the total area of Mixed Evergreen - Xeric.}
\label{tab:darea_megx}
\begin{tabular}{@{}llll@{}}
\toprule
\textbf{\begin{tabular}[c]{@{}l@{}}Summary Statistic \\ (disturbed area/timestep)\end{tabular}} & \textbf{Low Mortality} & \textbf{High Mortality} & \textbf{Any Mortality} \\ \midrule
Minimum       & 0.00	& 	0.00	& 0.00        \\
Maximum       & 54.72	& 	8.58	& 58.48         \\
Median        & 6.97	& 	0.54	& 7.78        \\
Mean          & 10.19	& 	1.07	& 11.27        \\ 
\textbf{Fire Rotation} 	& 51 	& 472 	& 46 	\\ \bottomrule
\end{tabular}
\end{table}

%%%%%%%%%%%%%%%%%%%%%%%%%%%%%%%%%%%%%%%%%%%%%%%%%%%%%%%%%%%%%%%%%%%%%%%%%%%%%
%%%%%%%%%%%%%%%%%%%%%%%%%%%%%%%%%%%%%%%%%%%%%%%%%%%%%%%%%%%%%%%%%%%%%%%%%%%%%
%%%%%%%%%%%%%%%%%%%%%%%%%%%%%%%%%%%%%%%%%%%%%%%%%%%%%%%%%%%%%%%%%%%%%%%%%%%%%
%%%%%%%%%%%%%%%%%%%%%%%%%%%%%%%%%%%%%%%%%%%%%%%%%%%%%%%%%%%%%%%%%%%%%%%%%%%%%
%%%%%%%%%%%%%%%%%%%%%%%%%%%%%%%%%%%%%%%%%%%%%%%%%%%%%%%%%%%%%%%%%%%%%%%%%%%%%

\clearpage
\subsection{Oak-Conifer Forest and Woodland} 

\begin{wrapfigure}{l}{0.5\textwidth}
\centering
    \includegraphics[width=0.48\textwidth]{/Users/mmallek/Tahoe/Report2/images/darea_ocfw.png}
    \caption{\small Disturbance trajectory for Oak-Conifer Forest and Woodland. High mortality fire in dark blue; low mortality fire in light blue.}
	\label{fig:darea_ocfw}
\end{wrapfigure}

Oak-Conifer Forest and Woodland (\textsc{ocfw})is a common cover type within the core project area, encompassing 23,279 ha and comprising roughly 13\% of the project area. The frequency and extent of simulated wildfires in oak-conifer forests and woodlands varied markedly among decades (Figure~\ref{fig:darea_ocfw} and Table~\ref{tab:darea_ocfw}). 

Wildfire was quite prevalent in this cover type. At least some area burned every fives, and at least 10\% of the cover type burned in about 70\% of the simulated timesteps. The median amount of land burned during the simulation was 21\%, and fires burned over 50\% of the cover type about once every 55 years. During one five year interval, over 80\% of the area in this type burned. High mortality wildfire was about one-third as common as low mortality. 

Under this wildfire regime, the return interval between fires (of any mortality level) varied widely from 17 years to over 500 years, with a median of 25 years (Figure~\ref{fig:preturn_ocfw}). As expected, median return interval and rotation values are much shorter for this cover type as compared to the mixed evergreen forests, which occupy similar elevations. Oak-conifer forests and woodlands had a low mortality fire rotation of 33 years and a high mortality fire rotation of 100 years (Table~\ref{tab:darea_ocfw}). 

In general, return intervals and canopy cover varied spatially across the forest and decreased with increasing TPI, reflecting our parameterization, which was based on the theory that higher, more southerly aspects are drier and more susceptible to fires. Canopy cover decreased by about 8\% when comparing minimum to maximum TPI, from an average of 58\% to an average of 53\% (Table~\ref{tab:tpi_cc}). 

Finally, \textsc{ocfw} stands embedded in a neighborhood containing cover types with longer return intervals exhibited longer return intervals, reflecting the importance of landscape context on fire regimes.

%%%
The age structure and dynamics of oak-conifer forests and woodlands reflected the interplay between disturbance and succession processes. \todo{Do we want anything on survivorship?} We focus here on simulation results following the equilibration period, and the 5$^{\text{th}}$ to 95$^{\text{th}}$ percentile simulated range of variability. 

The distribution of area among stand conditions within oak-conifer forests and woodlands fluctuated considerably over time, as expected \todo{add something about MEG not being expected?} (Figure~\ref{fig:covcond_ocfw}). For example, the percentage of oak-conifer forests and woodlands in the Late Development - Closed condition varied from 16\% to 38\%, reflecting the dynamic nature of this cover type when considered over century-long periods (Table~\ref{tab:covcond1}). Surprisingly for a cover type in which fuels are the largest contributor to disturbance and fire is relatively frequent, the open canopy conditions were the least common of all condition classes, ranging from 6\%--15\% for mid development and from 2\%--7\% for late development. The distribution of the other five condition classes was fairly even.

The seral-stage distribution appeared to be in dynamic equilibrium (i.e., the percentage in each stand condition varied about a stable mean). Our calculated current seral-stage distribution was never observed under the simulated HRV (Table~\ref{tab:covcond1}). The most notable departure was the shift from Mid Development, which is dominant in the current landscape, to Late Development conditions, which are almost nonexistent on the current landscape. The current proportions of all late development canopy cover levels are lower than at any point during the HRV.  The Early Development condition is within the HRV ($76^{\text{th}}$ percentile). The Mid Development - Moderate condition is too, but just barely, at the $94^{\text{th}}$ percentile.

The spatial configuration of stand conditions fluctuated markedly over time as well, although there was considerable variation in the magnitude of variability among configuration metrics \todo{make tables for all the class metrics?}. Area-weighted patch and core area exhibited the greatest variability over time. Because late conditions are nearly absent from the current landscape, configuration metrics consistently differ between current conditions and the simulated HRV. The HRV results for class-level metrics are consistent with the exception of one condition type, in the sense of their deviation from current conditions. For example, patches are currently smaller than during HRV, but tend to have larger core areas today. There were usually more patches in late development, fewer in mid development, and about the same in early development during the simulation than on the current landscape in comparison to the median, but current values are often within the HRV. Similarly, patches are now geometrically less complex and more aggregated than the median simulated values, but are not necessarily outside the HRV.


\begin{table}[!htbp]
\centering
\caption{Disturbed area summary statistics for Oak-Conifer Forest and Woodland. Proportions shown are relative to the total area of Oak-Conifer Forest and Woodland.}
\label{tab:darea_ocfw}
\begin{tabular}{@{}llll@{}}
\toprule
\textbf{\begin{tabular}[c]{@{}l@{}}Summary Statistic \\ (disturbed area/timestep)\end{tabular}} & \textbf{Low Mortality} & \textbf{High Mortality} & \textbf{Any Mortality} \\ \midrule
Minimum       & 0.39	& 0.00	& 0.47       \\
Maximum       & 59.90	& 22.77	& 81.34         \\
Median        & 11.93	& 3.82	& 16.22       \\
Mean          & 15.75	& 5.23	& 20.90        \\ 
\textbf{Fire Rotation} & 33	& 100	& 25 \\ \bottomrule
\end{tabular}
\end{table}


%%%%%%%%%%%%%%%%%%%%%%%%%%%%%%%%%%%%%%%%%%%%%%%%%%%%%%%%%%%%%%%%%%%%%%%%%%%%%
%%%%%%%%%%%%%%%%%%%%%%%%%%%%%%%%%%%%%%%%%%%%%%%%%%%%%%%%%%%%%%%%%%%%%%%%%%%%%
%%%%%%%%%%%%%%%%%%%%%%%%%%%%%%%%%%%%%%%%%%%%%%%%%%%%%%%%%%%%%%%%%%%%%%%%%%%%%
%%%%%%%%%%%%%%%%%%%%%%%%%%%%%%%%%%%%%%%%%%%%%%%%%%%%%%%%%%%%%%%%%%%%%%%%%%%%%
%%%%%%%%%%%%%%%%%%%%%%%%%%%%%%%%%%%%%%%%%%%%%%%%%%%%%%%%%%%%%%%%%%%%%%%%%%%%%
\clearpage
\subsection{Oak-Conifer Forest and Woodland - Ultramafic} 

\begin{wrapfigure}{l}{0.5\textwidth}
\centering
    \includegraphics[width=0.48\textwidth]{/Users/mmallek/Tahoe/Report2/images/darea_ocfwu.png}
    \caption{\small Disturbance trajectory for Oak-Conifer Forest and Woodland - Ultramafic. High mortality fire in dark blue; low mortality fire in light blue.}
	\label{fig:darea_ocfwu}
\end{wrapfigure}

Oak-Conifer Forest and Woodland - Ultramafic (\textsc{ocfw\_u})is a relatively uncommon cover type within the core project area, encompassing 1,060 ha and comprising roughly 0.6\% of the project area. The frequency and extent of simulated wildfires in ultramafic oak-conifer forests and woodlands varied markedly among decades (Figure~\ref{fig:darea_ocfwu} and Table~\ref{tab:darea_ocfwu}). 

Wildfire is much less common in this cover type compared to its non-ultramafic counterpart. All ultramafic oak-conifer forests and woodlands escaped fire 86 times during the simulation. In other words, the frequency of no fire was once every 27 years on average. The fire rotation is 46 years, so 0 ha burned is more likely than any amount of land burned. Much of this is due to the extreme rarity of high mortality fire, which never burned more than 5\% of this cover type, and actually averaged 0\% high mortality. In contrast, low mortality fire was fairly common, and over 10\% of the landscape burned about once every 13 years, which is more frequent than for the mixed evergreen types \todo{verify}. Fires burned over 50\% of the cover type only once every 155 years. Low mortality wildfire was nearly 25 times as common as high mortality. 

Under this wildfire regime, the grand mean return interval between fires (of any mortality level) varied widely from 21 years to over 500 years, with a median of 48 years (Figure~\ref{fig:preturn_ocfwu}). As expected, median return interval and rotation values are much longer for this cover type compared to the non-ultramafic variant. Ultramafic oak-conifer forests and woodlands had a low mortality fire rotation of 48 years and a high mortality fire rotation of 1192 years (Table~\ref{tab:darea_ocfwu}). 

In general, return intervals and canopy cover varied spatially across the ultramafic oak-conifer forests and woodlands, and increased with increasing TPI, which was not predicted. Canopy cover increased by about 3\% when comparing minimum to maximum TPI, the only focal cover type for which we observed an increase (Table~\ref{tab:tpi_cc}). 


%%%
The age structure and dynamics of ultramafic oak-conifer forests and woodlands reflected the interplay between disturbance and succession processes. We focus here on simulation results following the equilibration period, and the 5$^{\text{th}}$ to 95$^{\text{th}}$ percentiles of the simulated range of variability. 

The distribution of area among stand conditions within ultramafic oak-conifer forests and woodlands did not appear to reach equilibrium until approximately timestep 220 (Figure~\ref{fig:covcond_ocfwu}). When it did reach equilibrium, the vast majority of the cover type was in the Late Development - Open condition. It ranged from 46\&--93\%, but because the equilibration period for this cover type was so long, it may be more instructive to consider the $25^{\text{th}}--75^{\text{th}}$ percentile range, which is 81\&--91\%. This is consistent with the extremely low rate of high mortality fire just described, which is the only way to re-initiate a stand in our model. The resulting condition class distribution was dramatically different from the current distribution, which was primarily composed of Middle Development conditions. All conditions except Late Development - Moderate were well outside the HRV.

The seral-stage distribution appeared to eventually reach a dynamic equilibrium (i.e., the percentage in each stand condition varied about a stable mean). Our calculated current seral-stage distribution was never observed under the simulated HRV (Table~\ref{tab:covcond1}). The most notable departure was the huge shift away from all conditions to Late Development - Closed. The remaining conditions were fairly evenly distributed.

The spatial configuration of stand conditions fluctuated markedly over time as well, although there was considerable variation in the magnitude of variability among configuration metrics \todo{make tables for all the class metrics?}. Area-weighted patch and core area, patch density, mean similarity, and radius of gyration all exhibited high variability over time. 

While within the HRV, on the current landscape, patches and their cores are larger, more complex, and more numerous compared to the median for the simulated HRV. The current landscape also has more aggregated patches. However, we caution against drawing firm conclusions, since this type did not reach equilibrium until halfway through the simulation.

\begin{table}[!htbp]
\centering
\caption{Disturbed area summary statistics for Oak-Conifer Forest and Woodland - Ultramafic. Proportions shown are relative to the total area of Oak-Conifer Forest and Woodland - Ultramafic.}
\label{tab:darea_ocfwu}
\begin{tabular}{@{}llll@{}}
\toprule
\textbf{\begin{tabular}[c]{@{}l@{}}Summary Statistic \\ (disturbed area/timestep)\end{tabular}} & \textbf{Low Mortality} & \textbf{High Mortality} & \textbf{Any Mortality} \\ \midrule
Minimum       & 0.00	&   0.00	& 0.00       \\
Maximum       & 83.93	&   5.14	& 85.64         \\
Median        & 5.14	&   0.00	& 5.14       \\
Mean          & 10.28	&   0.00	& 10.28        \\ 
\textbf{Fire Rotation} & 48	& 1192	& 46 \\ \bottomrule
\end{tabular}
\end{table}

%%%%%%%%%%%%%%%%%%%%%%%%%%%%%%%%%%%%%%%%%%%%%%%%%%%%%%%%%%%%%%%%%%%%%%%%%%%%%
%%%%%%%%%%%%%%%%%%%%%%%%%%%%%%%%%%%%%%%%%%%%%%%%%%%%%%%%%%%%%%%%%%%%%%%%%%%%%
%%%%%%%%%%%%%%%%%%%%%%%%%%%%%%%%%%%%%%%%%%%%%%%%%%%%%%%%%%%%%%%%%%%%%%%%%%%%%
%%%%%%%%%%%%%%%%%%%%%%%%%%%%%%%%%%%%%%%%%%%%%%%%%%%%%%%%%%%%%%%%%%%%%%%%%%%%%
%%%%%%%%%%%%%%%%%%%%%%%%%%%%%%%%%%%%%%%%%%%%%%%%%%%%%%%%%%%%%%%%%%%%%%%%%%%%%
\clearpage
\subsection{Red Fir - Mesic} 

\begin{wrapfigure}{l}{0.5\textwidth}
\centering
    \includegraphics[width=0.48\textwidth]{/Users/mmallek/Tahoe/Report2/images/darea_rfrm.png}
    \caption{\small Disturbance trajectory for Red Fir - Mesic. High mortality fire in dark blue; low mortality fire in light blue.}
	\label{fig:darea_rfrm}
\end{wrapfigure}

Red Fir - Mesic (\textsc{ocfw})is a somewhat common cover type within the core project area, encompassing 8,563 ha and comprising roughly 5\% of the project area. The frequency and extent of simulated wildfires in mesic red fir forests varied markedly among decades (Figure~\ref{fig:darea_rfrm} and Table~\ref{tab:darea_rfrm}). 

Wildfire was fairly common in this cover type, although long periods could pass without any fire occurring. No fire burned mesic red fir forests 18 times during the simulation, which is equivalent to a 128 year interval for no fire. When fires did occur they were not usually very large. At least 10\% of these forests burned about once every 20 years, but fires burning over 50\% of the cover type only occured once in 330 years. The median and mean area burned was 4\% and 8\%, respectively. Low mortality fire was roughly 1.5 times as likely to occur as high mortality fire.

Under this wildfire regime, the return interval between fires (of any mortality level) varied widely from 21 years to over 500 years, with a median of 68 years (Figure~\ref{fig:preturn_rfrm}). Median return interval and rotation values tend to be longer in red fir forests compared to sierran mixed conifer forests, because their higher elevation corresponds to cooler and moister conditions. Mesic red fir forests had a low mortality fire rotation of 101 years and a high mortality fire rotation of 164 years (Table~\ref{tab:darea_rfrm}); the proximity of these two values indicates that neither high nor low mortality fires dominate the cover type. 

In general, return intervals and canopy cover varied spatially across the forest and decreased with increasing TPI, reflecting our parameterization, which was based on the theory that higher, more southerly aspects are drier and more susceptible to fires. Canopy cover decreased by about 11\% when comparing minimum to maximum TPI, from an average of 72\% to an average of 64\% (Table~\ref{tab:tpi_cc}). 

%%%
The age structure and dynamics of mesic red fir forests reflected the interplay between disturbance and succession processes. We focus here on simulation results following the equilibration period, and the 5$^{\text{th}}$ to 95$^{\text{th}}$ percentile simulated range of variability. 

The distribution of area among stand conditions within mesic red fir forests fluctuated considerably over time, as expected (Figure~\ref{fig:covcond_rfrm}). For example, the percentage of mesic red fir forests in the Early Development condition varied from 7\% to 32\%, reflecting the dynamic nature of this cover type when considered over century-long periods (Table~\ref{tab:covcond2}). As expected for mesic red fir forests, closed canopy conditions predominated. Proportion of the cover type occupied by Mid Development - Closed ranged from 21\%--49\% and the proportion occupied by Late Development - Closed ranged from 28\%--54\% for late development. Early Development, which includes post-fire chaparral fields, was the next most extensive cover type.

The seral-stage distribution appeared to be in dynamic equilibrium (i.e., the percentage in each stand condition varied about a stable mean). Our calculated current seral-stage distribution was never observed under the simulated HRV (Table~\ref{tab:covcond2}). The most notable departure was a shift from moderate canopy cover to closed canopy cover. Current levels of moderate canopy cover are much higher, and current levels of closed canopy cover much lower, than during the simulated HRV. Early Development and Late Development - Open are both within the HRV at the $79^{\text{th}}$ and $87^{\text{th}}$ percentiles, respectively. 
 The current proportions of all late development canopy cover levels are lower than at any point during the HRV.  The Early Development condition is within the HRV ($76^{\text{th}}$ percentile). The Mid Development - Moderate condition is too, but just barely, at the $94^{\text{th}}$ percentile.

The spatial configuration of stand conditions fluctuated markedly over time as well, although there was considerable variation in the magnitude of variability among configuration metrics \todo{make tables for all the class metrics?}. Several metrics exhibited high variability over time, including the area-weighted patch and core area, edge and patch density, radius of gyration, and contrast-weighted edge density. In general, current values for the class metrics are usually outside of the simulated HRV. Specifically, current patches tend to be smaller in both area and core area and more numerous. They also have more complex geometries and more edge than patches during the simulated HRV.

\begin{table}[!htbp]
\centering
\caption{Disturbed area summary statistics for Red Fir - Mesic. Proportions shown are relative to the total area of Red Fir - Mesic.}
\label{tab:darea_rfrm}
\begin{tabular}{@{}llll@{}}
\toprule
\textbf{\begin{tabular}[c]{@{}l@{}}Summary Statistic \\ (disturbed area/timestep)\end{tabular}} & \textbf{Low Mortality} & \textbf{High Mortality} & \textbf{Any Mortality} \\ \midrule
Minimum       & 0.00	& 	0.00	& 0.00        \\
Maximum       & 54.72	& 	8.58	& 58.48         \\
Median        & 6.97	& 	0.54	& 7.78        \\
Mean          & 10.19	& 	1.07	& 11.27        \\ 
\textbf{Fire Rotation} & 101	& 164	& 62 \\ \bottomrule
\end{tabular}
\end{table}


%%%%%%%%%%%%%%%%%%%%%%%%%%%%%%%%%%%%%%%%%%%%%%%%%%%%%%%%%%%%%%%%%%%%%%%%%%%%%
%%%%%%%%%%%%%%%%%%%%%%%%%%%%%%%%%%%%%%%%%%%%%%%%%%%%%%%%%%%%%%%%%%%%%%%%%%%%%
%%%%%%%%%%%%%%%%%%%%%%%%%%%%%%%%%%%%%%%%%%%%%%%%%%%%%%%%%%%%%%%%%%%%%%%%%%%%%
%%%%%%%%%%%%%%%%%%%%%%%%%%%%%%%%%%%%%%%%%%%%%%%%%%%%%%%%%%%%%%%%%%%%%%%%%%%%%
%%%%%%%%%%%%%%%%%%%%%%%%%%%%%%%%%%%%%%%%%%%%%%%%%%%%%%%%%%%%%%%%%%%%%%%%%%%%%
\clearpage
\subsection{Red Fir - Xeric} 

\begin{wrapfigure}{l}{0.5\textwidth}
\centering
    \includegraphics[width=0.48\textwidth]{/Users/mmallek/Tahoe/Report2/images/darea_rfrx.png}
    \caption{\small Disturbance trajectory for Red Fir - Xeric. High mortality fire in dark blue; low mortality fire in light blue.}
	\label{fig:darea_rfrx}
\end{wrapfigure}

Red Fir - Xeric (\textsc{ocfw})is a somewhat common cover type within the core project area, encompassing 7,493 ha and comprising roughly 5\% of the project area. The frequency and extent of simulated wildfires in xeric red fir forests varied markedly among decades (Figure~\ref{fig:darea_rfrx} and Table~\ref{tab:darea_rfrx}). 

Wildfire was fairly common in this cover type, and occurred much more frequently than in mesic red fir forests. While xeric red fir forests escaped fire completely eight times during the simulation, during a typical five-year period 5--10\% of the cover type burned. The disturbed area per timestep varied dramatically, from a minimum of 0\% to a maximum of 81\% (about 6,000 ha). More than 10\% of xeric red fir forest extent burned at about a 13 year interval, and fires burning over 50\% of the cover type occured once in 85 years. Low mortality fire was twice as likely as high mortality fire.

Under this wildfire regime, the return interval between fires (of any mortality level) varied widely from 20 years to over 500 years, with a median of 40 years (Figure~\ref{fig:preturn_rfrx}). Median return interval and rotation values tend to be longer in red fir forests compared to sierran mixed conifer forests, because their higher elevation corresponds to cooler and moister conditions. Xesic red fir forests had a low mortality fire rotation of 59 years and a high mortality fire rotation of 117 years (Table~\ref{tab:darea_rfrx}). These values are shorter in xeric red fir compared to mesic red fir, but much longer than lower elevation forest types such as \textsc{OCFW} or \textsc{SMC\_M}. 

In general, return intervals and canopy cover varied spatially across the forest and decreased with increasing TPI, reflecting our parameterization, which was based on the theory that higher, more southerly aspects are drier and more susceptible to fires. Canopy cover decreased by about 29\% when comparing minimum to maximum TPI, from an average of 41\% to an average of 29\% (Table~\ref{tab:tpi_cc}). \todo{make histogram for cover type of disturbed area?}

%%%
The age structure and dynamics of xeric red fir forests reflected the interplay between disturbance and succession processes. We focus here on simulation results following the equilibration period, and the 5$^{\text{th}}$ to 95$^{\text{th}}$ percentile simulated range of variability. 

The distribution of area among stand conditions within xeric red fir forests fluctuated considerably over time, as expected (Figure~\ref{fig:covcond_rfrx}). For example, the percentage of xeric red fir forests in the Early Development condition varied from 27\% to 48\%, reflecting the dynamic nature of this cover type when considered over century-long periods (Table~\ref{tab:covcond2}). Interestingly, although open canopy conditions dominated during middle development, the distribution of the three late development canopy conditions was roughly even. Early Development, which includes post-fire chaparral fields, was the single most extensive cover type. The current proportion of Early Development is easily within the simulated HRV, in the 32$^{\text{nd}}$ percentile.

The seral-stage distribution appeared to be in dynamic equilibrium (i.e., the percentage in each stand condition varied about a stable mean). Our calculated current seral-stage distribution was never observed under the simulated HRV (Table~\ref{tab:covcond2}). The most notable departure was a reduction in moderate and closed canopy cover and an increase in open canopy cover within the Mid Development stage. Current levels of moderate and closed canopy cover are much higher than ever observed during the simulated HRV. Late Development - Closed and Moderate are both within the HRV at the $35^{\text{th}}$ and $59^{\text{th}}$ percentiles, respectively. Late Development - Open is currently fairly rare on the landscape (3\%), but the simulated HRV is 7--15\%. 

The spatial configuration of stand conditions fluctuated markedly over time as well, although there was considerable variation in the magnitude of variability among configuration metrics \todo{make tables for all the class metrics?}. Area-weighted patch and core area exhibited the most variability over time. In general, current values for the class metrics are often completely outside or near the extremes of the simulated HRV. Specifically, current patches tend to be smaller in both area and core area and more numerous, with less complex geometries and more edge than patches during the simulated HRV.

\begin{table}[!htbp]
\centering
\caption{Disturbed area summary statistics for Red Fir - Xeric. Proportions shown are relative to the total area of Red Fir - Xeric.}
\label{tab:darea_rfrx}
\begin{tabular}{@{}llll@{}}
\toprule
\textbf{\begin{tabular}[c]{@{}l@{}}Summary Statistic \\ (disturbed area/timestep)\end{tabular}} & \textbf{Low Mortality} & \textbf{High Mortality} & \textbf{Any Mortality} \\ \midrule
Minimum       & 0.00	& 	0.00	& 0.00     \\
Maximum       & 54.52	& 	32.71	& 80.68        \\
Median        & 4.60	& 	2.18	& 6.78     \\
Mean          & 8.72	& 	4.36	& 13.33      \\ 
\textbf{Fire Rotation} & 59	& 117	& 39 \\ \bottomrule
\end{tabular}
\end{table}


%%%%%%%%%%%%%%%%%%%%%%%%%%%%%%%%%%%%%%%%%%%%%%%%%%%%%%%%%%%%%%%%%%%%%%%%%%%%%
%%%%%%%%%%%%%%%%%%%%%%%%%%%%%%%%%%%%%%%%%%%%%%%%%%%%%%%%%%%%%%%%%%%%%%%%%%%%%
%%%%%%%%%%%%%%%%%%%%%%%%%%%%%%%%%%%%%%%%%%%%%%%%%%%%%%%%%%%%%%%%%%%%%%%%%%%%%
%%%%%%%%%%%%%%%%%%%%%%%%%%%%%%%%%%%%%%%%%%%%%%%%%%%%%%%%%%%%%%%%%%%%%%%%%%%%%
%%%%%%%%%%%%%%%%%%%%%%%%%%%%%%%%%%%%%%%%%%%%%%%%%%%%%%%%%%%%%%%%%%%%%%%%%%%%%
\clearpage
\subsection{Sierran Mixed Conifer - Mesic} 

\begin{wrapfigure}{l}{0.5\textwidth}
\centering
    \includegraphics[width=0.48\textwidth]{/Users/mmallek/Tahoe/Report2/images/darea_smcm.png}
    \caption{\small Disturbance trajectory for Sierran Mixed Conifer - Mesic. High mortality fire in dark blue; low mortality fire in light blue.}
	\label{fig:darea_smcm}
\end{wrapfigure}

Sierran Mixed Conifer - Mesic (\textsc{smcm}) is the dominant cover type within the core project area, encompassing 57,853 ha and comprising roughly 32\% of the project area. The frequency and extent of simulated wildfires in sierran mixed conifer forests varied markedly among decades (Figure~\ref{fig:darea_smcm} and Table~\ref{tab:darea_smcm}). 

Wildfire was prevalent in this cover type. At least some area burned every five years, and at least 10\% of the cover type burned in about 65\% of the simulated timesteps. The median amount of land burned during on timestep was 18\%. Large extents of mesic mixed conifer forests burning was fairly uncommon, with a return interval of 165 years for disturbance across at least 50\% of the cover type. There was tremendous variability in the amount of area burned, from a minimum of 0.2\% to a maximum of 75\% (over 43,000 ha). Low mortality fire was typically about three times as likely as high mortality fire.

Under this wildfire regime, the return interval between fires (of any mortality level) varied widely from 18 years to over 500 years, with a median of 29 years (Figure~\ref{fig:preturn_smcm}). Median return interval and rotation values tend to be shorter in mixed conifer forests than in red fir forests, because their lower elevation corresponds to hotter and drier conditions. Mesic mixed conifer forests had a low mortality fire rotation of 39 years and a high mortality fire rotation of 115 years (Table~\ref{tab:darea_smcm}). Low mortality fires, the dominant disturbance type on these forests, are roughly three times as common as high mortality fires.

In general, return intervals and canopy cover varied spatially across the forest and decreased with increasing TPI, reflecting our parameterization, which was based on the theory that higher, more southerly aspects are drier and more susceptible to fires. Canopy cover decreased by about 9\% when comparing minimum to maximum TPI (Table~\ref{tab:tpi_cc}). \todo{make histogram for cover type of disturbed area?}

Finally, \textsc{smc\_m} stands embedded in a neighborhood containing cover types characterized by longer or shorter return intervals exhibited return intervals similar to those cover types, reflecting the importance of landscape context on fire regimes.

%%%
The age structure and dynamics of mesic mixed conifer forests reflected the interplay between disturbance and succession processes. We focus here on simulation results following the equilibration period, and the 5$^{\text{th}}$ to 95$^{\text{th}}$ percentile simulated range of variability. 

The distribution of area among stand conditions within mesic mixed conifer forests fluctuated over time, as expected (Figure~\ref{fig:covcond_smcm}). For example, the percentage of mesic mixed conifer forests in the Early Development condition varied from 8\% to 20\%, reflecting the dynamic nature of this cover type when considered over century-long periods (Table~\ref{tab:covcond2}). This condition is currently within the simulated HRV, in the 47$^{\text{th}}$ percentile. Mid - Development Closed was typically the most extensive condition class, but most of the condition classes (excepting Late Development - Moderate and Open) were common throughout the simulation. The shift towards closed canopies when stands reached the Late Development stage may be due to an increasing resilience to wildfire disturbances.

The seral-stage distribution appeared to be in dynamic equilibrium (i.e., the percentage in each stand condition varied about a stable mean). Our calculated current seral-stage distribution was never observed under the simulated HRV (Table~\ref{tab:covcond2}). The most notable departure was an increase in Mid Development - Closed extent and a decrease in Mid and Late Development - Moderate extent during the simulated HRV; these condition classes are currently all outside of the simulated HRV. The other two Late Development classes are within the HRV, with the closed canopy and open canopy conditions currently in the $76^{\text{th}}$ and $41^{\text{st}}$ percentiles, respectively. Late Development - Open is the least common condition on the landscape during both the HRV and today.

The spatial configuration of stand conditions fluctuated markedly over time as well, although there was considerable variation in the magnitude of variability among configuration metrics \todo{make tables for all the class metrics?}. Area-weighted patch and core area, and radius of gyration, exhibited the greatest variability over time. The class-level metrics for mesic mixed conifer forests do not consistently diverge from the HRV in one direction, nor are all classes outside the simulated HRV for our focal metrics. Instead, we observe that, for example, there are shifts in which patches have large extents: early development patches were much larger during the HRV, but late development - Moderate patches were smaller. We observed a similar pattern for core area and for \emph{Shape}, in which early development patches were more complex while late development - Moderate patches were less. Full results are available in \todo{Appendix}. Early development and Mid Development - Closed were more aggregated during the simulated HRV, but the remaining condition classes were less aggregated. Thus no clear pattern emerges within the mesic mixed conifer cover type regarding departure from the HRV.


\begin{table}[!htbp]
\centering
\caption{Disturbed area summary statistics for Sierran Mixed Conifer - Mesic. Proportions shown are relative to the total area of Sierran Mixed Conifer - Mesic.}
\label{tab:darea_smcm}
\begin{tabular}{@{}llll@{}}
\toprule
\textbf{\begin{tabular}[c]{@{}l@{}}Summary Statistic \\ (disturbed area/timestep)\end{tabular}} & \textbf{Low Mortality} & \textbf{High Mortality} & \textbf{Any Mortality} \\ \midrule
Minimum       & 0.19	& 	0.03	& 0.22       \\
Maximum       & 51.09	& 	24.42	& 74.53         \\
Median        & 10.54	& 	3.36	& 13.90       \\
Mean          & 13.46	& 	4.52	& 17.98        \\ 
\textbf{Fire Rotation} & 39	& 115	& 29 \\ \bottomrule
\end{tabular}
\end{table}


%%%%%%%%%%%%%%%%%%%%%%%%%%%%%%%%%%%%%%%%%%%%%%%%%%%%%%%%%%%%%%%%%%%%%%%%%%%%%
%%%%%%%%%%%%%%%%%%%%%%%%%%%%%%%%%%%%%%%%%%%%%%%%%%%%%%%%%%%%%%%%%%%%%%%%%%%%%
%%%%%%%%%%%%%%%%%%%%%%%%%%%%%%%%%%%%%%%%%%%%%%%%%%%%%%%%%%%%%%%%%%%%%%%%%%%%%
%%%%%%%%%%%%%%%%%%%%%%%%%%%%%%%%%%%%%%%%%%%%%%%%%%%%%%%%%%%%%%%%%%%%%%%%%%%%%
%%%%%%%%%%%%%%%%%%%%%%%%%%%%%%%%%%%%%%%%%%%%%%%%%%%%%%%%%%%%%%%%%%%%%%%%%%%%%
\clearpage
\subsection{Sierran Mixed Conifer - Ultramafic} 

\begin{wrapfigure}{l}{0.5\textwidth}
\centering
    \includegraphics[width=0.48\textwidth]{/Users/mmallek/Tahoe/Report2/images/darea_smcu.png}
    \caption{\small Disturbance trajectory for Sierran Mixed Conifer - Ultramafic. High mortality fire in dark blue; low mortality fire in light blue.}
	\label{fig:darea_ocfw}
\end{wrapfigure}

Sierran Mixed Conifer - Ultramafic (\textsc{smc\_u})is a relatively uncommon cover type within the core project area, encompassing 4,124 ha and comprising roughly 2\% of the project area. The frequency and extent of simulated wildfires in ultramafic sierran mixed conifer forests varied markedly among decades (Figure~\ref{fig:darea_smcu} and Table~\ref{tab:darea_smcu}). 

Wildfire is much less common in this cover type compared to its non-ultramafic counterpart. Ultramafic soils support scattered, but rarely dense stands of trees and shrubs, creating fuel discontinuities that stop fires from spreading easily. All ultramafic mixed conifer forests escaped fire 37 times during the simulation. In other words, the frequency of no fire was once every 62 years on average. The fire rotation is 69 years, so timesteps without fire are marginally more likely that timesteps with fire. Much of this is due to the rarity of high mortality fire, which ordinarily burned just 1\% of the landscape. In general, however, fire was reasonably common, and over 10\% of the landscape burned about once every 18 years, which is less common than for the mesic variant but fairly similar to the oak-conifer ultramafic type.Fires burned over 50\% of the cover type only once during the simulation, equivalent to a 2,305-year interval. Still, low mortality fire was only about twice as common as high mortality fire, in contrast to the much larger differential found for the other focal ultramafic type.

Under this wildfire regime, the return interval between fires (of any mortality level) varied widely from 20 years to over 500 years, with a median of 74 years (Figure~\ref{fig:preturn_smcu}). As expected, median return interval and rotation values are much longer for this cover type as compared to non-ultramafic mixed conifer forests, which occupy similar elevations. Ultramafic mixed conifer forests had a low mortality fire rotation of 106 years and a high mortality fire rotation of 196 years (Table~\ref{tab:darea_smcu}). 

In general, return intervals and canopy cover varied spatially across the forest and decreased with increasing TPI, reflecting our parameterization, which was based on the theory that higher, more southerly aspects are drier and more susceptible to fires. Canopy cover decreased by about 36\% when comparing minimum to maximum TPI, from an average of 39\% to an average of 25\% (Table~\ref{tab:tpi_cc}). 


%%%
The age structure and dynamics of ultramafic mixed conifer forests reflected the interplay between disturbance and succession processes. We focus here on simulation results following the equilibration period, and the 5$^{\text{th}}$ to 95$^{\text{th}}$ percentile simulated range of variability. 

The distribution of area among stand conditions within ultramafic mixed conifer forests fluctuated over time, as expected (Figure~\ref{fig:covcond_smcu}). For example, the percentage of ultramafic mixed conifer forests in the Mid Development - Open condition varied from 19\% to 30\%, reflecting the dynamic nature of this cover type when considered over century-long periods (Table~\ref{tab:covcond2}). During the simulation, patches in the Mid Development stage were most likely to be in an open canopy, then moderate, then open. Conversely, patches in the Late Development stage were most likely to be in a \emph{closed} canopy, then moderate, then open. Ultramafic soils present a challenge to vegetation, which may explain the dominance of open conditions at the Mid Development stage. However, because fire is relatively uncommon, the shift in dominance at the Late Development stage may reflect the additional time available to vegetation to grow into a closed canopy condition.

The seral-stage distribution appeared to be in dynamic equilibrium (i.e., the percentage in each stand condition varied about a stable mean). Our calculated current seral-stage distribution was never observed under the simulated HRV (Table~\ref{tab:covcond2}). The most notable departure was the decrease in area covered by Early Development and Late Development - Closed, and the upward shift in area covered by Mid Development Open, Late Development - Moderate, and Late Development - Closed observed during the simulated HRV, in contrast to the current conditions. The only cover type within the HRV was Mide Development - Moderate, which is in the $64^{\text{th}}$ percentile.

The spatial configuration of stand conditions fluctuated markedly over time as well, although there was considerable variation in the magnitude of variability among configuration metrics \todo{make tables for all the class metrics?}. Area-weighted patch and core area, as well as radius of gyration, exhibited the greatest variability over time. While there is not perfect correlation among the condition classes with respect to departure from the HRV, we can draw some general conclusions. For example, the current extent of ultramafic mixed conifer forests contains fewer patches that are more aggregated and have a higher area-weighted mean patch and core area compared to the simulated HRV. For other metrics, such as the measure of geometric complexity, \emph{Shape}, some condition classes are within the HRV, some are less complex, and some are more complex than during the HRV. 

\begin{table}[!htbp]
\centering
\caption{Disturbed area summary statistics for Sierran Mixed Conifer - Ultramafic. Proportions shown are relative to the total area of Sierran Mixed Conifer - Ultramafic.}
\label{tab:darea_smcu}
\begin{tabular}{@{}llll@{}}
\toprule
\textbf{\begin{tabular}[c]{@{}l@{}}Summary Statistic \\ (disturbed area/timestep)\end{tabular}} & \textbf{Low Mortality} & \textbf{High Mortality} & \textbf{Any Mortality} \\ \midrule
Minimum       & 0.00	& 0.00		& 0.00      \\
Maximum       & 32.14	& 21.13		& 52.83        \\
Median        & 2.64	& 1.32		& 4.40      \\
Mean          & 4.84	& 2.64		& 7.48       \\
\textbf{Fire Rotation} & 106	& 196	& 69 \\  \bottomrule
\end{tabular}
\end{table}


%%%%%%%%%%%%%%%%%%%%%%%%%%%%%%%%%%%%%%%%%%%%%%%%%%%%%%%%%%%%%%%%%%%%%%%%%%%%%
%%%%%%%%%%%%%%%%%%%%%%%%%%%%%%%%%%%%%%%%%%%%%%%%%%%%%%%%%%%%%%%%%%%%%%%%%%%%%
%%%%%%%%%%%%%%%%%%%%%%%%%%%%%%%%%%%%%%%%%%%%%%%%%%%%%%%%%%%%%%%%%%%%%%%%%%%%%
%%%%%%%%%%%%%%%%%%%%%%%%%%%%%%%%%%%%%%%%%%%%%%%%%%%%%%%%%%%%%%%%%%%%%%%%%%%%%
%%%%%%%%%%%%%%%%%%%%%%%%%%%%%%%%%%%%%%%%%%%%%%%%%%%%%%%%%%%%%%%%%%%%%%%%%%%%%
\clearpage
\subsection{Sierran Mixed Conifer - Xeric} 

\begin{wrapfigure}{l}{0.5\textwidth}
\centering
    \includegraphics[width=0.48\textwidth]{/Users/mmallek/Tahoe/Report2/images/darea_smcx.png}
    \caption{\small Disturbance trajectory for Sierran Mixed Conifer - Xeric. High mortality fire in dark blue; low mortality fire in light blue.}
	\label{fig:darea_ocfw}
\end{wrapfigure}

Sierran Mixed Conifer - Xeric (\textsc{smc\_x}) is the second most dominant cover type within the core project area, encompassing 52,198 ha and comprising roughly 29\% of the project area. The frequency and extent of simulated wildfires in xeric sierran mixed conifer forests varied markedly among decades (Figure~\ref{fig:darea_smcx} and Table~\ref{tab:darea_smcx}). 

Wildfire is quite prevalent in xeric mixed conifer forests; its overall fire rotation is the lowest of all cover types. At least some area burned during every five-year timestep, and at least 10\% of the cover type burned in about 75\% of the simulated timesteps, or about every 7 years. The median amount of land burned during the simulation was 17\%. Fires burned over 50\% of the cover type about once every 62 years, second only to oak-conifer forests and woodlands. During one five-year interval, 85\% of the xeric mixed conifer forest burned (around 44,400 ha). The minimum area burned was about 200 acres, which indicates a remarkable range of variability in disturbance extent. Low mortality fire was about 1.5 times as common as high mortality fire; both were a frequent occurence.

Under this wildfire regime, the return interval between fires (of any mortality level) varied widely from 18 years to over 500 years, with a median of 24 years (Figure~\ref{fig:preturn_smcx}). As expected, median return interval and rotation values are much longer for this cover type as compared to non-ultramafic mixed conifer forests, which occupy similar elevations. Xeric mixed conifer forests had a low mortality fire rotation of 40 years and a high mortality fire rotation of 62 years (Table~\ref{tab:darea_smcx}), which was by far the lowest high mortality rotation period of any cover type. 

In general, return intervals and canopy cover varied spatially across the forest and decreased with increasing TPI, reflecting our parameterization, which was based on the theory that higher, more southerly aspects are drier and more susceptible to fires. Canopy cover decreased by about 21\% when comparing minimum to maximum TPI, from an average of 31\% to an average of 24\% (Table~\ref{tab:tpi_cc}). 

Finally, \textsc{smc\_x} stands embedded in a neighborhood containing cover types characterized by longer or shorter return intervals exhibited return intervals similar to those cover types, reflecting the importance of landscape context on fire regimes.

%%%
The age structure and dynamics of xeric mixed conifer forests reflected the interplay between disturbance and succession processes. We focus here on simulation results following the equilibration period, and the 5$^{\text{th}}$ to 95$^{\text{th}}$ percentile simulated range of variability. 

The distribution of area among stand conditions within xeric mixed conifer forests fluctuated over time, as expected (Figure~\ref{fig:covcond_smcx}). For example, the percentage of xeric mixed conifer forests in the Early Development varied from 29\% to 49\%, reflecting the dynamic nature of this cover type when considered over century-long periods (Table~\ref{tab:covcond3}). During the simulation, Early Development and Mid Development - Open conditions dominated, in contrast to the current distribution, which is somewhat even across classes. The fairly short low mortality fire rotation may explain the predominance of open canopy conditions, while the fairly short high mortality fire rotation may explain the predominance of Early Development conditions, as well as the low proportion of the landscape in Late Development.

The seral-stage distribution appeared to be in dynamic equilibrium (i.e., the percentage in each stand condition varied about a stable mean). Our calculated current seral-stage distribution was never observed under the simulated HRV (Table~\ref{tab:covcond3}). In fact, none of the condition classes had a distribution within the simulated HRV. The most notable departure was the increase in Early Development and Mid Development - Open during the simulated HRV compared to the current landscape. We also observed a much lower proportion of xeric mixed conifer forest in Late Development - Closed during the simulation than in the current landscape.

The spatial configuration of stand conditions fluctuated markedly over time as well, although there was considerable variation in the magnitude of variability among configuration metrics \todo{make tables for all the class metrics?}. Area-weighted patch and core area, as well as edge density, exhibited the greatest variability over time. The magnitude of the shift from the median values during the simulated HRV is consistently high across class metrics, with many falling completely outside the HRV. However, the direction of that shift is not always consistent, breaking the cover type into two group: Early Development and Mid and Late Development - Open in one, and the remaining condition classes in the other. Since most of the area falls into the former category, our discussion will focus on how they differ from HRV. During the simulated HRV, patches were more numerous, with much larger average sizes and core areas. In addition, they were more aggregated and geometrically complex than the current landscape. 

\begin{table}[!htbp]
\centering
\caption{Disturbed area summary statistics for Sierran Mixed Conifer - Xeric. Proportions shown are relative to the total area of Sierran Mixed Conifer - Xeric.}
\label{tab:darea_smcx}
\begin{tabular}{@{}llll@{}}
\toprule
\textbf{\begin{tabular}[c]{@{}l@{}}Summary Statistic \\ (disturbed area/timestep)\end{tabular}} & \textbf{Low Mortality} & \textbf{High Mortality} & \textbf{Any Mortality} \\ \midrule
Minimum       & 0.28	& 0.03	  & 0.38    \\
Maximum       & 52.00	& 35.55	  & 84.90       \\
Median        & 9.98	& 6.26	  & 16.80    \\
Mean          & 12.83	& 8.42	  & 21.25     \\
\textbf{Fire Rotation} & 40	& 62	& 24 \\  \bottomrule
\end{tabular}
\end{table}





%%%%%%%%%%%%%%%%%%%%%%%%%%%%%%%%%%%%%%%%%%%%%%%%%%%%%%%%%%%%%%%%%%%%%%%%%%%%%
%%%%%%%%%%%%%%%%%%%%%%%%%%%%%%%%%%%%%%%%%%%%%%%%%%%%%%%%%%%%%%%%%%%%%%%%%%%%%
%%%%%%%%%%%%%%%%%%%%%%%%%%%%%%%%%%%%%%%%%%%%%%%%%%%%%%%%%%%%%%%%%%%%%%%%%%%%%
%%%%%%%%%%%%%%%%%%%%%%%%%%%%%%%%%%%%%%%%%%%%%%%%%%%%%%%%%%%%%%%%%%%%%%%%%%%%%
%%%%%%%%%%%%%%%%%%%%%%%%%%%%%%%%%%%%%%%%%%%%%%%%%%%%%%%%%%%%%%%%%%%%%%%%%%%%%
\clearpage
\section{Overall Landscape Assessment}

First, we note that during the HRV, the landscape was composed of larger and more extensive patches, as illustrated by Figure~\ref{fig:fragland_areashape}. 
Second, patches on the landscape were more aggregated at the cell-level during HRV, which is illustrated by the \emph{Contagion} metric. In addition, we observe increased dominance by certain cover types, as illustrated by smaller values for the \emph{Simpson's Evenness Index} durin the HRV (see Figure~\ref{fig:fragland_contagsiei}). However, despite being larger, more extensive, and aggregated, they do not show an associated increase in core area. This indicates that they are geometrically more complex, to the extent that little core area fits inside each patch. As an example, we highlight in Figure~\ref{fig:fragland_core} a patch of Sierran Mixed Conifer - Mesic Forest in the Mid Development - Closed condition, which is one of the largest patches on the landscape. Little room is available for cores due to the irregular shape of this patch.
