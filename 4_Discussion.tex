\chapter{Discussion}
\section{Scope}
The results of our HRV analyses must be interpreted within the scope and limitations of this study. Most importantly, our analyses were designed to simulate vegetation dynamics under a historic reference period. We chose the period from 1550 to 1850, representing the 300 years prior to European settlement, based on expert opinion (Safford, pers. comm. 20 September 2013). The arrival of European settlers to the Sierra Nevada was spurred primarily but not exclusively by the Gold Rush, and led to sweeping ecological changes that now have greatly altered many Sierran landscapes -- through fire suppression, grazing, road-building, timber cutting, recreation, and other activities (Meyer 2013, Safford 2013, \todo{OTHERS}). Climatically, this time frame does fall during the ``Little Ice Age.'' However, Safford (2013) argues that vegetation change did not change substantially during the time. The period prior to European settlement, then, is a suitable reference condition against which we can compare current landscape structure and dynamics. Moreover, it is frequently used in the western United States as the historical reference period for restoration planning (Safford 2013). The period is also up to several times the length of rotation periods identified for well-understood cover types within the project area. Finally, it is a timeframe for which we have a reasonable amount of specific information to enable us to model the system.

We do not argue here that the chosen reference period was a time of stasis, climatically, ecologically, or culturally. In Figure~\ref{pdsi} we illustrate how the Palmer Drought Severity index, a measure of climate variability, oscillates around an average value throughout the reference period. Multi-year droughts and El Niño/La Niña events also occurred over this time frame (Meyer 2013). Ecologically, our historical period occurred during a very long-term (on the scale of thousands of years) shift to a warmer and drier climate, with an associated shift toward species more tolerant of such conditions, such as yellow pine species, and away from species like white fir, which prefer more mesic conditions. A slow shift toward more frequent fire was also taking place (Safford 2013). Culturally, several Native American tribes were living throughout the project area during the reference period. Debate continues among scientists and researchers as to the extent to which those peoples managed vegetation through setting fires (Safford 2013, \todo{OTHERS}). Because we lack empirical evidence to distinguish between lightning-caused and human-caused fires during the reference period, we decided not to exclude any fire frequency or rotation data on the basis of not being reflective of ``natural'' conditions.

We emphasize that our choice of reference periods does not suggest that it should be our goal in management to recreate all of the ecological conditions and dynamics of this period. Complete achievement of such a goal would be impossible, given the climatic, cultural, and ecological changes that have occurred in the last century. It also would be unacceptable socially, economically, and politically. Nor do we suggest that the reference period was completely ``natural'' or preferable in all ways to today’s landscape. 

\begin{wrapfigure}{r}{0.5\textwidth}
\includegraphics[width=0.48\textwidth]{images/CALVEGmappingzones.png}
\caption{\small CALVEG Mapping Zones. These zones meet U.S. Forest Service standard at national and regional levels. These ecological provinces are associated with dozens of vegetation alliances, which are used to classify vegetation in spatial data products.} 
\label{calveg}
\end{wrapfigure}

However, the reference period proposed will allow us to compare current conditions to a baseline set of data on ecosystem conditions (composition, configuration, and disturbance processes) ``to develop an idea of trend over time and idea of the level of depature of altered ecosystems from their ``natural'' state'' (Safford 2013). The results presented here will complement the Natural Range of Variability assessments compiled by the Forest Service's Pacific Southwest Region Ecology group \todo{cite}. An understanding of natural landscape structures and variability during this reference period also provides a basis for forest management policies and associated actions that seek to mimic natural disturbance patterns  (Romme et al. 2000, Buse and Perera 2002).

The spatial scope of our project extends generally to the the northern Sierra Nevada. When deciding on land cover types, including determining xeric and mesic subtypes, our focus was to best represent the project area and the surrounding landscape. We used the CALVEG Mapping Zone boundary for the ``North Sierra'' (Figure~\ref{calveg}) as our focus for defining vegetation and disturbance, including susceptibility, response to fire, and fire size and distribution \todo{cite?}. The model could be applied, with some revision, to the east-side of the Sierra Nevada, or to the southern mountains.

\section{Limitations}

Because our study relied on the use of computer models, it is imperative that the limitations of these models be understood before applying the results in a management context. Here, we discuss several important limitations, some general to the modeling approach employed here and some specific to how we parameterized these models for application in the northern Sierra Nevada.

First, our approach relies heavily on the use of computer models, and while it is important to recognize the many advantages of models, it is critical to understand that models are abstract and simplified representations of reality. \textsc{RMLands}, in particular, simulates wildfires, but does not simulate all of the disturbance processes or all of the complex interactions among them that characterize real landscapes. Ultimately, the results of a model are constrained by the quality of input data. While \textsc{RMLands} utilizes a rich database, the data layers themselves are not perfect. For example, the vegetation cover layer is subject to human interpretation errors and objective classification errors, and is further limited by the spatial resolution of the grid. Thus, our results should not be interpreted as ``golden''. Rather, they should be used to help identify the most influential factors driving landscape change, critical empirical information needs, interesting system behavior, the limits of our understanding, a basis for exploring “what if” scenarios.

Second, it is important to realize that \textsc{RMLands} requires substantial parameterization before it can be applied to a particular landscape. To the extent possible, we have utilized local empirical data. However, we also drew on relevant scientific studies, often from other geographic locations, and relied heavily on expert opinion when scientific studies and local empirical data were not available. The source of information used to parameterize the models is fully documented and subject to review. Thus, our results should not be viewed as definitive, but rather as an informed estimate of the HRV based on our current scientific understanding. It is important to understand that our estimate of the HRV is subject to change as new scientific understanding or better data become available.

Third, this report (and \textsc{RMLands}) devotes more attention to upland vegetation types than to riparian or aquatic types; indeed, riparian and aquatic vegetation are covered only briefly. There are two reasons for this emphasis on upland vegetation in \textsc{RMLands}: (1) riparian and aquatic vegetation cover only a small (but ecologically critical!) portion of the total landscape, and (2) vegetation patterns and dynamics of riparian and aquatic vegetation are more complex, more variable, and more difficult to model in a straightforward fashion than are patterns and dynamics of upland vegetation. Additional research is needed to fully characterize the range of variability in riparian and aquatic ecosystems in this landscape. 

Fourth, this report (and \textsc{RMLands}) focuses on the effects of one major natural disturbance: fire. Other kinds of natural disturbances also occur, including insects and disease, wind-throw, ungulate and beaver herbivory, avalanches, and other forms of soil movement, but the impacts of these other disturbances tend to be localized in time or space and have far less impact on vegetation patterns over broad spatial and temporal scales than does fire. \todo{can we say this here?}


%%%%%%%%%%%%%%%%%%%%%%%%%%%%%%%%%%%%%%%%%%%%%%%%%%%%%%%%%%%%%%%%%%%%%%%%%%%%%
%%%%%%%%%%%%%%%%%%%%%%%%%%%%%%%%%%%%%%%%%%%%%%%%%%%%%%%%%%%%%%%%%%%%%%%%%%%%%
%%%%%%%%%%%%%%%%%%%%%%%%%%%%%%%%%%%%%%%%%%%%%%%%%%%%%%%%%%%%%%%%%%%%%%%%%%%%%
%%%%%%%%%%%%%%%%%%%%%%%%%%%%%%%%%%%%%%%%%%%%%%%%%%%%%%%%%%%%%%%%%%%%%%%%%%%%%
%%%%%%%%%%%%%%%%%%%%%%%%%%%%%%%%%%%%%%%%%%%%%%%%%%%%%%%%%%%%%%%%%%%%%%%%%%%%%
\clearpage
\section{Overall Landscape Assessment}

First, we note that during the HRV, the landscape was composed of larger and more extensive patches, as illustrated by Figure~\ref{fig:fragland_areashape}. This trend was heavily influenced by the presence of wildfires on the landscape, as high mortality fires in particular created large areas of early development (Figure~\ref{fig:patchmaps1-early}. However, we also observed large patches in the other condition classes, which were more likely to form long and/or convoluted patches that were nonetheless extensive. Some large patches have fairly simple shapes, as in the highlighted patch in the lower left of Figure ~\ref{fig:patchmaps1-mid}, while other are more complex, as in the highlighted patch above it.

\begin{figure}[!htbp]
  \centering
  \subfloat[][]{
    \centering
    \includegraphics[width=0.5\textwidth]{/Users/mmallek/Tahoe/Report2/images/695_large_ED.png}
    \label{fig:patchmaps1-early}
    }%
  \subfloat[][]{
    \includegraphics[width=0.5\textwidth]{/Users/mmallek/Tahoe/Report2/images/515_largeMD_2patches.png}
    \label{fig:patchmaps1-mid}
    }
  \caption{(a) A very large patch of Sierran Mixed Conifer - Mesic in Early Development during timestep 695. The patch highlighted with a dark black outline is 1,215 hectares, which is the second-largest patch during this timestep. (b) Two patches of Sierran Mixed Conifer - Mesic in Mid Development - Closed. These are two of the largest patches (in the top 25 out of over 40,000 patches) during timestep 515.} 
  \label{fig:patchmaps1}
\end{figure}

In addition, we observe increased dominance by certain cover-condition types, as illustrated by smaller values for the \emph{Simpson's Evenness Index} during the HRV (Figure~\ref{fig:fragland_contagsiei}). For example, within the Sierran Mixed Conifer - Xeric cover type, Early Development and Mid Development - Open were much more widespread during the simulated HRV than in the current landscape (Figure~\ref{fig:patchmaps2}). Because Sierran Mixed Conifer - Xeric is so widespread, this shift would directly influence \emph{Simpson's Evenness}, lowering its value.

\begin{figure}[!htbp]
  \centering
  \subfloat[][]{
    \centering
    \includegraphics[width=0.5\textwidth]{/Users/mmallek/Tahoe/Report2/images/ts0_smcx_match570.png}
    \label{fig:patchmaps2-ts0}
    }%
  \subfloat[][]{
    \includegraphics[width=0.5\textwidth]{/Users/mmallek/Tahoe/Report2/images/570_smcx_largeEDMD.png}
    \label{fig:patchmaps2-ts570}
    }
  \caption{Cover-Condition map focused on patches from Sierran Mixed Conifer - Xeric, showing increased dominance by certain cover-condition types during the HRV. (a) The current landscape. (b) The same region of the map during timestep 570. Note the contrast between the two maps with respect to the condition classes and size of individual patches.} 
  \label{fig:patchmaps2}
\end{figure}


Third, we find that patches on the landscape were more aggregated at the cell-level during HRV, which is illustrated by the \emph{Contagion} metric. In general, patches have low levels of both dispersion and interspersion. Of course, there are many ``edgy'' areas on the landscape, but this metric indicates that across the full landscape aggregation is more typical, particularly in comparison to the current landscape. Again, the homogeneity of post-fire early development stands likely aids in increasing the contagion value (Figure~\ref{fig:patchmaps3}). 

\begin{figure}[!htbp]
  \centering
  \subfloat[][]{
    \centering
    \includegraphics[width=0.5\textwidth]{/Users/mmallek/Tahoe/Report2/images/615_ts0_contag.png}
    \label{fig:patchmaps3-ts0}
    }%
  \subfloat[][]{
    \includegraphics[width=0.5\textwidth]{/Users/mmallek/Tahoe/Report2/images/615_contag.png}
    \label{fig:patchmaps3-ts615}
    }
  \caption{Cover-Condition map focused on patches from Sierran Mixed Conifer - Mesic. (a) The current landscape. (b) The same region of the map during timestep 615. Note the contrast between the two maps with respect to the contagion (at the cell level).} 
  \label{fig:patchmaps3}
\end{figure}

However, despite being larger, more extensive, and aggregated, they do not show an associated increase in core area (Figure~\ref{fig:fragland_core}). This indicates that they are geometrically more complex, to the extent that little core area fits inside each patch. The results for the landscape metric \emph{Shape} confirm this analysis (Figure~\ref{fig:fragland_areashape}). Especially among the largest patches on the landscape, convoluted shapes are common. Again, this is not to say that large and simple shapes do not occur---they do---but in comparison to the current landscape, complex shapes were characteristic of the simulated HRV.

\begin{figure}[!htbp]
  \centering
  \subfloat[][]{
    \centering
    \includegraphics[width=0.5\textwidth]{/Users/mmallek/Tahoe/Report2/images/555_convoluted.png}
    \label{fig:patchmaps4-ts555}
    }%
  \subfloat[][]{
    \includegraphics[width=0.5\textwidth]{/Users/mmallek/Tahoe/Report2/images/690_less_convoluted.png}
    \label{fig:patchmaps4-ts690}
    }
  \caption{Cover-Condition map focused on patches from Sierran Mixed Conifer - Xeric. (a) A large patch (2059 ha) from timestep 555 has little room is available for cores due to its irregular shape. (b) A large patch (828 ha) from timestep 690 is less geometrically complex, and thus has relatively more \emph{core area}.} 
  \label{fig:patchmaps4}
\end{figure}

\clearpage
\section{Management Implications}\todo{Evaluate all of this}
Our simulations indicate that the current landscape structure deviates substantially from its historic range of variability and that the level of ``departure'' varies in relation to differences among cover types. In general, the current landscape is dominated by mid- to late-successional forest and lacks the fire-dependent stand conditions and spatial heterogeneity in vegetation that was maintained by natural disturbances during the reference period. This landscape condition appear to be largely a legacy of the last century of land management practices, in particular fire exclusion (Romme et al. 2003). Indeed, Euro-American activities have altered the disturbance regime of many western forest landscapes, resulting in substantial changes in landscape structure and function (e.g., Baker 1992; Wallin et al. 1996; Baisan and Swetnam 1997; Agee 1999, McGarigal et al. 2001). 

Our findings are particularly interesting in light of increasing concern over anthropogenic habitat loss and fragmentation (Rochelle et al. 1999; Knight et al. 2000). Forest fragmentation has received considerable research attention in many regions of North America (e.g., Whitcomb et al. 1981; Robbins et al. 1989; Lehmkuhl and Ruggiero 1991; McGarigal and McComb 1995; Schmiegelow et al. 1997; Trzcinski et al. 1999; Villard et al. 1999). However, we are in the earliest stages of understanding the patterns, processes, and ecological significance of forest fragmentation in the southern Rocky Mountain region (Knight et al. 2000). It is not clear, for example, how the native biota responds to anthropogenic changes in landscape patterns caused by logging and road-building and disruption of natural disturbance regimes (e.g., fire suppression). This difficulty is exacerbated because Rocky Mountain landscapes are inherently very heterogeneous – a result of steep natural gradients in elevation, topography, and substrate – and forests in this region tend to be somewhat patchy even in the absence of human alterations (Hejl 1992).

Based on our results, it might be tempting for managers to reach the simple conclusion that the landscape is less fragmented today than during the reference period. However, this conclusion is not as straightforward as it might seem for the following reasons. First, fragmentation is a landscape-level process in which a specific habitat is progressively sub-divided into smaller, geometrically altered, and more isolated fragments as a result of both natural and human activities, and this process involves changes in landscape composition, structure, and function at many scales and occurs on a backdrop of a natural patch mosaic created by changing landforms and natural disturbances (McGarigal and McComb 1999). Of critical importance is the fact that fragmentation occurs to a specific habitat type, not the entire landscape mosaic, even though it happens at the landscape scale. Thus, landscapes don’t get fragmented, specific habitats do. In our study, we evaluated the spatial pattern - and by implication, the fragmentation - of many different patch types (defined by unique combinations of cover type and stand condition). Many of these patch types are indeed less fragmented in the current landscape than they were under the simulated HRV. This is true in general for most of the late-seral forest patch types. However, not all patch types are less fragmented in the current landscape. For example, many of the early-seral forest patch types are in fact much more fragmented in the current landscape than they were under the simulated HRV. Thus, conclusions about habitat fragmentation in the current landscape must be qualified with specific reference to one or more well-defined habitats.

Second, we evaluated vegetation patterns in the current landscape after excluding roads (i.e., we removed roads from the land cover map by filling in those areas with the abutting cover type), in order to be consistent with our simulation of landscape structure changes during the reference period. Yet, of all the novel kinds of disturbances that humans have introduced in the forests of the southern Rocky Mountains during the last century, roads may be the most ubiquitous and significant long-term legacy of our activities (Romme et al. 2003). Roads are unprecedented features in the ecological history of these landscapes (Forman 1995), and potentially affect many ecological processes (Forman and Alexander 1998; Trombulak and Frissell 2000). In particular, roads are linear landscape features that can create high-contrast edges and bisect patches. Consequently, roads can cause greater fragmentation of habitats than the direct loss of habitat from associated land use activities (Reed et al. 1996b; Tinker et al. 1997, McGarigal et al. 2001). Given the ubiquitous nature of roads and their disproportionate influence on landscape structure and function, any conclusions regarding departure in relation to habitat fragmentation that does not consider road impacts should be viewed with extreme caution. Note, the impacts of roads on landscape structure will be addressed in our evaluation of alternative land management scenarios in the next phase of this project.

Our simulations indicate that returning the landscape structure to a condition that falls within the simulated HRV would likely be a difficult and long-term undertaking if it were deemed desirable. We deduced this from the time it took the current landscape to equilibrate to the reference-period disturbance regime. The model equilibration period in many ways provides a direct measure of HRV departure; it is defined as the period required to return the initial landscape condition to a stable range of variation. It is a function of not only how far outside the stable range of variation the current landscape is, but also the speed at which disturbance and succession processes interact to affect a change in the landscape trajectory. Thus, we can infer that if management activities were designed to emulate natural disturbance processes, then it would take a length of time equal to the equilibration period to return the landscape to its HRV. In our simulations, most landscape structure metrics equilibrated within 100 years, although some metrics equilibrated faster and others slower. In particular, the configuration of the high-elevation conifer forest mosaic took considerably longer (up to 300 years) to equilibrate owing to the long return interval between disturbances and the relatively slow rate of stand development. It must be emphasized, however, that this does not imply that it should be our goal in management to recreate all of the ecological conditions and dynamics of the reference period. Complete achievement of such a goal would be impossible, given the climatic, cultural, and ecological changes that have occurred in the last century. Moreover, the extent and intensity of disturbance required to emulate the natural disturbance regime would be unacceptable socially, economically, and politically.