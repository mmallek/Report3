% !TEX root = master.tex

\section{Lodgepole Pine (LPN)}

\subsection{General Information}

\subsubsection{Cover Type Overview}

\paragraph{Lodgepole Pine (LPN)}

Crosswalks
\begin{itemize}
	\item EVeg: Regional Dominance Type 1
	\begin{itemize}
		\item Lodgepole Pine
	\end{itemize}

	\item LandFire BpS Model
	\begin{itemize}
		\item 0610581 Sierra Nevada Subalpine Lodgepole Pine Forest and Woodland – Wet
		\item 0610582 Sierra Nevada Subalpine Lodgepole Pine Forest and Woodland – Dry

	\end{itemize}

	\item Presettlement Fire Regime Type
	\begin{itemize}
		\item Lodgepole Pine
	\end{itemize}
\end{itemize}

\paragraph{Lodgepole Pine with Aspen (LPN-ASP)}
This type is created by overlaying the NRIS TERRA Inventory of Aspen on top of the EVeg layer. Where it intersects with LPN it is assigned to LPN-ASP.

Reviewed by Shana Gross, Ecologist, USDA Forest Service

\subsubsection{Vegetation Description}
\paragraph{Lodgepole Pine (LPN)}	P. contorta ssp. murrayana is the overwhelming dominant within its forest community, mixing occasionally with Abies magnifica, and with scattered Pinus jeffreyi  and Pinus monticola, and Tsuga mertensiana at higher elevations (Fites-Kaufman et al. 2007). Mature Sierran stands often contain significant seedlings and saplings. Understory characteristics are influenced by proximity to meadow and stream margins. Arctostaphylos and Ribes are common shrubs. Stands associated with meadow edges and streams may have a rich herbaceous layer consisting of grasses, forbs, and sedges. Species associations are likely very location specific. Plants present may include but are not limited to Cassiope, Vaccinium, Phyllodoce, Kalmia, Ceanothus, Chrysolepis, and Carex. Elsewhere, the understory may be virtually absent, consisting of scattered shrubs such as Quercus vaccinifolia, and herbs like Antennaria, Arabis, Eriogonum, and Gayophytum. Fast-moving streams within the cover type are generally characterized by relatively dense populations of Salix (Bartolome 1988, Fites-Kaufman et al. 2007, LandFire 2007a, LandFire 2007b).  

\paragraph{Lodgepole Pine with Aspen (LPN-ASP)}	When Populus tremuloides co-occurs with LPN on the west side of the Sierran crest, it is typically found in smaller patches, often less than 2 ha (5 acres) in size. Mature stands in which P. tremuloides are still dominant are usually relatively open. Average canopy closures range from 60 to 100 percent in young and intermediate-aged stands and from 25 to 60 percent in mature stands. The open nature of the stands results in substantial light penetration to the ground (Verner 1988).

\subsubsection{Distribution}
\paragraph{Lodgepole Pine (LPN)}	Open stands of P. contorta ssp. murrayana, which make up a widespread upper montane forest/woodland, tolerating both rocky soils and semisaturated meadow edges, in an elevational belt within and above the A. magnifica zone. These forests, strongly dominated by P. contorta ssp. murrayana, generally occur at elevations of about 1,830 to 2,400 m (6000 to 7875 ft) in the northern Sierra Nevada. Stands of P. contorta ssp. murrayana may reach much lower, however, with cold air drainage down glacial canyons (Fites-Kaufman et al. 2007, Anderson 1996). On infertile soils, P. contorta ssp. murrayana is often the only tree species that will grow (Lotan and Critchfield 1990).
More than any other Sierran conifer, P. contorta ssp. murrayana is relatively tolerant of poor soil aeration, and thus grows well around the margins of wet meadows and other moist areas. Many upper montane and subalpine meadows in the Sierra Nevada exhibit invasion of young P. contorta ssp. murrayana moving inward from their drier margins. It is not clear how much this process has been influenced by changes in fire frequency or grazing over the last 150 years (Fites-Kaufman et al. 2007).

\paragraph{Lodgepole Pine with Aspen (LPN-ASP)}		Sites supporting P. tremuloides are associated with added soil moisture, i.e., azonal wet sites. These sites are found throughout the LPN zone, often close to streams, lakes, and meadows. Other sites include rock reservoirs, springs and seeps. Terrain can be simple to complex (LandFire 2007c). 


\subsection{Disturbances}

\subsubsection{Wildfire}

\paragraph{Lodgepole Pine (LPN)} 	Wildfires tend to be high mortality, stand-replacing fires that initiate a process of post-fire forest succession. High mortality fires kill large as well as small trees, and may kill many of the shrubs and herbs as well, although below-ground organs of at least some individual shrubs and herbs survive and resprout. Low mortality fires tend to only kill small seedlings and depend on the herbaceous layer to carry fire.
	Unlike the Rocky Mountain subspecies of P. contorta (ssp. latifolia), P. contorta ssp. murrayana does not have serotinous cones (Fites-Kaufman et al. 2007). Following high mortality fire, it initially establishes in even-aged stands, but small-scale disturbances and the ability of the subspecies to regenerate in the absence of fire promote uneven-aged structure (Cope 1993, Gross 2013).
High mortality fire occurs at long intervals. Mixed severity fire is related to fire behavior across the often moist areas where P. contorta ssp. murrayana is found. Surface fires are more common on drier sites, although in general sparse fuels limit fire ignition and spread. Most fires are small (less than 1 ha) but very large fires covering hundreds of hectares do occur (LandFire 2007a, LandFire 2007b). This is due in part to the high susceptibility to fire mortality by P. contorta ssp. murrayana because of its thin bark and shallower roots. Postfire conditions provide an ideal seedbed, and P. contorta ssp. murrayana is an early post-fire colonizer (Cope 1993).

\paragraph{Lodgepole Pine with Aspen (LPN-ASP)}			Sites supporting P. tremuloides are maintained by stand-replacing disturbances that allow regeneration from below-ground suckers. Upland clones are impaired or suppressed by conifer ingrowth and overtopping and intensive grazing that inhibits growth. In a reference condition scenario, a few stands will advance toward conifer dominance, but in the current landscape scenario where fire has been reduced from reference conditions there are many more conifer-dominated mixed aspen stands (LandFire 2007c, Verner 1988). 

Estimates of fire rotations for these variants are available from the LandFire project and a few review papers. The LandFire project’s published fire return intervals are based on a series of associated models created using the Vegetation Dynamics Development Tool (VDDT). In VDDT, fires are specified concurrently with the transition that follows them. For example, a replacement fire causes a transition to the early development stage. In the RMLands model, such fires are classified as high mortality. However, in VDDT mixed severity fires may cause a transition to early development, a transition to a more open condition, or no transition at all. In this case, we categorize the first example as a high mortality fire, and the second and third examples as a low mortality fire. Based on this approach, we calculated fire rotations and the probability of high mortality fire for each of the LPN and LPN-ASP condition classes (Tables~\ref{tab:lpndesc_fire} and \ref{tab:lpnaspdesc_fire}). We computed overall target fire rotations based on values from Mallek et al. (2013) and Van de Water and Safford (2011). 



\begin{table}[]
\centering
\caption{Fire rotation (years) and proportion of high (versus low) mortality fires for Lodgepole Pine type. Values were derived from VDDT model 0610790 (LandFire 2007), Mallek et al. (2013), and Estes (personal communication). }
\label{tab:lpndesc_fire}
\begin{tabular}{@{}lcc@{}}
\toprule
\textbf{Condition}          & \textbf{Fire Rotation} & \textbf{Probability of High Mortality} \\ \midrule
Target                      & 52    & n/a        \\
Early Development – All     & 29    & 0.03       \\
Mid Development – Closed    & 59    & 0.41       \\
Mid Development – Moderate  & 27    & 0.15       \\
Mid Development – Open      & 18    & 0.07       \\
Late Development – Closed   & 37    & 0.26       \\
Late Development – Moderate & 24    & 0.13       \\
Late Development – Open     & 18    & 0.07       \\ \bottomrule
\end{tabular}
\end{table}

\begin{table}[]
\centering
\caption{Fire rotation (years) and proportion of high (versus low) mortality fires for Lodgepole Pine – Aspen type. Values were derived from VDDT model 0610790 (LandFire 2007) and Van de Water and Safford (pers. comm. 2013).}
\label{tab:lpnaspdesc_fire}
\begin{tabular}{@{}lcc@{}}
\toprule
\textbf{Condition}               & \textbf{Fire Rotation} & \textbf{Probability of High Mortality} \\ \midrule
Target                           & 52     & n/a        \\
Early Development – Aspen        & 29     & 0.03       \\
Mid Development – Aspen          & 59     & 0.41       \\
Mid Development – Aspen-Conifer  & 27     & 0.15       \\
Late Development – Conifer-Aspen & 24     & 0.13       \\
Late Development – Closed        & 37     & 0.26       \\ \bottomrule
\end{tabular}
\end{table}

\subsubsection{Other Disturbance}
Other disturbances are not currently modeled, but may, depending on the condition affected and mortality levels, reset patches to early development, maintain existing condition classes, or shift/accelerate succession to a more open condition. 

\subsection{Vegetation Seral Stages}


\subsubsection{Early Development (ED)}

\paragraph{Description}

\paragraph{Succession Transition}

\paragraph{Wildfire Transition}

\hrulefill


\subsubsection{Mid Development – Closed Canopy Cover (MDC)}

\paragraph{Description}

\paragraph{Succession Transition}

\paragraph{Wildfire Transition}

\hrulefill

\subsubsection{Mid Development – Moderate Canopy Cover (MDM)}

\paragraph{Description}

\paragraph{Succession Transition}

\paragraph{Wildfire Transition}

\hrulefill

\subsubsection{Mid Development – Open Canopy Cover (MDO)}

\paragraph{Description}

\paragraph{Succession Transition}

\paragraph{Wildfire Transition}

\hrulefill


\subsubsection{Late Development – Closed Canopy Cover (LDC)}

\paragraph{Description}

\paragraph{Succession Transition}

\paragraph{Wildfire Transition}

\hrulefill

\subsubsection{Late Development – Moderate Canopy Cover (LDM)}

\paragraph{Description}

\paragraph{Succession Transition}

\paragraph{Wildfire Transition}

\hrulefill

\subsubsection{Late Development – Open Canopy Cover (LDO)}

\paragraph{Description}

\paragraph{Succession Transition}

\paragraph{Wildfire Transition}

\hrulefill

\subsection{Condition Classification}

\subsection{Draft Model}
\begin{figure}[htbp]
\centering
\includegraphics[width=0.8\textwidth]{/Users/mmallek/Tahoe/Report3/images/state_trans_model.pdf}
\caption{Generic state and transition model for all non-shrub seral cover types. Boxes show seven condition classes and arrows depict transitions due to vegetation succession and high or low mortality fire.} 
\label{transmodel}
\end{figure}


\begin{thebibliography}
\bibitem{} 
\end{thebibliography}

