
\chapter{Historical Range of Variability Results}
\section{Disturbance Regime}

This report focuses on the effects of wildfire as a natural disturbance; the impacts of other natural disturbances during the reference period were likely localized in time or space and therefore probably had far less impact on vegetation patterns over broad spatial and temporal scales than did fire.\todo{Becky: is this true?} In the sections below, we describe the simulated disturbance regime in terms of its spatial extent and distribution, frequency, and temporal variability, for the landscape as a whole. Variations among vegetation types are described in Chapter~\ref{ch:covtype} and in Appendix~\ref{app:covtype_analysis}. We also acknowledge that some results were used to evaluate whether the model was correctly calibrated; specifically, fire rotation values were used in model calibration, as described in Chapter~\ref{ch:methods}. These rotation values are also an outcome of the model, and are therefore reported here. While this may seem a bit circular, it was a necessary part of the process of simulating the historical range of variability. In addition to the disturbance regime, this chapter includes results for the seral stage dynamics and analysis of the landscape configuration metrics from \textsc{Fragstats}.

%Finally, it is important to realize that while the information below is presented as ``results,'' it could have easily been presented in the methods section as ``model calibration.'' Key spatial and temporal aspects of the disturbance regime were evaluated during preliminary calibration runs, and we subsequently adjusted model parameters to effect desired changes. Thus, while the information presented below does in fact represent results (output) of the simulation, it also represents a set of targets used to calibrate the model (i.e., adjust model parameters to achieve desired results). While this may seem a bit circular, it was a necessary process for a complex model such as \textsc{RMLands}. Moreover, our real emphasis was on quantifying the vegetation patterns and dynamics resulting from these disturbance processes. 

\subsection{Disturbed Area} 

% 174830 eligible hectares
% 181553 hectares in core
% check math using Wildfire_darea_trajectory.csv
% redone 9/15
Approximately 96\% of the landscape was eligible for wildfire disturbance (all cover types except Barren and Water)\footnote{In this section we report values based on percent of eligible landscape. There are 181,550 hectares in the core project area, and 174,830 remain after excluding Barren and Water.}. As expected, the frequency and extent of simulated wildfires varied across timesteps (Figure~\ref{fig:darea}). Also, given the rotation interval and percent mortality expected over time on this landscape, large proportions of the project area burned each (5-year) timestep. High mortality fires do include the burning of early development vegetation, including chaparral, when it resets the successional process. In general, within a given timestep about a third of the disturbed area burned as high mortality. We further summarize the disturbance regime in Tables \ref{tab:darea_atleast} and \ref{tab:darea}. Figures~\ref{fig:darea_min_map}--\ref{fig:darea_mean_map} depict wildfire disturbances during the timesteps representing the $5^{th}$ percentile, $50^{th}$ percentile, $95^{th}$ percentile, and mean area burned. 

% created 9/17
\begin{table}[!htbp]
\centering
\caption{Summary of disturbed area in terms of proportion of the landscape burned (any level of mortality) during the simulation (after the equilibration period). For each benchmark proportion of the landscape, we list the number of timesteps during the simulation when that extent burned, the proportion of timesteps that represents, the interval in timesteps calculated from the proportion (i.e. approximately every 4 timesteps, at least 25\% of the landscape burned.), and the interval in years calculated from the interval in timesteps (5 years to a timestep).}
\label{tab:darea_atleast}
\begin{tabular}{@{}lllll@{}}
\toprule
                           & at least 1\%     & at least 10\%    & at least 25\%    & at least 50\% \\ \midrule
Number of timesteps        & 459              & 313              & 115              & 13            \\
Proportion of timesteps    & 1.00             & 0.68             & 0.25             & 0.03          \\
Interval (timesteps)       & 1.00             & 1.47             & 4.01             & 35.46         \\
Interval (years)           & 5.02             & 7.36             & 20.04            & 177.31        \\ \bottomrule
\end{tabular}
\end{table}\todo{Need to improve this caption}

% redone 9/16
\begin{table}[!htbp]
\centering
\caption{Summary statistics for area disturbed by wildfire during the simulation. Values are expressed as percentage and areal extent (in hectares) of the landscape eligible for disturbance that was actually disturbed.}
\label{tab:darea}
\begin{tabular}{@{}llll@{}}
\toprule
\textbf{\begin{tabular}[c]{@{}l@{}}Summary Statistic \\ (disturbed area/timestep)\end{tabular}}    & \textbf{Low Mortality}   & \textbf{High Mortality}    & \textbf{Any Mortality}   \\
\midrule                      %Low                      %high                 %any
$5^{th}$ percentile         &   2.72 (4,763)        & 0.71 (1,244)     &    3.54 (6,184)         \\
$50^{th}$ percentile        &   10.47 (18,300)      & 3.75 (6,563)     &    14.04 (24,544)         \\
$95^{th}$ percentile        &   31.29 (54,703)      & 21.43 (37,461)   &    45.88 (80,209)          \\
   Mean                     &   13.20 (23,079)      & 4.87 (8,512)     &    18.07 (31,592)         \\
\bottomrule
\end{tabular}
\end{table}

%redone 9/13
\begin{figure}[!htbp]
\centering
\includegraphics[height=0.3\textheight]{/Users/mmallek/Documents/Thesis/Plots/darea/hrv_all.png}
\caption{Disturbance trajectory for wildfire during the simulation. The first timestep is 40 because we excluded earlier timesteps as equilibration. Dark blue values represent high mortality fire, while light blue values represent low mortality fire and are stacked on top of high mortality.}
\label{fig:darea}
\end{figure}


\clearpage

% background color 24, 15, 41, 0
\begin{figure}[!htbp]
  \centering
  \subfloat[][]{
    \centering
    \includegraphics[width=0.5\textwidth]{/Users/mmallek/Documents/Thesis/maps/hrv-wfmort-5th.pdf}
    \label{fig:darea_min}
  }%
  \subfloat[][]{
    \includegraphics[width=0.5\textwidth]{/Users/mmallek/Documents/Thesis/maps/hrv-distid-5th.pdf}
    \label{fig:distid_min}
  }
  \caption{Maps of area burned during the timestep in the \textbf{$5^{th}$ percentile for area burned (3.54\%)} during the simulation. (a) Map by mortality level. Red indicates high mortality fire, while orange indicates low mortality fire. (b) Map showing each individual fire in a different color.}
  \label{fig:darea_min_map}
\end{figure}

\begin{figure}[!htbp]
  \centering
  \subfloat[][]{
    \centering
    \includegraphics[width=0.5\textwidth]{/Users/mmallek/Documents/Thesis/maps/hrv-wfmort-95th.pdf}
    \label{fig:darea_max}
  }%
  \subfloat[][]{
    \includegraphics[width=0.5\textwidth]{/Users/mmallek/Documents/Thesis/maps/hrv-distid-95th.pdf}
    \label{fig:distid_max}
  }
  \caption{Maps of area burned during the timestep with the \textbf{$95^{th}$ percentile for area burned (45.88\%)} during the simulation. (a) Map by mortality level. Red indicates high mortality fire, while orange indicates low mortality fire. (b) Map showing each individual fire in a different color.}
  \label{fig:darea_max_map}
\end{figure}

\begin{figure}[!htbp]
  \centering
  \subfloat[][]{
    \centering
    \includegraphics[width=0.5\textwidth]{/Users/mmallek/Documents/Thesis/maps/hrv-wfmort-median.pdf}
    \label{fig:darea_median}
  }%
  \subfloat[][]{
    \includegraphics[width=0.5\textwidth]{/Users/mmallek/Documents/Thesis/maps/hrv-distid-median.pdf}
    \label{fig:distid_median}
  }
  \caption{Maps of area burned during the timestep with the \textbf{median total area burned (14.04\%)} during the simulation. (a) Map by mortality level. Red indicates high mortality fire, while orange indicates low mortality fire. (b) Map showing each individual fire in a different color.}
  \label{fig:darea_median_map}
\end{figure}

\begin{figure}[!htbp]
  \centering
  \subfloat[][]{
    \centering
    \includegraphics[width=0.5\textwidth]{/Users/mmallek/Documents/Thesis/maps/hrv-wfmort-mean.pdf}
    \label{fig:darea_mean}
  }%
  \subfloat[][]{
    \includegraphics[width=0.5\textwidth]{/Users/mmallek/Documents/Thesis/maps/hrv-distid-mean.pdf}
    \label{fig:distid_mean}
  }
  \caption{Maps of area burned during the timestep with the \textbf{mean total area burned (18.07\%)} during the simulation. (a) Map by mortality level. Red indicates high mortality fire, while orange indicates low mortality fire. (b) Map showing each individual fire in a different color.}
  \label{fig:darea_mean_map}
\end{figure}

\clearpage

%%%%%%%%%%%%%%%%%%%%%%%%%%%%%%%%%%%%%%%%%%%%%%%%%%%%%%%%%%%%%%%%%%%%%%%%%%%%%%%%%%%%%%%%%%%%%%%%%%%%%%%%%%%%%%%%%%%%%%%%%%%%%%%%%%%%

\subsubsection{Sierran Mixed Conifer - Mesic}
Sierran Mixed Conifer - Mesic (\textsc{smc\_m}) is the dominant cover type within the core project area, encompassing 57,853 ha and comprising roughly 32\% of the project area. Wildfire was prevalent in this cover type. The frequency and extent of burned area is similar to that for the landscape as a whole (Figure \ref{fig:darea_smcm}). We summarize the disturbance regime in Tables \ref{tab:darea_smcm} and \ref{tab:darea_atleast_smcm}.


\begin{figure}[!htbp]
  \centering
  \subfloat[][]{
    \centering
    \includegraphics[width=0.5\textwidth]{/Users/mmallek/Documents/Thesis/Plots/darea/hrv_smcm.png}
    }%
  \subfloat[][]{
    \includegraphics[width=0.5\textwidth]{/Users/mmallek/Documents/Thesis/Plots/darea/hrv_hist_smcm.png}
    }
  \caption{\small (a) Disturbance trajectory for Sierran Mixed Conifer - Mesic. High mortality fire in dark blue; low mortality fire in light blue. (b) Histogram of disturbed hectares with density curve overlaid.} 
  \label{fig:darea_smcm}
\end{figure}

\begin{table}[!htbp]
\centering
\caption{\small Disturbed area summary statistics for Sierran Mixed Conifer - Mesic. Proportions shown are relative to the total area of Sierran Mixed Conifer - Mesic.}
\label{tab:darea_smcm}
\begin{tabular}{@{}llll@{}}
\toprule
\textbf{\begin{tabular}[c]{@{}l@{}}Summary Statistic \\ (disturbed area/timestep)\end{tabular}} & \textbf{Low Mortality} & \textbf{High Mortality} & \textbf{Any Mortality} \\ \midrule
Minimum       & 0.31  & 0.03  & 0.37  \\
Maximum       & 53.29 & 20.78 & 72.95 \\
Median        & 10.96 & 3.18  & 14.48 \\
Mean          & 14.00 & 3.98  & 18.03 \\
\textbf{Fire Rotation} & 36       & 125       & 28   \\ \bottomrule
\end{tabular}
\end{table}

\begin{table}[!htbp]
\centering
\caption{Summary of disturbed area in terms of proportion of the amount of \textsc{smc\_m} burned (any level of mortality) during the simulation (after the equilibration period). For each benchmark proportion of the landscape, we list the number of timesteps during the simulation when that extent burned, the proportion of timesteps that represents, the interval in timesteps calculated from the proportion (i.e. approximately every 4 timesteps, at least 25\% of the landscape burned.), and the interval in years calculated from the interval in timesteps (5 years to a timestep).}
\label{tab:darea_atleast_smcm}
\begin{tabular}{@{}lllll@{}}
                        & at least 1\% & at least 10\% & at least 25\% & at least 50\% \\ \midrule
Number of timesteps     & 458          & 311           & 126           & 15            \\
Proportion of timesteps & 0.99         & 0.67          & 0.27          & 0.03          \\
Interval (timesteps)    & 1.01         & 1.48          & 3.66          & 30.73         \\
Interval (years)        & 5.03         & 7.41          & 18.29         & 153.67       \\ \bottomrule
\end{tabular}
\end{table}

%%%%%%%%%%%%%%%%%%%%%%%%%%%%%%%%%%%%%%%%%%%%%%%%%%%%%%%%%%%%%%%%%%%%%%%%%%%%%%%%%%%%%%%%%%%%%%%%%%%%%%%%%%%%%%%%%%%%%%%%%%%%%%%%%%%%


\subsubsection{Sierran Mixed Conifer - Xeric}
Sierran Mixed Conifer - Xeric (\textsc{smc\_x}) is the second most dominant cover type within the core project area, encompassing 52,198 ha and comprising roughly 29\% of the project area. Wildfire was prevalent in this cover type. The frequency and extent of burned area is similar to that for the landscape as a whole (Figure \ref{fig:darea_smcx}). We summarize the disturbance regime in Tables \ref{tab:darea_smcx} and \ref{tab:darea_atleast_smcx}

\begin{figure}[!htbp]
  \centering
  \subfloat[][]{
    \centering
    \includegraphics[width=0.5\textwidth]{/Users/mmallek/Documents/Thesis/Plots/darea/hrv_smcx.png}
    }%
  \subfloat[][]{
    \includegraphics[width=0.5\textwidth]{/Users/mmallek/Documents/Thesis/Plots/darea/hrv_hist_smcx.png}
    }
  \caption{\small (a) Disturbance trajectory for Sierran Mixed Conifer - Xeric. High mortality fire in dark blue; low mortality fire in light blue. (b) Histogram of disturbed hectares with density curve overlaid.} 
  \label{fig:darea_smcx}
\end{figure}

\begin{table}[!htbp]
\centering
\caption{\small Disturbed area summary statistics for Sierran Mixed Conifer - Xeric. Proportions shown are relative to the total area of Sierran Mixed Conifer - Xeric.}
\label{tab:darea_smcx}
\begin{tabular}{@{}llll@{}}
\toprule
\textbf{\begin{tabular}[c]{@{}l@{}}Summary Statistic \\ (disturbed area/timestep)\end{tabular}} & \textbf{Low Mortality} & \textbf{High Mortality} & \textbf{Any Mortality} \\ \midrule
Minimum       & 0.65  & 0.07  & 0.93  \\
Maximum       & 54.14 & 36.11 & 86.29 \\
Median        & 10.93 & 5.99  & 17.94 \\
Mean          & 13.78 & 7.84  & 21.71 \\
\textbf{Fire Rotation} & 36       & 64        & 23 \\  \bottomrule
\end{tabular}
\end{table}

\begin{table}[!htbp]
\centering
\caption{Summary of disturbed area in terms of proportion of the amount of \textsc{smc\_x} burned (any level of mortality) during the simulation (after the equilibration period). For each benchmark proportion of the landscape, we list the number of timesteps during the simulation when that extent burned, the proportion of timesteps that represents, the interval in timesteps calculated from the proportion (i.e. approximately every 4 timesteps, at least 25\% of the landscape burned.), and the interval in years calculated from the interval in timesteps (5 years to a timestep).}
\label{tab:darea_atleast_smcx}
\begin{tabular}{@{}lllll@{}}
                        & at least 1\% & at least 10\% & at least 25\% & at least 50\% \\ \midrule
Number of timesteps     461          & 347           & 148           & 27            \\
Proportion of timesteps 1.00         & 0.75          & 0.32          & 0.06          \\
Interval (timesteps)    1.00         & 1.33          & 3.11          & 17.07         \\
Interval (years)        5.00         & 6.64          & 15.57         & 85.37         \\ \bottomrule
\end{tabular}
\end{table}


%%%%%%%%%%%%%%%%%%%%%%%%%%%%%%%%%%%%%%%%%%%%%%%%%%%%%%%%%%%%%%%%%%%%%%%%%%%%%%%%%%%%%%%%%%%%%%%%%%%%%%%%%%%%%%%%%%%%%%%%%%%%%%%%%%%%
%%%%%%%%%%%%%%%%%%%%%%%%%%%%%%%%%%%%%%%%%%%%%%%%%%%%%%%%%%%%%%%%%%%%%%%%%%%%%%%%%%%%%%%%%%%%%%%%%%%%%%%%%%%%%%%%%%%%%%%%%%%%%%%%%%%%

\subsection{Effect of Climate} 
Climate does have a positive relationship with disturbed area, as expected (Figure \ref{fig:climate_darea}). We show here a fitted line, but note the heteroskedastic variance about the mean. The relationship is weakly positive. During wetter-than-average years, we see less disturbed area. Over 20\% of the landscape burned only in timesteps during which the climate parameter was at least 0.63. However, over 50\% of the landscape burned in several timesteps when the climate parameter was less than 1, which is the average value over the historical period. Overall we observe that as climate shifts from wet to drought, the disturbed area increases. %Climate also has a weak effect on the size of individual fires (Figure \ref{fig:climate_dsize}). 
Fire size is also influenced by vegetation susceptibility and the specified disturbance size distribution. For this reason, large areas may burn in relatively ``wet'' years. Figure \ref{fig:compare_clim_darea} illustrates the climate parameter values and disturbed area proportion of the landscape for a subset of timesteps during the simulation.

\begin{figure}[!htbp]
  \centering
    \includegraphics[width=0.4\textwidth]{/Users/mmallek/Documents/Thesis/Plots/darea/hrv_climdarea.png}
  \caption{Plot of the climate parameter and disturbed area value for each timestep of the simulation (excluding the equilibration period). A linear model has been fit to the data and is shown as a blue line; the grey shaded area represents the 95\% confidence interval around the mean.}
  \label{fig:climate_darea}
\end{figure}


% updated 9/13
\begin{figure}[!htbp]
\centering
\includegraphics[width=0.8\textwidth]{/Users/mmallek/Documents/Thesis/Plots/darea/climate_darea_vert.png}
\caption{Climate parameter and proportion of eligible landscape disturbed by wildfire for timesteps 250 to 310 of the simulation, illustrating the wide variability in both climate parameter values and disturbed area per timestep. Purple lines are intended to aid in visualization of the climate paramter value and proportion of landscape burned during a particular timestep.}
\label{fig:compare_clim_darea}
\end{figure}

\clearpage


\subsection{Effect of Topographic Position}

The topographic position index value for a given cell acts as an input into the susceptibility and mortality values otherwise defined for that cover type and condition class combination. In general, cells with smaller TPI values had reduced susceptibility and mortality. Early development and open canopy conditions tend to result from fire, and we predicted that an increase in fires and in the likelihood of high mortality fire would lead to a decrease in the average canopy cover values for cells with large TPI values. Table~\ref{tab:tpi_cc} in Appendix \ref{app:full-results} displays the results for this simulation for the nine most common cover types. All  show decreased average canopy cover as TPI increases, with the decrease ranging from 5.7\% in xeric mixed evergreen forest to 28.8\% in ultramafic oak-conifer forests. Figure \ref{fig:tpi_cc_smc} shows the plotted data and fitted linear regression line for mesic and xeric sierran mixed conifer forests. Figure \ref{fig:averagecc} is a map displaying average canopy cover across the landscape for the full simulated HRV timeframe, excluding the equilibration period. 

% figure redone
\begin{figure}[!htbp]
\centering
\includegraphics[width=0.8\textwidth]{/Users/mmallek/Documents/Thesis/maps/hrv_tpi.pdf}
\caption{Smoothed visualization of the average canopy cover across the project area over the course of the simulation. Higher percent cover is shown in dark blue, transitioning to red where average percent cover was low. Water is shown in blue; barren is shown in grey.}
\label{fig:averagecc}
\end{figure}

% figure redone
\begin{figure}[!htbp]
\centering
\includegraphics[width=.8\textwidth]{/Users/mmallek/Documents/Thesis/Plots/tpi/hrv-facet-smc.png}
\caption{Average canopy cover for Sierran Mixed Conifer Mesic and Xeric during the simulated HRV. Each blue point represents one pixel of an individual cover type on the landscape grid. The black line is the result of a linear regression fit to the data. Table \ref{tab:tpi_cc} provides the numerical representation of the shift from minimum to maximum TPI values for each cover type. (a) Sierran Mixed Conifer - Mesic; (b) Sierran Mixed Conifer - Xeric.} 
\label{fig:tpi_cc_smc}
\end{figure}


%redone 9/15
\begin{table}[!htbp]
\centering
\caption{The percent change in canopy cover from the minimum TPI value for that cover type to the maximum TPI value. Results for Sierran Mixed Conifer Mesic and Xeric shown here; results for other focal cover types available in Appendix~\ref{app:full-results}}.
\label{tab:tpi_cc_smcs}
\begin{tabular}{@{}lrrrrr@{}}
\toprule
\small \textbf{\begin{tabular}[c]{@{}l@{}}Cover \\ Name\end{tabular}} & \small \textbf{\begin{tabular}[c]{@{}l@{}}Minimum \\ TPI\end{tabular}} & \small \textbf{\begin{tabular}[c]{@{}l@{}}Maximum \\ TPI\end{tabular}} & \small \textbf{\begin{tabular}[c]{@{}l@{}}Average Canopy \\Cover at \\ Minimum TPI\end{tabular}} & \small \textbf{\begin{tabular}[c]{@{}l@{}}Average Canopy \\ Cover at \\ Maximum TPI\end{tabular}}  & \small \textbf{\begin{tabular}[c]{@{}l@{}}Percent \\ Change in \\ Canopy \\ Cover\end{tabular}} \\ \midrule
\textsc{smc\_m   }    & -300                 & 300                  & 55.5       & 50.4              & -9.3      \\
\textsc{smc\_x   }    & -300                 & 300                  & 27.6       & 21.9              & -20.5     \\ \bottomrule
\end{tabular}
\end{table}

\newpage
\subsection{Fire Rotation} 
We present here the results for Sierran Mixed Conifer Mesic and Xeric. Full results are presented in Appendix \ref{app:full-results}.

\begin{table}[!htbp]
\centering
\caption{Fire rotation for Sierran Mixed Conifer Mesic and Xeric.}
\begin{tabular}{@{}lrrr@{}}
\toprule
\begin{tabular}[c]{@{}l@{}}Land Cover \\ Type\end{tabular}     & \begin{tabular}[c]{@{}l@{}}Low Mortality \\ Fire Rotation\end{tabular} & \begin{tabular}[c]{@{}l@{}}High Mortality \\ Fire Rotation\end{tabular} & \begin{tabular}[c]{@{}l@{}}All Fires \\ Rotation\end{tabular} \\ \midrule
\textsc{smc\_m   }    & 35                          & 113                          & 27                 \\
\textsc{smc\_x   }    & 34                          & 72                           & 23                 \\
\emph{Full Landscape    }      &\emph{ 38}                   & \emph{103}                   & \emph{28  }        \\ \bottomrule
\end{tabular}
\end{table}


\subsection{Population Return Interval}
Overall, the point-specific return interval (grand mean) for all eligible cells ranged from 17 years to \textgreater 2500 years (cells that never burned during the simulation) for both classes of wildfire mortality (Figure \ref{fig:preturn}). The median return interval across all cover types was 42 years for low mortality fire, 111 year for high mortality fire, and 29 years for any fire. The population return interval plots and maps specific to Sierran Mixed Conifer Mesic and Xeric follow (Figures~\ref{fig:preturn_smcm} and \ref{fig:preturn_smcx}). The other seven focal cover types\todo{where is this established??} are included in Appendix~\ref{app:full-results}. We compare the current landscape's seral stage distribution to the simulated distribution and compute the HRV departure index in Tables \ref{tab:covcond1} and \ref{tab:covcond2}.\todo{figure out how to update this with most info in appendix}

% first plot redone 9/13
% second plot not redone yet
\begin{figure}[!htbp]
  \centering
  \subfloat[][]{
    \centering
    \includegraphics[height=.4\textheight]{/Users/mmallek/Documents/Thesis/Plots/preturn/hrv-total.png}
    \label{fig:preturn_plot}
  }%
  \qquad
  \subfloat[][]{
    \includegraphics[height=.4\textheight]{/Users/mmallek/Tahoe/Report2/images/fri_all.png}
    \label{fig:preturn_map}
  }
  \caption{(a) Population return interval (average number of years between fires) distribution for the full landscape under study. The population return interval is the point-specific interval, sometimes described as the ``grand mean'' for a given point. (b) Spatial depiction of fire return intervals across the landscape, for all cover types, in terms of fire return interval. The value at any given cell is the point-specific return interval.}
  \label{fig:preturn}
\end{figure}

% first plot updated 9/13
\begin{figure}[!htbp]
  \centering
  \subfloat[][]{
    \centering
    \includegraphics[width=0.5\textwidth]{/Users/mmallek/Documents/Thesis/Plots/preturn/hrv-smcm.png}
    }%
  \subfloat[][]{
    \includegraphics[width=0.5\textwidth]{/Users/mmallek/Tahoe/Report2/images/fri_smcm.png}
    }
  \caption{(a) Population return interval (average number of years between fires) distribution for Sierran Mixed Conifer - Mesic.  (b) Spatial depiction of fire return intervals across the landscape. Cover types other than Sierran Mixed Conifer - Mesic are partially obscured in grey. The value at any given cell is the point-specific return interval, which ranges from 18 years to \textgreater 500 years.}
\label{fig:preturn_smcm}
\end{figure}

%first plot redone 9/13
\begin{figure}[!htbp]
  \centering
  \subfloat[][]{
    \centering
    \includegraphics[width=0.5\textwidth]{/Users/mmallek/Documents/Thesis/Plots/preturn/hrv-smcx.png}
    }%
  \subfloat[][]{
    \includegraphics[width=0.5\textwidth]{/Users/mmallek/Tahoe/Report2/images/fri_smcx.png}
    }
  \caption{(a) Population return interval (average number of years between fires) distribution for Sierran Mixed Conifer - Xeric.  (b) Spatial depiction of fire return intervals across the landscape. Cover types other than Sierran Mixed Conifer - Xeric are partially obscured in grey. The value at any given cell is the point-specific return interval, which ranges from 17 years to \textgreater 500 years.}
\label{fig:preturn_smcx}
\end{figure}

\clearpage


%%%%%%%%%%%%%%%%%%%%%%%%%%%%%%%%
%%%%%%%%%%%%%%%%%%%%%%%%%%%%%%%%
%%%%%%%%%%%%%%%%%%%%%%%%%%%%%%%%

%\pagebreak[4]
\section{Vegetation Response}
\label{subsec:HRVvegresponse}

\subsection{Landscape Composition} \todo{when I redid the plots I did them starting at year 40. should i redo them with the line added in? can probably modify rmlstats call to do this.}


The seral stage distribution for each cover type varied over time. Evidence of both high mortality fire, which triggers a transition to early development conditions for all cover types, and low mortality fire, which can thin a stand and cause a transition to a more open canopy condition (within the same developmental stage), are visible in examining the output grids. Figure \ref{fig:covcondmaps} illustrates these changes for a sequence of four timesteps during the simulation.

\begin{figure}[!htbp]
  \centering
  \subfloat[][]{
    \includegraphics[width=0.5\textwidth]{/Users/mmallek/Tahoe/Report2/images/covcond660.png}
  }%
  \subfloat[][]{
    \includegraphics[width=0.5\textwidth]{/Users/mmallek/Tahoe/Report2/images/covcond665.png}
  }\\%
  \subfloat[][]{
    \includegraphics[width=0.5\textwidth]{/Users/mmallek/Tahoe/Report2/images/covcond670.png}
    }
  \subfloat[][]{
    \centering
    \includegraphics[width=0.5\textwidth]{/Users/mmallek/Tahoe/Report2/images/covcond675.png}
  }%
  \caption{A sequence of four timesteps during the simulation, showing changes in condition classes over time. Here we highlight the dominant cover type, Sierran Mixed Conifer - Mesic, and its classes, in order to illustrate the dynamics that play out over many years. (a) Timestep 660 (b) Timestep 665 (c) Timestep 670 (d) Timestep 675.}
  \label{fig:covcondmaps}
\end{figure}

In general, the seral stage distribution appeared to be in dynamic equilibrium, despite considerable variability from timestep to timestep. The exception is the Oak-Conifer Forest and Woodland cover type, which did not appear to reach equilibrium until around timestep 220.\todo{review} The seral stage dynamics and current seral stage distribution plots specific to Sierran Mixed Conifer Mesic and Xeric follow (Figures~\ref{fig:covcond_smcm} and \ref{fig:covcond_smcx}). Plots and tabular results for the other seven focal types are included in Appendix~\ref{app:full-results}, section~\ref{app:sec:seraldynamics}.

\subsubsection{Sierran Mixed Conifer - Mesic}
The age structure and dynamics of mesic mixed conifer forests illustrates the interaction between disturbance and succession processes. We focus our analysis on the 5$^{\text{th}}$ to 95$^{\text{th}}$ percentile range of variability for our simulation (excluding the equilibration period). %

The distribution of area among stand conditions within mesic mixed conifer forests fluctuated over time, as expected (Figure~\ref{fig:covcond_smcm}). For example, the percentage of mesic mixed conifer forests in the Early Development condition varied from 7\%--21\%, reflecting the dynamic nature of this cover type (Table~\ref{tab:covcond2}). This condition is currently within the simulated HRV (56$^{\text{th}}$ percentile). Mid Development - Closed was typically the most extensive condition class (11\%-29\%), but most of the condition classes were common throughout the simulation. The shift towards closed canopies when stands reached the Late Development stage may be due to an increasing resilience to wildfire disturbances by stands of that age: wildfires may burn the understory without significantly affecting overstory canopy cover. %

The seral-stage distribution appeared to be in dynamic equilibrium (i.e., the percentage in each stand condition varied about a stable mean). Our calculated current seral-stage distribution was never observed under the simulated HRV (Table~\ref{tab:covcond2}). The most notable departure was an increase in Mid Development - Closed and Late Development - Open extent, and a decrease in Mid Development - Moderate extent during the simulated HRV. These condition classes are currently all outside of the simulated HRV. In fact, Late Development - Open is rare on the current landscape (3.6\%), but present in similar proportions to the other classes during the HRV. The other two Late Development classes are within the HRV, with the closed canopy and moderate canopy conditions currently in the $86^{\text{th}}$ and $54^{\text{th}}$ percentiles, respectively. 

\begin{figure}[!htbp]
  \centering
  \subfloat[][]{
    \centering
    \includegraphics[width=0.6\textwidth]{/Users/mmallek/Documents/Thesis/Plots/covcond-dynamics/hrv_covcond_smcm.png}
    }%
  \subfloat[][]{
    \includegraphics[height=2.65in]{/Users/mmallek/Tahoe/R/Rplots/November2014/covcond_current_smcm.png}
    }
  \caption{(a) Cover-Condition dynamics for Sierran Mixed Conifer - Mesic. The black vertical line at 40 timesteps marks the end of the equilibration period used in this study. (b) Current seral stage distribution for Sierran Mixed Conifer - Mesic.} 
  \label{fig:covcond_smcm}
\end{figure}

%\begin{landscape}

\begin{figure}[!htbp]
  \centering
    \includegraphics[width=\textwidth]{/Users/mmallek/Documents/Thesis/Plots/covcond-bycover/SMCM-HRV-boxplots-.png}
  \caption{Boxplots showing the range of variability for each seral stage over the course of the simulation, excluding the equilibration period. Boxplots were modified so that whiskers extend from the $5^{\text{th}} - 95^{\text{th}}$ percentiles of the observed results. Thick black bars in line with the boxplots denote the current proportion of mesic mixed conifer forests in a given seral stage.} 
  \label{fig:covcond_smcm_boxplots}
\end{figure}


\begin{table}[!htbp]
\centering
\caption{Range of variation in landscape structure, illustrating the cover-condition class dynamics for Sierran Mixed Conifer - Mesic (\textsc{smc\_m}). For condition class abbreviations, see Table \ref{condtable}.}
\label{tab:covcond_smcm}
\begin{tabular}{@{}rrrrrr|rrr@{}}
\toprule
\footnotesize \textbf{\begin{tabular}[c]{@{}l@{}}Condition \\ Class\end{tabular}}  & \footnotesize \textbf{srv5\%} & \footnotesize \textbf{srv25\%} & \footnotesize \textbf{srv50\%} & \footnotesize \textbf{srv75\%} & \footnotesize \textbf{srv95\%}  & \footnotesize \textbf{\begin{tabular}[c]{@{}l@{}}Current\\ \%cover\end{tabular}} & \textbf{\begin{tabular}[c]{@{}l@{}}Current\\ \%srv\end{tabular}} & \textbf{\begin{tabular}[c]{@{}l@{}}Departure\\ Index\end{tabular}} \\ \midrule
\footnotesize \textsc{early\_all}        & \footnotesize  7.75        & \footnotesize 12.34   & \footnotesize 15.11     & \footnotesize 18.68   & \footnotesize 24.74     & \footnotesize 14.98    & \footnotesize 48    & \footnotesize -4      \\
\footnotesize \textsc{mid\_cl   }        & \footnotesize  21.52       & \footnotesize 26.15   & \footnotesize 29.69     & \footnotesize 32.58   & \footnotesize 37.01     & \footnotesize 9.74     & \footnotesize 0     & \footnotesize -100     \\
\footnotesize \textsc{mid\_mod  }        & \footnotesize  6.8         & \footnotesize 7.98    & \footnotesize 9.03      & \footnotesize 10.3    & \footnotesize 12.63     & \footnotesize 17.97    & \footnotesize 100   & \footnotesize 100     \\
\footnotesize \textsc{mid\_op   }        & \footnotesize  6.68        & \footnotesize 9.2     & \footnotesize 11.21     & \footnotesize 13.08   & \footnotesize 16.15     & \footnotesize 16.29    & \footnotesize 96    & \footnotesize 92     \\
\footnotesize \textsc{late\_cl  }        & \footnotesize  5.31        & \footnotesize 9.54    & \footnotesize 12.87     & \footnotesize 17.2    & \footnotesize 22.91     & \footnotesize 23.23    & \footnotesize 97    & \footnotesize 94      \\
\footnotesize \textsc{late\_mod }        & \footnotesize  8.56        & \footnotesize 10.32   & \footnotesize 11.24     & \footnotesize 12.56   & \footnotesize 14.41     & \footnotesize 14.18    & \footnotesize 95    & \footnotesize 90      \\
\footnotesize \textsc{late\_op  }        & \footnotesize  4.96        & \footnotesize 7.39    & \footnotesize 9.26      & \footnotesize 12.12   & \footnotesize 14.95     & \footnotesize 3.6      & \footnotesize 1     & \footnotesize -98      \\
  \hline
\end{tabular}
\end{table}

%\end{landscape}
%%%%%%%%%%%%%%%%%%%%%%%%%%%%%%%%%%%%%%%%%%%%%%%%%%%%%%%%%%%%%%%%%%%%%%%%%%%%%%%%%%%%%%%%%%%%%%%%
\subsubsection{Sierran Mixed Conifer - Xeric}


\begin{figure}[!htbp]
  \centering
  \subfloat[][]{
    \centering
    \includegraphics[width=0.6\textwidth]{/Users/mmallek/Documents/Thesis/Plots/covcond-dynamics/hrv_covcond_smcx.png}
    }%
  \subfloat[][]{
    \includegraphics[height=2.65in]{/Users/mmallek/Tahoe/R/Rplots/November2014/covcond_current_smcx.png}
    }
  \caption{(a) Cover-Condition dynamics for Sierran Mixed Conifer - Xeric. The black vertical line at 40 timesteps marks the end of the equilibration period used in this study. (b) Current seral stage distribution for Sierran Mixed Conifer - Xeric.} 
  \label{fig:covcond_smcx}
\end{figure}

The\todo{rewrite this section to just point out what is interesting} age structure and dynamics of xeric mixed conifer forests illustrates the interaction between disturbance and succession processes. We focus our analysis on the 5$^{\text{th}}$ to 95$^{\text{th}}$ percentile range of variability for our simulation (excluding the equilibration period). %

The distribution of area among stand conditions within xeric mixed conifer forests fluctuated over time, as expected (Figure~\ref{fig:covcond_smcx}). For example, the percentage of xeric mixed conifer forests in the Early Development varied from 24\% to 44\%, reflecting the dynamic nature of this cover type (Table~\ref{tab:covcond3}). During the simulation, Early Development (which includes post-fire chaparral fields) and Mid Development - Open conditions dominated, in contrast to the current distribution, which is somewhat even across classes (although late development, open canopy stands are currently quite rare).  %

The seral-stage distribution appeared to be in dynamic equilibrium (i.e., the percentage in each stand condition varied about a stable mean). Our calculated current seral-stage distribution was never observed under the simulated HRV (Table~\ref{tab:covcond3}). In fact, none of the condition classes had a distribution within the simulated HRV. The most dramatic departure was the increase in Early Development and Mid Development - Open during the simulated HRV compared to the current landscape (currently at 19\% and 11\%, respectively). We also observed a much lower proportion of xeric mixed conifer forest in Late Development - Closed during the simulation than in the current landscape (25\%). The decline in the extent of Late Development forests is primarily due to the frequency of high mortality fire, which inhibits stands from succeeding to those stages. As stated above, this cover type experienced the most fire during the simulated HRV. High mortality fire directly led to the increase in Early Development conditions, and the dominance of open canopies within the middle and late development stages is explained by the high frequency of low mortality fire.



\begin{figure}[!htbp]
  \centering
    \includegraphics[width=\textwidth]{/Users/mmallek/Documents/Thesis/Plots/covcond-bycover/SMCX-HRV-boxplots-.png}
  \caption{Boxplots showing the range of variability for each seral stage over the course of the simulation, excluding the equilibration period. Boxplots were modified so that whiskers extend from the $5^{\text{th}} - 95^{\text{th}}$ percentiles of the observed results. Thick black bars in line with the boxplots denote the current proportion of mesic mixed conifer forests in a given seral stage.} 
  \label{fig:covcond_smcx_boxplots}
\end{figure}

\begin{table}[!htbp]
\centering
\caption{Range of variation in landscape structure, illustrating the cover-condition class dynamics for Sierran Mixed Conifer - Xeric (\textsc{smc\_x}). For condition class abbreviations, see Table \ref{condtable}.}
\label{tab:covcond_smcx}
\begin{tabular}{@{}rrrrrr|rrr@{}}
\toprule
\footnotesize \textbf{\begin{tabular}[c]{@{}l@{}}Condition \\ Class\end{tabular}}  & \footnotesize \textbf{srv5\%} & \footnotesize \textbf{srv25\%} & \footnotesize \textbf{srv50\%} & \footnotesize \textbf{srv75\%} & \footnotesize \textbf{srv95\%}  & \footnotesize \textbf{\begin{tabular}[c]{@{}l@{}}Current\\ \%cover\end{tabular}} & \textbf{\begin{tabular}[c]{@{}l@{}}Current\\ \%srv\end{tabular}} & \textbf{\begin{tabular}[c]{@{}l@{}}Departure\\ Index\end{tabular}} \\ \midrule
\footnotesize \textsc{early\_all}      & \footnotesize 25.2          & \footnotesize 29.63    & \footnotesize 34.53    & \footnotesize 38.95    & \footnotesize 42.82     & \footnotesize 19.48       & \footnotesize  0      & \footnotesize -100    \\
\footnotesize \textsc{mid\_cl   }      & \footnotesize 0.02          & \footnotesize 0.06     & \footnotesize 0.13     & \footnotesize 0.36     & \footnotesize 1.07      & \footnotesize 11.96       & \footnotesize  100    & \footnotesize 100      \\
\footnotesize \textsc{mid\_mod  }      & \footnotesize 0.9           & \footnotesize 1.62     & \footnotesize 2.88     & \footnotesize 4.35     & \footnotesize 7.6       & \footnotesize 14.92       & \footnotesize  100    & \footnotesize 100    \\
\footnotesize \textsc{mid\_op   }      & \footnotesize 26.55         & \footnotesize 30.59    & \footnotesize 33.79    & \footnotesize 36.58    & \footnotesize 39.36     & \footnotesize 11.48       & \footnotesize  0      & \footnotesize -100    \\
\footnotesize \textsc{late\_cl  }      & \footnotesize 1.19          & \footnotesize 2.51     & \footnotesize 3.81     & \footnotesize 5.99     & \footnotesize 8.69      & \footnotesize 24.72       & \footnotesize  100    & \footnotesize 100      \\
\footnotesize \textsc{late\_mod }      & \footnotesize 5.83          & \footnotesize 7.49     & \footnotesize 9.16     & \footnotesize 10.71    & \footnotesize 13.03     & \footnotesize 13.31       & \footnotesize  97     & \footnotesize 94     \\
\footnotesize \textsc{late\_op  }      & \footnotesize 9.39          & \footnotesize 12.4     & \footnotesize 15       & \footnotesize 17.42    & \footnotesize 22.45     & \footnotesize 4.13        & \footnotesize  0     & \footnotesize  -100  \\ \bottomrule 
\end{tabular}
\end{table}

%%%%%%%%%%%%%%%%%%%%%%%%%%%%%%%%%%%%%%%%%%%%%%%%%%%%%%%%%%%%%%%%%%%%%%%%%%%%%%%%%%%%%%%%%%%%%%%%
%%%%%%%%%%%%%%%%%%%%%%%%%%%%%%%%%%%%%%%%%%%%%%%%%%%%%%%%%%%%%%%%%%%%%%%%%%%%%%%%%%%%%%%%%%%%%%%%

\subsection{Landscape Configuration}
One of the principal purposes of gaining a better quantitative understanding of the historic reference period is to know whether recent human activities have caused landscapes to move outside their historic range of variability (Landres et al.1999, Swetnam et al. 1999). We summarized the structure and patterns in the landscape using a suite of statistical measures calculated using \textsc{Fragstats}. Table \ref{tab:fragland} shows the range of variability for the simulation period as well as the current value, SRV percentile for the current value (``Current SRV Percentile''), and the departure index. We show here a subset of metrics most useful for understanding patch characteristics in the study area. See Appendix \ref{app:metricdescriptions} for a detailed description of \textsc{Fragstats} metrics. At the landscape-level, most computed metrics have values outside the HRV. In Figures \ref{fig:fragland_areashape}, \ref{fig:fragland_contagsiei}, and \ref{fig:fragland_core} we highlight a further subset of the metrics from Table \ref{tab:fragland} for the purposes of discussing the landscape under the simulated historic period as compared to the present day. %Figures for all metrics are included in \todo{the appendix?}. *I don't think we need to include all the figures


%%% Decided that this departure index didn't make sense
%The departure index indicates the distance from the 50$^{\text{th}}$ percentile value. This provides an uncapped index, allowing for interpretation of ``how far'' outside the 90\% range of variability a current landscape metric is. 
%If the current value is less than the median from the simulation, the departure index is given by:
%$$ \frac{ \text{observed} - \text{median} }{ \text{median} - 5^{th} \text{percentile} }  $$
%If the current value is greater than the median from the simulation, the departure index is given by:
%$$ \frac{ \text{observed} - \text{median} }{95^{th} \text{percentile} - \text{median}  }  $$
%If the current value is outside the HRV, its departure index value will be greater than 100 or less than -100. 

% repaired table 9/13
\begin{landscape}
\begin{table}[!htbp]
\centering
\caption{Range of variability during the simulation for a selected suite of landscape configuration metrics calculated using \textsc{Fragstats}. The landscape metrics listed here are described in detail in the 
\textsc{Fragstats} methods section. 
\textsc{te} = total edge;
\textsc{area\_am} = area-weighted mean patch size; 
\textsc{gyrate\_am} = area-weighted mean patch radius of gyration (correlation length); 
\textsc{shape\_am} = area-weighted mean patch shape index; 
\textsc{core\_am} = area-weighted mean patch core area; 
\textsc{simi\_mn} = mean similarity; 
\textsc{cwed} = contrast-weighted edge density; 
\textsc{econ\_am} = area-weighted mean edge contrast; 
\textsc{contag} = contagion; 
\textsc{siei} = Simpson's evenness index; 
\textsc{ai} = aggregation index.}
\label{tab:fragland}
\begin{tabular}{@{}llllll|lll@{}}
\toprule
\textbf{\begin{tabular}[c]{@{}l@{}}Landscape\\ Metric\end{tabular}}  & \textbf{srv5\%} & \textbf{srv25\%} & \textbf{srv50\%} & \textbf{srv75\%} & \textbf{srv95\%}  & \textbf{\begin{tabular}[c]{@{}l@{}}Current\\ Value\end{tabular}} & \textbf{\begin{tabular}[c]{@{}l@{}}Current\\ \%SRV\end{tabular}} & \textbf{\begin{tabular}[c]{@{}l@{}}Departure\\ Index\end{tabular}} \\ \midrule
\small \textsc{te}              & $2.19 \times 10^7$  & $2.21 \times 10^7$ & $2.23 \times 10^7$ & $2.25 \times 10^7$ & $2.27 \times 10^7$    & $2.34 \times 10^7$      & 100      & 100  \\
\small \textsc{area\_am}         & 156.549  & 166.016  & 174.884  & 184.448  & 205.209    & 119.985       & 0        & -100 \\
\small \textsc{gyrate\_am}       & 693.361  & 705.323  & 715.921  & 730.824  & 758.915    & 620.951       & 0        & -100 \\
\small \textsc{shape\_am}        & 3.56     & 3.621    & 3.667    & 3.727    & 3.847      & 3.243         & 0        & -100 \\
\small \textsc{core\_am}         & 135.146  & 141.964  & 149.582  & 157.587  & 169.545    & 106.71        & 0        & -100 \\
\small \textsc{simi\_mn}         & 2333.717 & 2456.329 & 2531.906 & 2629.83  & 2794.671   & 2095.764      & 0        & -100 \\
\small \textsc{cwed}             & 40.608   & 41.114   & 41.51    & 41.95    & 42.564     & 36.092        & 0        & -100 \\
\small \textsc{econ\_am}         & 32.793   & 33.163   & 33.458   & 33.833   & 34.401     & 27.756        & 0        & -100 \\
\small \textsc{contag}           & 53.943   & 54.455   & 54.744   & 55.064   & 55.523     & 51.172        & 0        & -100 \\
\small \textsc{siei}             & 0.946    & 0.949    & 0.951    & 0.953    & 0.956      & 0.971         & 100      & 100  \\
\small \textsc{ai}               & 81.531   & 81.699   & 81.821   & 81.974   & 82.168     & 80.963        & 0        & -100 \\ \bottomrule
\end{tabular}
\end{table}
\end{landscape}

\clearpage
\begin{figure}[!htbp]
  \centering
  \subfloat[][]{
    \centering
\includegraphics[width=0.5\textwidth]{/Users/mmallek/Documents/Thesis/Plots/fragland-hrv/AREA_AM1.png}
    }%
  \subfloat[][]{
\includegraphics[width=0.5\textwidth]{/Users/mmallek/Documents/Thesis/Plots/fragland-hrv/SHAPE_AM1.png}
  }
\caption{Landscape \textsc{Fragstats} Metrics. Left, Area-weighted Mean Patch Area. Right, Area-weighted Mean Shape. We use the area-weighted metrics to reduce the influence of the many extremely small patches. The average patch size is larger, and the average patch shape more complex, than the current landscape.} 
\label{fig:fragland_areashape}
\end{figure}

\begin{figure}[!htbp]
  \centering
  \subfloat[][]{
    \centering
\includegraphics[width=0.5\textwidth]{/Users/mmallek/Documents/Thesis/Plots/fragland-hrv/CONTAG1.png}
    }%
  \subfloat[][]{
\includegraphics[width=0.5\textwidth]{/Users/mmallek/Documents/Thesis/Plots/fragland-hrv/SIEI1.png}
  }
\caption{Landscape \textsc{Fragstats} Metrics. (a) Contagion, a metric describing patch dispersion and interspersion. The landscape during the HRV is much more contagious than the current landscape. (b) Simpson's Evenness Index, which indicates the distance from maximum diversity, or evenness, in the landscape patches. Values for Simpson's Evenness are near 1 during the HRV and in the present landscape, but the HRV values are well below the current conditions.} 
\label{fig:fragland_contagsiei}
\end{figure}

\begin{figure}[!htbp]
  \centering
  \includegraphics[width=0.5\textwidth]{/Users/mmallek/Documents/Thesis/Plots/fragland-hrv/CORE_AM1.png}
\caption{Landscape \textsc{Fragstats} Metrics. Results for the Area-weighted Mean Core Area, a measure of interior habitat available at the patch level. During the HRV, the average patch contains more core area than in the current landscape.} 
\label{fig:fragland_core}
\end{figure}
