%\bibliography{bibliography.bib} %The files containing all the articles and books you ever referenced. 
\chapter{Methods}
\label{ch:methods}
\section{Study Area}
\label{sec:studyarea}

\begin{figure}
\includegraphics[width=\textwidth]{/Users/mmallek/Tahoe/Report3/images/ecoregionprojectarea.jpg}
\caption{The Sierra Nevada Ecoregion is outlined in brown. The project landscape (outlined in green) is located in the northern extent of the Sierra Nevada on the Tahoe National Forest, comprising the Yuba River watershed.}
\label{projectarea}
\end{figure}

The Sierra Nevada is a major North American mountain range, located east of California's Central Valley and extending from Fredonyer Pass in the north to southern Kern County in the south. Much of the Sierra Nevada is reserved as federally-held public land, managed by the U.S. Forest Service, Bureau of Land Management, and the National Park Service. The Plumas and Tahoe National Forests are located in the northern portion of the Sierra Nevada. The project landscape (see Figure~\ref{projectarea}) is located on the northern part of the Tahoe National Forest, on the Yuba River and Sierraville Ranger Districts, and comprises about 181,550 hectares. It is composed of a set of three HUC-5 watersheds, the Upper North Yuba River, the Middle Yuba River, and the Lower North Yuba River, collectively referred to in this document as the Upper Yuba River watershed. 

The topography of the project landscape consists of rugged mountains incised by two major and a few minor river drainages. Elevation ranges from about 350 to 2500 meters. The area receives 30--260 cm of precipitation annually, most of which falls as snow in the middle to upper elevations (Storer and Usinger 1963). Some areas in the mid-elevation band receive high precipitation compared to the region, resulting in patches of exceptionally productive forest \citetext{Alan\ Doerr,\ pers.\ comm.}. Vegetation is tremendously diverse and changes slowly along an elevational gradient and in response to local changes in drainage, aspect, and soil structure. Grasslands, chaparral, oak woodlands, mixed conifer forests, and subalpine forests are all found within the study area. %Many species exhibit fire-adapted traits, such as resprouting from roots after a fire (e.g. tanoak), fire-induced germination (e.g. manzanita), or thick bark (e.g. ponderosa pine). 

Prior to European settlement, wildfire was a major source of disturbance on the landscape, shaping the composition and configuration of vegetation in the Forest. Fires were primarily lightning-caused, although indigenous peoples are thought to have set fires for vegetation management, especially in the lower elevations. In general, fire was frequent, with a mean rotation as short as 20 years in Ponderosa Pine-dominated forests. Wetter mixed conifer areas are predicted to have had a mean fire rotation of 30 years. Fire rotation is thought to increase gradually with elevation. For example, mesic Red Fir forests, which exist around 2,000 feet higher in elevation than Ponderosa Pine forests, had a mean fire rotation of 60 years. Variance around these means can be significant, as some parts of the forest experience fire much more frequently, while other escape fire for long periods. In general, regardless of vegetation type, high mortality fires were thought to be rare, with the vast majority of fires killing under 70\% of overstory trees. Under this disturbance regime, stand-replacing fire initiated early development conditions on the landscape. Low mortality fire tended to open forest canopies, especially in more xeric parts of the forest, while vegetation succession closed them again. The rarity of high mortality fire allowed large forest stands to succeed into late development and old growth conditions \citep{SNEP1996,Mallek2013,Safford2014,SNEP1996a}.

The arrival of Europeans in the 1850s sparked a transformation of this landscape as people harvested timber, extracted gold using hydraulic mining techniques, and suppressed wildfires \citep{Storer1963}. Forestry, mining, grazing, and dozens of recreational activities, including hunting, mountain biking, and hiking continue to take place on the Tahoe National Forest. Fifteen allotments exist within the project area for both cattle and sheep grazing. In addition, 231,368 hectares inside of the project area have non-Forest Service ownership. Many of these lands were privately held, often by timber companies, before the Forest was created. In addition, many public lands were given to the Central Pacific Railroad in the late 19th century, and this ``checkerboard'' ownership pattern persists today (Figure~\ref{ownership}). Mining of gold and other minerals also continues. These activities affect and interact with ongoing vegetation succession and disturbance processes in the area \citep{USDAForestService2014}.

\begin{figure}[!htbp]
\centering
\includegraphics[width=0.3\textheight]{/Users/mmallek/Tahoe/Report3/images/ownership2.png}
\caption{Map distinguishing between National Forest lands and lands held by other entities (including private, industry, and other public land). Forest lands are in green, with other ownership in tan. The boundary of the study area is in black; this image shows the 10 km buffer.} 
\label{ownership}
\end{figure}

The Tahoe National Forest has an active fire management program, and maintains a cooperative agreement with the State of California to help fight wildfires on the non-Forest Service lands located with the Forest boundary. In the recent history of the forest, very few acres have burned, with the exception of 1960, when approximately 100,000 acres burned on the Forest. The low burned acreage corresponds with fairly high fire starts (both human- and lightning-caused) \citep{USDAForestService1990}. Fire suppression has lead to an increase in the amount of dead and downed fuels in the forest \citep{USDAForestService2004}.

Logging has been a major human impact on the Forest since European settlement. Clearcutting, shelterwood, salvage cutting, and plantation management have been major components of timber management on the TNF for several decades. More recently, group cut strategies replaced clearcutting as a management alternative. Together with fire suppression, timber management on the Forest has effected changes in forest composition. In general, shade intolerant species such as ponderosa pine, sugar pine, and Douglas fir have become less common, while shade tolerant species, especially white fir, have become more predominant \citep{USDAForestService1990}. Between 1988 and 2002, timber sales in the Sierra Nevada dropped drastically, but on the Tahoe National Forests timber sale levels have fluctuated both up and down (although annual sawtimber sold has decreased similarly to other Sierran Forests) \citep{USDAForestService2004}.


\section{Modeling Framework}
\label{sec:modelframe}

\subsection{RMLands}
\textsc{RMLands} is a spatially-explicit, stochastic, landscape-level disturbance and succession model capable of simulating fine-grained processes over large spatial and long temporal extents \citep{McGarigal2005}. It is grid-based but simulates disturbance and succession at the patch level, and employs a dichotomous high vs. low mortality effect of fire in place of the uniformly low, mixed, and high severity regimes used elsewhere \citep{McGarigal2012}. \textsc{RMLands} was originally developed to simulate Rocky Mountain Landscapes (hence the name), specifically the San Juan National Forest. Historical range of variability analyses were completed in 2005 for the San Juan National Forest and the Uncompaghre Plateau (McGarigal and Romme 2005a, McGarigal and Romme 2005b). \textsc{RMLands} has also been used to simulate wildfire and vegetation succession in northern Idaho \citep{Cushman2011}. Outputs from the model are readable by the landscape pattern analysis software \textsc{Fragstats}, which facilitates the landscape configuration analysis.


Because this study relied on the use of computer models, we note here some limitations that should be understood before applying the results in a management context. First, while it is important to recognize the many advantages of models, it is critical to understand that models are abstract and simplified representations of reality. \textsc{RMLands}, in particular, simulates wildfires, but does not simulate all of the disturbance processes or all of the complex interactions among them that characterize real landscapes. Ultimately, the results of a model are constrained by the quality of input data, which are not perfect. For example, the vegetation cover layer is subject to human interpretation errors and objective classification errors, and is further limited by the spatial resolution of the grid. The most appropriate use of the results is therefore to help identify the most influential factors driving landscape change, implications of our simulated disturbance and succession regime, and areas where further research is needed to delineate key parameters.

Second, it is important to realize that \textsc{RMLands} requires substantial parameterization before it can be applied to a particular landscape. To the extent possible, we have utilized local empirical data. However, we also drew on relevant scientific studies, often from other geographic locations, and relied heavily on expert opinion when scientific studies and local empirical data were not available. Our estimate of the HRV is subject to change as new scientific understanding or better data become available.

Third, the Sierra Nevada vegetation is extremely diverse and complex in its spatial arrangement and scale of mapping. In this report we limit our results to an evaluation of the full landscape and of its two most extensive land cover types, which together comprise 63\% of the study area. Results for the next seven most extensive types are included in the appendices. In general, our confidence in the results decline as the extent of a cover type declines, because the results are statistical and large samples are needed. We do not provide results for cover types that extend across less than 1000 ha of the study area.


Fourth, this report (and \textsc{RMLands}) focuses on the effects of one major natural disturbance: fire. Other kinds of natural disturbances also occur, including insects and disease, wind-throw, ungulate and beaver herbivory, avalanches, and other forms of soil movement, but the impacts of these other disturbances tend to be localized in time or space and have far less impact on vegetation patterns over broad spatial and temporal scales than does fire.\todo{can we say this here?}
 


\subsection{Input Layers}
\label{subsec:hrvinputlayers}

All input layers to \textsc{RMLands} must be custom-built to work with the software. For technical details on the data structure requirements of \textsc{RMLands}, see Appendix \ref{app:inputs}. A brief overview of each input layer is included below.

\subsubsection{Cover} Cover type is based on the potential or current natural vegetation of a site and includes both natural and anthropogenic cover types. For example, cover types include not only Lodgepole Pine, Sierran Mixed Conifer, and Red Fir, but also Barren and Agriculture. Succession pathways are defined uniquely for each cover type, susceptibility to natural disturbances varies among cover types, and suitability or eligibility for various vegetation treatments varies among cover types. Cover is a static/constant grid and therefore provides a fixed template upon which disturbance and succession processes play out over time. 

The source for the cover layer is the Region 5 Existing Vegetation Layer (``EVeg''), first mapped to the \textsc{calveg} classification developed by the Region's Ecology Program in 1978. When deciding on land cover types, including determining xeric and mesic subtypes, our focus was to best represent the project area and the surrounding landscape. We used the CALVEG Mapping Zone boundary for the ``North Sierra'' (Figure~\ref{calveg}) as our focus for defining vegetation and disturbance, including susceptibility, response to fire, and fire size and distribution. Within the project area, the EVeg layer was developed based on three separate efforts: a satellite-based imagery analysis in 2000, and two orthoimagery analysis completed by contracting firms in 2005 \citetext{Alan\ Doerr,\ pers.\ comm.}. Generally, specific cover type names were derived from the California Fire Return Interval Departure (FRID) report by \citet{VandeWater2011}

\begin{wrapfigure}{r}{0.5\textwidth}
\includegraphics[width=0.48\textwidth]{images/CALVEGmappingzones.png}
\caption{\small CALVEG Mapping Zones. These zones meet U.S. Forest Service standard at national and regional levels. These ecological provinces are associated with dozens of vegetation alliances, which are used to classify vegetation in spatial data products.} 
\label{calveg}
\end{wrapfigure}

\paragraph{Alternative Cover Layers}
The original intent of our team was to utilize two separate cover layers: one for the historical reference period, and one for the current period to be used in projections of future scenarios. Two layers were identified as potentially suitable for the historic analysis: a map created from forest survey and inventory efforts under Albert Wieslander conducted between 1928 and 1940 (``Wieslander'') \citep{Thorne2006}, and a map of Potential Natural Vegetation created by a Forest Service Enterprise Team for the Tahoe National Forest in the 2000s (Forest Service internal GIS data). Our intent was to use the PNV, Wieslander, or a combination thereof to derive the land cover layer for the HRV phase of the project. 

In order to validate the historical maps, we needed to develop a crosswalk between the vegetation type methodologies for the EVeg, PNV, and Wieslander maps. We also examined the spatial consistency in cover types across the maps. With significant assistance from the Tahoe National Forest, we attempted to create a crosswalk from these maps to the set of land cover types to be used in the project. However, we were unable to develop a consistent and comprehensive set of rules for this purpose. A major reason for this is that both the PNV and Wieslander maps used species lists, rather than assemblages (as in \textsc{calveg} and LandFire). For example, Sierran mixed conifer forests do not appear as a dominant ``cover type'' in the PNV map. The Wieslander maps do contain an internal crosswalk to a mixed conifer alliance, but only rarely. 

In addition, the PNV map contained a more significant error: we learned that, for the purposes of the modeling used to create the PNV map, ``potential natural vegetation'' meant the so-called ``climax'' community that would develop in the complete absence of disturbance, regardless of whether that disturbance was human-caused or natural. Since we are seeking to mimic the natural historic range of variability, we decided to discard this layer. The Wieslander map had its own issues. Most problematic was the non-systematic spatial error of up to 300 meters, which meant it would not be suitable for comparing specific locations. In addition, crosswalking precisely was impossible because coded vegetation was not necessarily in order of most prevalent vegetation, but instead prioritized tree species over shrubs, and commercially important trees over others. As an example from the handbook states, a plot consisting of 75\% \emph{Quercus kelloggi} (black oak), 15\% \emph{Pinus ponderosa} (ponderosa pine), and 10\% \emph{Pinus lambertiana} (grey pine) would be coded as ponderosa pine, grey pine, black oak. Consequently, the Wieslander map is also not a reliable predictor of land cover type without extensive review of the original data and maps, which would be beyond the scope of this project. 

To confirm these problems, we examined the overlap in land cover types between different maps in ArcGIS. In general, the overlap between EVeg and either the PNV or the Wieslander layers was no better than random, and in many cases it was worse. We decided, in conjunction with Tahoe National Forest staff, to proceed using only the EVeg map, and omit the calibration period of the model from our analysis of the characteristics of the HRV. This ensured that our analysis of future management scenarios and comparison of spatial metrics between those results and the HRV results was credible.
% in retrospect I wonder if we should have analyzed the configuration more. in the end the biggest problem was probably the lack of crosswalk, since a precise spatial equivalence wasn't assumed.

\paragraph{Selection of Specific Cover Types}
In the early stages of this project, the team created a suite of land cover types based roughly on the Wildlife Habitat Relationships (WHR) types used in California and by Forest Service managers and planners. These consisted of the WHR types with a few additional types where additional specificity or refinement was desired. For example, Red Fir was split up into two subtypes. The original concept was to begin with the WHR types and modify them as needed based on other attributes in the EVeg layer. However, creating a crosswalk from WHR to the project-specific types also proved problematic. First, we realized that the WHR values were actually derived from the \textsc{calveg} species alliances included in the EVeg layer, but the methodology used was unavailable or missing. The crosswalks we did find were not mutually exclusive and all-inclusive, and do not always make ecological sense \citep{Keeler-Wolf2007,DeBecker1988,Game2005}. This is probably due in part to the fact that WHR is not a mapping classification. It is always derived secondarily. So, we were unable to create consistent rules for mapping from WHR to other types. Others have encountered similar issues:
%
\begin{quote}
WHR has been less successful in differentiating between vegetation types. Because the habitat types are inconsistently defined, a broad familiarity with its detailed descriptions is needed to differentiate among types of similar structure. Although mappers have constructed rules for discriminating among types, difficulties still remain because species dominance varies substantially within some types and broad overlaps in dominant plants occur among types. Other problems arise due to the small number of classes and the inconsistencies in scale among them \citep[p.~23]{Keeler-Wolf2007}
\end{quote}
%
In collaboration with National Forest staff we decided to instead base our land cover types on, at the first order, Presettlement Fire Regime (PFR) types as defined in the Fire Return Interval Departure (FRID) report by \citet{VandeWater2011}. The PFR types, as part of the FRID, were developed through the scientific process and underwent peer review. We used the methodology from the FRID rather than using the second-order WHR classification and trying to reverse-engineer it to fit into our custom land cover types. Thus we created a new structure of cover types in a nested regime, moving from PFR (the coarsest aggregation of \textsc{calveg} types, which included a direct crosswalk from them to PFR types), to Biophysical Settings from LandFire (which were also crosswalked to PFR types in the FRID report), and finally to various local types not otherwise represented, such as xeric and mesic variants of cover types like Montane Hardwood, and aspen variants, such as Red Fir - Aspen. A mutually exclusive and all-inclusive crosswalk for each land cover type used in this analysis to a single LandFire Biophysical Setting and Presettlement Fire Regime type thus exists.

Extensive geoprocessing was required to prepare the EVeg layer for use in \textsc{RMLands}. Beyond converting the vector data to a raster format, further analysis was required to distinguish east- and west-side areas from one another, and generate the cover type modifications that the team agreed on. Aspen types were created by overlaying an aspen layer onto the vegetation layer and creating combined types (``type - Aspen'')where appropriate. Areas mapped as a vegetation type characteristic of early seral (e.g. chaparral) were analyzed and assigned an appropriate forested cover type. Ultramafic\footnote{Ultramafic soils are those created from the weathering of igneous rocks, brought to the earth's surface as magma, where they then cooled. Ultramafic soils are typically shallow, rocky, and nutrient deficient, with high levels of metals uncommon in other soils. Only a few species of plants have evolved to live on them, many of which are endemic to such soils. Plants that do grow mature more slowly and cover the land less continuously than the same plant would on better soil. In the study area, the most common ultramafic rock is serpentine \citep{Safford2004}.} types were created by overlaying a geology layer onto the vegetation layer and performing a similar processing step to create ``type-ultramafic''. Finally, for the Sierran Mixed Conifer and Red Fir cover types, which cover broad swaths of land across elevation and aspect, a xeric to mesic gradient was developed in conjunction with local experts and applied, creating ``type - Mesic'' and ``type - Xeric''. 

Ultimately, 31 cover types were generated for the buffered project area, as listed in Table~\ref{covertable} and shown in Figure~\ref{fig:inputlayermaps1}.\footnote{Larger images of all of the input layers are included in Appendix \ref{app:inputs}.}. %A thorough description of geoprocessing steps necessary to recreate this data layer will be available soon. 
As Table~\ref{covertable} shows, most cover types occupy a small extent of the project area. The cover types with an extent of less than 1000 ha within the core project area may have statistically unreliable results; this problem increases as the extent of given cover type decreases. We caution against attempting to make inferences for these less common cover types. However, because the nine cover types that do occur over at least 1000 ha represent approximately 93\% of the core project area, we have high confidence in the landscape-level results. These nine cover types were considered our focal cover types, and were all fully analyzed as part of the historical range of variability assessment. For space and continuity, in the main body of this report we discuss only the two most common cover types, Sierran Mixed Conifer - Mesic and Serrian Mixed Conifer - Xeric. Results for the other seven cover types are included in the appropriate appendices. 

%%%%%%%%%%%%%%%%%%%%%%
%%% COVER TABLE %%%%%%
%%%%%%%%%%%%%%%%%%%%%%
\small
\begin{table}[!htbp]
\caption{List of land cover types developed for this project. Included are the cover type abbreviation, full cover type name, and total area in the buffered project landscape in both acres and hectares.}
\label{covertable}
\begin{tabular}{@{}lllll@{}}
\toprule
\textbf{\begin{tabular}[c]{@{}l@{}}Land \\ Cover \\ Value\end{tabular}} & \textbf{\begin{tabular}[c]{@{}l@{}}Land Cover \\ Abbreviation\end{tabular}} & \textbf{Land Cover Name}    & \textbf{\begin{tabular}[c]{@{}l@{}}Area \\ Core Only\\ (Hectares)\end{tabular}} & \textbf{\begin{tabular}[c]{@{}l@{}}Area\\ Core+Buffer\\ (Hectares)\end{tabular}} \\ \midrule
\rowcolor[HTML]{CAD6BA} 1    & \textsc{agr     }     & Agriculture                                  & 16          & 5,416     \\
2                            & \textsc{bar     }     & Barren                                       & 2665        & 8,751     \\
\rowcolor[HTML]{CAD6BA} 3    & \textsc{cmm     }     & Curl-leaf Mountain Mahogany                  & 18          & 41        \\
4                            & \textsc{grass   }     & Grassland                                    & 1379        & 4,617     \\\rowcolor[HTML]{CAD6BA} 
5                            & \textsc{lpn     }     & Lodgepole Pine                               & 837         & 2,816     \\
6                            & \textsc{lpn\_asp}     & Lodgepole Pine with Aspen      & 8             & 31   \\
\rowcolor[HTML]{CAD6BA} 7    & \textsc{lsg     }     & Black and Low Sagebrush                      & 0           & 5         \\
8                            & \textsc{med     }     & Meadow                                       & 1201        & 3,435     \\
\rowcolor[HTML]{CAD6BA} 9    & \textsc{meg\_m  }     & Mixed Evergreen - Mesic                      & 7273        & 13,547    \\
10                           & \textsc{meg\_u  }     & Mixed Evergreen - Ultramafic                 & 604         & 1,655     \\
\rowcolor[HTML]{CAD6BA} 11   & \textsc{meg\_x  }     & Mixed Evergreen - Xeric                      & 6768        & 13,771    \\
12                           & \textsc{mrip    }     & Montane Riparian                             & 732         & 2,216     \\
\rowcolor[HTML]{CAD6BA} 13   & \textsc{oak     }     & Oak Woodland                                 & 19          & 4,186     \\
14                           & \textsc{ocfw    }     & Oak-Conifer Forest and Woodland              & 23729       & 56,941    \\
\rowcolor[HTML]{CAD6BA} 15   & \textsc{ocfw\_u }     & \begin{tabular}[c]{@{}l@{}}Oak-Conifer Forest and \\ Woodland -  Ultramafic\end{tabular} & 1060   & 2,185   \\
16                           & \textsc{rfr\_asp}     & Red Fir with Aspen                           & 0     		    & 34             \\
\rowcolor[HTML]{CAD6BA} 17   & \textsc{rfr\_m  }     & Red Fir - Mesic                              & 8,563  	      & 19,626         \\
18                           & \textsc{rfr\_u  }     & Red Fir - Ultramafic                         & 294   		    & 321            \\
\rowcolor[HTML]{CAD6BA} 19   & \textsc{rfr\_x  }     & Red Fir - Xeric                              & 7,493  	      & 9,989          \\
20                           & \textsc{sage    }     & Big Safebrush                                & 0     		    & 1,600          \\
\rowcolor[HTML]{CAD6BA} 21   & \textsc{scn     }     & Subalpine Conifer                            & 638   		    & 12,543         \\
22                           & \textsc{scn\_asp}     & Subalpine Conifer with Aspen                 & 0     		    & 6              \\
\rowcolor[HTML]{CAD6BA} 23   & \textsc{smc\_asp}     & Sierran Mixed Conifer with Aspen             & 58    		    & 121            \\
24                           & \textsc{smc\_m  }     & Sierran Mixed Conifer - Mesic                & 57,853 	      & 133,920        \\
\rowcolor[HTML]{CAD6BA} 25   & \textsc{smc\_u  }     & Sierran Mixed Conifer - Ultramafic           & 4,124  	      & 9,774          \\
26                           & \textsc{smc\_x  }     & Sierran Mixed Conifer - Xeric                & 52,198 	      & 91,443         \\
\rowcolor[HTML]{CAD6BA} 27   & \textsc{urb     }     & Urban                                        & 114   		    & 782            \\
28                           & \textsc{wat     }     & Water                                        & 4,058  	      & 8,212          \\
\rowcolor[HTML]{CAD6BA} 29   & \textsc{wwp     }     & Western White Pine                           & 273   		    & 510            \\
30                           & \textsc{ypn     }     & Yellow Pine                                  & 0     		    & 10,499         \\
\rowcolor[HTML]{CAD6BA} 31   & \textsc{ypn\_asp}     & Yellow Pine with Aspen                       & 0     		    & 3              \\ \bottomrule
\end{tabular}
\end{table}
\normalsize

\subsubsection{Seral Stage}
Seral stage classes combine developmental stage and canopy cover, and are defined for all cover types that undergo succession. Seral stages in this application are based on LandFire structural classes, and were further modified in collaboration with local experts on the Tahoe National Forest. In \textsc{RMLands}, susceptibility to and mortality from natural disturbances varies among seral stages. Unlike the cover grid, the seral stage grid changes dynamically over time in response to simulated succession and disturbance events. The combination of cover type and seral stage forms the basis for characterizing vegetation patterns and dynamics.

The source for the condition layer is the Region 5 Existing Vegetation Layer, mapped to the \textsc{calveg} classification. The \textsc{calveg} classification was developed by the Region's Ecology Program in 1978. Within the project area, the Existing Vegetation Layer was developed based on three separate efforts: a satellite-based imagery analysis in 2000, and two orthoimagery analysis completed by contracting firms in 2005. All members of the team discussed potential attributes to be used for this classification, and identified attributes for tree diameter at breast height and cover from above to classify pixels into early, middle, or late development, and open, moderate, and closed canopy. In this application, aspen and shrub types have condition classes that differ from that of the remaining forest types. The other forested types use a consistent set of seral stages.

Extensive geoprocessing was required to prepare this layer for \textsc{RMLands}. Beyond converting the vector data to a raster format, further analysis was required to update the layer to a year 2010 condition. Spatial data on wildfire and timber management history was used to provide a more accurate assessment of seral stage based on estimated stand age. In addition, areas currently mapped as chaparral in the Existing Vegetation Layer were assigned to the early development stage. The full set of seral stages is provided in Table~\ref{condtable} and depicted in Figure~\ref{fig:inputlayermaps1}.

%%%%%%%%%%%%%%%%%%%%%%
%%% CONDITION TABLE %%
%%%%%%%%%%%%%%%%%%%%%%

\begin{table}[!htbp]
\centering
\caption{List of condition classes developed for this project. Condition classes describe developmental stage (e.g. ``early'') and canopy closure (e.g. ``open''). Included are the condition class codes, abbreviations, and full names.}
\label{condtable}
\begin{tabular}{@{}lll@{}}
\toprule
\textbf{\begin{tabular}[c]{@{}l@{}}Condition Class \\ Value\end{tabular}} & \textbf{\begin{tabular}[c]{@{}l@{}}Condition Class \\ Abbreviation\end{tabular}} & \textbf{\begin{tabular}[c]{@{}l@{}}Condition Class  \\ Name\end{tabular}} \\ \midrule
\rowcolor[HTML]{CAD6BA} 
0                              & \textsc{ns}                                    & Non-Seral                     				 \\
10                             & \textsc{early\_all }                           & Early Development - All                   	 \\
\rowcolor[HTML]{CAD6BA} 
20                             & \textsc{mid\_cl    }                           & Mid Development - Closed                    \\
21                             & \textsc{mid\_mod   }                           & Mid Development - Moderate                  \\
\rowcolor[HTML]{CAD6BA} 
22                             & \textsc{mid\_op    }                           & Mid Development - Open                      \\
30                             & \textsc{late\_cl   }                           & Late Development - Closed                   \\
\rowcolor[HTML]{CAD6BA} 
31                             & \textsc{late\_mod   }                          & Late Development - Moderate                 \\
32                             & \textsc{late\_cl     }                         & Late Development - Open                     \\
\rowcolor[HTML]{CAD6BA} 
40                             & \textsc{early\_asp  }                          & Early Development - Aspen                   \\
41                             & \textsc{mid\_asp   }                           & Mid Development - Aspen                     \\
\rowcolor[HTML]{CAD6BA} 
42                             & \textsc{mid\_ac    }                           & Mid Development - Aspen Conifer             \\
43                             & \textsc{late\_ca    }                          & Late Development - Conifer Aspen            \\ \bottomrule
\end{tabular}
\end{table}



\subsubsection{Age}
Age represents the number of years since the last stand-replacing disturbance (high mortality wildfire). Because the characteristic species of a given cover type may not immediately establish after a stand-replacing fire, it is likely that the age value is larger than the actual age of the oldest individuals in a stand. Several of the cover types in this area may go through a chaparral-dominated early development stage; in those cases the oldest trees in the stand could be decades older than the formal stand age. In \textsc{RMLands}, age is used to trigger potential successional transitions and to calculate susceptibility to disturbance. In this application, we rounded all modeled and derived ages to the nearest five years (the length of one timestep).

In the HRV analysis, the initial age value assigned to a given cell is not necessarily important to the outcome of the simulation, due to the exclusion of the first (in our case) 40 timesteps from the analyzed results. In the future scenario analysis, the initial age value carries more weight, because the total simulation length is only 18 timesteps.

In this application, we used data from stand exams dating to the 1960s and from recent Forest Service Region 5 Ecology group survey plots to estimate stand age across the buffered project area. We then interpolated that information across the landscape. Due to insufficient data, we were unable to disaggregate the data below the landscape scale to cover type or another more finely resolved classification. We also acknowledge that the stand exam and modern veg plots do not constitute a true sample and were conducted almost exclusively in mid-mature and mature stands of commercially viable trees, thus skewing the results to some unquantifiable degree.

We updated the interpolated data with wildfire and timber management history, and assigned ages to types coded as chaparral in the Existing Vegetation layer to the midpoint of the age spread of early development for the forest cover type to which it was converted. Remaining ages out of compliance with allowed ages for the corresponding condition of a given cell were modified to be in compliance, based on the assumption that the condition class assignment was more accurate that the interpolated age information. The input Age layer, showing the map at timestep 0, is shown in Figure~\ref{fig:inputlayermaps1}.


\subsubsection{Condition-Age}
Condition-Age represents the age since transitioning to the current condition. In \textsc{RMLands} it affects most transitions between condition classes: typically there is a threshold condition age below which transitions do not occur. After creating both the condition and age layers, we used a Python function to derive condition-age based on the youngest possible age for a cell of that cover and condition. For example, if we determine that a particular cell on the landscape has a cover type of Lodgepole Pine, condition of Mid Development Closed, and age of 50 years, we take the minimum age for that cover-condition combination (10 years old), and subtract it from the age to arrive at a condition-age of 40. The final, original map for the initial condition-age is shown in Figure~\ref{fig:inputlayermaps1}.


\subsubsection{Topographic Position Index}
Our topographic position index (TPI) combines heat load, which is based on aspect and slope, with slope position (Figure~\ref{fig:inputlayermaps2}). High values for TPI are correlated with locations on steep, south and west-facing, upper slopes. Low values are correlated with locations on gentle, north and east-facing, valley bottoms. Values in between occur along a gradient of these characteristics. The TPI is scaled to the project area and the region immediately surrounding it, and is therefore a local index only. We use TPI to adjust vegetation susceptibility and mortality because we expect susceptibility and mortality to be higher when TPI is high. In the Results chapter, we plot the average canopy cover (over the simulated period) against TPI and calculate the proportional effect of TPI on cover.



\subsubsection{Elevation} 
Elevation represents the height above sea level in meters. In \textsc{RMLands}, the elevation layer affects disturbance spread. The elevation grid used in this analysis was a digital elevation model (DEM) provided by the Tahoe National Forest GIS staff and rescaled from 10 m$^2$ to 30 m$^2$ pixels. It is shown as a map in Figure~\ref{fig:inputlayermaps2}.



\subsubsection{Slope} 
Slope represents the steepness of a cell as measured in percent and is derived from the elevation layer. Slope is used in GIS preprocessing to define cover types and eligibility for various vegetation treatments. Within \textsc{RMLands}, slope affects disturbance spread. The slope for the study area was derived from the elevation layer described above, and is shown in Figure~\ref{fig:inputlayermaps2}.


\subsubsection{Aspect} Aspect represents the direction a cell is facing in terms of eight cardinal directions. Flat aspects are also recognized. Within \textsc{RMLands}, aspect affects disturbance spread. The aspect for the study area was derived from the elevation layer described above, and is shown in Figure~\ref{fig:inputlayermaps2}. 


\subsubsection{Streams} 
Streams represents linear hydrological features, classified as small, medium or large based on stream order. In this application, the streams layer was created from a line coverage containing hydrography data, including an attribute for stream size or order, by converting to a grid based on the stream size attribute. Streams may inhibit the spread of wildfire in \textsc{RMLands}, depending on both stream size and potential wildfire size. In order to function as a barrier, cells in the stream raster coded as streams much share a side, rather than only a vertex. The stream layer was used to overwrite any cells not coded as ``Water'' in the cover type layer, such that all ``large'' streams are represented in the cover type layer. \todo{check this} The final streams input layer is shown in Figure~\ref{fig:inputlayermaps2}.


\subsubsection{Buffer/Core} 
This layer identifies and distinguishes the ``core'' project area from the 10 km ``buffer'' added to allow wildfires to initiate outside of and burn beyond the formal project area. Without a buffer, edge effects would alter results for all aspects of the disturbance regime, as well as resulting landscape composition and configuration. A 10 km buffer was selected arbitrarily, but has in the past been sufficient to offset edge effects (McGarigal, personal communication). Because the simulation plays out on the full extent of the core plus the buffer, all input grids are developed to that larger extent as well. To create this raster, the original project polygon was buffered in ArcMap by 10 km, then converted to raster using the same procedure as for other layers. The buffer and core are easily distinguished in Figures~\ref{fig:inputlayermaps1}--\ref{fig:inputlayermaps2}: the core is the interior area delinated by a thick black line, while the buffer is area outside of this line, displayed at a decreased brightness level.

\begin{figure}[!htbp]
  \centering
  \subfloat[][]{
    \centering
	\includegraphics[height=0.2\textheight]{/Users/mmallek/Tahoe/Report2/images/cover.png}
    \label{fig:covermap}
  } \qquad
  \subfloat[][]{
    \includegraphics[height=0.2\textheight]{/Users/mmallek/Tahoe/Report2/images/condition.png}
    \label{fig:conditionmap}
  } \\
	\subfloat[][]{
    \includegraphics[height=0.2\textheight]{/Users/mmallek/Tahoe/Report2/images/age.png}
    \label{fig:agemap}
  } \qquad
	\subfloat[][]{
    \includegraphics[height=0.2\textheight]{/Users/mmallek/Tahoe/Report2/images/condage.png}
    \label{fig:condagemap}
  } 

  \caption{RMLands input layers. (a) Cover type map (b) Seral stage map (c) Age map at Timestep 0 (d) Condition-Age map at Timestep 0 }
  \label{fig:inputlayermaps1}
\end{figure}

\begin{figure}[!htbp]
  \centering
  \subfloat[][]{
    \centering
	\includegraphics[height=0.2\textheight]{/Users/mmallek/Tahoe/Report2/images/tpi.png}
    \label{fig:tpimap}
  } \qquad
  \subfloat[][]{
    \includegraphics[height=0.2\textheight]{/Users/mmallek/Tahoe/Report2/images/elevation.png}
    \label{fig:elevationmap}
  } \\
	\subfloat[][]{
    \includegraphics[height=0.2\textheight]{/Users/mmallek/Tahoe/Report2/images/slope.png}
    \label{fig:slopemap}
  } \qquad
	\subfloat[][]{
    \includegraphics[height=0.2\textheight]{/Users/mmallek/Tahoe/Report2/images/aspect.png}
    \label{fig:aspectmap}
  } \\
  	\subfloat[][]{
    \includegraphics[height=0.2\textheight]{/Users/mmallek/Tahoe/Report2/images/streams.png}
    \label{fig:streamsmap}
  } 

  \caption{RMLands input layers. (a) Topographic Position Index (b) Elevation (c) Slope (d) Aspect (e) Streams}
  \label{fig:inputlayermaps2}
\end{figure}


\subsection{Model Parameterization}
\label{subsec:hrvmodelparam}

\subsubsection{State and Transition Models}
We have created a detailed cover type description document for each cover type in the simulated landscape that experiences transitions between cover class. These documents describe crosswalks to other data layers, detailed accounts of the multiple species characteristic of the cover type, cover type distribution, relationship and response to wildfire, predicted fire return intervals, plus descriptions of each seral stage present within the cover type and their succession and wildfire transition conditions and rates (Appendix \ref{sec:covertypedesc}). Each detailed document can be summarized as a state and transition model for a particular cover type, which is implemented in the model by specifying susceptibility to wildfire, rules for vegetational succession, and rules for transitions after a fire event. Figure~\ref{transmodel} shows a generic example state and transition model for the forested cover types.

\begin{figure}[htbp]
\centering
\includegraphics[width=0.8\textwidth]{/Users/mmallek/Tahoe/Report3/images/state_trans_model.pdf}
\caption{Generic state and transition model for all non-shrub seral cover types. Boxes show seven condition classes and arrows depict transitions due to vegetation succession and high or low mortality fire.} 
\label{transmodel}
\end{figure}

To illustrate the parameterization, in the following tables we present values for the Sierran Mixed Conifer - Mesic cover type model. The target fire return interval for this cover type is 29 years. A fire return interval index is used as the parameter controlling the relative susceptibility to fire of the seral stages within an individual cover type. In addition, a probability of high severity fire leading to at least 70\% overstory mortality is specified for each seral stage (Table~\ref{smcm_fri_phm}). We also specified transition probabilities for natural succession between the early, middle, and late stages of development, as well as between closed, moderate, and open canopy cover. This type of succession also depends on the time in the current seral stage both in terms of the early-middle-late sequence (\emph{Development-Age}) and the specific stage-canopy cover combination (\emph{Seral Stage-Age}) (Table~\ref{smcm_vegtrans}). Finally, probabilities are specified for vegetation transitions after wildfire (Table~\ref{smcm_firetrans}). We calculated these values using the VDDT models associated with the LandFire project, modifying some based on local expert opinion. From the VDDT models, we used the probabilities of a transition to early seral conditions, a more open canopy condition, or of no transition. We ignored the classified type of fire (as replacement, mixed, or low severity), focusing instead on the outcome from fire in terms of the seral stage, if any, to which a cell transitioned after wildfire.

% edited 2015-9
\begin{table}[htbp]
\small
\centering
\caption{Relative susceptibility to fire and probability of high severity fire (at least 70\% overstory tree mortality) probabilities for Sierran Mixed Conifer - Mesic.}
\label{smcm_fri_phm}
\begin{tabular}{lcc}
\hline
\textbf{Seral Stage}    & \textbf{Relative Susceptibility to Fire} & \textbf{Probability of High Severity Fire} \\ \hline
Early All     				& 5.5        & 1                 \\
Mid Closed    				& 2.4        & 0.23              \\
Mid Moderate  				& 1.6        & 0.17              \\
Mid Open      				& 1.3        & 0.14              \\
Late Closed   				& 4.3        & 0.37              \\
Late Moderate 				& 1.6        & 0.14              \\
Late Open     				& 1          & 0.09              \\ 
\emph{Target Fire Return Interval}    			& \emph{29 years}  &   \\ \hline
\end{tabular}

\end{table}

\begin{table}[!htbp]
\small
\centering
\caption{Timeframes for transitions between seral stages in \textsc{RMLands} for Sierran Mixed Conifer - Mesic. ``Early to Mid'' and ``Mid to Late'' times are based on the time in a developmental stage, regardless of disturbance history. ``Open to Moderate'' and ``Moderate to Closed'' times are based on the time in a seral stage since a disturbance.}
\label{smcm_vegtrans}
\begin{tabular}{cccc}
\hline
\textbf{Condition Class Transition} & \textbf{Minimum (years)} & \textbf{Average (years)} & \textbf{Maximum (years)} \\ \hline
Early to Mid 	& 20      & 26      & 40      \\
Mid to Late 	& 100     & 113     & 150     \\
\begin{tabular}[c]{@{}c@{}}Open to Moderate or\\ Moderate to Closed\end{tabular}  & 15      & 21      &    ---     \\ \hline
\end{tabular}

\end{table}


\begin{table}[!htbp]
\small
\centering
\caption{Transition probabilities for Sierran Mixed Conifer - Mesic following low mortality fire.}
\label{smcm_firetrans}
\begin{tabular}{lcc}
\hline
\textbf{Seral Stage Transition} & \textbf{Probability}\\
\hline
Mid Closed to Mid Moderate     	& 0.53   	\\
Mid Moderate to Mid Open    	& 0.36		\\
Late Closed to Late Moderate	& 0.54    \\
Late Moderate to Late Open     	& 0.24    \\
\hline
\end{tabular}
\end{table}

Transitions between Early and Middle Development, and between Middle and Late Development are governed by the time in the Early or Middle stage (canopy cover usually does not affect these probabilities). These transitions may begin at the minimum time in a specified \emph{Development-Age}, and proceed at rates that vary across cover types. Table~\ref{smcm_vegtrans} displays the average \emph{Seral Stage-Age} of transition. If a cell reaches the maximum stage-age listed, its probability of transitioning goes to 1. Transitions between the canopy cover types occur within one developmental stage: e.g., between Middle Development Open and Middle Development Moderate, but not between Middle Development Open and Late Development Moderate. These transitions are governed by the time in the full condition class specification since the last disturbance. This ``years since'' value may be affected by a low mortality fire, a transition between developmental stages, or a transition between canopy cover levels. Similarly to the developmental transitions, the shift from, for example, Middle Development Open to Middle Development Closed may begin when the minimum time is reached, and also proceeds at rates that vary across cover types. Table~\ref{smcm_vegtrans} shows the average \emph{Seral Stage-Age} of transition. However, no maximum age is specified for this type of transition.

\subsubsection{Disturbance Parameters} 
\label{subsubsec:distparams}

%\begin{adjustwidth}{5ex}{0pt}
\begin{itemize}
\item \emph{Climate:} The climate parameters are based on a rescaling of the Palmer Drought Severity Index (PDSI). PDSI is a long-term measure of drought, on the scale of months to years. It is based on precipitation and temperature and incorporates soil moisture. Resconstructed PDSI values for summer months during the historic period of this project (1550-1850) are available from the National Oceanic and Atmospheric Administration \todo{cite}. We used \citet{Zhangetal.2004} and \citet{Cook2004}. These data are summarized at large scales; for example, the \citet{Cook2004} data are calculated for a grid with points spaced at 2.5\textdegree. We selected the five closest points to the center of the project area from these two datasets and calculated the inverse distance-weighted mean of the values. We then converted the yearly data into five-year averages to align with the five-year timesteps in our model. By recentering the mean value around 1 and then taking the inverse, we create a dataset in which a value of 1 is neither wetter nor dryer than average, values between 0 and 1 represent wetter-than-normal timesteps, and values greater than 1 represent dryer-than-normal timesteps (Figure~\ref{pdsi}). Climate interacts with other disturbance parameters in \textsc{RMLands}, including initiation, susceptibility, and spread.

\begin{figure}[htbp]
\centering
\includegraphics[height=0.3\textheight]{/Users/mmallek/Tahoe/Report3/Images/pdsi_hrv.png}
\caption{Palmer Drought Severity Index, rescaled, inverted, and presented as a 5-year average for the ``historic'' period in this study (1550-1850).} 
\label{pdsi}
\end{figure}

%\medskip

%\noindent 
\item \emph{Susceptibility:}\todo{explain this better} Cover and seral stage are both inputs to susceptibility. Cover modifies susceptibility via the ability to specify the influence of topographic position on susceptibility (Table~\ref{covtpi}. The magnitude of this effect is estimated as a potential reduction in susceptibility of 30\% between the minimum and maximum TPI values used in the model. This is specified as part of the 4-parameter logistic function 
$$c + \frac{L-c}{1+e^{k(x_0-x)}}$$
in which $c= 0.7$, $L=1$, slope $k=1$, inflection point $x_0=0$, and $x$ is equal to the TPI value at a given cell. It therefore simplifies to 
$$\frac{1}{1.7+e^{-x}}$$

\begin{table}[htbp]
\small
\centering
\caption{Cover types whose susceptibility is modified by Topographic Position Index.}
\label{covtpi}
\begin{tabular}{ll}
\hline
\multicolumn{2}{c}{\textbf{Cover Types with TPI Adjustment}} \\
\hline
Grassland     									& Red Fir - Mesic   					\\
Lodgepole Pine    								& Red Fir - Ultramafic					\\
Mixed Evergreen - Mesic							& Red Fir - Xeric    					\\
Mixed Evergreen - Ultramafic     				& Sierran Mixed Conifer - Mesic    		\\
Mixed Evergreen - Xeric 						& Sierran Mixed Conifer - Ultramafic 	\\
Montane Riparian								& Sierran Mixed Conifer - Xeric 		\\
Oak Woodland 									& Western White Pine					\\
Oak-Conifer Forest and Woodland 				& Yellow Pine 							\\
Oak-Conifer Forest and Woodland - Ultramafic 	&										\\
\hline
\end{tabular}

\end{table}

Seral stage further modifies susceptibility. We use the Weibull cumulative distribution function and specify a scale parameter $\lambda$ (mean return interval), shape parameter $k$, and the reset point for the function (\emph{age since high mortality disturbance} or \emph{age since any disturbance}). The mean return interval for the seral stage is used as a calibration parameter and was initially set as equal to the values provided in analogous LandFire Biophysical Setting types \citep{Landfire2007}. Some modifications were made based on consultation with Forest Service staff. All mean return intervals within a cover type are modified as a group and kept relative to one another even as the magnitude of the return intervals is adjusted. The relative difference for Sierra Mixed Conifer - Mesic is show in Table~\ref{smcm_fri_phm}; these values for each cover type are included in the cover type description documents (Appendix \ref{sec:covertypedesc}). We set $k=3$ for all cover types and seral stages. We selected between ``age since high mortality disturbance'' and ``age since any disturbance'' based on whether wildfires in that cover type are climate-driven (in which case we select the former) or fuels-driven (in which case we select the latter) (Figure~\ref{howdriven}).

\begin{table}[htbp]
\small
\centering
\caption{Cover types sorted by whether wildfire disturbance in them is characterized by fuels present or overarching climatic conditions. If the likelihood of wildfire depends on the accumulation of fuels, the value of $x$ (``time since'') reverts to 0 after any disturbance. If the likelihood of wildfire depends primarily on climate and weather conditions, the value of $x$ reverts to 0 only after a high mortality disturbance.}
\label{howdriven}
\begin{tabular}{ll}
\hline
\textbf{Fuel-Driven Cover Types} 				& \textbf{Climate-Driven Cover Types}	\\
\hline
Curl-leaf Mountain Mahogany 					& Agriculture   						\\
Grassland     									& Big Sagebrush 						\\
Lodgepole Pine    								& Black and Low Sagebrush				\\
Meadow											& Lodgepole Pine with Aspen 			\\
Mixed Evergreen - Mesic							& Montane Riparian						\\
Mixed Evergreen - Ultramafic     				& Red Fir with Aspen   					\\
Mixed Evergreen - Xeric 						& Red Fir - Mesic    					\\
Oak Woodland 									& Red Fir - Ultramafic 					\\
Oak-Conifer Forest and Woodland 				& Red Fir - Xeric 						\\ 	
Oak-Conifer Forest and Woodland - Ultramafic 	& Subalpine Conifer 					\\
Sierran Mixed Conifer - Ultramafic 				& Subalpine Conifer with Aspen 			\\
Sierran Mixed Conifer - Xeric 					& Sierran Mixed Conifer with Aspen 		\\
Urban 											& Sierran Mixed Conifer - Mesic 		\\
Yellow Pine 									& Western White Pine 					\\
												& Yellow Pine with Aspen 				\\
\hline
\end{tabular}
\end{table}


%\medskip

%\noindent 
\item \emph{Initiation:} In \textsc{RMLands}, parameters for initiation are used as calibration parameters. The probability of wildfire initiation is a function of its susceptibility to wildfire and the climate modifier value for that timestep, and is applied at the cell level. The ignition calibration coefficient refers to the number of ignitions per 100,000 ha per year. For the HRV simulation, we set this coefficient at 42. We applied the coefficient evenly across the landscape based on local expert knowledge of lighting strike locations in the area. Fires may be initiated anywhere within the project area or the 10 km buffer around it. The total area cover within that boundary is 409,410.7 ha, so up to 860 fire starts were possible during each 5-year timestep in our simulation (not all potential ignitions result in fire). Climate also influences initiation.

%\medskip

%\noindent 
\item \emph{Spread:} The probability of fire spread in \textsc{RMLands} is a function of climate, susceptibility to wildfire, potential wildfire size, wind, spotting, relative elevation, and presence of streams. The first two are described above. The disturbance size distribution that regulates potential fire size was created by analyzing the size distribution of all mapped fires in the Northern Sierra \textsc{calveg} mapping zone and west of the Sierran crest, available from the Forest Service and the California Department of Forestry and Fire Protection, which goes back to approximately 1900. 

Wind is incorporated in two parts. First, a prevailing wind direction for the fire is selected probabilistically from the eight cardinal directions. To compute the wind distribution values, we first consulted local experts to determine the dates of fire season (May 15 to October 15) and burning period times (1000 hours to 1800 hours). We then downloaded all available historic wind direction data from 6 local weather stations (Rice Canyon, Saddleback, Downieville, White Cloud, Emigrant Gap, and Blue Canyon, Figure~\ref{weather}). Data from all weather stations was weighted equally. After the wind direction is selected, fires are able to grow in all directions, but are relatively more likely to spread with wind than against it. We parameterized the influence of \emph{relative wind} as a reduction in spread likelihood. Thus, spread in the same direction as wind has a neutral effect, spread at $\ang{45}$ angles is reduced by 30\%, spread at $\ang{90}$  angles is reduced by 70\%, spread at $\ang{135}$ angles is reduced by 90\%, and spread opposite the prevailing wind direction is reduced by 95\%. 

\begin{figure}[htbp]
\centering
\includegraphics[width=0.3\textheight]{/Users/mmallek/Tahoe/Report3/images/weather.png}
\caption{Weather stations used to inform wind direction parameters.}
\label{weather}
\end{figure}

Relative elevation also modifies spreading potential. We parameterized the model such that spread downhill is extremely unlikely. Spotting and the extent to which streams act as barriers to spread are affected by the fire size. As fires become larger, their probability of spotting and spotting distance increases. Similarly, streams function as a barrier to smaller fires, but large fires are able to spread past streams regardless of size. This decision is based on the idea that large fires are more influenced by wind and climatic conditions. Stream size does impact smaller fires; the largest streams and rivers are usually an effective barrier to smaller fires, although even fairly small fires often spread past intermittent and small perennial streams. 


\item \emph{Mortality:} Cover and seral stage are both inputs to mortality. The effect of topographic position is tied to cover: the mortality of certain cover types (Table~\ref{covtpi}) is affected by cover. The magnitude of this effect is estimated as a potential reduction in mortality of 30\% between the minimum and maximum TPI values used in the model. This is specified as part of the 4-parameter logistic function 
$$c + \frac{L-c}{1+e^{k(x_0-x)}}$$
in which $c= 0.7$, $L=1$, slope $k=1$, inflection point $x_0=0$, and $x$ is equal to the TPI value at a given cell. It therefore simplifies to 
$$\frac{1}{1.7+e^{-x}}$$

\begin{table}[htbp]
\small
\centering
\caption{Cover types whose mortality is modified by Topographic Position Index.}
\begin{tabular}{ll}
\hline
\multicolumn{2}{c}{\textbf{Cover Types with TPI Adjustment}} \\
\hline
Grassland     									& Red Fir - Mesic   					\\
Lodgepole Pine    								& Red Fir - Ultramafic					\\
Mixed Evergreen - Mesic							& Red Fir - Xeric    					\\
Mixed Evergreen - Ultramafic     				& Sierran Mixed Conifer - Mesic    		\\
Mixed Evergreen - Xeric 						& Sierran Mixed Conifer - Ultramafic 	\\
Montane Riparian								& Sierran Mixed Conifer - Xeric 		\\
Oak Woodland 									& Western White Pine					\\
Oak-Conifer Forest and Woodland 				& Yellow Pine 							\\
Oak-Conifer Forest and Woodland - Ultramafic 	&										\\
\hline
\end{tabular}
\end{table}

Seral stage further modifies mortality. We extracted the likelihood of mortality from the VDDT models built during the LandFire project. However, our accounting method differs slightly. VDDT characterizes fire severity (e.g. mixed) independently from a particular outcome (e.g. resetting to early seral conditions). We do not include ``mixed severity'' or ``stand-thinning'' fire \citep{Keeley2000} in our model. Instead, we characterize fire dichotomously, based on their outcomes: fires are either high mortality or low mortality events. We define low mortality fires are those in which less than 70\% of overstory trees are killed, while high mortality fires are those in which more that 70\% of overstory trees are killed. 

To interpret the VDDT models, we analyzed not only the fire type (replacement, mixed, or surface), but also the change in seral stage that occurred as a consequence of that fire. I classified the probabilities specified by the VDDT model as associated with either high or low mortality fires. High mortality fires are those that result in conversion to early seral (regardless of whether they are called ``replacement'' or ``mixed''). All other fires are considered low mortality. The probability of a high mortality outcome from fire was calculated by dividing the summed probabilities of high mortality fires as defined above by the summed probabilities of all fires. As an example, these probabilities for Sierran Mixed Conifer - Mesic are provided in Table~\ref{smcm_fri_phm} (Subsection~\ref{subsec:hrvmodelparam}).

\end{itemize}
%\end{adjustwidth}

\subsection{Model Calibration}\todo{rewrite this section for clarity. Explain that goal is to “tune” the model.Explain why we do model calibration in general and then why we chose our calibration parameters as such. Why calibration to rotation?.}
Calibrating the model was done by manipulating the ignition calibration coefficient and the relative magnitude of the scale parameter $\lambda$ in the Weibull cumulative distribution function under Susceptibility. We first manipulated the ignition calibration coefficient and reran the simulation until the output cover type rotation values were within 10\% of the target values. Target values were based on empirical published values and local expert opinion. 

We then modified the ``fire return interval'' values for all seral stages (under a given cover type), manipulating them as a set by multiplying by a constant. We then evaluated the simulation results at the cover type (rather than seral stage) level. Adjustments were made to the set of seral stages as necessary to produce more or less fire. For example, the target rotation for Sierran Mixed Conifer - Mesic was 29 years. As part of calibration, we adjusted the input condition class-based mean fire return interval up or down, eventually arriving at an increase by a factor of 24 from the original calculated ratio values. That is, each initial scale parameter value was multiplied by 24 in order to modify the susceptibility of Sierran Mixed Conifer - Mesic to fire without changing the relative susceptibility among Sierran Mixed Conifer - Mesic condition classes. We came very close to achieving our target cover type rotation values through several iterations of this method, and as a final step made a small modification to the ignitition calibration coefficient.

As described in Section \ref{subsubsec:distparams}, we specified a target set of disturbance sizes. Because wildfire has many stochastic components, we did not expect the model results to match these targets exactly. Figure \ref{fig:dsize} compares the observed and target disturbance size distribution, and was used to evaluate whether the model was functioning correctly. We analyzed the difference between the first and second bins, and determined that the difference can be attributed primarily to ``spillover'' effects. Although fires have a size potential drawn from a distribution, they can grow somewhat larger than that potential size due to the fact that the fire size is evaluated after each iteration of fire growth. If a fire has a maximum potential size of 25 hectares, and is at 24 hectares after an iteration of growth, then it may grow to 27 hectares (for example) in one iteration, and thus be put into the bin for 25-125 ha fires.\todo{This section should be shorter. I'm also wondering if I should delete it entirely.}

% updated 9/13
\begin{figure}[!htbp]
  \centering
    \centering
    \includegraphics[width=0.4\textwidth]{/Users/mmallek/Documents/Thesis/Plots/dsize/hrv.png}
  \caption{(a) Side by side barplot of the observed and target wildfire size distribution for our 500-timestep long run of the model.}
  \label{fig:dsize}
\end{figure}

\subsection{Model Execution}
During the calibration phase of the model, a typical simulation would be three runs of 200 timesteps each. The equilibration period of 40 timesteps was chosen based on visual analysis of the disturbed area and rotation plots from the combined runs. Once calibration was complete, we conducted one 500 timestep-long run in order to capture multiple disturbance and succession cycles across the most common cover types. Each timestep represents five years. The five-year timestep was chosen based on the short fire return intervals recorded from dendrochronology analysis in the literature and our desire to capture these very short return intervals in the simulation.

\subsection{Data Analysis}
\label{subsec:dataanalysis}

\paragraph{Disturbance Regime} We quantified the following overall temporal and spatial characteristics of the wildfire disturbance regime:
\begin{itemize}
	\item \emph{Disturbed Area:} We calculated disturbed area for each timestep, divided into low mortality and high mortality disturbance, and summed to produce an ``any mortality'' statistic. We summarize the results for minimum, maximum, mean, and median area disturbed as a proportion of the total area eligible for disturbance for the full simulation excluding the equilibration period (460 timesteps, or 2300 years). We also present maps of the landscape illustrating the minimum, maximum, and median area burned during the simulation, and a 4-timestep sequence illustrating a time series of wildfire disturbance. Finally, we present the distribution of wildfire extents during the simulation, excluding the equilibration period, as a histogram.
	\item \emph{Disturbance Frequency:} We calculated the number of years between disturbances exceeding a particular threshold in total disturbed area. We report the frequency of timesteps during which at least 10\%, 25\%, or 50\% of the landscape experienced wildfire.
	\item \emph{Climate Effect:} Climate interacts with several components of the model. We present plots illustrating the value of the climate parameter by timesteps concurrently with the area disturbed per timestep. It is not practical to further illustrate its effect everywhere, and in some cases its influence is not easily separated from the other inputs to the model. 
	\item \emph{Rotation Period:} We calculated the rotation period---the number of years required to burn an area equivalent to the total eligible area---for each cover type within the project area. Because it is based on the full landscape results, and is not a sample, it is equivalent to the cell-specific grand mean return interval for a given cover type across the landscape. We report the rotation values for low mortality fire, high mortality fire, and any fire for each of the nine focal cover types individually and the study area as a whole.
	\item \emph{Return Interval:} We summarized the cell-specific population mean return interval---the average number of years between disturbances at a single cell---and present it as the distribution of the percentage of eligible cells that experienced each possible mean return interval. We use histograms to visualize the distribution of this return interval for low mortality fire, high mortality fire, and any fire, along with their median values. As with the rotation period, this method is based on the full landscape results and is equivalent to the cell-specific grand mean return interval for a given cover type across the landscape. We report these data for the nine focal cover types individually and the study area as a whole. Finally, we present this result spatially as a map showing the population fire return interval for the full landscape and for our nine focal cover types.
\end{itemize}

\paragraph{Vegetation Response} 

\begin{itemize}
\item \emph{Landscape Composition:} We quantified the distribution and dynamics of landscape composition by cover type. For our single 2500 year simulation (with 200 year equilibration period), we summarize the results in a table and in stacked bar plots. For the tabular results, we present the mean, median, minimum, maximum, 5th, 25th, 75th and 95th percentiles of the distribution. We compared the current landscape condition (i.e., proportion of cover type in each condition class) to this simulated historic range of variability to determine whether the current landscape deviates, and to what degree, from the HRV. We visualize the proportion of the total area of a given cover type occurring as each seral stage, for each timestep in the model. In addition, we show a simple plot of the current seral stage distribution, allowing a visual comparison between current conditions and the historical range of variability in the distribution of the seral stages.

\item \emph{Landscape Structure and Patterns:} We used \textsc{Fragstats} \citep{McGarigal2012} to compute several landscape-level and class-level metrics that summarize landscape structure over the course of the simulation. We present the results in a series of tables and figures. The descriptions in Appendix~\ref{app:metricdescriptions} are intended as a general introduction to the \textsc{Fragstats} metrics; for a much more detailed and mathematical description of all \textsc{Fragstats} metrics, see the \href{http://www.umass.edu/landeco/research/fragstats/documents/fragstats.help.4.2.pdf}{documentation}. Each metric is computed on the study area for a single timestep and the results are displayed in tabular format by quantiles and in graphical format with line graphs and boxplots. Table~\ref{tab:fraglist} lists all \textsc{Fragstats} metrics computed, their abbreviation, and whether they are computed at the landscape- or class-level; bolded metrics were selected as focal metrics to provide a simple and understandable explanation of the characteristics of landscape structure during the simulated HRV.

\begin{table}[]
\centering
\caption{All \textsc{Fragstats} metrics computed for the study area during the simulated HRV. An `X' in the landscape or class column denotes whether that metric is calculated at that level. Metrics in bold are those we selected to emphasize in order to provide a parsimonious explanation of the variability in landscape structure during the simulated HRV.}
\label{tab:fraglist}
\begin{tabular}{@{}llcc@{}}
\cmidrule(r){1-3}
{\bf Metric  }                         & {\bf Abbreviation }   & {\bf Landscape-level} & {\bf Class-level} \\ \cmidrule(r){1-3}
Percentage of Landscape               & pland           &                 & X           \\
Patch Density                         & pd              & X               & X           \\
Total Edge                            & te              & X               & X           \\
Edge Density                          & ed              & X               & X           \\
{\bf Area-Weighted Mean Area}         & {\bf area\_am}  & {\bf X}         & {\bf X}     \\
Mean Area                             & area\_mn        & X               & X           \\
Area-Weighted Mean Radius of Gyration & gyrate\_am      & X               & X           \\
{\bf Area-Weighted Mean Shape}        & {\bf shape\_am} & {\bf X}         & {\bf X}     \\
Mean Shape                            & shape\_mn       & X               & X           \\
{\bf Area-Weighted Mean Core Area}    & {\bf core\_am}  & {\bf X}         & {\bf X}     \\
Area-Weighted Mean Core Area Index    & cai\_am         & X               & X           \\
Mean Similarity Index                 & simi\_mn        & X               & X           \\
Contrast-Weighted Edge Density        & cwed            & X               & X           \\
Total Edge Contrast Index             & teci            & X               & X           \\
Area-weighted Mean Edge Contrast      & econ\_am        & X               & X           \\
Mean Edge Contrast                    & econ\_mn        & X               & X           \\
{\bf Contagion}                       & {\bf contag}    & {\bf X}         & {\bf }      \\
{\bf Clumpiness Index}                & {\bf clumpy}    & {\bf }          & {\bf X}     \\
Interspersion and Juxtaposition Index & iji             & X               & X           \\
Patch Richness                        & pr              & X               &             \\
Simpson’s Diversity Index             & sidi            & X               &             \\
{\bf Simpson’s Evenness Index}        & {\bf siei}      & {\bf X}         & {\bf }      \\
Aggregation Index                     & ai              & X               & X           \\ \cmidrule(r){1-3}
\end{tabular}
\end{table}

\item We also quantified the departure from the HRV conditions by calculating a departure index for each cover-seral stage type and each \textsc{Fragstats} metric. We summarized the distribution of each metric calculated over the length of the simulation, minus the equilibration period. We computed the 5$^{\text{th}}$, 25$^{\text{th}}$, 50$^{\text{th}}$, 75$^{\text{th}}$, and 95$^{\text{th}}$ percentiles of the distribution of observed values. The current percentile for the statistical range of variability refers to the place within the 0--100$^{\text{th}}$ percentile of the observed, simulated HRV. If the current landscape is outside of the HRV, its current \%SRV value is noted as either 0 or 100. The departure index indicates the distance from the 50$^{\text{th}}$ percentile value for a given metric. It is computed based on the Current SRV Percentile, which is the percentile value under the SRV that the current metric value is equivalent to. A value of 0 means that the current value is identical to the median from the simulated HRV, and a value of either less than -95 or greater than 95 means that the current value is below or above and outside the simulated HRV. The departure index is computed by subtracting 50 from the current value's percentile under the simulated range of variability (SRV) then dividing by 50 and multiplying by 100. Thus, for the landscape metric \emph{Patch Density}, 19.507 is equivalent to the 32$^{\text{nd}}$ percentile of observations during the HRV simulation, and the departure index is $(39-50)/50*100 = -22$). This value is within the HRV for the landscape. However, the landscape metric \emph{Edge Density} is 100, because $128.875 > 125.316$, the largest value observed during the HRV simulation. Edge density at the landscape level is outside the HRV.

% moved descriptions to appendix

\end{itemize}

