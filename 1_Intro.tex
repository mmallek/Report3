\chapter{Introduction}

\section{Background and Overview}

Historic range of variability (HRV), defined here as the variance in disturbance processes and landscape composition and configuration over the 300 years prior to European settlement, is a useful tool in landscape planning (Nonaka and Spies 2005). HRV analysis is intended to help conceptualize the mechanisms behind large-scale ecosystem functions and provide a basis from which to make predictions about how a given ecosystem will react to disturbances in the future (Nonaka and Spies 2005, Landres et al. 1999). U.S. Forest Service managers and planners are directed to ground decisions in the context of ``maintain[ing] or restor[ing] ecological conditions that are similar to the biological and physical range of expected variability'' (36 CFR 219.4). On the Tahoe National Forest, an opportunity to model HRV and future management scenarios to inform restoration planning arose after the 1999 Pendola Fire, which burned 3,000 acres on the Tahoe National Forest (Becky Estes, personal communication).

Although empirical data may sometimes be available on some variables affecting HRV, the time scales and broad spatial extents under study require simulation in order to incorporate all parameters of interest and therefore derive a meaningful quantification of HRV (Swetnam et al. 1999, Mladenoff and Baker 1999). Range of variability analyses can and have been conducted using literature searches exclusively, including within the Sierra Nevada (Hugh Safford et al., unpublished report). However, the results of such analyses depend on the assumption that an aggregation of many small studies is sufficient to address long-term, large-scale questions, and require researchers to accept many unknowns about research methodologies. Moreover, in landscapes severely impacted by European settlement, such as those of the northern Sierra Nevada, we can never observe trajectories in which fire suppression is not part of the equation (Keane 2012). In the absence of consistent and complete data, simulations provide a broad and flexible foundation from which to make inferences about the HRV of an area and compare current conditions to the HRV.

This project provides an analysis of the historic range of variability for the Upper Yuba River Watershed and a preliminary set of future management scenarios to aid in planning on the Tahoe National Forest. The quantitative assessment of the HRV of this landscape provides managers with a statistical, ecosystem-level analysis of the disturbance and succession processes that characterize this portion of the northern Sierra Nevada. Quantification of HRV provides managers with a neutral assessment of the current departure from HRV, which they can use to prioritize certain vegetation types, disturbance processes, or their intersection for restoration or maintenance. Because the simulation captures landscape changes over hundreds of years, far longer than the planning cycle, the results allow managers to ground near-term plans and expectations within a larger context. 

RMLands is a spatially-explicit, stochastic, landscape-level disturbance and succession model capable of simulating fine-grained processes over large spatial and long temporal extents (McGarigal et al. 2001). It is grid-based but simulates disturbance and succession at the patch level, and employs a dichotomous high vs. low mortality effect of fire in place of the uniformly low, mixed, and high severity regimes used elsewhere (McGarigal and Romme 2012). It also simulates a variety of vegetation treatments that can be customized by area (e.g. the wildland-urban interface) or cover type and condition class (e.g., Sierran Mixed Conifer - Mesic closed canopy only). Originally developed for use in the Rocky Mountains of southern Colorado to provide a quantitative description of HRV (McGarigal and Romme 2012), RMLands has also been used to simulate wildfire and vegetation succession in northern Idaho (Cushman et al. 2011). It links with a landscape pattern analysis program (FragStats) for comprehensive spatial analysis of landscape composition and configuration.  

\section{Project Objectives}
The overall purpose of this project is to quantify the historical range of variability in landscape structure in the Yuba River watershed on the Tahoe National Forest and evaluate the relative effects of  alternative future land management scenarios on landscape structure. The specific objectives are as follows:
\begin{enumerate}
	\item Synthesize the empirical and expert knowledge on disturbance and succession processes characteristic of the pre-settlement period in the ecoregion containing the Yuba River watershed.
	\item Based on \#1 above, simulate landscape dynamics in the Yuba River watershed using the Rocky Mountain Landscape Simulator (RMLands).
	\item Based on the simulation above, quantify the HRV in landscape structure in the Yuba River watershed.
	\item Based on the HRV results above, establish the desired future range of variability in landscape structure in the Yuba River watershed and then design and simulate alternative future land management scenarios in RMLands aimed at achieving these desired future conditions.
\end{enumerate}


\section{Report Organization}