% !TEX root = master.tex

\chapter{Introduction}
\setcitestyle{notesep={:},aysep={}}

\emph{Not yet implemented, but suggested by Jack: buff out the intro more to ``give away the ending'' so that it's more clear what the point of everything is. Also to be implemented generally is improving graphics for readability; some of this can be done after the defense though. I can provide any figure as a simple pdf/png as needed.}
\section{Background and Overview}

Historic range of variability (HRV), defined here as the variance in disturbance processes and landscape composition and configuration over the 300 years prior to European settlement, is a useful tool in landscape planning \citep{Nonaka2005}. HRV analysis is intended to help conceptualize the mechanisms behind large-scale ecosystem functions and provide a basis from which to make predictions about how a given ecosystem will react to disturbances in the future \citep{Nonaka2005,Landres1999}. U.S. Forest Service managers and planners are directed to ground decisions in the context of ``maintain[ing] or restor[ing] ecological conditions that are similar to the biological and physical range of expected variability'' (36 CFR 219.4). On the Tahoe National Forest, an opportunity to model HRV and future management scenarios to inform restoration planning arose after the 1999 Pendola Fire, which burned 3,000 acres on the Tahoe National Forest. This work is a collaborative effort between two landscape ecologists from the University of Massachusetts -- Amherst (Kevin McGarigal and Maritza Mallek), staff on the Tahoe National Forest in California (Terri Walsh, Marilyn Tierney, Alan Doerr, and Travis Thane), and ecologists from the USFS Region 5 Ecology program (Hugh Safford and Becky Estes).\todo{add more about Forest Planning}

Although empirical data may sometimes be available on some variables affecting HRV, the time scales and broad spatial extents under study require simulation in order to incorporate all parameters of interest and therefore derive a meaningful quantification of HRV \citep{Swetnam1999,Mladenoff1999}. Range of variability analyses can and have been conducted using literature searches exclusively, including within the Sierra Nevada \citetext{Hugh\ Safford,\ unpublished\ report}. However, the results of such analyses depend on the assumption that an aggregation of many small studies is sufficient to address long-term, large-scale questions, and require researchers to accept many unknowns about research methodologies. Moreover, in landscapes severely impacted by European settlement, such as those of the northern Sierra Nevada, we can never observe trajectories in which fire suppression is not part of the equation \citep{Keane2012}. In the absence of consistent and complete data, simulations provide a broad and flexible foundation from which to make inferences about the HRV of an area and compare current conditions to the HRV.

This project provides an analysis of the historic range of variability for the Upper Yuba River Watershed and a preliminary set of future management scenarios to aid in planning on the Tahoe National Forest. The quantitative assessment of the HRV of this landscape provides managers with a statistical, ecosystem-level analysis of the disturbance and succession processes that characterize this portion of the northern Sierra Nevada. Quantification of HRV provides managers with a neutral assessment of the current departure from HRV, which they can use to prioritize certain vegetation types, disturbance processes, or their intersection for restoration or maintenance. Because the simulation captures landscape changes over hundreds of years, far longer than the planning cycle, the results allow managers to ground near-term plans and expectations within a larger context. 

In restoration planning efforts, it is logical to look back to the last known period during which a dynamic but resilient landscape existed. The arrival of European settlers to the Sierra Nevada led to sweeping ecological changes that now have greatly altered many Sierran landscapes -- through fire suppression, grazing, road-building, timber cutting, recreation, and other activities \citep{Meyer2013,Safford2013}\todo{others?}.  The period prior to European settlement, then, is a suitable reference condition against which we can compare current landscape structure and dynamics. Moreover, it is frequently used in the western United States as the historical reference period for restoration planning \citep{Safford2013}. The period is also up to several times the length of rotation periods identified for well-understood cover types within the project area. Finally, it is a timeframe for which we have a reasonable amount of specific information to enable us to model the system. 

We are mindful of the fact that this reference period overlaps the ``Little Ice Age,'' which may temper the utility of the results as specific management targets, but does not diminish their usefulness in other ways \citep{Safford2013}. The chosen reference period was a not time of stasis, climatically, ecologically, or culturally. The oscillation of the Palmer Drought Severity Index, a measure of climate variability in terms of precipitation and temperature, over time illustrates this (Figure~\ref{pdsi}). Multi-year droughts and El Ni\~no/La Ni\~na events also occurred over this time frame \citep{Meyer2013}. Ecologically, our historical period occurred during a very long-term (on the scale of thousands of years) shift to a warmer and drier climate, with an associated shift toward species more tolerant of such conditions, such as yellow pine species, and away from species like white fir, which prefer more mesic conditions. A slow shift toward more frequent fire was also taking place \citep{Safford2013}. Culturally, several Native American tribes were living throughout the project area during the reference period. Debate continues among scientists and researchers as to the extent to which those peoples managed vegetation through setting fires \citep{Safford2013}\todo{others?}. Because we lack empirical evidence to distinguish between lightning-caused and human-caused fires during the reference period, we decided not to exclude any fire frequency or rotation data on the basis of not being reflective of ``natural'' conditions. 

We emphasize that our choice of reference periods does not suggest that it should be our goal in management to recreate all of the ecological conditions and dynamics of this period. Such a goal is probably not possible, nor potentially desireable in light of current climate change, ecological shifts, and social realities. However, the reference period proposed will allow us to compare current conditions to a baseline set of data on ecosystem conditions (composition, configuration, and disturbance processes) ``to develop an idea of trend over time and idea of the level of depature of altered ecosystems from their ``natural'' state'' \citep{Safford2013}. The results presented here will complement the Natural Range of Variability assessments compiled by the Forest Service's Pacific Southwest Region Ecology group \citep[e.g.,][]{Safford2013,Merriam2013,Meyer2013a,Meyer2013,Estes2013,Estes2013a,Gross2013}. An understanding of natural landscape structures and variability during this reference period also provides a basis for forest management policies and associated actions that seek to mimic natural disturbance patterns \cite{Romme2000,Buse2002}. 



% why we care about climate too
In addition to the HRV analysis, the need to explore and understand the ramifications of climate change on the disturbance regime and its relationship to the forest has become more well recognized. As important as it is to understand the characteristics of the historical period, the future climate will differ from the historic climate. Fires have become more common and proportionally more severe in the last few decades, and and this is anticipated to continue under warmer and drier climate change scenarios (McKenzie et al 2004, Westerling and Bryant 2008). Where the focus of management efforts had been restoration in the past, now adaptation to ensure resilient ecosystems is the primary objective of managers (Stephens et al 2010). By simulating both the historical and a range of potential future climate scenarios, we are able to isolate the effect of climate on the disturbance regime, and evaluate the difference between results generated under the historical versus future scenarios, as well as place the current landscape within the context of both. This provides a more complete picture of the range of potential variation in the landscape than either an HRV or future climate analaysis alone.


\section{Project Objectives}
The overall purpose of this project is to quantify the historical range of variability in landsca pe structure in the Yuba River watershed on the Tahoe National Forest and evaluate the relative effects of  alternative future land management scenarios on landscape structure. The specific objectives are as follows:
\begin{enumerate}
	\item Synthesize the empirical and expert knowledge on disturbance and succession processes characteristic of the pre-settlement period in the ecoregion containing the Yuba River watershed.
	\item Based on the synthesis above, simulate landscape dynamics using the Rocky Mountain Landscape Simulator (\textsc{RMLands}).
	\item Based on the simulation above, quantify the HRV in the disturbance regime, landscape composition, and landscape configuration of our focal ecoregion.
	\item Using a suite of future climate scenarios, simulate landscape dynamics and quantify the FRV in the disturbance regime, landscape composition, and landscape configuration of our focal ecoregion.
	\item Interpret the results and provide management recommendations where appropriate.
\end{enumerate}


\section{Report Organization}

We begin with a detailed presentation of the methodology behind this project for the historic range of variability and the future management scenarios. This includes the development of the input spatial data layers, selection of values for model parameterization, model calibration and execution, and the suite of tools used to conduct the analysis. Next we present the results for the HRV and the future scenarios, focusing first on the disturbance regime and second on the vegetation response. We focus on these results by analysis method, then in the subsequent chapter, Analysis by Cover Type, discuss the results in more detail for each of the cover types independently. The next chapter, Discussion, includes an description of the scope and limitations of our simulation and its results, as well as an overall assessment of the landscape under the HRV, as compared to the current conditions. We include management implications of this study. Finally, we present the results of the FRV analysis.\footnote{In order to facilitate publication of the FRV report, it has been written as a self-contained document. As a result, some redundancy may exist between the the chapters leading up to the FRV chapter.}
