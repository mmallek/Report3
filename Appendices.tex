% !TEX root = master.tex
\appendix


\chapter{Input Layers to \textsc{RMLands}}
\label{app:inputs}

\paragraph{Technical details on \textsc{RMLands} Data Structure}
\label{app:rmlspecs}
\textsc{RMLands} uses raster GeoTiffs (.tif files) as its data structure. Rasters are based on uniform square units called cells (or pixels). Each cell represents an actual portion of geographic space. In this application, we use the Universal Transverse Mercator (UTM) projection, Zone 10 North. The extent of the raster is rectangular although the area of study is not. Cells outside of the buffered project area are assigned a null value.\footnote{Latitude and longitude are commonly pictured when describing coordinates. In such cases the X value refers to longitude and Y refers to latitude. However, because we use UTMs in this project, the correct convention is actually that the X value is the Easting and the Y value is the Northing. For simplicity we discuss X and Y only in this document.} In the Yuba River watershed landscape, each grid cell is 30 meters on a side (i.e., 900 m$^2$ or 0.09 ha), and the input grid measures 2910 by 2245 pixels. \textsc{RMLands} requires that all input grids are perfectly aligned. We accomplished this by setting the Extent and Snap Raster to the same parameters whenever we manipulated the layers in ArcMap. This ``base'' spatial layer was created by taking the primary elevation layer used on the Tahoe National Forest, resampling it to a 30 meter grid, and clipping its extent to match that of the buffered project area. Each cell is assigned a single class value, where valid class values are positive non-zero integers. Integer values are mapped to more descriptive class names using csv files with names identical to the grid name. All grids are created in ArcMap and saved as GeoTiff files before being loaded into to the model. 

\section{Input Layer Maps}
\label{app:sec:inputmaps}

% cover layer
\begin{figure}[!htbp]
\centering
\includegraphics[height=0.4\textheight]{/Users/mmallek/Tahoe/Report2/images/cover.png}
\caption{Cover Type Map for the project area. Also shows the 10 km buffer from the project area boundary. See Table~\ref{covertable} for full land cover type names.}
%\label{covermap}
\end{figure}

% condition layer
\begin{figure}[!htbp]
\centering
\includegraphics[height=0.4\textheight]{/Users/mmallek/Tahoe/Report2/images/condition.png}
\caption{Condition Class Map for the project area. Also shows the 10 km buffer from the project area boundary.} 
\label{conditionmap}
\end{figure}

% age layer
\begin{figure}[htbp]
\centering
\includegraphics[height=0.4\textheight]{/Users/mmallek/Tahoe/Report2/images/age.png}
\caption{Age map at Timestep 0 for the project area. Also shows the 10 km buffer from the project area boundary.} 
\label{agemap}
\end{figure}

% condition-age layer
\begin{figure}[htbp]
\centering
\includegraphics[height=0.4\textheight]{/Users/mmallek/Tahoe/Report2/images/condage.png}
\caption{Condition-Age map at Timestep 0 for the project area. Also shows the 10 km buffer from the project area boundary.} 
\label{condagemap}
\end{figure}

% tpi 
\begin{figure}[htbp]
\centering
\includegraphics[height=0.4\textheight]{/Users/mmallek/Tahoe/Report2/images/tpi.png}
\caption{Topographic Position Index for the project area. Also shows the 10 km buffer from the project area boundary.} 
\label{tpimap}
\end{figure}

% elevation layer
\begin{figure}[htbp]
\centering
\includegraphics[height=0.4\textheight]{/Users/mmallek/Tahoe/Report2/images/elevation.png}
\caption{Elevation for the project area. Also shows the 10 km buffer from the project area boundary.} 
\label{elevationmap}
\end{figure}

% slope layer
\begin{figure}[htbp]
\centering
\includegraphics[height=0.4\textheight]{/Users/mmallek/Tahoe/Report2/images/slope.png}
\caption{Slope for the project area, which ranges from flat to 126\%. Also shows the 10 km buffer from the project area boundary.} 
\label{slopemap}
\end{figure}

% aspect layer
\begin{figure}[htbp]
\centering
\includegraphics[height=0.4\textheight]{/Users/mmallek/Tahoe/Report2/images/aspect.png}
\caption{Aspect for the project area. Also shows the 10 km buffer from the project area boundary.} 
\label{aspectmap}
\end{figure}

% streams layer
\begin{figure}[htbp]
\centering
\includegraphics[height=0.4\textheight]{/Users/mmallek/Tahoe/Report2/images/streams.png}
\caption{Streams in the project area. Also shows the 10 km buffer from the project area boundary.} 
\label{streamsmap}
\end{figure}

%%%%%%%%%%%%%%%%%%%%%%%%%%%%%%%%%%%%%%%%%%%%%%%%%%%%%%%%%%%%%%%%%%%%%%%%%%%%%%%%%%%%%%%%%%%%%%%%%%%%%%%%%%%%%%%%%%%%%%%%%%%%%%%%%%%%%%%%%%%%%%%%%%%%%%%%%%%%%%%%%%%%%%%%%%%%%%%%%%%%%%%%%%%%%%%%%%%%%%%%%%%%%%%%%%%%%%%%%%%%%%%%%%%%%%%%%%%%%%%%%%%%%%%%%%%%%%%%%%%%%%%%%%%%%%%%%%%%%%%%%%%%%%%%%%%%%%%%%%%%%%%%%%%%%%%%%%%%%%%%%%%%%%%%%%%%%%%%%%%%%%%%%%%%%%%%%%%%%%%%%%%%%%%%%%%%%%%%%%%%%%%%%%%%%%%%
\chapter{Cover Type Descriptions}
\label{sec:covertypedesc}

% !TEX root = master.tex

\section{Big Sagebrush (SAGE)}
\label{sage-description}
\subsection{General Information}

\subsubsection{Cover Type Overview}

\paragraph{Big Sagebrush (SAGE)}

Crosswalks
\begin{itemize}
	\item EVeg: Regional Dominance Type 1
	\begin{itemize}
		\item Bitterbrush 
		\item Basin Sagebrush
		\item Great Basin Mixed Scrub
		\item Bitterbrush – Sagebrush
	\end{itemize}

	\item LandFire BpS Model
	\begin{itemize}
		\item 0610800 Inter-Mountain Basins Big Sagebrush Shrubland
	\end{itemize}

	\item Presettlement Fire Regime Type
	\begin{itemize}
		\item Big Sagebrush
	\end{itemize}
\end{itemize}

Reviewed by Michele Slaton, GIS Specialist, Inyo National Forest, USDA Forest Service

\subsubsection{Vegetation Description}
The Big Sagebrush landcover type is typified by large, open, discontinuous stands of Artemisia tridentata of fairly uniform height. A. tridentata tends to have a single short, thick, stem that branches into a nearly globular crown (Neal 1988). Ericameria nauseosa is a frequent associate or co-dominant (LandFire 2007).

Shrub canopy cover generally ranges from very open, widely spaced, small plants to large, closely spaced plants with canopies touching. Cover may be greater at higher elevations and in areas receiving more precipitation. In addition to a deep root system, A. tridentata has a well-developed system of lateral roots near the soil surface (LandFire 2007, Neal 1988). Consequently, well-established sagebrush plants exclude most other shrubs in an area up to three times their crown area. Forbs and graminoids are often more abundant beneath these crowns (Slaton pers. comm. 2013). This produces stands of shrubs of very uniform size and spacing (Neal 1988).

Often the habitat is composed of pure stands of A. tridentata, but many stands include other species of Artemisia, Ericameria, Tetradymia, Ribes, Prunus, Cercocarpus, and Purshia. In communities not fully occupied by Artemisia, various amounts of herbaceous understory are found. Perennial forb cover is usually less than 10\% with perennial grass cover reaching 20-25\% on the more productive sites. Pseudoroegneria spicata may be a dominant species following replacement fires and a co-dominant after 20 years. Elymus elymoides and Oryzopsis hymenoides are common on more xeric sites. Festuca, Stipa, Poa, and Leymus are among the more common grasses. Percent cover and species richness of understory are determined by site limitations. Pinus monophylla and Juniperus osteosperma may be present, especially in areas protected from fire (Neal 1988, LandFire 2007).


\subsubsection{Distribution}
This widespread system is common to the Basin and Range province. It ranges in elevation from 900 m to 2450+ m (3000 ft - 8000+ ft) and occurs on well-drained soils on foothills, terraces, slopes, and plateaus. It is found on deeper soils (LandFire 2007).

\subsection{Disturbances}


\subsubsection{Wildfire}
Wildfires tend to be high mortality, stand-replacing fires that initiate a process of post-fire forest succession. High mortality fires kill large as well as small shrubs, and may kill many of the forbs and grasses as well, although below-ground organs of at least some individual shrubs and herbs survive and re-sprout. 

Replacement fires generally occur where shrub canopy exceeds 25\% cover, or where grass cover is greater than 15\% and shrub cover is greater than 20\%. Surface fires occur in areas dominated by grasses but are otherwise uncommon (LandFire 2007). A tridentata does not sprout after burning but most of the other shrubs common to the type do (Neal 1988). For the last several decades, post-settlement converstion to Bromus tectorum has become common and results in changes to fire frequency and vegetation dynamics. Extended periods of fire suppression or absence can lead to P. monophylla-J. osteosperma encroachment and subsequent decline of other shrubs and herbaceous plants (LandFire 2007). 

Estimates of fire rotations are available from the LandFire project and a review paper (LandFire 2007, Van de Water and Safford 2011). The LandFire project’s published fire return intervals are based on a series of associated models created using the Vegetation Dynamics Development Tool (VDDT). In VDDT, fires are specified concurrently with the transition that follows them. For example, a replacement fire causes a transition to the early development stage. In the RMLands model, such fires are classified as high mortality. However, in VDDT mixed severity fires may cause a transition to early development, a transition to a more open condition, or no transition at all. In this case, we categorize the first example as a high mortality fire, and the second and third examples as a low mortality fire. Based on this approach, we calculated fire rotations and the probability of high mortality fire for each of the three SAGE condition classes (Table~\ref{tab:sagedesc_fire}). 

\subsubsection{Other Disturbance}
Other disturbances are not currently modeled, but may, depending on the condition affected and mortality levels, reset patches to early development, maintain existing condition classes, or shift/accelerate succession to a more open condition. 

\begin{table}[]
\small
\centering
\caption{Fire rotations (years) and probability of high versus low mortality fires. Values were derived from BpS model 0610800 (LandFire 2007), Van de Water and Safford (2011), and Safford (pers. comm. 2013).}
\label{tab:sagedesc_fire}
\begin{tabular}{@{}lcc@{}}
\toprule
\textbf{Condition}         & \multicolumn{1}{l}{\textbf{Fire Rotation}} & \multicolumn{1}{l}{\textbf{\begin{tabular}[c]{@{}l@{}}Probability of \\ High Mortality\end{tabular}}} \\ \midrule
Target                     & 115      & n/a       \\
Early Development – All    & 200      & 0         \\
Mid Development – Moderate & 125      & 1         \\
Late Development – Closed  & 100      & 0.9       \\ \bottomrule
\end{tabular}
\end{table}



\subsection{Vegetation Seral Stages}
We recognize three separate seral stages for SAGE: Early Development (ED), Mid Development – Moderate Canopy Cover (MDM), and Late Development – Closed Canopy Cover (LDC). Our seral stages are an alternative to “successional” classes that imply a linear progression of states and tend not to incorporate disturbance. The condition classes identified here are derived from a combination of successional processes and anthropogenic and natural disturbance, and are intended to represent a composition and structural condition that can be arrived at from multiple other conditions described for that landcover type. Thus our condition classes incorporate age, size, canopy cover, and vegetation composition as well as relative seral stages. In general, the delineation of stages has originated from the LandFire biophysical setting model descriptive of a given landcover type; however, condition classes are not necessarily identical to the classes identified in those models.

\subsubsection{Early Development (ED)}

\paragraph{Description} A. tridentata does not sprout after burning but most of the other shrubs common to the type do. Consequently, for as long as 20 years after fire the vegetative community may be dominated by Chrysothamnus, Tetradymia, and grasses. A very hot fire in a degraded site may result in a seral community dominated by annual grasses and forbs. Perennial bunchgrasses frequently survive fires and become dominant (Neal 1988). Canopy cover is less than 40\%, but shrub cover may be as little as 10\%. Fuel loading is discontinuous (LandFire 2007).

\paragraph{Succession Transition} In the absence of disturbance, patches in this condition will transition to MDM at 20 years. 

\paragraph{Wildfire Transition} High mortality wildfire is not modeled for this condition class Low mortality wildfire (100\% of fires in this condition) maintains the patch in the ED condition. 

\hrulefill


\subsubsection{Mid Development – Moderate Canopy Cover (MDM)}

\paragraph{Description} A. tridentata usually reaches fairly stable dominance 10 to 20 years after disturbance, with or without an understory of perennial bunchgrass. A. tridentata usually remains dominant indefinitely or until the next disturbance (Neal 1988). Shrub density is sufficient in old stands to carry the fire without fine fuels. Shrubs and herbaceous vegetation can be codominant. Generally, shrub cover averages 30\% (LandFire 2007).

\paragraph{Succession Transition} At 40 years without disturbance, patches in this condition will transition to LDC. 

\paragraph{Wildfire Transition} High mortality wildfire (90\% of fires in this condition) recycles the patch through the ED condition. Low mortality wildfire (10\%) maintains the patch in the MDM condition.

\hrulefill


\subsubsection{Late Development – Closed Canopy Cover (LDC)}

\paragraph{Description} Shrublands with some encroachment from P. monophylla and J. osteosperma possible. Wildfire has not occurred for at least 60 years. Tree species cover is highly variable. In the continued absence of disturbance, shrub cover will decline (LandFire 2007).

\paragraph{Succession Transition} In the absence of disturbance, patches in this condition will maintain. 

\paragraph{Wildfire Transition} High mortality wildfire (100\% of fires in this condition) recycles the patch through the ED condition. Low mortality wildfire is not modeled for this condition class.

\hrulefill

\subsection{Condition Classification}
Because condition classification was done through orthophoto analysis, no polygons are assigned to Late condition, which is actually not an Artemisia-dominated condition. Polygons are assigned to MDO or MDC based on a 20\% break point. Open conditions have less than 20\% cover and closed conditions have greater than 20\% cover. Polygons with a Null value for shrub cover are assigned to ED.

\subsection{Draft Model}
\begin{figure}[htbp]
\centering
\includegraphics[width=0.8\textwidth]{/Users/mmallek/Tahoe/Report3/images/state_trans_model.pdf}
\caption{Generic state and transition model for all non-shrub seral cover types. Boxes show seven condition classes and arrows depict transitions due to vegetation succession and high or low mortality fire.} 
\label{sage_transmodel}
\end{figure}


%\begin{thebibliography}
%\bibitem{LandFire} LandFire. “Biophysical Setting Models.” Biophysical Setting 0610800: Inter-Mountain Basins Big Sagebrush Shrubland. 2007. LANDFIRE Project, U.S. Department of Agriculture, Forest Service; U.S. Department of the Interior. <http://www.landfire.gov/national_veg_models_op2.php>. Accessed 9 November 2012.

%\bibitem{Neal} Neal, Donald L. “Sagebrush (SGB).” A Guide to Wildlife Habitats of California, edited by Kenneth E. Mayer and William F. Laudenslayer. California Deparment of Fish and Game, 1988. <http://www.dfg.ca.gov/biogeodata/cwhr/pdfs/SGB.pdf>. Accessed 4 December 2012.

%\bibitem{Safford} Safford, Hugh. Regional Ecologist, USDA Forest Service. Personal communication, 15 August 2013.

%\bibitem{VandeWater} Van de Water, Kip M. and Hugh D. Safford. “A Summary of Fire Frequency Estimates for California Vegetation Before Euro-American Settlement.” Fire Ecology 7.3 (2011): 26-57. doi: 10.4996/fireecology.0703026.
%\end{thebibliography}

% !TEX root = master.tex
\newpage
\section{Black and Low Sagebrush (LSG)}
\label{lsg-description}

\subsection*{General Information}

\subsubsection{Cover Type Overview}

\textbf{Black and Low Sagebrush (LSG)}
\newline
Crosswalks
\begin{itemize}
	\item EVeg: Regional Dominance Type 1
	\begin{itemize}
		\item Low Sagebrush
		\item Black Sagebrush
	\end{itemize}

	\item LandFire BpS Model
	\begin{itemize}
		\item 0610790: Great Basin Xeric Mixed Sagebrush Shrubland
	\end{itemize}

	\item Presettlement Fire Regime Type
	\begin{itemize}
		\item Black and Low Sagebrush
	\end{itemize}
\end{itemize}

\noindent Reviewed by Michele Slaton, GIS Specialist, Inyo National Forest, USDA Forest Service

\subsubsection{Vegetation Description}
\paragraph{Black and Low Sagebrush (LSG)}	This landcover type is generally dominated by broad-leaved, evergreen shrubs of short stature, typically averaging about 15\% cover. Depending on site conditions, crowns may touch. Deciduous shrubs and small trees are sometimes sparsely scattered within this type. The ground cover of grasses and forbs is typically a sparse 5-15\% cover (Verner 1988). LSG may be dominated by either \emph{Artemisia arbuscula} or \emph{Artemisia nova}, often in association with \emph{Chrysothamnus viscidiflorus}, \emph{Purshia tridentata}, or \emph{Artemisia tridentata}; \emph{A. nova} is also commonly associated with \emph{Krascheninnikovia} and \emph{Ephedra}. \emph{Juniperus occidentalis} may be sparsely scattered in stands dominated by \emph{Artemisia arbuscula}, and \emph{Juniperus osteosperma} and \emph{Pinus monophylla} are sometimes scattered in stands dominated by Artemisia nova. A rich variety of forbs is usually present, including \emph{Eriogonum}, \emph{Erigeron}, \emph{Phlox}, \emph{Castilleja}, \emph{Sphaeralcea}, and \emph{Lupinus}. Common grasses include \emph{Poa}, \emph{Pseudoroegneria}, \emph{Elymus}, \emph{Stipa} and \emph{Festuca}. The abundance and distribution of associated plants is highly influenced by soils and precipitation (Verner 1988, LandFire 2007).

\subsubsection{Distribution}
Stands of \emph{A. arbuscula} are usually found on shallow soils with impaired drainage in the transition zone between the wetter bottom and open timber on the mountainsides. The type also occurs on terraces with hardpan or heavy clay soils. In mosaics formed with \emph{P. tridentata}, \emph{A. arbuscula} occurs on harsher sites with shallow, well-drained soils, while \emph{P. tridentata} occupies areas with deeper soils. Soils typically associated with stands of A. nova are shallow, contain a high percentage of gravel, and are rich in mineral carbonates. It is prevalent on limestone soils (Verner 1988).

\emph{A. arbuscula} communities are generally restricted to elevated arid plains along the eastern flanks of the Sierra Nevada. \emph{A. nova} can occur in subalpine areas, at elevations above 2420 m (8000 ft). Stands dominated by \emph{A. arbuscula} range in elevation from 1210 to 2740 m (4000-9000 ft) (Verner 1988).


\subsection*{Disturbances}

\subsubsection{Wildfire}
Wildfires tend to be high mortality, stand-replacing fires that initiate a process of post-fire forest succession. High mortality fires kill large as well as small trees, and may kill many of the shrubs and herbs as well, although below-ground organs of at least some individual shrubs and herbs survive and re-sprout. 

A. nova generally supports more fire than other dwarf sagebrushes. Stand-replacing fire is rare due to relatively low fuel loads and herbaceous cover. Bare ground acts as a micro-barrier to fire between low-statured shrubs. Stand-replacing fires can occur in this type when successive years of above average precipitation are followed by an average or dry year. Stand-replacing fires predominate in the late successional class where the herbaceous component has diminished or where trees dominate (LandFire 2007).

Although it is not included in this iteration of the model, scientists have noted that \emph{Bromus tectorum} has invaded most of these communities, altering successional pathways and disturbance regimes. It burns readily and is an early-season post-fire colonizer (Verner 1988).

Estimates of fire rotations are available from the LandFire project and a review paper (LandFire 2007, Van de Water and Safford 2011). The LandFire project’s published fire return intervals are based on a series of associated models created using the Vegetation Dynamics Development Tool (VDDT). In VDDT, fires are specified concurrently with the transition that follows them. For example, a replacement fire causes a transition to the early development stage. In the RMLands model, such fires are classified as high mortality. However, in VDDT mixed severity fires may cause a transition to early development, a transition to a more open seral stage, or no transition at all. In this case, we categorize the first example as a high mortality fire, and the second and third examples as a low mortality fire. Based on this approach, we calculated fire rotations and the probability of high mortality fire for each of the three LSG seral stages (Table 1). We computed the overall target fire rotation of 82 years based on values from Van de Water and Safford (2011). 




\begin{table}[]
\small
\centering
\caption{Fire rotation (years) and proportion of high (versus low) mortality fires. Values were derived from VDDT model 0610790 (LandFire 2007) and Van de Water and Safford (2011). }
\label{tab:lsgdesc_fire}
\begin{tabular}{@{}lcc@{}}
\toprule
\textbf{Condition}         & \multicolumn{1}{l}{\textbf{Fire Rotation}} & \multicolumn{1}{l}{\textbf{\begin{tabular}[c]{@{}l@{}}Probability of \\ High Mortality\end{tabular}}} \\ \midrule
Target                     & 82     & n/a    \\
Early Development – All    & 250     & 1      \\
Mid Development – Moderate & 149     & 1      \\
Late Development – Closed  & 63     & 0.31    \\ \bottomrule
\end{tabular}
\end{table}

\subsubsection{Other Disturbance}
Other disturbances are not currently modeled, but may, depending on the seral stage affected and mortality levels, reset patches to early development, maintain existing seral stages, or shift/accelerate succession to a more open seral stage. 

\subsection*{Vegetation Seral Stages}
We recognize three separate seral stages for LSG: Early Development (ED), Mid Development – Moderate Canopy Cover (MDM), and Late Development – Closed Canopy Cover (LDC). Our seral stages are an alternative to ``successional'' classes that imply a linear progression of states and tend not to incorporate disturbance. The seral stages identified here are derived from a combination of successional processes and anthropogenic and natural disturbance, and are intended to represent a composition and structural condition that can be arrived at from multiple other conditions described for that landcover type. Thus our seral stages incorporate age, size, canopy cover, and vegetation composition. In general, the delineation of stages has originated from the LandFire biophysical setting model descriptive of a given landcover type; however, seral stages are not necessarily identical to the classes identified in those models.

\subsubsection{Early Development (ED)} 

\paragraph{Description} Early seral community dominated by herbaceous vegetation, including \emph{Poa}, \emph{Pseudoroegneria}, and \emph{Achnatherum}. Shrub canopy is less than 20\%. Fire-tolerant shrubs, such as \emph{Chrysothamnus} species are initial sprouters post-fire (LandFire 2007).

\paragraph{Succession Transition} In the absence of disturbance, patches in this seral stage will transition to MDM at 20 years. 

\paragraph{Wildfire Transition} High mortality wildfire (100\% of fires in this seral stage) recycles the patch through the ED seral stage. Low mortality wildfire is not modeled for this seral stage.

\noindent\hrulefill


\subsubsection{Mid Development – Moderate Canopy Cover (MDM)}

\paragraph{Description} Mid-seral community with a mixture of herbaceous and shrub vegetation. Vegetation present likely includes \emph{A. nova}, \emph{A. arbuscula}, \emph{Poa}, \emph{Achnatherum}, and \emph{Pseudoroegneria}.  Shrub cover less than 25\% (LandFire 2007).

\paragraph{Succession Transition} After 120 years without high mortality disturbance, patches in this seral stage will transition to LDC. 

\paragraph{Wildfire Transition} High mortality wildfire (100\% of fires in this seral stage) recycles the patch through the ED seral stage. Low mortality wildfire is not modeled for this seral stage.

\noindent\hrulefill


\subsubsection{Late Development – Closed Canopy Cover (LDC)} 

\paragraph{Description} Late seral community with an increased presence of conifer trees (up to 40\% cover). The degree of tree canopy closure differs depending on whether it is an \emph{A. arbuscula} (closure likely under 15\%) or an A. nova (closure up to 40\%) community. In \emph{A. arbuscula} communities a mixture of herbaceous and shrub vegetation with over 10\% shrub cover would still be present. In \emph{A. nova} communities the herbaceous and shrub component would be greatly reduced (less than 1\% cover). Vegetation present includes \emph{A. nova}, \emph{A. arbuscula}, \emph{Juniperus}, \emph{P. monophylla} and \emph{Achnatherum} (LandFire 2007).

\paragraph{Succession Transition} In the absence of disturbance, this class will maintain. 

\paragraph{Wildfire Transition} High mortality wildfire (31\% of fires in this seral stage) recycles the patch through the ED seral stage. Low mortality wildfire (69\%) maintains the LDO seral stage.

\noindent\hrulefill

\subsection*{Condition Classification}
Because seral stageification was done through orthophoto analysis, no polygons will be assigned to the LDC seral stage, which is actually not an \emph{Artemisia}-dominated seral stage. Only 3 polygons were assigned to LSG. Typical fields used to assign early-mid-late seral stage (overstory tree diameter) are null for shrubs. Cover is available. Polygons with cover less than 50\% are assigned to MD and polygons with cover greater than 50\% are assigned to LDO.

\subsection*{Draft Model}
\begin{figure}[htbp]
\centering
\includegraphics[width=0.8\textwidth]{/Users/mmallek/Tahoe/Report3/images/state_trans_model.pdf}
\caption{Generic state and transition model for all non-shrub seral cover types. Boxes show seven seral stages and arrows depict transitions due to vegetation succession and high or low mortality fire.} 
\label{lsg_transmodel}
\end{figure}

\clearpage
\subsection*{References}
\begin{hangparas}{.25in}{1} LandFire. ``Biophysical Setting Models.'' Biophysical Setting 0610790: Great Basin Xeric Mixed Sagebrush Shrubland. 2007. LANDFIRE Project, U.S. Department of Agriculture, Forest Service; U.S. Department of the Interior. \burl{http://www.landfire.gov/national\_veg\_models\_op2.php}. Accessed 9 November 2012.

Van de Water, Kip M. and Hugh D. Safford. ``A Summary of Fire Frequency Estimates for California Vegetation Before Euro-American Settlement.'' \emph{Fire Ecology} 7.3 (2011): 26-57. doi: 10.4996/fireecology.0703026.

Verner, Jared. ``Low Sage (LSG).'' \emph{A Guide to Wildlife Habitats of California}, edited by Kenneth E. Mayer and William F. Laudenslayer. California Deparment of Fish and Game, 1988. \burl{http://www.dfg.ca.gov/biogeodata/cwhr/pdfs/SGB.pdf}. Accessed 4 December 2012.
\end{hangparas}



% !TEX root = master.tex
\newpage
\section{Curl-leaf Mountain Mahogany (CMM)}

\subsection*{General Information}

\subsubsection{Cover Type Overview}

\textbf{Curl-leaf Mountain Mahogany (CMM)}
\newline
Crosswalks
\begin{itemize}
	\item EVeg: Regional Dominance Type 1
	\begin{itemize}
		\item Curl-leaf Mountain Mahogany
	\end{itemize}

	\item LandFire BpS Model
	\begin{itemize}
		\item 0610620: Inter-Mountain Basin Curl-leaf Mountan Mahogany Woodland and Shrubland
	\end{itemize}

	\item Presettlement Fire Regime Type
	\begin{itemize}
		\item Curl-leaf Mountain Mahogany
	\end{itemize}
\end{itemize}

\noindent Reviewed by Becky Estes, Central Sierra Province Ecologist, USDA Forest Service

\subsubsection{Vegetation Description}
This landcover type is characterized by the dominance or co-dominance of \emph{Cercocarpus ledifolius}. Other shrubs such as \emph{Artemisia}, \emph{Arctostaphylos}, \emph{Ceanothus}, and \emph{Ephedra} may be present. \emph{C. ledifolius} is both a primary early successional colonizer rapidly invading bare mineral soils after disturbance and the dominant long-lived species. Depending on the effects of a given fire on the seed bank, in some cases it could take 10 years to recolonize. Where \emph{C. ledifolius} has reestablished quickly after fire, \emph{Chrysothamnus nauseosus} may codominate. Litter and shading by woody plants inhibits the establishment of \emph{C. ledifolius}, particularly in late seral stages where canopy cover is high. Reproduction often appears more dependent upon geographic variables (slope, aspect, and elevation) than biotic factors. \emph{Artemisia arbuscula} and \emph{Artemisia nova} are infrequently associated. \emph{Symphoricarpos}, \emph{Amelanchier}, and \emph{Ribes} are present on cooler, moister sites. \emph{Pinus monophylla}, \emph{Juniperus}, \emph{Pseudotsuga menziesii}, \emph{Abies magnifica}, \emph{Abies concolor}, and \emph{Pinus jeffreyi} may have sporadic presence at very low densities. In older stands the understory may consist largely of \emph{Leptodactylon pungens} (LandFire 2007, Gucker 2006).

\subsubsection{Distribution}
\emph{C. ledifolius} communities are usually found on upper slopes and ridges between 2130 and 3200 m (7000-10,500 ft), although northern stands may occur as low as 600 m (200 ft). It is more common on northwestern and northeastern aspects. Most stands occur on rocky, shallow soils and outcrops, with mature stand cover from 10-55\%. In the absence of fire, old stands may occur on somewhat deeper soils, with more than 55\% cover (LandFire 2007).

\subsection*{Disturbances}

\subsubsection{Wildfire}
Wildfires tend to be high mortality, stand-replacing fires that initiate a process of post-fire forest succession. High mortality fires kill large as well as small trees, and may kill many of the shrubs and herbs as well, although below-ground organs of at least some individual shrubs and herbs survive and re-sprout. 

\emph{C. ledifolius} is easily killed by fire and does not resprout. However, it is a primary early successional colonizer, rapidly invading bare mineral soils after disturbance. Fires are not common in early seral stages, when there is little fuel, except in chaparral-dominated stands. Stand-replacing fires are more common in mid-seral stands, where herbs and smaller shrubs provide ladder fuels. When surface fire is relatively common, stands will adopt a savanna-like woodland structure with an understory characterized by \emph{Ribes}, \emph{L. pungens}, and various grasses. Trees can become very old and will rarely show fire scars. In late, closed stands, the absence of herbs and small forbs makes fire uncommon, requiring extreme winds and drought conditions. However, stands that do burn often experience high mortality fire (LandFire 2007).

Estimates of fire rotations are available from the LandFire project and a review paper (LandFire 2007, Van de Water and Safford 2011). The LandFire project's published fire return intervals are based on a series of associated models created using the Vegetation Dynamics Development Tool (VDDT). In VDDT, fires are specified concurrently with the transition that follows them. For example, a replacement fire causes a transition to the early development stage. In the RMLands model, such fires are classified as high mortality. However, in VDDT mixed severity fires may cause a transition to early development, a transition to a more open seral stage, or no transition at all. In this case, we categorize the first example as a high mortality fire, and the second and third examples as a low mortality fire. Based on this approach, we calculated fire rotations and the probability of high mortality fire for each of the three CMM seral stages (Table 1). We computed the overall target fire rotation of 76 years based on values from Van de Water and Safford (2011). 




\begin{table}[]
\small
\centering
\caption{Fire rotation (years) and proportion of high (versus low) mortality fires. Values were derived from VDDT model 0610790 (LandFire 2007) and Van de Water and Safford (2011). }
\label{tab:cmmdesc_fire}
\begin{tabular}{@{}lcc@{}}
\toprule
\textbf{Condition}         & \multicolumn{1}{l}{\textbf{Fire Rotation}} & \multicolumn{1}{l}{\textbf{\begin{tabular}[c]{@{}l@{}}Probability of \\ High Mortality\end{tabular}}} \\ \midrule
Target                     & 76  & n/a      \\
Early Development - All    & 83  & 0.17        \\
Mid Development - Moderate & 17  & 0.67        \\
Late Development - Closed  & 500  & 1      \\ \bottomrule
\end{tabular}
\end{table}

\subsubsection{Other Disturbance}
Other disturbances are not currently modeled, but may, depending on the seral stage affected and mortality levels, reset patches to early development, maintain existing seral stages, or shift/accelerate succession to a more open seral stage. 

\subsection*{Vegetation Seral Stages}
We recognize three separate seral stages for CMM: Early Development (ED), Mid Development - Moderate Canopy Cover (MDM), and Late Development - Closed Canopy Cover (LDC). Our seral stages are an alternative to ``successional'' classes that imply a linear progression of states and tend not to incorporate disturbance. The seral stages identified here are derived from a combination of successional processes and anthropogenic and natural disturbance, and are intended to represent a composition and structural condition that can be arrived at from multiple other conditions described for that landcover type. Thus our seral stages incorporate age, size, canopy cover, and vegetation composition. In general, the delineation of stages has originated from the LandFire biophysical setting model descriptive of a given landcover type; however, seral stages are not necessarily identical to the classes identified in those models.

\subsubsection{Early Development (ED)}

\paragraph{Description} \emph{C. ledifolius} seedlings rapidly invade bare mineral soils after fire. Litter and shading by woody plants inhibits establishment. Bunchgrasses and disturbance-tolerant forbs and resprouting shrubs, such as \emph{Symphoricarpos}, may be present. \emph{Ericameria} and \emph{Artemisia} seedlings are likely present. Vegetation composition will affect fire behavior, especially if chaparral species like \emph{Arctostaphylos} or \emph{Ceanothus} are present (LandFire 2007).

\paragraph{Succession Transition} In the absence of disturbance, patches in this seral stage will transition to MDM upon reaching 20 years of age. 

\paragraph{Wildfire Transition} High mortality wildfire (17\% of fires in this seral stage) recycles the patch through the ED seral stage. No transition occurs as a result of low mortality fire.

\noindent\hrulefill


\subsubsection{Mid Development - Moderate Canopy Cover (MDM)}

\paragraph{Description} \emph{C. ledifolius} may co-dominate with mature \emph{Artemisia}, \emph{Purshia}, \emph{Symphoricarpos}, or \emph{Ericameria}. Few \emph{C. ledifolius} seedlings are present. Canopy cover is variable (LandFire 2007).

\paragraph{Succession Transition} After 120 years in this stage, patches in this seral stage will transition to LDC.

\paragraph{Wildfire Transition} High mortality wildfire (67\% of fires in this seral stage) recycles the patch through the ED seral stage. No transition occurs as a result of low mortality fire.

\noindent\hrulefill


\subsubsection{Late Development - Closed Canopy Cover (LDC)}

\paragraph{Description} Moderate to high cover of large shrub- or tree-like \emph{C. ledifolius}. When low mortality fire is relatively frequent, late-successional \emph{C. ledifolius} may exhibit evidence of infrequent fire scars on older trees. Patches may consist of open savanna-like woodlands with an herbaceous-dominated understory. Other shrub species may be abundant, but decadent. When low mortality fire is absent, very few other shrubs are present, and herbaceous cover is low. Duff may be very deep, and scattered trees may occur. \emph{C. ledifolius} trees reach very old age in the absence of stand-replacing fire, potentially living over 1000 years (LandFire 2007).

\paragraph{Succession Transition} In the absence of disturbance, patches in this seral stage will remain in this seral stage. 

\paragraph{Wildfire Transition} High mortality wildfire (100\% of fires in this seral stage) recycles the patch through the ED seral stage.

\noindent\hrulefill

\subsection*{Condition Classification}
To create the initial cover and seral stage layer (2010), polygons were randomly assigned to seral stages based on a 20:10:70 distribution for early/mid/late development (based on an analysis of past fire in the project area). Random numbers between 0 and 1 were generated using numpy for Python and used to assign each CMM polygon to a seral stage.

\subsection*{Draft Model}
\begin{figure}[htbp]
\centering
\includegraphics[width=0.8\textwidth]{/Users/mmallek/Documents/Thesis/statetransmodel/StateTransitionModel/shrub.png}
\caption{State and Transition Model for Curl-leaf Mountain Mahogany. Each dark grey box represents one of the three seral stages for this landcover type. Three stages of development are represented: early, middle, and late. We describe the middle development stage as characterized by moderate canopy cover and the late development stage as characterized by closed canopy cover, but these are not hard and fast rules. Transitions between states/seral stages may occur as a result of high mortality fire, low mortality fire, or succession. Specific pathways for each are denoted by the appropriate color line and arrow: red lines relate to high mortality fire, orange lines relate to low mortality fire, and green lines relate to natural succession.} 
\label{cmm_transmodel}
\end{figure}

\clearpage
\subsection*{References}
\begin{hangparas}{.25in}{1} Gucker, Corey L. ``Cercocarpus ledifolius'' \emph{Fire Effects Information System}, U.S. Department of Agriculture, Forest Service, Rocky Mountain Research Station, Fire Sciences Laboratory, 2006.  \burl{http://www.fs.fed.us/database/feis/} [Accessed 29 July 2013.]. 

LandFire. ``Biophysical Setting Models.'' Biophysical Setting 0610790: Great Basin Xeric Mixed Sagebrush Shrubland. 2007. LANDFIRE Project, U.S. Department of Agriculture, Forest Service; U.S. Department of the Interior. \burl{http://www.landfire.gov/national\_veg\_models\_op2.php}. Accessed 9 November 2012.

Van de Water, Kip M. and Hugh D. Safford. ``A Summary of Fire Frequency Estimates for California Vegetation Before Euro-American Settlement.'' \emph{Fire Ecology} 7.3 (2011): 26-57. doi: 10.4996/fireecology.0703026.

\end{hangparas}






