% !TEX root = master.tex
\appendix


\chapter{Input Layers to \textsc{RMLands}}
\label{app:inputs}

\paragraph{Technical details on \textsc{RMLands} Data Structure}
\label{app:rmlspecs}
\textsc{RMLands} uses raster GeoTiffs (.tif files) as its data structure. Rasters are based on uniform square units called cells (or pixels). Each cell represents an actual portion of geographic space. In this application, we use the Universal Transverse Mercator (UTM) projection, Zone 10 North. The extent of the raster is rectangular although the area of study is not. Cells outside of the buffered project area are assigned a null value.\footnote{Latitude and longitude are commonly pictured when describing coordinates. In such cases the X value refers to longitude and Y refers to latitude. However, because we use UTMs in this project, the correct convention is actually that the X value is the Easting and the Y value is the Northing. For simplicity we discuss X and Y only in this document.} In the Yuba River watershed landscape, each grid cell is 30 meters on a side (i.e., 900 m$^2$ or 0.09 ha), and the input grid measures 2910 by 2245 pixels. \textsc{RMLands} requires that all input grids are perfectly aligned. We accomplished this by setting the Extent and Snap Raster to the same parameters whenever we manipulated the layers in ArcMap. This ``base'' spatial layer was created by taking the primary elevation layer used on the Tahoe National Forest, resampling it to a 30 meter grid, and clipping its extent to match that of the buffered project area. Each cell is assigned a single class value, where valid class values are positive non-zero integers. Integer values are mapped to more descriptive class names using csv files with names identical to the grid name. All grids are created in ArcMap and saved as GeoTiff files before being loaded into to the model. 

\section{Input Layer Maps}
\label{app:sec:inputmaps}

% cover layer
\begin{figure}[!htbp]
\centering
\includegraphics[height=0.4\textheight]{/Users/mmallek/Tahoe/Report2/images/cover.png}
\caption{Cover Type Map for the project area. Also shows the 10 km buffer from the project area boundary. See Table~\ref{covertable} for full land cover type names.}
%\label{covermap}
\end{figure}

% condition layer
\begin{figure}[!htbp]
\centering
\includegraphics[height=0.4\textheight]{/Users/mmallek/Tahoe/Report2/images/condition.png}
\caption{Condition Class Map for the project area. Also shows the 10 km buffer from the project area boundary.} 
\label{conditionmap}
\end{figure}

% age layer
\begin{figure}[htbp]
\centering
\includegraphics[height=0.4\textheight]{/Users/mmallek/Tahoe/Report2/images/age.png}
\caption{Age map at Timestep 0 for the project area. Also shows the 10 km buffer from the project area boundary.} 
\label{agemap}
\end{figure}

% condition-age layer
\begin{figure}[htbp]
\centering
\includegraphics[height=0.4\textheight]{/Users/mmallek/Tahoe/Report2/images/condage.png}
\caption{Condition-Age map at Timestep 0 for the project area. Also shows the 10 km buffer from the project area boundary.} 
\label{condagemap}
\end{figure}

% tpi 
\begin{figure}[htbp]
\centering
\includegraphics[height=0.4\textheight]{/Users/mmallek/Tahoe/Report2/images/tpi.png}
\caption{Topographic Position Index for the project area. Also shows the 10 km buffer from the project area boundary.} 
\label{tpimap}
\end{figure}

% elevation layer
\begin{figure}[htbp]
\centering
\includegraphics[height=0.4\textheight]{/Users/mmallek/Tahoe/Report2/images/elevation.png}
\caption{Elevation for the project area. Also shows the 10 km buffer from the project area boundary.} 
\label{elevationmap}
\end{figure}

% slope layer
\begin{figure}[htbp]
\centering
\includegraphics[height=0.4\textheight]{/Users/mmallek/Tahoe/Report2/images/slope.png}
\caption{Slope for the project area, which ranges from flat to 126\%. Also shows the 10 km buffer from the project area boundary.} 
\label{slopemap}
\end{figure}

% aspect layer
\begin{figure}[htbp]
\centering
\includegraphics[height=0.4\textheight]{/Users/mmallek/Tahoe/Report2/images/aspect.png}
\caption{Aspect for the project area. Also shows the 10 km buffer from the project area boundary.} 
\label{aspectmap}
\end{figure}

% streams layer
\begin{figure}[htbp]
\centering
\includegraphics[height=0.4\textheight]{/Users/mmallek/Tahoe/Report2/images/streams.png}
\caption{Streams in the project area. Also shows the 10 km buffer from the project area boundary.} 
\label{streamsmap}
\end{figure}

%%%%%%%%%%%%%%%%%%%%%%%%%%%%%%%%%%%%%%%%%%%%%%%%%%%%%%%%%%%%%%%%%%%%%%%%%%%%%%%%%%%%%%%%%%%%%%%%%%%%%%%%%%%%%%%%%%%%%%%%%%%%%%%%%%%%%%%%%%%%%%%%%%%%%%%%%%%%%%%%%%%%%%%%%%%%%%%%%%%%%%%%%%%%%%%%%%%%%%%%%%%%%%%%%%%%%%%%%%%%%%%%%%%%%%%%%%%%%%%%%%%%%%%%%%%%%%%%%%%%%%%%%%%%%%%%%%%%%%%%%%%%%%%%%%%%%%%%%%%%%%%%%%%%%%%%%%%%%%%%%%%%%%%%%%%%%%%%%%%%%%%%%%%%%%%%%%%%%%%%%%%%%%%%%%%%%%%%%%%%%%%%%%%%%%%%
\chapter{Cover Type Descriptions}
\label{sec:covertypedesc}

% !TEX root = master.tex

\section{Big Sagebrush (SAGE)}
\label{sage-description}
\subsection{General Information}

\subsubsection{Cover Type Overview}

\paragraph{Big Sagebrush (SAGE)}

Crosswalks
\begin{itemize}
	\item EVeg: Regional Dominance Type 1
	\begin{itemize}
		\item Bitterbrush 
		\item Basin Sagebrush
		\item Great Basin Mixed Scrub
		\item Bitterbrush – Sagebrush
	\end{itemize}

	\item LandFire BpS Model
	\begin{itemize}
		\item 0610800 Inter-Mountain Basins Big Sagebrush Shrubland
	\end{itemize}

	\item Presettlement Fire Regime Type
	\begin{itemize}
		\item Big Sagebrush
	\end{itemize}
\end{itemize}

Reviewed by Michele Slaton, GIS Specialist, Inyo National Forest, USDA Forest Service

\subsubsection{Vegetation Description}
The Big Sagebrush landcover type is typified by large, open, discontinuous stands of Artemisia tridentata of fairly uniform height. A. tridentata tends to have a single short, thick, stem that branches into a nearly globular crown (Neal 1988). Ericameria nauseosa is a frequent associate or co-dominant (LandFire 2007).

Shrub canopy cover generally ranges from very open, widely spaced, small plants to large, closely spaced plants with canopies touching. Cover may be greater at higher elevations and in areas receiving more precipitation. In addition to a deep root system, A. tridentata has a well-developed system of lateral roots near the soil surface (LandFire 2007, Neal 1988). Consequently, well-established sagebrush plants exclude most other shrubs in an area up to three times their crown area. Forbs and graminoids are often more abundant beneath these crowns (Slaton pers. comm. 2013). This produces stands of shrubs of very uniform size and spacing (Neal 1988).

Often the habitat is composed of pure stands of A. tridentata, but many stands include other species of Artemisia, Ericameria, Tetradymia, Ribes, Prunus, Cercocarpus, and Purshia. In communities not fully occupied by Artemisia, various amounts of herbaceous understory are found. Perennial forb cover is usually less than 10\% with perennial grass cover reaching 20-25\% on the more productive sites. Pseudoroegneria spicata may be a dominant species following replacement fires and a co-dominant after 20 years. Elymus elymoides and Oryzopsis hymenoides are common on more xeric sites. Festuca, Stipa, Poa, and Leymus are among the more common grasses. Percent cover and species richness of understory are determined by site limitations. Pinus monophylla and Juniperus osteosperma may be present, especially in areas protected from fire (Neal 1988, LandFire 2007).


\subsubsection{Distribution}
This widespread system is common to the Basin and Range province. It ranges in elevation from 900 m to 2450+ m (3000 ft - 8000+ ft) and occurs on well-drained soils on foothills, terraces, slopes, and plateaus. It is found on deeper soils (LandFire 2007).

\subsection{Disturbances}


\subsubsection{Wildfire}
Wildfires tend to be high mortality, stand-replacing fires that initiate a process of post-fire forest succession. High mortality fires kill large as well as small shrubs, and may kill many of the forbs and grasses as well, although below-ground organs of at least some individual shrubs and herbs survive and re-sprout. 

Replacement fires generally occur where shrub canopy exceeds 25\% cover, or where grass cover is greater than 15\% and shrub cover is greater than 20\%. Surface fires occur in areas dominated by grasses but are otherwise uncommon (LandFire 2007). A tridentata does not sprout after burning but most of the other shrubs common to the type do (Neal 1988). For the last several decades, post-settlement converstion to Bromus tectorum has become common and results in changes to fire frequency and vegetation dynamics. Extended periods of fire suppression or absence can lead to P. monophylla-J. osteosperma encroachment and subsequent decline of other shrubs and herbaceous plants (LandFire 2007). 

Estimates of fire rotations are available from the LandFire project and a review paper (LandFire 2007, Van de Water and Safford 2011). The LandFire project’s published fire return intervals are based on a series of associated models created using the Vegetation Dynamics Development Tool (VDDT). In VDDT, fires are specified concurrently with the transition that follows them. For example, a replacement fire causes a transition to the early development stage. In the RMLands model, such fires are classified as high mortality. However, in VDDT mixed severity fires may cause a transition to early development, a transition to a more open condition, or no transition at all. In this case, we categorize the first example as a high mortality fire, and the second and third examples as a low mortality fire. Based on this approach, we calculated fire rotations and the probability of high mortality fire for each of the three SAGE condition classes (Table~\ref{tab:sagedesc_fire}). 

\subsubsection{Other Disturbance}
Other disturbances are not currently modeled, but may, depending on the condition affected and mortality levels, reset patches to early development, maintain existing condition classes, or shift/accelerate succession to a more open condition. 

\begin{table}[]
\small
\centering
\caption{Fire rotations (years) and probability of high versus low mortality fires. Values were derived from BpS model 0610800 (LandFire 2007), Van de Water and Safford (2011), and Safford (pers. comm. 2013).}
\label{tab:sagedesc_fire}
\begin{tabular}{@{}lcc@{}}
\toprule
\textbf{Condition}         & \multicolumn{1}{l}{\textbf{Fire Rotation}} & \multicolumn{1}{l}{\textbf{\begin{tabular}[c]{@{}l@{}}Probability of \\ High Mortality\end{tabular}}} \\ \midrule
Target                     & 115      & n/a       \\
Early Development – All    & 200      & 0         \\
Mid Development – Moderate & 125      & 1         \\
Late Development – Closed  & 100      & 0.9       \\ \bottomrule
\end{tabular}
\end{table}



\subsection{Vegetation Seral Stages}
We recognize three separate seral stages for SAGE: Early Development (ED), Mid Development – Moderate Canopy Cover (MDM), and Late Development – Closed Canopy Cover (LDC). Our seral stages are an alternative to “successional” classes that imply a linear progression of states and tend not to incorporate disturbance. The condition classes identified here are derived from a combination of successional processes and anthropogenic and natural disturbance, and are intended to represent a composition and structural condition that can be arrived at from multiple other conditions described for that landcover type. Thus our condition classes incorporate age, size, canopy cover, and vegetation composition as well as relative seral stages. In general, the delineation of stages has originated from the LandFire biophysical setting model descriptive of a given landcover type; however, condition classes are not necessarily identical to the classes identified in those models.

\subsubsection{Early Development (ED)}

\paragraph{Description} A. tridentata does not sprout after burning but most of the other shrubs common to the type do. Consequently, for as long as 20 years after fire the vegetative community may be dominated by Chrysothamnus, Tetradymia, and grasses. A very hot fire in a degraded site may result in a seral community dominated by annual grasses and forbs. Perennial bunchgrasses frequently survive fires and become dominant (Neal 1988). Canopy cover is less than 40\%, but shrub cover may be as little as 10\%. Fuel loading is discontinuous (LandFire 2007).

\paragraph{Succession Transition} In the absence of disturbance, patches in this condition will transition to MDM at 20 years. 

\paragraph{Wildfire Transition} High mortality wildfire is not modeled for this condition class Low mortality wildfire (100\% of fires in this condition) maintains the patch in the ED condition. 

\hrulefill


\subsubsection{Mid Development – Moderate Canopy Cover (MDM)}

\paragraph{Description} A. tridentata usually reaches fairly stable dominance 10 to 20 years after disturbance, with or without an understory of perennial bunchgrass. A. tridentata usually remains dominant indefinitely or until the next disturbance (Neal 1988). Shrub density is sufficient in old stands to carry the fire without fine fuels. Shrubs and herbaceous vegetation can be codominant. Generally, shrub cover averages 30\% (LandFire 2007).

\paragraph{Succession Transition} At 40 years without disturbance, patches in this condition will transition to LDC. 

\paragraph{Wildfire Transition} High mortality wildfire (90\% of fires in this condition) recycles the patch through the ED condition. Low mortality wildfire (10\%) maintains the patch in the MDM condition.

\hrulefill


\subsubsection{Late Development – Closed Canopy Cover (LDC)}

\paragraph{Description} Shrublands with some encroachment from P. monophylla and J. osteosperma possible. Wildfire has not occurred for at least 60 years. Tree species cover is highly variable. In the continued absence of disturbance, shrub cover will decline (LandFire 2007).

\paragraph{Succession Transition} In the absence of disturbance, patches in this condition will maintain. 

\paragraph{Wildfire Transition} High mortality wildfire (100\% of fires in this condition) recycles the patch through the ED condition. Low mortality wildfire is not modeled for this condition class.

\hrulefill

\subsection{Condition Classification}
Because condition classification was done through orthophoto analysis, no polygons are assigned to Late condition, which is actually not an Artemisia-dominated condition. Polygons are assigned to MDO or MDC based on a 20\% break point. Open conditions have less than 20\% cover and closed conditions have greater than 20\% cover. Polygons with a Null value for shrub cover are assigned to ED.

\subsection{Draft Model}
\begin{figure}[htbp]
\centering
\includegraphics[width=0.8\textwidth]{/Users/mmallek/Tahoe/Report3/images/state_trans_model.pdf}
\caption{Generic state and transition model for all non-shrub seral cover types. Boxes show seven condition classes and arrows depict transitions due to vegetation succession and high or low mortality fire.} 
\label{sage_transmodel}
\end{figure}


%\begin{thebibliography}
%\bibitem{LandFire} LandFire. “Biophysical Setting Models.” Biophysical Setting 0610800: Inter-Mountain Basins Big Sagebrush Shrubland. 2007. LANDFIRE Project, U.S. Department of Agriculture, Forest Service; U.S. Department of the Interior. <http://www.landfire.gov/national_veg_models_op2.php>. Accessed 9 November 2012.

%\bibitem{Neal} Neal, Donald L. “Sagebrush (SGB).” A Guide to Wildlife Habitats of California, edited by Kenneth E. Mayer and William F. Laudenslayer. California Deparment of Fish and Game, 1988. <http://www.dfg.ca.gov/biogeodata/cwhr/pdfs/SGB.pdf>. Accessed 4 December 2012.

%\bibitem{Safford} Safford, Hugh. Regional Ecologist, USDA Forest Service. Personal communication, 15 August 2013.

%\bibitem{VandeWater} Van de Water, Kip M. and Hugh D. Safford. “A Summary of Fire Frequency Estimates for California Vegetation Before Euro-American Settlement.” Fire Ecology 7.3 (2011): 26-57. doi: 10.4996/fireecology.0703026.
%\end{thebibliography}

% !TEX root = master.tex
\newpage
\section{Black and Low Sagebrush (LSG)}
\label{lsg-description}

\subsection*{General Information}

\subsubsection{Cover Type Overview}

\textbf{Black and Low Sagebrush (LSG)}
\newline
Crosswalks
\begin{itemize}
	\item EVeg: Regional Dominance Type 1
	\begin{itemize}
		\item Low Sagebrush
		\item Black Sagebrush
	\end{itemize}

	\item LandFire BpS Model
	\begin{itemize}
		\item 0610790: Great Basin Xeric Mixed Sagebrush Shrubland
	\end{itemize}

	\item Presettlement Fire Regime Type
	\begin{itemize}
		\item Black and Low Sagebrush
	\end{itemize}
\end{itemize}

\noindent Reviewed by Michele Slaton, GIS Specialist, Inyo National Forest, USDA Forest Service

\subsubsection{Vegetation Description}
\paragraph{Black and Low Sagebrush (LSG)}	This landcover type is generally dominated by broad-leaved, evergreen shrubs of short stature, typically averaging about 15\% cover. Depending on site conditions, crowns may touch. Deciduous shrubs and small trees are sometimes sparsely scattered within this type. The ground cover of grasses and forbs is typically a sparse 5-15\% cover (Verner 1988). LSG may be dominated by either \emph{Artemisia arbuscula} or \emph{Artemisia nova}, often in association with \emph{Chrysothamnus viscidiflorus}, \emph{Purshia tridentata}, or \emph{Artemisia tridentata}; \emph{A. nova} is also commonly associated with \emph{Krascheninnikovia} and \emph{Ephedra}. \emph{Juniperus occidentalis} may be sparsely scattered in stands dominated by \emph{Artemisia arbuscula}, and \emph{Juniperus osteosperma} and \emph{Pinus monophylla} are sometimes scattered in stands dominated by Artemisia nova. A rich variety of forbs is usually present, including \emph{Eriogonum}, \emph{Erigeron}, \emph{Phlox}, \emph{Castilleja}, \emph{Sphaeralcea}, and \emph{Lupinus}. Common grasses include \emph{Poa}, \emph{Pseudoroegneria}, \emph{Elymus}, \emph{Stipa} and \emph{Festuca}. The abundance and distribution of associated plants is highly influenced by soils and precipitation (Verner 1988, LandFire 2007).

\subsubsection{Distribution}
Stands of \emph{A. arbuscula} are usually found on shallow soils with impaired drainage in the transition zone between the wetter bottom and open timber on the mountainsides. The type also occurs on terraces with hardpan or heavy clay soils. In mosaics formed with \emph{P. tridentata}, \emph{A. arbuscula} occurs on harsher sites with shallow, well-drained soils, while \emph{P. tridentata} occupies areas with deeper soils. Soils typically associated with stands of A. nova are shallow, contain a high percentage of gravel, and are rich in mineral carbonates. It is prevalent on limestone soils (Verner 1988).

\emph{A. arbuscula} communities are generally restricted to elevated arid plains along the eastern flanks of the Sierra Nevada. \emph{A. nova} can occur in subalpine areas, at elevations above 2420 m (8000 ft). Stands dominated by \emph{A. arbuscula} range in elevation from 1210 to 2740 m (4000-9000 ft) (Verner 1988).


\subsection*{Disturbances}

\subsubsection{Wildfire}
Wildfires tend to be high mortality, stand-replacing fires that initiate a process of post-fire forest succession. High mortality fires kill large as well as small trees, and may kill many of the shrubs and herbs as well, although below-ground organs of at least some individual shrubs and herbs survive and re-sprout. 

A. nova generally supports more fire than other dwarf sagebrushes. Stand-replacing fire is rare due to relatively low fuel loads and herbaceous cover. Bare ground acts as a micro-barrier to fire between low-statured shrubs. Stand-replacing fires can occur in this type when successive years of above average precipitation are followed by an average or dry year. Stand-replacing fires predominate in the late successional class where the herbaceous component has diminished or where trees dominate (LandFire 2007).

Although it is not included in this iteration of the model, scientists have noted that \emph{Bromus tectorum} has invaded most of these communities, altering successional pathways and disturbance regimes. It burns readily and is an early-season post-fire colonizer (Verner 1988).

Estimates of fire rotations are available from the LandFire project and a review paper (LandFire 2007, Van de Water and Safford 2011). The LandFire project’s published fire return intervals are based on a series of associated models created using the Vegetation Dynamics Development Tool (VDDT). In VDDT, fires are specified concurrently with the transition that follows them. For example, a replacement fire causes a transition to the early development stage. In the RMLands model, such fires are classified as high mortality. However, in VDDT mixed severity fires may cause a transition to early development, a transition to a more open seral stage, or no transition at all. In this case, we categorize the first example as a high mortality fire, and the second and third examples as a low mortality fire. Based on this approach, we calculated fire rotations and the probability of high mortality fire for each of the three LSG seral stages (Table 1). We computed the overall target fire rotation of 82 years based on values from Van de Water and Safford (2011). 




\begin{table}[]
\small
\centering
\caption{Fire rotation (years) and proportion of high (versus low) mortality fires. Values were derived from VDDT model 0610790 (LandFire 2007) and Van de Water and Safford (2011). }
\label{tab:lsgdesc_fire}
\begin{tabular}{@{}lcc@{}}
\toprule
\textbf{Condition}         & \multicolumn{1}{l}{\textbf{Fire Rotation}} & \multicolumn{1}{l}{\textbf{\begin{tabular}[c]{@{}l@{}}Probability of \\ High Mortality\end{tabular}}} \\ \midrule
Target                     & 82     & n/a    \\
Early Development – All    & 250     & 1      \\
Mid Development – Moderate & 149     & 1      \\
Late Development – Closed  & 63     & 0.31    \\ \bottomrule
\end{tabular}
\end{table}

\subsubsection{Other Disturbance}
Other disturbances are not currently modeled, but may, depending on the seral stage affected and mortality levels, reset patches to early development, maintain existing seral stages, or shift/accelerate succession to a more open seral stage. 

\subsection*{Vegetation Seral Stages}
We recognize three separate seral stages for LSG: Early Development (ED), Mid Development – Moderate Canopy Cover (MDM), and Late Development – Closed Canopy Cover (LDC). Our seral stages are an alternative to ``successional'' classes that imply a linear progression of states and tend not to incorporate disturbance. The seral stages identified here are derived from a combination of successional processes and anthropogenic and natural disturbance, and are intended to represent a composition and structural condition that can be arrived at from multiple other conditions described for that landcover type. Thus our seral stages incorporate age, size, canopy cover, and vegetation composition. In general, the delineation of stages has originated from the LandFire biophysical setting model descriptive of a given landcover type; however, seral stages are not necessarily identical to the classes identified in those models.

\subsubsection{Early Development (ED)} 

\paragraph{Description} Early seral community dominated by herbaceous vegetation, including \emph{Poa}, \emph{Pseudoroegneria}, and \emph{Achnatherum}. Shrub canopy is less than 20\%. Fire-tolerant shrubs, such as \emph{Chrysothamnus} species are initial sprouters post-fire (LandFire 2007).

\paragraph{Succession Transition} In the absence of disturbance, patches in this seral stage will transition to MDM at 20 years. 

\paragraph{Wildfire Transition} High mortality wildfire (100\% of fires in this seral stage) recycles the patch through the ED seral stage. Low mortality wildfire is not modeled for this seral stage.

\noindent\hrulefill


\subsubsection{Mid Development – Moderate Canopy Cover (MDM)}

\paragraph{Description} Mid-seral community with a mixture of herbaceous and shrub vegetation. Vegetation present likely includes \emph{A. nova}, \emph{A. arbuscula}, \emph{Poa}, \emph{Achnatherum}, and \emph{Pseudoroegneria}.  Shrub cover less than 25\% (LandFire 2007).

\paragraph{Succession Transition} After 120 years without high mortality disturbance, patches in this seral stage will transition to LDC. 

\paragraph{Wildfire Transition} High mortality wildfire (100\% of fires in this seral stage) recycles the patch through the ED seral stage. Low mortality wildfire is not modeled for this seral stage.

\noindent\hrulefill


\subsubsection{Late Development – Closed Canopy Cover (LDC)} 

\paragraph{Description} Late seral community with an increased presence of conifer trees (up to 40\% cover). The degree of tree canopy closure differs depending on whether it is an \emph{A. arbuscula} (closure likely under 15\%) or an A. nova (closure up to 40\%) community. In \emph{A. arbuscula} communities a mixture of herbaceous and shrub vegetation with over 10\% shrub cover would still be present. In \emph{A. nova} communities the herbaceous and shrub component would be greatly reduced (less than 1\% cover). Vegetation present includes \emph{A. nova}, \emph{A. arbuscula}, \emph{Juniperus}, \emph{P. monophylla} and \emph{Achnatherum} (LandFire 2007).

\paragraph{Succession Transition} In the absence of disturbance, this class will maintain. 

\paragraph{Wildfire Transition} High mortality wildfire (31\% of fires in this seral stage) recycles the patch through the ED seral stage. Low mortality wildfire (69\%) maintains the LDO seral stage.

\noindent\hrulefill

\subsection*{Condition Classification}
Because seral stageification was done through orthophoto analysis, no polygons will be assigned to the LDC seral stage, which is actually not an \emph{Artemisia}-dominated seral stage. Only 3 polygons were assigned to LSG. Typical fields used to assign early-mid-late seral stage (overstory tree diameter) are null for shrubs. Cover is available. Polygons with cover less than 50\% are assigned to MD and polygons with cover greater than 50\% are assigned to LDO.

\subsection*{Draft Model}
\begin{figure}[htbp]
\centering
\includegraphics[width=0.8\textwidth]{/Users/mmallek/Tahoe/Report3/images/state_trans_model.pdf}
\caption{Generic state and transition model for all non-shrub seral cover types. Boxes show seven seral stages and arrows depict transitions due to vegetation succession and high or low mortality fire.} 
\label{lsg_transmodel}
\end{figure}

\clearpage
\subsection*{References}
\begin{hangparas}{.25in}{1} LandFire. ``Biophysical Setting Models.'' Biophysical Setting 0610790: Great Basin Xeric Mixed Sagebrush Shrubland. 2007. LANDFIRE Project, U.S. Department of Agriculture, Forest Service; U.S. Department of the Interior. \burl{http://www.landfire.gov/national\_veg\_models\_op2.php}. Accessed 9 November 2012.

Van de Water, Kip M. and Hugh D. Safford. ``A Summary of Fire Frequency Estimates for California Vegetation Before Euro-American Settlement.'' \emph{Fire Ecology} 7.3 (2011): 26-57. doi: 10.4996/fireecology.0703026.

Verner, Jared. ``Low Sage (LSG).'' \emph{A Guide to Wildlife Habitats of California}, edited by Kenneth E. Mayer and William F. Laudenslayer. California Deparment of Fish and Game, 1988. \burl{http://www.dfg.ca.gov/biogeodata/cwhr/pdfs/SGB.pdf}. Accessed 4 December 2012.
\end{hangparas}



% !TEX root = master.tex
\newpage
\section{Curl-leaf Mountain Mahogany (CMM)}

\subsection*{General Information}

\subsubsection{Cover Type Overview}

\textbf{Curl-leaf Mountain Mahogany (CMM)}
\newline
Crosswalks
\begin{itemize}
	\item EVeg: Regional Dominance Type 1
	\begin{itemize}
		\item Curl-leaf Mountain Mahogany
	\end{itemize}

	\item LandFire BpS Model
	\begin{itemize}
		\item 0610620: Inter-Mountain Basin Curl-leaf Mountan Mahogany Woodland and Shrubland
	\end{itemize}

	\item Presettlement Fire Regime Type
	\begin{itemize}
		\item Curl-leaf Mountain Mahogany
	\end{itemize}
\end{itemize}

\noindent Reviewed by Becky Estes, Central Sierra Province Ecologist, USDA Forest Service

\subsubsection{Vegetation Description}
This landcover type is characterized by the dominance or co-dominance of \emph{Cercocarpus ledifolius}. Other shrubs such as \emph{Artemisia}, \emph{Arctostaphylos}, \emph{Ceanothus}, and \emph{Ephedra} may be present. \emph{C. ledifolius} is both a primary early successional colonizer rapidly invading bare mineral soils after disturbance and the dominant long-lived species. Depending on the effects of a given fire on the seed bank, in some cases it could take 10 years to recolonize. Where \emph{C. ledifolius} has reestablished quickly after fire, \emph{Chrysothamnus nauseosus} may codominate. Litter and shading by woody plants inhibits the establishment of \emph{C. ledifolius}, particularly in late seral stages where canopy cover is high. Reproduction often appears more dependent upon geographic variables (slope, aspect, and elevation) than biotic factors. \emph{Artemisia arbuscula} and \emph{Artemisia nova} are infrequently associated. \emph{Symphoricarpos}, \emph{Amelanchier}, and \emph{Ribes} are present on cooler, moister sites. \emph{Pinus monophylla}, \emph{Juniperus}, \emph{Pseudotsuga menziesii}, \emph{Abies magnifica}, \emph{Abies concolor}, and \emph{Pinus jeffreyi} may have sporadic presence at very low densities. In older stands the understory may consist largely of \emph{Leptodactylon pungens} (LandFire 2007, Gucker 2006).

\subsubsection{Distribution}
\emph{C. ledifolius} communities are usually found on upper slopes and ridges between 2130 and 3200 m (7000-10,500 ft), although northern stands may occur as low as 600 m (200 ft). It is more common on northwestern and northeastern aspects. Most stands occur on rocky, shallow soils and outcrops, with mature stand cover from 10-55\%. In the absence of fire, old stands may occur on somewhat deeper soils, with more than 55\% cover (LandFire 2007).

\subsection*{Disturbances}

\subsubsection{Wildfire}
Wildfires tend to be high mortality, stand-replacing fires that initiate a process of post-fire forest succession. High mortality fires kill large as well as small trees, and may kill many of the shrubs and herbs as well, although below-ground organs of at least some individual shrubs and herbs survive and re-sprout. 

\emph{C. ledifolius} is easily killed by fire and does not resprout. However, it is a primary early successional colonizer, rapidly invading bare mineral soils after disturbance. Fires are not common in early seral stages, when there is little fuel, except in chaparral-dominated stands. Stand-replacing fires are more common in mid-seral stands, where herbs and smaller shrubs provide ladder fuels. When surface fire is relatively common, stands will adopt a savanna-like woodland structure with an understory characterized by \emph{Ribes}, \emph{L. pungens}, and various grasses. Trees can become very old and will rarely show fire scars. In late, closed stands, the absence of herbs and small forbs makes fire uncommon, requiring extreme winds and drought conditions. However, stands that do burn often experience high mortality fire (LandFire 2007).

Estimates of fire rotations are available from the LandFire project and a review paper (LandFire 2007, Van de Water and Safford 2011). The LandFire project's published fire return intervals are based on a series of associated models created using the Vegetation Dynamics Development Tool (VDDT). In VDDT, fires are specified concurrently with the transition that follows them. For example, a replacement fire causes a transition to the early development stage. In the RMLands model, such fires are classified as high mortality. However, in VDDT mixed severity fires may cause a transition to early development, a transition to a more open seral stage, or no transition at all. In this case, we categorize the first example as a high mortality fire, and the second and third examples as a low mortality fire. Based on this approach, we calculated fire rotations and the probability of high mortality fire for each of the three CMM seral stages (Table 1). We computed the overall target fire rotation of 76 years based on values from Van de Water and Safford (2011). 




\begin{table}[]
\small
\centering
\caption{Fire rotation (years) and proportion of high (versus low) mortality fires. Values were derived from VDDT model 0610790 (LandFire 2007) and Van de Water and Safford (2011). }
\label{tab:cmmdesc_fire}
\begin{tabular}{@{}lcc@{}}
\toprule
\textbf{Condition}         & \multicolumn{1}{l}{\textbf{Fire Rotation}} & \multicolumn{1}{l}{\textbf{\begin{tabular}[c]{@{}l@{}}Probability of \\ High Mortality\end{tabular}}} \\ \midrule
Target                     & 76  & n/a      \\
Early Development - All    & 83  & 0.17        \\
Mid Development - Moderate & 17  & 0.67        \\
Late Development - Closed  & 500  & 1      \\ \bottomrule
\end{tabular}
\end{table}

\subsubsection{Other Disturbance}
Other disturbances are not currently modeled, but may, depending on the seral stage affected and mortality levels, reset patches to early development, maintain existing seral stages, or shift/accelerate succession to a more open seral stage. 

\subsection*{Vegetation Seral Stages}
We recognize three separate seral stages for CMM: Early Development (ED), Mid Development - Moderate Canopy Cover (MDM), and Late Development - Closed Canopy Cover (LDC). Our seral stages are an alternative to ``successional'' classes that imply a linear progression of states and tend not to incorporate disturbance. The seral stages identified here are derived from a combination of successional processes and anthropogenic and natural disturbance, and are intended to represent a composition and structural condition that can be arrived at from multiple other conditions described for that landcover type. Thus our seral stages incorporate age, size, canopy cover, and vegetation composition. In general, the delineation of stages has originated from the LandFire biophysical setting model descriptive of a given landcover type; however, seral stages are not necessarily identical to the classes identified in those models.

\subsubsection{Early Development (ED)}

\paragraph{Description} \emph{C. ledifolius} seedlings rapidly invade bare mineral soils after fire. Litter and shading by woody plants inhibits establishment. Bunchgrasses and disturbance-tolerant forbs and resprouting shrubs, such as \emph{Symphoricarpos}, may be present. \emph{Ericameria} and \emph{Artemisia} seedlings are likely present. Vegetation composition will affect fire behavior, especially if chaparral species like \emph{Arctostaphylos} or \emph{Ceanothus} are present (LandFire 2007).

\paragraph{Succession Transition} In the absence of disturbance, patches in this seral stage will transition to MDM upon reaching 20 years of age. 

\paragraph{Wildfire Transition} High mortality wildfire (17\% of fires in this seral stage) recycles the patch through the ED seral stage. No transition occurs as a result of low mortality fire.

\noindent\hrulefill


\subsubsection{Mid Development - Moderate Canopy Cover (MDM)}

\paragraph{Description} \emph{C. ledifolius} may co-dominate with mature \emph{Artemisia}, \emph{Purshia}, \emph{Symphoricarpos}, or \emph{Ericameria}. Few \emph{C. ledifolius} seedlings are present. Canopy cover is variable (LandFire 2007).

\paragraph{Succession Transition} After 120 years in this stage, patches in this seral stage will transition to LDC.

\paragraph{Wildfire Transition} High mortality wildfire (67\% of fires in this seral stage) recycles the patch through the ED seral stage. No transition occurs as a result of low mortality fire.

\noindent\hrulefill


\subsubsection{Late Development - Closed Canopy Cover (LDC)}

\paragraph{Description} Moderate to high cover of large shrub- or tree-like \emph{C. ledifolius}. When low mortality fire is relatively frequent, late-successional \emph{C. ledifolius} may exhibit evidence of infrequent fire scars on older trees. Patches may consist of open savanna-like woodlands with an herbaceous-dominated understory. Other shrub species may be abundant, but decadent. When low mortality fire is absent, very few other shrubs are present, and herbaceous cover is low. Duff may be very deep, and scattered trees may occur. \emph{C. ledifolius} trees reach very old age in the absence of stand-replacing fire, potentially living over 1000 years (LandFire 2007).

\paragraph{Succession Transition} In the absence of disturbance, patches in this seral stage will remain in this seral stage. 

\paragraph{Wildfire Transition} High mortality wildfire (100\% of fires in this seral stage) recycles the patch through the ED seral stage.

\noindent\hrulefill

\subsection*{Condition Classification}
To create the initial cover and seral stage layer (2010), polygons were randomly assigned to seral stages based on a 20:10:70 distribution for early/mid/late development (based on an analysis of past fire in the project area). Random numbers between 0 and 1 were generated using numpy for Python and used to assign each CMM polygon to a seral stage.

\subsection*{Draft Model}
\begin{figure}[htbp]
\centering
\includegraphics[width=0.8\textwidth]{/Users/mmallek/Documents/Thesis/statetransmodel/StateTransitionModel/shrub.png}
\caption{State and Transition Model for Curl-leaf Mountain Mahogany. Each dark grey box represents one of the three seral stages for this landcover type. Three stages of development are represented: early, middle, and late. We describe the middle development stage as characterized by moderate canopy cover and the late development stage as characterized by closed canopy cover, but these are not hard and fast rules. Transitions between states/seral stages may occur as a result of high mortality fire, low mortality fire, or succession. Specific pathways for each are denoted by the appropriate color line and arrow: red lines relate to high mortality fire, orange lines relate to low mortality fire, and green lines relate to natural succession.} 
\label{cmm_transmodel}
\end{figure}

\clearpage
\subsection*{References}
\begin{hangparas}{.25in}{1} Gucker, Corey L. ``Cercocarpus ledifolius'' \emph{Fire Effects Information System}, U.S. Department of Agriculture, Forest Service, Rocky Mountain Research Station, Fire Sciences Laboratory, 2006.  \burl{http://www.fs.fed.us/database/feis/} [Accessed 29 July 2013.]. 

LandFire. ``Biophysical Setting Models.'' Biophysical Setting 0610790: Great Basin Xeric Mixed Sagebrush Shrubland. 2007. LANDFIRE Project, U.S. Department of Agriculture, Forest Service; U.S. Department of the Interior. \burl{http://www.landfire.gov/national\_veg\_models\_op2.php}. Accessed 9 November 2012.

Van de Water, Kip M. and Hugh D. Safford. ``A Summary of Fire Frequency Estimates for California Vegetation Before Euro-American Settlement.'' \emph{Fire Ecology} 7.3 (2011): 26-57. doi: 10.4996/fireecology.0703026.

\end{hangparas}


% !TEX root = master.tex
\newpage
\section{Lodgepole Pine (LPN)}

\subsection*{General Information}

\subsubsection{Cover Type Overview}

\textbf{Lodgepole Pine (LPN)}
\newline
Crosswalks
\begin{itemize}
	\item EVeg: Regional Dominance Type 1
	\begin{itemize}
		\item Lodgepole Pine
	\end{itemize}

	\item LandFire BpS Model
	\begin{itemize}
		\item 0610581 Sierra Nevada Subalpine Lodgepole Pine Forest and Woodland – Wet
		\item 0610582 Sierra Nevada Subalpine Lodgepole Pine Forest and Woodland – Dry

	\end{itemize}

	\item Presettlement Fire Regime Type
	\begin{itemize}
		\item Lodgepole Pine
	\end{itemize}
\end{itemize}

\paragraph{Lodgepole Pine with Aspen (LPN-ASP)}
This type is created by overlaying the NRIS TERRA Inventory of Aspen on top of the EVeg layer. Where it intersects with LPN it is assigned to LPN-ASP.

\noindent Reviewed by Shana Gross, Ecologist, USDA Forest Service

\subsubsection{Vegetation Description}
\paragraph{Lodgepole Pine (LPN)} \emph{P. contorta} ssp. \emph{murrayana} is the overwhelming dominant within its forest community, mixing occasionally with \emph{Abies magnifica}, and with scattered \emph{Pinus jeffreyi}  and \emph{Pinus monticola}, and \emph{Tsuga mertensiana} at higher elevations (Fites-Kaufman et al. 2007). Mature Sierran stands often contain significant seedlings and saplings. Understory characteristics are influenced by proximity to meadow and stream margins. \emph{Arctostaphylos} and \emph{Ribes} are common shrubs. Stands associated with meadow edges and streams may have a rich herbaceous layer consisting of grasses, forbs, and sedges. Species associations are likely very location specific. Plants present may include but are not limited to \emph{Cassiope}, \emph{Vaccinium}, \emph{Phyllodoce}, \emph{Kalmia}, \emph{Ceanothus}, \emph{Chrysolepis}, and \emph{Carex}. Elsewhere, the understory may be virtually absent, consisting of scattered shrubs such as \emph{Quercus vaccinifolia}, and herbs like \emph{Antennaria}, \emph{Arabis}, \emph{Eriogonum}, and \emph{Gayophytum}. Fast-moving streams within the cover type are generally characterized by relatively dense populations of \emph{Salix} (Bartolome 1988, Fites-Kaufman et al. 2007, LandFire 2007a, LandFire 2007b).  

\paragraph{Lodgepole Pine with Aspen (LPN-ASP)}	When \emph{Populus tremuloides} co-occurs with LPN on the west side of the Sierran crest, it is typically found in smaller patches, often less than 2 ha (5 acres) in size. Mature stands in which \emph{P. tremuloides} are still dominant are usually relatively open. Average canopy closures range from 60 to 100 percent in young and intermediate-aged stands and from 25 to 60 percent in mature stands. The open nature of the stands results in substantial light penetration to the ground (Verner 1988).

\subsubsection{Distribution}
\paragraph{Lodgepole Pine (LPN)}	Open stands of \emph{P. contorta} ssp. \emph{murrayana}, which make up a widespread upper montane forest/woodland, tolerating both rocky soils and semisaturated meadow edges, in an elevational belt within and above the A. magnifica zone. These forests, strongly dominated by \emph{P. contorta} ssp. \emph{murrayana}, generally occur at elevations of about 1,830 to 2,400 m (6000 to 7875 ft) in the northern Sierra Nevada. Stands of \emph{P. contorta} ssp. \emph{murrayana} may reach much lower, however, with cold air drainage down glacial canyons (Fites-Kaufman et al. 2007, Anderson 1996). On infertile soils, \emph{P. contorta} ssp. \emph{murrayana} is often the only tree species that will grow (Lotan and Critchfield 1990).
More than any other Sierran conifer, \emph{P. contorta} ssp. \emph{murrayana} is relatively tolerant of poor soil aeration, and thus grows well around the margins of wet meadows and other moist areas. Many upper montane and subalpine meadows in the Sierra Nevada exhibit invasion of young \emph{P. contorta} ssp. \emph{murrayana} moving inward from their drier margins. It is not clear how much this process has been influenced by changes in fire frequency or grazing over the last 150 years (Fites-Kaufman et al. 2007).

\paragraph{Lodgepole Pine with Aspen (LPN-ASP)}		Sites supporting \emph{P. tremuloides} are associated with added soil moisture, i.e., azonal wet sites. These sites are found throughout the LPN zone, often close to streams, lakes, and meadows. Other sites include rock reservoirs, springs and seeps. Terrain can be simple to complex (LandFire 2007c). 


\subsection*{Disturbances}

\subsubsection{Wildfire}

\paragraph{Lodgepole Pine (LPN)} 	Wildfires tend to be high mortality, stand-replacing fires that initiate a process of post-fire forest succession. High mortality fires kill large as well as small trees, and may kill many of the shrubs and herbs as well, although below-ground organs of at least some individual shrubs and herbs survive and resprout. Low mortality fires tend to only kill small seedlings and depend on the herbaceous layer to carry fire.

Unlike the Rocky Mountain subspecies of \emph{P. contorta} (ssp. \emph{latifolia}), \emph{P. contorta} ssp. \emph{murrayana} does not have serotinous cones (Fites-Kaufman et al. 2007). Following high mortality fire, it initially establishes in even-aged stands, but small-scale disturbances and the ability of the subspecies to regenerate in the absence of fire promote uneven-aged structure (Cope 1993, Gross 2013).

High mortality fire occurs at long intervals. Mixed severity fire is related to fire behavior across the often moist areas where \emph{P. contorta} ssp. \emph{murrayana} is found. Surface fires are more common on drier sites, although in general sparse fuels limit fire ignition and spread. Most fires are small (less than 1 ha) but very large fires covering hundreds of hectares do occur (LandFire 2007a, LandFire 2007b). This is due in part to the high susceptibility to fire mortality by \emph{P. contorta} ssp. \emph{murrayana} because of its thin bark and shallower roots. Postfire conditions provide an ideal seedbed, and \emph{P. contorta} ssp. \emph{murrayana} is an early post-fire colonizer (Cope 1993).

\paragraph{Lodgepole Pine with Aspen (LPN-ASP)}			Sites supporting \emph{P. tremuloides} are maintained by stand-replacing disturbances that allow regeneration from below-ground suckers. Upland clones are impaired or suppressed by conifer ingrowth and overtopping and intensive grazing that inhibits growth. In a reference condition scenario, a few stands will advance toward conifer dominance, but in the current landscape scenario where fire has been reduced from reference conditions there are many more conifer-dominated mixed aspen stands (LandFire 2007c, Verner 1988). 

Estimates of fire rotations for these variants are available from the LandFire project and a few review papers. The LandFire project’s published fire return intervals are based on a series of associated models created using the Vegetation Dynamics Development Tool (VDDT). In VDDT, fires are specified concurrently with the transition that follows them. For example, a replacement fire causes a transition to the early development stage. In the RMLands model, such fires are classified as high mortality. However, in VDDT mixed severity fires may cause a transition to early development, a transition to a more open seral stage, or no transition at all. In this case, we categorize the first example as a high mortality fire, and the second and third examples as a low mortality fire. Based on this approach, we calculated fire rotations and the probability of high mortality fire for each of the LPN and LPN-ASP seral stages (Tables~\ref{tab:lpndesc_fire} and \ref{tab:lpnaspdesc_fire}). We computed overall target fire rotations based on values from Mallek et al. (2013) and Van de Water and Safford (2011). 



\begin{table}[]
\centering
\caption{Fire rotation (years) and proportion of high (versus low) mortality fires for Lodgepole Pine type. Values were derived from VDDT model 0610790 (LandFire 2007), Mallek et al. (2013), and Estes (personal communication). }
\label{tab:lpndesc_fire}
\begin{tabular}{@{}lcc@{}}
\toprule
\textbf{Condition}          & \textbf{Fire Rotation} & \textbf{Probability of High Mortality} \\ \midrule
Target                      & 52    & n/a        \\
Early Development – All     & 29    & 0.03       \\
Mid Development – Closed    & 59    & 0.41       \\
Mid Development – Moderate  & 27    & 0.15       \\
Mid Development – Open      & 18    & 0.07       \\
Late Development – Closed   & 37    & 0.26       \\
Late Development – Moderate & 24    & 0.13       \\
Late Development – Open     & 18    & 0.07       \\ \bottomrule
\end{tabular}
\end{table}

\begin{table}[]
\centering
\caption{Fire rotation (years) and proportion of high (versus low) mortality fires for Lodgepole Pine – Aspen type. Values were derived from VDDT model 0610790 (LandFire 2007) and Van de Water and Safford (pers. comm. 2013).}
\label{tab:lpnaspdesc_fire}
\begin{tabular}{@{}lcc@{}}
\toprule
\textbf{Condition}               & \textbf{Fire Rotation} & \textbf{Probability of High Mortality} \\ \midrule
Target                           & 52     & n/a        \\
Early Development – Aspen        & 29     & 0.03       \\
Mid Development – Aspen          & 59     & 0.41       \\
Mid Development – Aspen-Conifer  & 27     & 0.15       \\
Late Development – Conifer-Aspen & 24     & 0.13       \\
Late Development – Closed        & 37     & 0.26       \\ \bottomrule
\end{tabular}
\end{table}

\subsubsection{Other Disturbance}
Other disturbances are not currently modeled, but may, depending on the seral stage affected and mortality levels, reset patches to early development, maintain existing seral stages, or shift/accelerate succession to a more open seral stage. 

\subsection*{Vegetation Seral Stages}
We recognize seven separate seral stages for LPN: Early Development (ED), Mid Development – Open Canopy Cover (MDO), Mid Development – Moderate Canopy Cover, Mid Development – Closed Canopy Cover (MDC), Late Development – Open Canopy Cover (LDO), Late Development – Moderate Canopy Cover (LDM), and Late Development – Closed Canopy Cover (LDC). The LPN-ASP variant is assigned to five seral stages: Early Development – Aspen (EDA), Mid Development – Aspen (MDA), Mid Development – Aspen with Conifer (MDAC), Late Development – Conifer with Aspen (LDCA), and Late Development – Closed Canopy Cover (LDC).

Our seral stages are an alternative to ``successional'' classes that imply a linear progression of states and tend not to incorporate disturbance. The seral stages identified here are derived from a combination of successional processes and anthropogenic and natural disturbance, and are intended to represent a composition and structural condition that can be arrived at from multiple other conditions described for that landcover type. Thus our seral stages incorporate age, size, canopy cover, and vegetation composition. In general, the delineation of stages has originated from the LandFire biophysical setting model descriptive of a given landcover type; however, seral stages are not necessarily identical to the classes identified in those models.


\subsubsection{Lodgepole Pine}

\paragraph{Early Development (ED)}

\paragraph{Description} Grasses, forbs, low shrubs, and sparse to moderate cover of trees (primarily \emph{P. contorta} ssp. \emph{murrayana}) seedlings/saplings with an open canopy. This seral stage is characterized by the recruitment of a new cohort of early successional, shade-intolerant tree species into an open area created by a stand-replacing disturbance. 


A short period of herbaceous productivity precedes closure of the tree canopy on productive sites. The prolific seed output, establishment, and seedling growth of \emph{P. contorta} ssp. \emph{murrayana} makes the period of herbaceous production short (Bartolome 1988). \emph{P. contorta} ssp. \emph{murrayana} regeneration density ranges from moderate to dog hair thickets (LandFire 2007a).


\paragraph{Succession Transition} In the absence of disturbance, patches in this seral stage will begin transitioning to MDC at 10 years at a rate of 0.6 per time step. At 40 years, all patches will succeed. On average, patches remain in early development for 18 years.

\paragraph{Wildfire Transition} High mortality wildfire (3\% of fires in this seral stage) recycles the patch through the Early Development seral stage. No transition occurs as a result of low mortality fire.

\noindent\hrulefill


\paragraph{Mid Development – Open Canopy Cover (MDO)}

\paragraph{Description} Sparse ground cover of grasses, forbs, and shrubs. Mid-maturity \emph{P. contorta} ssp. \emph{murrayana} where surface fire or other disturbance has opened the stand. Canopy cover ranges from 10-50\% (LandFire 2007a).
Continued recruitment into stands produces overstocking and slow growth of the overcrowded trees. This overcrowding may make them susceptible to insects, although others have argued that the more vigorously growing trees are more likely to be attacked. Beetle infestation creates large quantities of fuel that increase the probability of wildfire (Bartolome 1988).


\paragraph{Succession Transition} Patches in this seral stage may stay in this seral stage under low mortality disturbance, but after 10 years without fire they begin transitioning to MDM at a rate of 0.8 per time step. Succession to LDO occurs once the patch has been in mid development for 50 years. The rate of succession per time step is 0.5. At 100 years, all stands will succeed to LDO. On average, patches remain in mid development for 54 years.

\paragraph{Wildfire Transition} High mortality wildfire (7\% of fires in this seral stage) recycles the patch through the Early Development seral stage. Low mortality wildfire (93\%) maintains the patch in MDO.

\noindent\hrulefill

\paragraph{Mid Development – Moderate Canopy Cover (MDM)}

\paragraph{Description} Sparse ground cover of grasses, forbs, and shrubs. Mid-maturity \emph{P. contorta} ssp. \emph{murrayana} where surface fire or other disturbance has opened the stand. Canopy cover ranges from 10-50\% (LandFire 2007a).

Continued recruitment into stands produces overstocking and slow growth of the overcrowded trees. This overcrowding may make them susceptible to insects, although others have argued that the more vigorously growing trees are more likely to be attacked. Beetle infestation creates large quantities of fuel that increase the probability of wildfire (Bartolome 1988).


\paragraph{Succession Transition} Patches in this seral stage may stay in this seral stage under low mortality disturbance, but after 10 years without fire they begin transitioning to MDC at a rate of 0.8 per time step. Succession to LDM occurs once the patch has been in mid development for 45 years. The rate of succession per time step is 0.55. At 90 years, all stands will succeed to LDM.

\paragraph{Wildfire Transition} High mortality wildfire (15\% of fires in this seral stage) recycles the patch through the Early Development seral stage. Low mortality wildfire (85\%) triggers a transition to MDO 68\% of the time; otherwise, it remains in MDM.

\noindent\hrulefill

\paragraph{Mid Development – Closed Canopy Cover (MDC)}

\paragraph{Description} Sparse ground cover of grasses, forbs, and shrubs; mid-maturity \emph{P. contorta} ssp. \emph{murrayana} undergoing intrinsic stand thinning. Considerable surface fuel from tree mortality from previous fire. Canopy cover is greater than 50\% (LandFire 2007a).

Continued recruitment into stands produces overstocking and slow growth of the overcrowded trees. This overcrowding may make them susceptible to insects, although others have argued that the more vigorously growing trees are more likely to be attacked. Beetle infestation creates large quantities of fuel that increase the probability of wildfire. (Bartolome 1988).


\paragraph{Succession Transition} After 40 years in a MD seral stage without a wildfire-triggered transition, patches in this seral stage will begin transitioning to LDC. The rate of succession per time step is 0.6. At 80 years, all patches succeed to LDC.

\paragraph{Wildfire Transition} High mortality wildfire (41\% of fires in this seral stage) recycles the patch through the Early Development seral stage. Low mortality wildfire (59\%) triggers a transition to MDM.

\noindent\hrulefill


\paragraph{Late Development – Open Canopy Cover (LDO)}

\paragraph{Description} Areas that have experienced one or more low severity understory fires that had reduced stand density or old stands that have not experienced fire but have been thinned by other processes (tree falls, etc.). Stands are uneven aged. Canopy cover ranges from 10-50\% (LandFire 2007a).

\paragraph{Succession Transition} Patches in this seral stage may maintain under low mortality disturbance, but after 25 years without fire, these patches succeed to LDM at a rate of 0.7 per timestep.

\paragraph{Wildfire Transition} High mortality wildfire (7\% of fires in this seral stage) recycles the patch through the Early Development seral stage. Low mortality wildfire (93\%) maintains the patch in LDO.

\noindent\hrulefill

\paragraph{Late Development – Moderate Canopy Cover (LDM)}

\paragraph{Description} Sparse ground cover of grasses, forbs, and shrubs. Mid-maturity \emph{P. contorta} ssp. \emph{murrayana} where surface fire or other disturbance has opened the stand. Canopy cover ranges from 10-50\% (LandFire 2007a).

Continued recruitment into stands produces overstocking and slow growth of the overcrowded trees. This overcrowding may make them susceptible to insects, although others have argued that the more vigorously growing trees are more likely to be attacked. Beetle infestation creates large quantities of fuel that increase the probability of wildfire (Bartolome 1988).


\paragraph{Succession Transition} Patches in this seral stage may stay in this seral stage under low mortality disturbance, but after 25 years without fire they begin transitioning to LDC at a rate of 0.7 per time step. 

\paragraph{Wildfire Transition} High mortality wildfire (13\% of fires in this seral stage) recycles the patch through the Early Development seral stage. Low mortality wildfire (87\%) triggers a transition to LDO 73\% of the time; otherwise, it remains in LDM.

\noindent\hrulefill

\paragraph{Late Development – Closed Canopy Cover (LDC)}

\paragraph{Description} Old \emph{P. contorta} ssp. \emph{murrayana} stands where fire has had minimal influence. Canopy cover exceeds 50\%.

\paragraph{Succession Transition} This class will maintain in the absence of disturbance.

\paragraph{Wildfire Transition} High mortality wildfire (26\% of fires in this seral stage) recycles the patch through the Early Development seral stage. Low mortality wildfire (73\%) triggers a transition to LDM.

\noindent\hrulefill
\noindent\hrulefill

\subsubsection{Aspen Variant}

\paragraph{Early Development – Aspen (ED–A)}

\paragraph{Description} Grasses, forbs, low shrubs, and sparse to moderate cover of tree seedlings/saplings (primarily \emph{P. tremuloides}) with an open canopy. This seral stage is characterized by the recruitment of a new cohort of early successional, shade-intolerant tree species into an open area created by a stand-replacing disturbance. 

Following disturbance, succession proceeds rapidly from an herbaceous layer to shrubs and trees, which invade together (Verner 1988). \emph{P. tremuloides} suckers over 6ft tall develop within about 10 years (LandFire 2007c). 


\paragraph{Succession Transition} Unless it burns, a patch in the early seral stage persists for 10 years, at which point it transitions to MD-A.

\paragraph{Wildfire Transition} High mortality wildfire (3\% of fires in this seral stage) recycles the patch through the Early Development seral stage. No transition occurs as a result of low mortality fire.

\noindent\hrulefill


\paragraph{Mid Development – Aspen (MD–A)}

\paragraph{Description} \emph{P. tremuloides} trees 5-16'' DBH. Canopy cover is highly variable, and can range from 40-100\%. These patches range in age from 10 to 110 years. Some understory conifers, predominantly \emph{P. contorta} ssp. \emph{murrayana}, are encroaching, but \emph{P. tremuloides} is still the dominant component of the stand (LandFire 2007c).

\paragraph{Succession Transition} MD-A persists for at least 50 years in the absence of fire, after which patches begin transitioning to MD-AC at a rate of 0.6 per timestep. After 100 years all remaining MD-A patches transition to MD-AC. 

\paragraph{Wildfire Transition} High mortality wildfire (41\% of fires in this seral stage) recycles the patch through the Early Development seral stage. No transition occurs as a result of low mortality fire.

\noindent\hrulefill

\paragraph{Mid Development – Aspen with Conifer (MD–AC)}

\paragraph{Description} These stands have been protected from fire since the last stand-replacing disturbance. \emph{P. tremuloides} trees are predominantly 16'' DBH and greater. Conifers (predominantly \emph{P. contorta} ssp. \emph{murrayana}) are present and becoming increasingly dominant over the \emph{P. tremuloides}. Conifers are pole to medium-sized, and conifer cover is at least 40\% (LandFire 2007c).

\paragraph{Succession Transition} MD-AC persists for 100 years in the absence of high mortality fire, after which patches transition to LDC. 

\paragraph{Wildfire Transition} High mortality wildfire (15\% of fires in this seral stage) returns the patch to ED-A. Low mortality wildfire (85\%) maintains the patch in MD–AC.

\noindent\hrulefill

\paragraph{Late Development – Closed (LDC)}

\paragraph{Description} Some \emph{P. tremuloides} continue to be present in the understory, but large\emph{ P. contorta} ssp. \emph{murrayana} are now the dominant tree species, having overtopped the \emph{P. tremuloides}. Smaller conifers are present in the midstory as well (LandFire 2007a). This seral stage is analogous to the LDC seral stage for the LPN variant.

\paragraph{Succession Transition} Patches in this seral stage will maintain in the absence of disturbance.

\paragraph{Wildfire Transition} High mortality wildfire (26\% of fires in this seral stage) will return the patch to ED–A. Low mortality wildfire (74\%) opens the stand up to LD-CA.

\noindent\hrulefill


\paragraph{Late Development – Conifer with Aspen (LD–CA)}

\paragraph{Description} If stands are sufficiently protected from fire such that conifer species overtop \emph{P. tremuloides} and become large, they may be able to withstand some fire that more sensitive \emph{P. tremuloides} cannot. When this occurs, it creates a patch characterized by late development conifers, such as \emph{P. contorta} ssp. \emph{murrayana}, and early seral \emph{P. tremuloides}. 

\paragraph{Succession Transition} LD-CA persists for 70 years in the absence of any fire, at which point patches transition to LDC. 

\paragraph{Wildfire Transition} High mortality wildfire (13\% of fires in this seral stage) returns the patch to ED-A. Low mortality wildfire (87\%) maintains the stand in LD-CA. 

\noindent\hrulefill

\subsection*{Condition Classification}

\begin{table}[]
\small
\centering
\caption{Classification of seral stage for LPN. Diameter at Breast Height (DBH) and Cover From Above (CFA) values taken from EVeg polygons. DBH categories are: null, 0-0.9'', 1-4.9'', 5-9.9'', 10-19.9'', 20-29.9'', 30''+. CFA categories are null, 0-10\%, 10-20\%, … , 90-100\%. Each row in the table below should be read with a boolean AND across each column.}
\label{lpn_classification}
\begin{tabular}{@{}lrrrrr@{}}
\toprule
\textbf{\begin{tabular}[l]{@{}l@{}}Cover \\ Condition\end{tabular}} & \textbf{\begin{tabular}[r]{@{}r@{}}Overstory Tree \\ Diameter 1 \\ (DBH)\end{tabular}} & \textbf{\begin{tabular}[r]{@{}r@{}}Overstory Tree \\ Diameter 2 \\ (DBH)\end{tabular}} & \textbf{\begin{tabular}[r]{@{}r@{}}Total Tree\\ CFA (\%)\end{tabular}} & \textbf{\begin{tabular}[r]{@{}r@{}}Conifer \\ CFA (\%)\end{tabular}} & \textbf{\begin{tabular}[r]{@{}r@{}}Hardwood \\ CFA (\%)\end{tabular}} \\ \midrule
Early All & 0-4.9'' & any & any & any & any \\
Mid Open & 5-9.9'' & any & 0-40 & any & any \\
Mid Moderate & 5-9.9'' & any & 40-70 & any & any \\
Mid Closed & 5-9.9'' & any & 70-100 & any & any \\
Late Open & 10''+ & any & 0-40 & any & any \\
Late Moderate & 10''+ & any & 40-70 & any & any \\
Late Closed & 10''+ & any & 70-100 & any & any \\ \bottomrule
\end{tabular}
\end{table}

LPN-ASP seral stages were assigned manually using NAIP 2010 Color IR imagery to assess seral stage.

\subsection*{Draft Model}
\begin{figure}[htbp]
\centering
\includegraphics[width=0.8\textwidth]{/Users/mmallek/Tahoe/Report3/images/state_trans_model.pdf}
\caption{State and Transition Model for Lodgepole Pine Forest (not inclusive of the aspen variant). Each dark grey box represents one of the seven seral stages for this landcover type. Each column of boxes represents a stage of development: early, middle, and late. Each row of boxes represents a different level of canopy cover: closed (70-100\%), moderate (40-70\%), and open (0-40\%). Transitions between states/seral stages may occur as a result of high mortality fire, low mortality fire, or succession. Specific pathways for each are denoted by the appropriate color line and arrow: red lines relate to high mortality fire, orange lines relate to low mortality fire, and green lines relate to natural succession.} 
\label{transmodel_lpn}
\end{figure}

\clearpage
\subsection*{References}
\begin{hangparas}{.25in}{1} 
Bartolome, James W. ``Lodgepole Pine (LPN).'' \emph{A Guide to Wildlife Habitats of California}, edited by Mayer, Kenneth E. and William F. Laudenslayer. California Deparment of Fish and Game. 1988. \burl{http://www.dfg.ca.gov/biogeodata/cwhr/pdfs/LPN.pdf}. Accessed 4 December 2012.

``CalVeg Zone 1.'' Vegetation Descriptions. \emph{Vegetation Classification and Mapping}.  11 December 2008. U.S. Forest Service. \burl{http://www.fs.usda.gov/Internet/FSE\_DOCUMENTS/fsbdev3\_046448.pdf}. Accessed 2 April 2013.

Cope, Amy B. 1993. ``Pinus contorta var. murrayana.'' In: Fire Effects Information System, [Online].  U.S. Department of Agriculture, Forest Service,  Rocky Mountain Research Station, Fire Sciences Laboratory (Producer).  \burl{http://www.fs.fed.us/database/feis/} [Accessed 4 December 2012].

Fites-Kaufman, Jo Ann, Phil Rundel, Nathan Stephenson, and Dave A. Wixelman. ``Montane and Subalpine Vegetation of the Sierra Nevada and Cascade Ranges.'' In \emph{Terrestrial Vegetation of California, 3rd Edition}, edited by Michael Barbour, Todd Keeler-Wolf, and Allan A. Schoenherr, 456-501. Berkeley and Los Angeles: University of California Press, 2007. 

Gross, Shana. Ecologist, USDA Forest Service. Personal communication, 3 July 2013.

Lotan, James E. and William B. Critchfield. ``Lodgepole Pine.'' Russell M. Burns and Barbara H. Honkala, tech. coords. Silvics of North America, vol 1. Conifers; Glossary. Agriculture handbook no.654. Washington, D.C.: U.S. Dept. of Agriculture, Forest Service, 1990. 

LandFire. ``Biophysical Setting Models.'' Biophysical Setting 0610581: Sierra Nevada Subalpine Lodgepole Pine Forest and Woodland. 2007a. LANDFIRE Project, U.S. Department of Agriculture, Forest Service; U.S. Department of the Interior. \burl{http://www.landfire.gov/national\_veg\_models\_op2.php}. Accessed 9 November 2012.

LandFire. ``Biophysical Setting Models.'' Biophysical Setting 0610582: Sierra Nevada Subalpine Lodgepole Pine Forest and Woodland. 2007b. LANDFIRE Project, U.S. Department of Agriculture, Forest Service; U.S. Department of the Interior. \burl{http://www.landfire.gov/national\_veg\_models\_op2.php}. Accessed 9 November 2012.

LandFire. ``Biophysical Setting Models.'' Biophysical Setting 0610610: Inter-Mountain Basins Aspen-Mixed Conifer Forest and Woodland. 2007c. LANDFIRE Project, U.S. Department of Agriculture, Forest Service; U.S. Department of the Interior. \burl{http://www.landfire.gov/national\_veg\_models\_op2.php}. Accessed 7 January 2013.

Safford, Hugh S. Regional Ecologist, USDA Forest Service. Personal communication, 5 May 2013.

Skinner, Carl N. and Chi-Ru Chang. ``Fire Regimes, Past and Present.'' \emph{Sierra Nevada Ecosystem Project: Final report to Congress, vol. II, Assessments and scientific basis for management options}. Davis: University of California, Centers for Water and Wildland Resources, 1996.

Van de Water, Kip M. and Hugh D. Safford. ``A Summary of Fire Frequency Estimates for California Vegetation Before Euro-American Settlement.'' \emph{Fire Ecology} 7.3 (2011): 26-57. doi: 10.4996/fireecology.0703026.

Verner, Jared. ``Aspen (ASP).'' ).'' \emph{A Guide to Wildlife Habitats of California}, edited by Kenneth E. Mayer and William F. Laudenslayer. California Deparment of Fish and Game, 1988. \burl{http://www.dfg.ca.gov/biogeodata/cwhr/pdfs/ASP.pdf}. Accessed 4 December 2012.

\end{hangparas}


% !TEX root = master.tex
\newpage
\section{Mixed Evergreen Forest (MEG)}

\subsection*{General Information}

\subsubsection{Cover Type Overview}

\paragraph{Mixed Evergreen Forest (MEG)}

Crosswalks
\begin{itemize}
	\item EVeg: Regional Dominance Type 1
	\begin{itemize}
		\item Interior Mixed Hardwood
		\item California Bay
		\item Canyon Live Oak
		\item Madrone
		\item Bigleaf Maple
		\item Interior Live Oak
		\item Montane Mixed Hardwood 
		\item Pacific Douglas Fir
		\item Tanoak
	\end{itemize}

	\item EVeg: Regional Dominance Type 2
	\begin{itemize}
		\item Tanoak (regardless of RD Type 1 value, and therefore inclusive of all potential Type 1 vegetation types)
	\end{itemize}

	\item LandFire BpS Model
	\begin{itemize}
		\item 0610430 Mediterranean California Mixed Evergreen Forest
	\end{itemize}

	\item Presettlement Fire Regime Type
	\begin{itemize}
		\item Mixed Evergreen Forest
	\end{itemize}
\end{itemize}

\paragraph{Mesic Modifer (MEG\_M)}
This type is created by intersecting a binary xeric/mesic layer with the existing vegetation layer. MEG cells that intersect with mesic cells are assigned to the mesic modifier.
\paragraph{Xeric Modifer (MEG\_X)}
This type is created by intersecting a binary xeric/mesic layer with the existing vegetation layer. MEG cells that intersect with xeric cells are assigned to the xeric modifier.
\paragraph{Ultramafic Modifer (MEG\_U)}
This type is created by intersecting an ultramafic soils/geology layer with the existing vegetation layer. Where ultramafic cells intersect with MEG they are assigned to the ultramafic modifier.

Reviewed by Kyle Merriam, Sierra-Cascade Province Ecologist, USDA Forest Service; Becky Estes, Central Sierra Province Ecologist, USDA Forest Service


\subsubsection{Vegetation Description}
\paragraph{Mixed Evergreen Forest (MEG)} 	This landcover type forms a complex mosaic of forest due to the geologic, topographic, and successional variation typical within its range. This type is characterized by a combination of coniferous and broadleaved trees. Characteristic trees include Pseudotsuga menziesii, Quercus chrysolepis, Notholithocarpus densiflorus,\footnote{Tan oak was known as Lithocarpus densiflorus for over 90 years before botanists renamed it Notholithocarpus densiflorus in 2008 (Manos et al. 2008). Some sources and database continue to use the old name and plant symbol.}  Arbutus menziesii, Umbellularia californica, and Chrysolepis chrysophylla. Species composition is primarily determined by the environmental gradients of temperature and moisture availability. Quercus kelloggii is found on drier sites on inland portion of the range. Pinus lambertiana and Pinus ponderosa can be present in this type. These stands tend to have dense or diverse shrub understories with Ceanothus, Corylus, Gaultheria, Morella, Rhododendron, Ribes, Rubus, Toxicodendron diversilobum, and Vaccinium. Grass species include Bromus, Festuca, and Hierochloe. Polystichum munitum and Pteridium aquilinum var. pubescens sometimes grow abundantly. Carex spp. are present in some places (LandFire 2007, McDonald 1988, Tappeiner 1990).

\begin{adjustwidth}{2cm}{}
\textbf{Mesic Modifer (MEG\_M)}
Deep mesic soils support aggregations that include a lower or midstory layer of dense, sclerophyllous, broad-leaved evergreen trees like N. densiflorus and Arbutus menziesii, with an irregular, often open, higher layer of tall needle-leaved evergreen trees, typically P. menziesii. A small number of pole and sapling trees occur throughout stands. On wetter sites, shrub layers are well developed, often with 100\% cover. Cover of the herbaceous layer under the shrubs can be up to 10 percent. At higher elevations, the shrubs disappear and the herb layer is often 100\%. Diversity of tree size typically increases with stand age, along with tree spacing. Young stands have closely spaced and uniformly distributed trees, whereas older stands have a more patchy stem distribution. Snags and downed logs, an important structural component of this habitat, increase in density or volume with stand age (Raphael 1988). Potential additional conifer associates include Abies concolor, Pinus lambertiana, Calocedrus decurrens, and Pinus ponderosa (Tappeiner 1990). A large variety of shrubs, forbs, grasses, sedges, and ferns, along with N. densiflorus sprouts, can become aggressive on burned or cutover areas. This is especially true in areas where high severity fires have locally eliminated conifer seed sources (Tappeiner 1990).

\medskip
\noindent \textbf{Xeric Modifer (MEG\_X)}
A pronounced hardwood tree layer is typical, with an infrequent and poorly developed shrub stratum, and a sparse herbaceous layer (McDonald 1988). Characteristic oaks include Q. chrysolepis, Q. wislizeni, Q. kelloggi, and Quercus garryana. Q. chrysolepis and Q. wislizeni are the most common oaks in the project area. They may individually form almost pure stands on steep canyon slopes and rocky ridgetops throughout the Sierra Nevada, or co-occur. They have tremendously variable growth forms, ranging from shrubs with multiple trunks on rocky, steep slopes, to magnificently spreading tall trees on deeper soils in moister areas. Both are evergreen with dense canopies (Allen-Diaz et al. 2007). Tree spacing is close (3-4 m) on better sites, and wider (8-10 m) on poor sites. In general, snags and downed woody material are sparse. Lower elevation associates are Pinus sabiniana, Pinus attenuata, N. densiflorus, A. menziesii, Quercus wislizeni, C. chrysophylla, and scrubby U. californica (McDonald 1988).

\medskip
\noindent \textbf{Ultramafic Modifer (MEG\_U)}
Notholithocarpus densiflorus var. echinoides, or dwarf tanoak, growns on ultramafic and other less productive sites (Estes 2013). It is unclear if the 2 varieties differ genetically or if the small stature of dwarf tanoak is due to unproductive site conditions. Ecology literature does not usually distinguish between the 2 infrataxa (Fryer 2008). However, its identification is pertinent to management decisions. While N. lithocarpus is generally protected as an oak species, the dwarf variety may be classified as a shrub and therefore subject to treatment or removal. Typically, P. menziesii attains less dominance and may replaced by open stands of various conifers, such as Pinus ponderosa, Pinus sabiniana, or Pinus jeffreyi. Trees occur within a generally open grassland or shrubland. The shrub layer is likely to include Quercus vaccinifolia, N. densiflorus, U. californica, Quercus breweri, and Rhamnus. Common grasses include Stipa, Festuca, and Danthonia (LandFire 2007b, McDonald 1988, O’Geen et al. 2007, Raphael 1988). 

\end{adjustwidth}

\subsubsection{Distribution}
\paragraph{Mixed Evergreen Forest}		This highly variable cover type occurs in the Sierra Nevada on all aspects at elevations of 350 m (1150 ft) to over 1700 m (5575 ft) (LandFire 2007a). Soil depth classes range from shallow to deep. The large number of species in the type, both conifer and hardwood, allow it to occupy and persist in a wide range of environments. Good soils and poor, steep slopes and slight, frequently disturbed and pristine all are at least adequate habitats for one or more species (McDonald 1988).

A xeric-mesic gradient was developed based on four variables: 1) aspect, 2) potential evapotranspiration, 3) topographic wetness index, and 4) soil water storage. The variables were standardized by z-score such that higher values correspond to more mesic environments. Thus, potential evapotranspiration was inverted to maintain this balance. The four variables were combined with equal weights. This final variables was split into xeric vs. mesic, with xeric occupying the negative end of the range up to $\frac{1}{4}$ standard deviation below the mean (zero) and mesic occupying the remaining portion of the spectrum.

\begin{adjustwidth}{2cm}{}
\textbf{Mesic Modifer }
Soils are deep, well-drained, and loamy, sandy, or gravelly. Found in valleys, coves, ravines, along streams, and on north as well as east slopes. It typically occurs in areas that are cool and moist sites in areas where precipitation is highest most likely in the form of rain and snow.

\medskip
\noindent \textbf{Xeric Modifer}
Q. chrysolepis and associates are found on a wide range of slopes, especially those that are moderate to steep. Soils are for the most part rocky, alluvial, coarse textured, poorly developed, and well drained. 

\medskip
\noindent \textbf{Ultramafic Modifer} Ultramafics have been mapped at various spatial densities throughout the elevational range of the textbf landcover type. Low to moderate elevations in ultramafic and serpentinized areas often produce soils low in essential minerals like calcium potassium, and nitrogen, and have excessive accumulations of heavy metals such as nickel and chromium. These sites vary widely in the degree of serpentinization and effects on their overlying plant communities (``CalVeg Zone 1'' 2011). Note, the terms ``ultramafic rock'' and ``serpentine'' are broad terms used to describe a number of different but related rock types, including serpentinite, peridotite, dunite, pyroxenite, talc and soapstone, among others (O’Geen et al. 2007). 

\end{adjustwidth}

%%%

\subsection*{Disturbances}
\subsubsection{Wildfire}

\paragraph{Mixed Evergreen Forest}		Fire is the dominant disturbance event. Wildfires are common and frequent; mortality depends on vegetation vulnerability and wildfire intensity. Low mortality fires kill small trees and may consume above-ground portions of small oaks, shrubs and herbs, but do not kill large trees or below-ground organs of most oaks, shrubs and herbs which promptly resprout. High mortality fires kill trees of all sizes and may kill many of the shrubs and herbs as well. However, high mortality fires typically kill only the above ground portions of the oaks, shrubs and herbs; consequently, most oaks, shrubs and herbs promptly resprout from surviving below ground organs.

The vast majority of fires occur in late summer or early fall and are associated with lightning storms. Native American burns locally increased the frequency and may have been extensive prior to 1850. However, research also suggests that fire frequencies actually increased after European settlement (Merriam, pers. comm. 2013). Fires in the past were often large in area due to the high number of ignition points associated with fire events, and created patches of varying age and species composition (LandFire 2007a). 

Hardwoods typically provide the greatest cover after fire due to root-crown sprouting. Depending upon fire severity many hardwoods may have epicormic sprouting well into the crown. Species composition, density and interspecific competition within stands contributes to multiple pathways following disturbance. If fire has been absent from an area for an extended period of time, some conifers may be able to establish and persist even with the return of frequent low severity fire. But, if low severity fire is frequent after a stand-replacing fire, conifers will be more or less excluded and hardwoods will dominate (LandFire 2007a).

Estimates of fire rotations for these variants are available from the LandFire project and a few review papers. The LandFire project’s published fire return intervals are based on a series of associated models created using the Vegetation Dynamics Development Tool (VDDT). In VDDT, fires are specified concurrently with the transition that follows them. For example, a replacement fire causes a transition to the early development stage. In the RMLands model, such fires are classified as high mortality. However, in VDDT mixed severity fires may cause a transition to early development, a transition to a more open condition, or no transition at all. In this case, we categorize the first example as a high mortality fire, and the second and third examples as a low mortality fire. Based on this approach, we calculated fire rotations and the probability of high mortality fire for each of the MEG condition classes across the three variants (Tables 1–3). We computed overall target fire rotations based on expert input from Safford and Estes, values from Mallek et al. (2013), and Van de Water and Safford (2011). 

\begin{adjustwidth}{2cm}{}
\textbf{Mesic Modifer }
N. densiflorus is adapted to ignite easily. In the lower montane zone of the Sierra Nevada where N. densiflorus occurs, the historic fire regime was characterized by dormant season fires of mostly low to moderate severity (Tappeiner 1990). In stands with high N. densiflorus cover, N. densiflorus may dominate the stand for many years before conifers re-establish. Patchy, stand-replacement fires were most common on north-facing slopes and during extended droughts. N. densiflorus seedlings and saplings are typically top-killed by even low severity surface fire. Large trees usually survive moderate-severity fire, bearing fire scars afterward. Even N. densiflorus with thick bark (3-10 cm) typically sustain bole damage from fire. Relative to associated conifers, mature P. menziesii is fairly resistant to surface fires. Crown fires cause extensive mortality (Tappeiner 1990).

\medskip
\noindent \textbf{Xeric Modifer} Q. chrysolepis has loose, dead, flaky bark that catches fire readily and burns intensely. Occasional fire often changes a stand of Q. chrysolepis to Q. wislizeni¬–chaparral, but without fire for sufficient time, trees again develop. Where fire is frequent, this oak becomes scarce or even drops out of the montane hardwood community (McDonald 1988).

\medskip
\noindent \textbf{Ultramafic Modifer} Historically, these woodland types had frequent low-severity fire (Fire Regime I). However, now there is higher susceptibility to stand replacing fire because of fire exclusion.

\end{adjustwidth}

%%%


\begin{table}[]
\small
\centering
\caption{Fire rotation (years) and proportion of high (versus low) mortality fires for Mixed Evergreen Forest – Mesic. Values were derived from VDDT model 0610790 (LandFire 2007a) and Safford and Estes (personal communication). }
\label{tab:megmdesc_fire}
\begin{tabular}{@{}lcc@{}}
\toprule
\textbf{Condition}         & \multicolumn{1}{l}{\textbf{Fire Rotation}} & \multicolumn{1}{l}{\textbf{\begin{tabular}[c]{@{}l@{}}Probability of \\ High Mortality\end{tabular}}} \\ \midrule
Target                      & 50            & n/a                           \\
Early Development – All     & 68            & 1                             \\
Mid Development – Closed    & 46            & 0.11                          \\
Mid Development – Moderate  & 26            & 0.11                          \\
Mid Development – Open      & 18            & 0.11                          \\
Late Development – Closed   & 44            & 0.21                          \\
Late Development – Moderate & 25            & 0.11                          \\
Late Development – Open     & 17            & 0.11   						\\ \bottomrule
\end{tabular}
\end{table}

\begin{table}[]
\small
\centering
\caption{Fire rotation (years) and proportion of high (versus low) mortality fires for Mixed Evergreen Forest – Xeric. Values were derived from VDDT model 0610790 (LandFire 2007a), and Safford and Estes (personal communication). }
\label{tab:megxdesc_fire}
\begin{tabular}{@{}lcc@{}}
\toprule
\textbf{Condition}         & \multicolumn{1}{l}{\textbf{Fire Rotation}} & \multicolumn{1}{l}{\textbf{\begin{tabular}[c]{@{}l@{}}Probability of \\ High Mortality\end{tabular}}} \\ \midrule
Target                      & 40            & n/a     \\
Early Development – All     & 85            & 1       \\
Mid Development – Closed    & 39            & 0.10    \\
Mid Development – Moderate  & 22            & 0.10    \\
Mid Development – Open      & 15            & 0.10    \\
Late Development – Closed   & 37            & 0.10    \\
Late Development – Moderate & 21            & 0.10    \\
Late Development – Open     & 15            & 0.03 	  \\ \bottomrule
\end{tabular}
\end{table}

\begin{table}[]
\small
\centering
\caption{Fire rotation (years) and proportion of high (versus low) mortality fires for Mixed Evergreen Forest – Ultramafic. Values were derived from VDDT model 0711700 (LandFire 2007b), and Safford and Estes (personal communication). }
\label{tab:megudesc_fire}
\begin{tabular}{@{}lcc@{}}
\toprule
\textbf{Condition}         & \multicolumn{1}{l}{\textbf{Fire Rotation}} & \multicolumn{1}{l}{\textbf{\begin{tabular}[c]{@{}l@{}}Probability of \\ High Mortality\end{tabular}}} \\ \midrule
Target                      & 50            & n/a                           \\
Early Development – All     & 68            & 1                             \\
Mid Development – Closed    & 46            & 0.11                          \\
Mid Development – Moderate  & 26            & 0.11                          \\
Mid Development – Open      & 18            & 0.11                          \\
Late Development – Closed   & 44            & 0.21                          \\
Late Development – Moderate & 25            & 0.11                          \\
Late Development – Open     & 17            & 0.11   						\\ \bottomrule
\end{tabular}
\end{table}

%%%

\subsubsection{Other Disturbance}
Other disturbances are not currently modeled, but may, depending on the condition affected and mortality levels, reset patches to early development, maintain existing condition classes, or shift/accelerate succession to a more open condition. All of the tree species associated with this vegetation type are susceptible to a wide variety of pathogens and insects (such as sudden oak death for N. densiflorus, which is caused by the pathogen Phytophthora ramorum).

\subsection*{Vegetation Seral Stages}
We recognize seven separate condition classes for MEG: Early Development (ED), Mid Development – Open Canopy Cover (MDO), Mid Development – Moderate Canopy Cover, Mid Development – Closed Canopy Cover (MDC), Late Development – Open Canopy Cover (LDO), Late Development – Moderate Canopy Cover (LDM), and Late Development – Closed Canopy Cover (LDC). We use condition classes not in the sense of fire regime condition classes, but as an alternative to ``successional'' classes that imply a linear progression of states and tend not to incorporate disturbance. The condition classes identified here are derived from a combination of successional processes and anthropogenic and natural disturbance, and are intended to represent a composition and structural condition that can be arrived at from multiple other conditions described for that landcover type. Thus our condition classes incorporate age, size, canopy cover, and vegetation composition as well as relative seral stages. In general, the delineation of stages has originated from the LandFire biophysical setting model descriptive of a given landcover type; however, condition classes are not necessarily identical to the classes identified in those models. 

\paragraph{Early Development (ED)}

\paragraph{Description} This condition is characterized by the diversity of species establishing and reestablishing into an open area created by a stand-replacing disturbance. 

\begin{adjustwidth}{2cm}{}
\textbf{Mesic Modifer } On mesic sites, abundant grasses, forbs, low shrubs, found under sparse to moderate cover of trees (primarily P. menziesii and N. densiflorus) seedlings/saplings with an open canopy. Seedling establishment of P. menziesii following fire is dependent on the spacing and number of surviving seed trees. Seedling establishment following large stand-replacing fires may be slow if seed trees are killed over extensive areas. Or, if there are numerous, well-spaced surviving seed trees within the burned area, a new cohort of seedlings can quickly establish (Uchytil 1991). Nearly all N. densiflorus burls sprout after fire, and survivorship is high. Q. chrysolepis, if present, also sprouts readily, and shrubs such as Mahonia, Gaultheria, and Rhododendron may be significant. Shrub growth from seed banks, e.g. Ceanothus integerrimus, can also be high (LandFire 2007a). Thus, N. densiflorus and other shrubs usually dominante the initial condition if P. menziesii isn’t able to seed in quickly (Raphael 1988).

\medskip
\noindent \textbf{Xeric Modifer}  On xeric sites, grasses, forbs, low shrubs, and sparse cover of tree seedlings and saplings are found under an open canopy. Forest openings contain a dense cover of hardwood sprouts. Sprouting shrubs such as M. aquifolium, Gaultheria shallon, and Rhododendron may be significant. Shrub growth from seed banks, e.g. Ceanothus integerrimus, can also be high (LandFire 2007a). 


\medskip
\noindent \textbf{Ultramafic Modifer}  On ultramafic sites, P. menziesii may be stunted and slow-growing, and N. densiflorus var. echinoides may be present. Grasses like Festuca, Danthonia, and Acnatherum, or else chaparral shrubs establish. Scattered Pinus ponderosa, Pinus sabiniana, or Pinus jeffreyi may also be present (LandFire 2007b).

\end{adjustwidth}

\paragraph{Succession Transition}
\begin{adjustwidth}{2cm}{}
\textbf{Mesic and Xeric Modifer } In the absence of disturbance, patches in this condition class will begin transitioning to MDM at 20 years. The rate of succession per time step is 0.8. At 40 years, all patches will succeed. On average, patches remain in ED for 26 years.


\medskip
\noindent \textbf{Ultramafic Modifer} Succession may be delayed. Thus, in the absence of disturbance, patches in this condition will begin transitioning to MDM after 30 years and may be delayed in the ED condition for as long as 80 years. A patch in this condition succeeds at a rate of 0.4 per time step. On average, patches remain in ED for 43 years.

\end{adjustwidth}

\paragraph{Wildfire Transition} High mortality wildfire (100\% of fires in this condition) recycles the patch through the ED condition. Low mortality wildfire is not modeled for this condition class.

\noindent\hrulefill


\paragraph{Mid Development – Open Canopy Cover (MDO)}

\paragraph{Description}
\begin{adjustwidth}{2cm}{}
\textbf{Mesic Modifer } Sparse ground cover of grasses, forbs, and shrubs; open tree canopy cover (primarily P. menziesii and N. densiflorus). Other Quercus and Arctostaphylos species may also be present. In this stage, hardwoods are dominant, but P. menziesii and possibly other conifers are established or establishing under the predominantly N. densiflorus canopy (LandFire 2007a, McDonald 1988). 


\medskip
\noindent \textbf{Xeric Modifer}  Sparse ground cover of grasses, forbs, and shrubs; open tree canopy cover, primarily hardwoods such as Q. chrysolepis and Q. kelloggii. Conifers such as P. menziesii are present at low densities in emergent status. The shrub understory is still a significant presence (LandFire 2007a). 


\medskip
\noindent \textbf{Ultramafic Modifer}  Ultramafic sites are characterized by open P. menziesii, Pinus ponderosa, Pinus sabiniana, or Pinus jeffreyi stands with an understory comprised of N. densiflorus var. echinoides or Q. chrysolepis as well as grasses, forbs, and shrubs (LandFire 2007b).

\end{adjustwidth}
\paragraph{Succession Transition}
\begin{adjustwidth}{2cm}{}
\textbf{Mesic and Xeric Modifer } Patches in this condition may stay in this condition under low mortality disturbance, but after 15 years without fire they begin transitioning to MDM at a rate of 0.8 per time step. After 20 years in a mid development stage, patches in this condition will begin transitioning to LDO. The rate of succession per time step is 0.8. At 40 years, all patches succeed. On average, patches remain in the mid development stage for 26 years.


\medskip
\noindent \textbf{Ultramafic Modifer}  Succession may be delayed. Thus, in the absence of low mortality disturbance, patches in the MDO condition will begin transitioning to MDM after 20 years in MDO at a rate of 0.7 per timestep. Patches in this condition will begin transitioning to LDO after 30 years in a mid development stage, and may be delayed in this stage for as long as 80 years. A patch in this condition succeeds at a rate of 0.4 per time step. On average, patches remain in the mid development stage for 43 years.

\end{adjustwidth}
\paragraph{Wildfire Transition}
\begin{adjustwidth}{2cm}{}
\textbf{Mesic Modifer } High mortality wildfire (11\% of fires in this condition) recycles the patch through the ED condition. Low mortality wildfire (89\%) does not effect a change in the MDO condition.


\medskip
\noindent \textbf{Xeric Modifer}  High mortality wildfire (10\% of fires in this condition) recycles the patch through the ED condition. Low mortality wildfire (90\%) does not effect a change in the MDO condition.


\medskip
\noindent \textbf{Ultramafic Modifer} High mortality wildfire (11\% of fires) recycles the patch through the ED condition. Low mortality wildfire (89\%) does not effect a change in the MDO condition.

\end{adjustwidth}

\noindent\hrulefill

\paragraph{Mid Development – Moderate Canopy Cover (MDM)}

\paragraph{Description}
\begin{adjustwidth}{2cm}{}
\textbf{Mesic Modifer } Sparse ground cover of grasses, forbs, and shrubs; moderate tree canopy cover (primarily P. menziesii and N. densiflorus). Other Quercus and Arctostaphylos species may also be present. In this stage, hardwoods are dominant, but P. menziesii and possibly other conifers are established or establishing under the predominantly N. densiflorus canopy (LandFire 2007a, McDonald 1988). 


\medskip
\noindent \textbf{Xeric Modifer} Sparse ground cover of grasses, forbs, and shrubs; moderate tree canopy cover, primarily hardwoods such as Q. chrysolepis and Q. kelloggii. Conifers such as P. menziesii are present at low densities in emergent status. The shrub understory is still a significant presence (LandFire 2007a). 


\medskip
\noindent \textbf{Ultramafic Modifer}  Ultramafic sites are characterized by open P. menziesii, Pinus ponderosa, Pinus sabiniana, or Pinus jeffreyi stands with an understory comprised of N. densiflorus var. echinoides or Q. chrysolepis as well as grasses, forbs, and shrubs (LandFire 2007b).

\end{adjustwidth}
\paragraph{Succession Transition}
\begin{adjustwidth}{2cm}{}
\textbf{Mesic and Xeric Modifer } Patches in this condition may stay in this condition under low mortality disturbance, but after 15 years without fire they begin transitioning to MDC at a rate of 0.8 per time step. After 20 years in a mid development stage, patches in this condition will begin transitioning to LDM. The rate of succession per time step is 0.8. At 40 years, all patches succeed. On average, patches remain in the mid development stage for 26 years.


\medskip
\noindent \textbf{Ultramafic Modifer} Succession may be delayed. Thus, in the absence of low mortality disturbance, patches in the MDM condition will begin transitioning to MDC after 20 years in MDM at a rate of 0.7 per timestep. Patches in this condition will begin transitioning to LDM after 30 years in a mid development stage, and may be delayed in this stage for as long as 80 years. A patch in this condition succeeds at a rate of 0.4 per time step. On average, patches remain in the mid development stage for 43 years.

\end{adjustwidth}
\paragraph{Wildfire Transition}
\begin{adjustwidth}{2cm}{}
\textbf{Mesic Modifer } High mortality wildfire (11\% of fires in this condition) recycles the patch through the ED condition. Low mortality wildfire (89\%) triggers a transition to MDM 14\% of the time; otherwise, it remains in MDC.

\medskip
\noindent \textbf{Xeric Modifer} High mortality wildfire (10\% of fires in this condition) recycles the patch through the ED condition. Low mortality wildfire (90\%) triggers a transition to MDM 14\% of the time; otherwise, it remains in MDC.

\medskip
\noindent \textbf{Ultramafic Modifer} High mortality wildfire (11\% of fires) recycles the patch through the ED condition. Low mortality wildfire (89\%) triggers a transition to MDM 13\% of the time; otherwise, it remains in MDC.

\end{adjustwidth}
\noindent\hrulefill

\paragraph{Mid Development – Closed Canopy Cover (MDC)}

\paragraph{Description}
\begin{adjustwidth}{2cm}{}
\textbf{Mesic Modifer } Sparse ground cover of grasses, forbs, and shrubs; closed tree canopy cover (primarily P. menziesii and N. densiflorus). Other Quercus and Arctostaphylos species may also be present. In this stage, hardwoods are dominant, but P. menziesii and possibly other conifers are established or establishing under the predominantly N. densiflorus canopy (LandFire 2007a, McDonald 1988). 

\medskip
\noindent \textbf{Xeric Modifer} Sparse ground cover of grasses, forbs, and shrubs; closed tree canopy cover, primarily hardwoods such as Q. chrysolepis and Q. kelloggii. Conifers such as P. menziesii are present at low densities in emergent status. The shrub understory is still a significant presence (LandFire 2007a). 

\medskip
\noindent \textbf{Ultramafic Modifer} Ultramafic sites are characterized by open P. menziesii, Pinus ponderosa, Pinus sabiniana, or Pinus jeffreyi stands with an understory comprised of N. densiflorus var. echinoides or Q. chrysolepis as well as grasses, forbs, and shrubs (LandFire 2007b).

\end{adjustwidth}
\paragraph{Succession Transition}
\begin{adjustwidth}{2cm}{}
\textbf{Mesic and Xeric Modifer } After 20 years in a mid development stage, patches in this condition will begin transitioning to LDC. The rate of succession per time step is 0.8. At 40 years, all patches succeed. On average, patches remain in the mid development stage for 26 years.

\medskip
\noindent \textbf{Ultramafic Modifer} Succession may be delayed. Patches in this condition will begin transitioning to LDC after 30 years in a mid development stage, and may be delayed in this stage for as long as 80 years. A patch in this condition succeeds at a rate of 0.4 per time step. On average, patches remain in the mid development stage for 43 years.

\end{adjustwidth}
\paragraph{Wildfire Transition}
\begin{adjustwidth}{2cm}{}
\textbf{Mesic Modifer } High mortality wildfire (11\% of fires in this condition) recycles the patch through the ED condition. Low mortality wildfire (89\%) triggers a transition to MDM 22\% of the time; otherwise, it remains in MDC.

\medskip
\noindent \textbf{Xeric Modifer} High mortality wildfire (10\% of fires in this condition) recycles the patch through the ED condition. Low mortality wildfire (90\%) triggers a transition to MDM 20\% of the time; otherwise, it remains in MDC.

\medskip
\noindent \textbf{Ultramafic Modifer} High mortality wildfire (11\% of fires) recycles the patch through the ED condition. Low mortality wildfire (89\%) triggers a transition to MDM 22\% of the time; otherwise, it remains in MDC.

\end{adjustwidth}
\noindent\hrulefill


\paragraph{Late Development – Open Canopy Cover (LDO)}

\paragraph{Description}
\begin{adjustwidth}{2cm}{}
\textbf{Mesic Modifer } Overstory of large and very large trees, primarily P. menziesii. Canopy cover less than 40\%. P. lambertiana also occurs. N. densiflorus is tolerant of both full sun and shade, and usually dominates the subcanopy at this stage. Co-dominance of the upper canopy with P. menziesii is uncommon but possible after extended periods without disturbance (Uchytil 1991, LandFire 2007a). There is also some evidence that the senescence of late development N. densiflorus may cause openings in the canopy and allow for continued P. menziesii dominance (Estes pers. comm. 2013). Quercus and Arctostaphylos species may also be present in the sub-canopy (LandFire 2007a).

\medskip
\noindent \textbf{Xeric Modifer}  Overstory of large and very large trees, often with canopy cover less than 40\%. P. menziesii, Q. chrysolepis, and Arctostaphylos mewukka may occur. Conifers are taller and larger than in MD and clearly form the upper canopy layer here. Shrubs persist in openings but those in shade are likely to begin senescing (LandFire 2007a). On ultramafic sites, large Pinus ponderosa may additionally be present. Grass savannah persists on sites experiencing low intensity fire (with Festuca, Achnatherum, and Danthonia). Where fire is less frequent, chaparral shrubland develops (with Arctostaphylos and Quercus breweri) (LandFire 2007b).

\medskip
\noindent \textbf{Ultramafic Modifer} On ultramafic sites, large Pinus ponderosa, Pinus sabiniana, or Pinus jeffreyi may be present along with P. menziesii and N. densiflorus var. echinoides. Grass savannah persists on sites experiencing low intensity fire (with Festuca, Achnatherum, and Danthonia). Where fire is less frequent, chaparral shrubland develops (with Arctostaphylos and Quercus breweri) (LandFire 2007b).

\end{adjustwidth}
\paragraph{Succession Transition}
\begin{adjustwidth}{2cm}{}
\textbf{Mesic and Xeric Modifer } Patches in this condition may stay in this condition under low mortality disturbance, but after 15 years without fire they begin transitioning to LDM at a rate of 0.8 per time step. 

\medskip
\noindent \textbf{Ultramafic Modifer} Succession may be delayed. Thus, in the absence of low mortality disturbance, patches in the LDO condition will begin transitioning to LDM after 20 years in LDO at a rate of 0.7 per timestep. 

\end{adjustwidth}
\paragraph{Wildfire Transition}
\begin{adjustwidth}{2cm}{}
\textbf{Mesic Modifer } High mortality wildfire (11\% of fires in this condition) recycles the patch through the ED condition. Low mortality wildfire (89\%) does not effect a change in the MDO condition. 

\medskip
\noindent \textbf{Xeric Modifer} High mortality wildfire (3\% of fires in this condition) recycles the patch through the ED condition. Low mortality wildfire (97\%) does not effect a change in the MDO condition.

\medskip
\noindent \textbf{Ultramafic Modifer} High mortality wildfire (11\% of fires) recycles the patch through the ED condition. Low mortality wildfire (89\%) does not effect a change in the MDO condition.

\end{adjustwidth}
\noindent\hrulefill

\paragraph{Late Development – Moderate Canopy Cover (LDM)}

\paragraph{Description}
\begin{adjustwidth}{2cm}{}
\textbf{Mesic Modifer } Overstory of large and very large trees, primarily P. menziesii. Canopy cover between 40\% and 60\%. P. lambertiana also occurs. N. densiflorus is tolerant of both full sun and shade, and usually dominates the subcanopy at this stage. Co-dominance of the upper canopy with P. menziesii is uncommon but possible after extended periods without disturbance (Uchytil 1991, LandFire 2007a). There is also some evidence that the senescence of late development N. densiflorus may cause openings in the canopy and allow for continued P. menziesii dominance (Estes pers. comm. 2013). Quercus and Arctostaphylos species may also be present in the sub-canopy (LandFire 2007a).

\medskip
\noindent \textbf{Xeric Modifer} Overstory of large and very large trees, often with canopy cover between 40\% and 60\%. P. menziesii, Q. chrysolepis, and Arctostaphylos mewukka may occur. Conifers are taller and larger than in MD and clearly form the upper canopy layer here. Shrubs persist in openings but those in shade are likely to begin senescing (LandFire 2007a). On ultramafic sites, large Pinus ponderosa may additionally be present. Grass savannah persists on sites experiencing low intensity fire (with Festuca, Achnatherum, and Danthonia). Where fire is less frequent, chaparral shrubland develops (with Arctostaphylos and Quercus breweri) (LandFire 2007b).

\medskip
\noindent \textbf{Ultramafic Modifer} On ultramafic sites, large Pinus ponderosa, Pinus sabiniana, or Pinus jeffreyi may be present along with P. menziesii and N. densiflorus var. echinoides. Grass savannah persists on sites experiencing low intensity fire (with Festuca, Achnatherum, and Danthonia). Where fire is less frequent, chaparral shrubland develops (with Arctostaphylos and Quercus breweri) (LandFire 2007b).

\end{adjustwidth}
\paragraph{Succession Transition}
\begin{adjustwidth}{2cm}{}
\textbf{Mesic and Xeric Modifer } Patches in this condition may stay in this condition under low mortality disturbance, but after 15 years without fire they begin transitioning to LDC at a rate of 0.8 per time step. 

\medskip
\noindent \textbf{Ultramafic Modifer} Succession may be delayed. Thus, in the absence of low mortality disturbance, patches in the LDM condition will begin transitioning to LDC after 20 years in LDM at a rate of 0.7 per timestep. 

\end{adjustwidth}
\paragraph{Wildfire Transition}
\begin{adjustwidth}{2cm}{}
\textbf{Mesic Modifer } High mortality wildfire (11\% of fires in this condition) recycles the patch through the ED condition. Low mortality wildfire (89\%) triggers a transition to LDO 17\% of the time; otherwise, it remains in LDM.

\medskip
\noindent \textbf{Xeric Modifer} High mortality wildfire (10\% of fires in this condition) recycles the patch through the ED condition. Low mortality wildfire (90\%) triggers a transition to LDO 15\% of the time; otherwise, it remains in LDM.

\medskip
\noindent \textbf{Ultramafic Modifer}  High mortality wildfire (11\% of fires) recycles the patch through the ED condition. Low mortality wildfire (89\%) triggers a transition to LDO 17\% of the time; otherwise, it remains in LDM.

\end{adjustwidth}
\noindent\hrulefill

\paragraph{Late Development – Closed Canopy Cover (LDC)}

\paragraph{Description}
\begin{adjustwidth}{2cm}{}
\textbf{Mesic Modifer } Overstory of large and very large trees, primarily P. menziesii. Canopy cover exceeds 70\%. P. lambertiana also occurs. N. densiflorus is tolerant of both full sun and shade, and usually dominates the subcanopy at this stage. Co-dominance of the upper canopy with P. menziesii is uncommon but possible after extended periods without disturbance (Uchytil 1991, LandFire 2007a). There is also some evidence that the senescence of late development N. densiflorus may cause openings in the canopy and allow for continued P. menziesii dominance (Estes pers. comm. 2013). Quercus and Arctostaphylos species may also be present in the sub-canopy (LandFire 2007a).

\medskip
\noindent \textbf{Xeric Modifer} Overstory of large and very large trees, often with canopy cover over 70\%. P. menziesii, Q. chrysolepis, and Arctostaphylos mewukka may occur. Conifers are taller and larger than in MD and clearly form the upper canopy layer here. Shrubs persist in openings but those in shade are likely to begin senescing (LandFire 2007a). On ultramafic sites, large Pinus ponderosa may additionally be present. Grass savannah persists on sites experiencing low intensity fire (with Festuca, Achnatherum, and Danthonia). Where fire is less frequent, chaparral shrubland develops (with Arctostaphylos and Quercus breweri) (LandFire 2007b).

\medskip
\noindent \textbf{Ultramafic Modifer} On ultramafic sites, large Pinus ponderosa, Pinus sabiniana, or Pinus jeffreyi may be present along with P. menziesii and N. densiflorus var. echinoides. Grass savannah persists on sites experiencing low intensity fire (with Festuca, Achnatherum, and Danthonia). Where fire is less frequent, chaparral shrubland develops (with Arctostaphylos and Quercus breweri) (LandFire 2007b).

\end{adjustwidth}

\paragraph{Succession Transition}
\begin{adjustwidth}{2cm}{}
\textbf{Mesic, Xeric, and Ultramafic Modifer } In the absence of disturbance, patches in this condition will remain in this condition. 

\end{adjustwidth}

\paragraph{Wildfire Transition}
\begin{adjustwidth}{2cm}{}
\textbf{Mesic Modifer } High mortality wildfire (21\% of fires in this condition) recycles the patch through the ED condition. Low mortality wildfire (79\%) triggers a transition to LDM 26\% of the time; otherwise, it remains in LDC.

\medskip
\noindent \textbf{Xeric Modifer} High mortality wildfire (21\% of fires in this condition) recycles the patch through the ED condition. Low mortality wildfire (79\%) triggers a transition to LDM 24\% of the time; otherwise, it remains in LDC.

\medskip
\noindent \textbf{Ultramafic Modifer} High mortality wildfire (11\% of fires) recycles the patch through the ED condition. Low mortality wildfire (79\%) triggers a transition to LDM 26\% of the time; otherwise, it remains in LDC.

\end{adjustwidth}
\noindent\hrulefill

\subsection*{Seral Stage Classification}
\begin{table}[]
\small
\centering
\caption{Classification of seral stage for MEG. Diameter at Breast Height (DBH) and Cover From Above (CFA) values taken from EVeg polygons. DBH categories are: null, 0-0.9'', 1-4.9'', 5-9.9'', 10-19.9'', 20-29.9'', 30''+. CFA categories are null, 0-10\%, 10-20\%, … , 90-100\%. Each row in the table below should be read with a boolean AND across each column.}
\label{meg_classification}
\begin{tabular}{@{}lrrrrr@{}}
\toprule
\textbf{\begin{tabular}[l]{@{}l@{}}Cover \\ Condition\end{tabular}} & \textbf{\begin{tabular}[r]{@{}r@{}}Overstory Tree \\ Diameter 1 \\ (DBH)\end{tabular}} & \textbf{\begin{tabular}[r]{@{}r@{}}Overstory Tree \\ Diameter 2 \\ (DBH)\end{tabular}} & \textbf{\begin{tabular}[r]{@{}r@{}}Total Tree\\ CFA (\%)\end{tabular}} & \textbf{\begin{tabular}[r]{@{}r@{}}Conifer \\ CFA (\%)\end{tabular}} & \textbf{\begin{tabular}[r]{@{}r@{}}Hardwood \\ CFA (\%)\end{tabular}} \\ \midrule
Early All        & 0-4.9''         & any & any    & any & any \\
Mid Open         & 5-19.9''        & any & 0-40   & any & any \\
Mid Moderate     & 5-19.9''        & any & 40-70  & any & any \\
Mid Closed       & 5-19.9''        & any & 70-100 & any & any \\
Late Open        & 20-40''+        & any & 0-40   & any & any \\
Late Moderate    & 20-40''+        & any & 40-70  & any & any \\
Late Closed      & 20-40''+        & any & 70-100 & any & any \\ \bottomrule
\end{tabular}
\end{table}

\subsection*{Draft Model}
\begin{figure}[htbp]
\centering
\includegraphics[width=0.8\textwidth]{/Users/mmallek/Tahoe/Report3/images/state_trans_model.pdf}
\caption{State and Transition Model for Mixed Evergreen Forest. Each dark grey box represents one of the seven condition classes for this landcover type. Each column of boxes represents a stage of development: early, middle, and late. Each row of boxes represents a different level of canopy cover: closed (70-100\%), moderate (40-70\%), and open (0-40\%). Transitions between states/condition classes may occur as a result of high mortality fire, low mortality fire, or succession. Specific pathways for each are denoted by the appropriate color line and arrow: red lines relate to high mortality fire, orange lines relate to low mortality fire, and green lines relate to natural succession.} 
\label{meg_transmodel}
\end{figure}

\clearpage
\subsection*{References}
\begin{hangparas}{.25in}{1} 
Allen-Diaz, Barbara, Richard Standiford, and Randall D. Jackson. ``Oak Woodlands and Forests.'' In Terrestrial Vegetation of California, 3rd Edition, edited by Michael Barbour, Todd Keeler-Wolf, and Allan A. Schoenherr, 313-338. Berkeley and Los Angeles: University of California Press, 2007. 

``CalVeg Zone 1.'' Vegetation Descriptions. Vegetation Classification and Mapping.  11 December 2008. U.S. Forest Service. \burl{http://www.fs.usda.gov/Internet/FSE\_DOCUMENTS/fsbdev3\_046448.pdf}. Accessed 2 April 2013.

Estes, Becky, Province Ecologist, USDA Forest Service. Personal communication, 15 August 2013 and 3 September 2013.

LandFire. ``Biophysical Setting Models.'' Biophysical Setting 0610430: Mediterranean California Mixed Evergreen Forest. 2007a. LANDFIRE Project, U.S. Department of Agriculture, Forest Service; U.S. Department of the Interior. \burl{http://www.landfire.gov/national\_veg\_models\_op2.php}. Accessed 9 November 2012.

LandFire. ``Biophysical Setting Models.'' Biophysical Setting 0711700: Klamath-Siskiyou Xeromorphic Serpentine Savanna and Chaparral. 2007b. LANDFIRE Project, U.S. Department of Agriculture, Forest Service; U.S. Department of the Interior. \burl{http://www.landfire.gov/national\_veg\_models\_op2.php}. Accessed 30 November 2012.

Mallek, Chris, Hugh Safford, Joshua Viers, and Jay Miller. ``Modern departures in fire severity and area vary by forest type, Sierra Nevada and southern Cascades, California, USA.'' Ecosphere 4.12 (2013): art153. doi: http://www.esajournals.org/doi/pdf/10.1890/ES13-00217.1. 

Manos, P. S., C. H. Cannon, and S. H. Oh. ``Phylogenetic relationships and taxonomic status of the paleoendemic Fagaceae of western North America: recognition of a new genus, Notholithocarpus.'' Madroño 55.3 (2008): 181–190. doi: 10.3120/0024-9637-55.3.181

McDonald, Philip M. ``Montane Hardwood (MHW).'' A Guide to Wildlife Habitats of California, edited by Kenneth E. Mayer and William F. Laudenslayer. California Deparment of Fish and Game, 1988. \burl{http://www.dfg.ca.gov/biogeodata/cwhr/pdfs/MHW.pdf}. Accessed 4 December 2012.

Merriam, Kyle. Province Ecologist, USDA Forest Service. Personal communication, 9 July 2013.

O’Geen, Anthony T., Randy A. Dahlgren, and Daniel Sanchez-Mata. ``California Soils and Examples of Ultramafic Vegetation.'' In Terrestrial Vegetation of California, 3rd Edition, edited by Michael Barbour, Todd Keeler-Wolf, and Allan A. Schoenherr, 71-106. Berkeley and Los Angeles: University of California Press, 2007. 

Raphael, Martin G. ``Douglas-Fir (DFR).'' A Guide to Wildlife Habitats of California, edited by Kenneth E. Mayer and William F. Laudenslayer. California Deparment of Fish and Game, 1988. \burl{http://www.dfg.ca.gov/biogeodata/cwhr/pdfs/DFR.pdf}. Accessed 4 December 2012.

Safford, Hugh. Regional Ecologist, USDA Forest Service. Personal communication, 15 August 2013.

Skinner, Carl N. and Chi-Ru Chang. ``Fire Regimes, Past and Present.'' Sierra Nevada Ecosystem Project: Final report to Congress, vol. II, Assessments and scientific basis for management options. Davis: University of California, Centers for Water and Wildland Resources, 1996.

Tappeiner, John C., Philip M. McDonald, Douglass F. Roy. ``Tanoak.'' Silvics of North America: 2. Hardwoods. Agriculture Handbook 654. Burns, Russell M., and Barbara H. Honkala, tech. cords. U.S. Department of Agriculture, Forest Service, 1990. \burl{http://www.na.fs.fed.us/spfo/pubs/silvics\_manual/volume\_2/quercus/chrysolepis.htm}. Accessed 7 December 2012.

Uchytil, Ronald J. ``Pseudotsuga menziesii var. menziesii''.  Fire Effects Information System, U.S. Department of Agriculture, Forest Service,  Rocky Mountain Research Station, Fire Sciences Laboratory, 1991. \burl{http://www.fs.fed.us/database/feis/plants/tree/quekel/all.html}. Accessed 21 December 2012.

Van de Water, Kip M. and Hugh D. Safford. ``A Summary of Fire Frequency Estimates for California Vegetation Before Euro-American Settlement.'' Fire Ecology 7.3 (2011): 26-57. doi: 10.4996/fireecology.0703026.
\end{hangparas}


