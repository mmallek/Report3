\appendix


\chapter{Input Layers to \textsc{RMLands}}
\label{app:inputs}

\section{Technical details on \textsc{RMLands} software}
\label{app:rmlspecs}

\paragraph{Data Structure} \textsc{RMLands} uses raster GeoTiffs (.tif files) as its data structure. Rasters are based on uniform square units called cells (or pixels). Each cell represents an actual portion of geographic space. In this application, we use the Universal Transverse Mercator (UTM) projection, Zone 10 North. The extent of the raster is rectangular although the area of study is not. Cells outside of the buffered project area are assigned a null value.\footnote{Latitude and longitude are commonly pictured when describing coordinates. In such cases the X value refers to longitude and Y refers to latitude. However, because we use UTMs in this project, the correct convention is actually that the X value is the Easting and the Y value is the Northing. For simplicity we discuss X and Y only in this document.} In the Yuba River watershed landscape, each grid cell is 30 meters on a side (i.e., 900 m$^2$ or 0.09 ha), and the input grid measures 2910 by 2245 pixels. \textsc{RMLands} requires that all input grids are perfectly aligned. We accomplished this by setting the Extent and Snap Raster to the same parameters whenever we manipulated the layers in ArcMap. This ``base'' spatial layer was created by taking the primary elevation layer used on the Tahoe National Forest, resampling it to a 30 meter grid, and clipping its extent to match that of the buffered project area. Each cell is assigned a single class value, where valid class values are positive non-zero integers. Integer values are mapped to more descriptive class names using csv files with names identical to the grid name. All grids are created in ArcMap and saved as GeoTiff files before being loaded into to the model. 

\section{Input Layer Maps}
\label{app:sec:inputmaps}

% cover layer
\begin{figure}[!htbp]
\centering
\includegraphics[height=0.4\textheight]{/Users/mmallek/Tahoe/Report2/images/cover.png}
\caption{Cover Type Map for the project area. Also shows the 10 km buffer from the project area boundary. See Table~\ref{covertable} for full land cover type names.}
%\label{covermap}
\end{figure}

% condition layer
\begin{figure}[!htbp]
\centering
\includegraphics[height=0.4\textheight]{/Users/mmallek/Tahoe/Report2/images/condition.png}
\caption{Condition Class Map for the project area. Also shows the 10 km buffer from the project area boundary.} 
\label{conditionmap}
\end{figure}

% age layer
\begin{figure}[htbp]
\centering
\includegraphics[height=0.4\textheight]{/Users/mmallek/Tahoe/Report2/images/age.png}
\caption{Age map at Timestep 0 for the project area. Also shows the 10 km buffer from the project area boundary.} 
\label{agemap}
\end{figure}

% condition-age layer
\begin{figure}[htbp]
\centering
\includegraphics[height=0.4\textheight]{/Users/mmallek/Tahoe/Report2/images/condage.png}
\caption{Condition-Age map at Timestep 0 for the project area. Also shows the 10 km buffer from the project area boundary.} 
\label{condagemap}
\end{figure}

% tpi 
\begin{figure}[htbp]
\centering
\includegraphics[height=0.4\textheight]{/Users/mmallek/Tahoe/Report2/images/tpi.png}
\caption{Topographic Position Index for the project area. Also shows the 10 km buffer from the project area boundary.} 
\label{tpimap}
\end{figure}

% elevation layer
\begin{figure}[htbp]
\centering
\includegraphics[height=0.4\textheight]{/Users/mmallek/Tahoe/Report2/images/elevation.png}
\caption{Elevation for the project area. Also shows the 10 km buffer from the project area boundary.} 
\label{elevationmap}
\end{figure}

% slope layer
\begin{figure}[htbp]
\centering
\includegraphics[height=0.4\textheight]{/Users/mmallek/Tahoe/Report2/images/slope.png}
\caption{Slope for the project area, which ranges from flat to 126\%. Also shows the 10 km buffer from the project area boundary.} 
\label{slopemap}
\end{figure}

% aspect layer
\begin{figure}[htbp]
\centering
\includegraphics[height=0.4\textheight]{/Users/mmallek/Tahoe/Report2/images/aspect.png}
\caption{Aspect for the project area. Also shows the 10 km buffer from the project area boundary.} 
\label{aspectmap}
\end{figure}

% streams layer
\begin{figure}[htbp]
\centering
\includegraphics[height=0.4\textheight]{/Users/mmallek/Tahoe/Report2/images/streams.png}
\caption{Streams in the project area. Also shows the 10 km buffer from the project area boundary.} 
\label{streamsmap}
\end{figure}

%%%%%%%%%%%%%%%%%%%%%%%%%%%%%%%%%%%%%%%%%%%%%%%%%%%%%%%%%%%%%%%%%%%%%%%%%%%%%%%%%%%%%%%%%%%%%%%%%%%%%%%%%%%%%%%%%%%%%%%%%%%%%%%%%%%%%%%%%%%%%%%%%%%%%%%%%%%%%%%%%%%%%%%%%%%%%%%%%%%%%%%%%%%%%%%%%%%%%%%%%%%%%%%%%%%%%%%%%%%%%%%%%%%%%%%%%%%%%%%%%%%%%%%%%%%%%%%%%%%%%%%%%%%%%%%%%%%%%%%%%%%%%%%%%%%%%%%%%%%%%%%%%%%%%%%%%%%%%%%%%%%%%%%%%%%%%%%%%%%%%%%%%%%%%%%%%%%%%%%%%%%%%%%%%%%%%%%%%%%%%%%%%%%%%%%%
\chapter{Cover Type Descriptions}
\label{sec:covertypedesc}






