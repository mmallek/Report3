% !TEX root = master.tex

\section{Big Sagebrush (SAGE)}
\label{sage-description}
\subsection{General Information}

\subsubsection{Cover Type Overview}

\paragraph{Big Sagebrush (SAGE)}

Crosswalks
\begin{itemize}
	\item EVeg: Regional Dominance Type 1
	\begin{itemize}
		\item Bitterbrush 
		\item Basin Sagebrush
		\item Great Basin Mixed Scrub
		\item Bitterbrush – Sagebrush
	\end{itemize}

	\item LandFire BpS Model
	\begin{itemize}
		\item 0610800 Inter-Mountain Basins Big Sagebrush Shrubland
	\end{itemize}

	\item Presettlement Fire Regime Type
	\begin{itemize}
		\item Big Sagebrush
	\end{itemize}
\end{itemize}

Reviewed by Michele Slaton, GIS Specialist, Inyo National Forest, USDA Forest Service

\subsubsection{Vegetation Description}
The Big Sagebrush landcover type is typified by large, open, discontinuous stands of Artemisia tridentata of fairly uniform height. A. tridentata tends to have a single short, thick, stem that branches into a nearly globular crown (Neal 1988). Ericameria nauseosa is a frequent associate or co-dominant (LandFire 2007).

Shrub canopy cover generally ranges from very open, widely spaced, small plants to large, closely spaced plants with canopies touching. Cover may be greater at higher elevations and in areas receiving more precipitation. In addition to a deep root system, A. tridentata has a well-developed system of lateral roots near the soil surface (LandFire 2007, Neal 1988). Consequently, well-established sagebrush plants exclude most other shrubs in an area up to three times their crown area. Forbs and graminoids are often more abundant beneath these crowns (Slaton pers. comm. 2013). This produces stands of shrubs of very uniform size and spacing (Neal 1988).

Often the habitat is composed of pure stands of A. tridentata, but many stands include other species of Artemisia, Ericameria, Tetradymia, Ribes, Prunus, Cercocarpus, and Purshia. In communities not fully occupied by Artemisia, various amounts of herbaceous understory are found. Perennial forb cover is usually less than 10\% with perennial grass cover reaching 20-25\% on the more productive sites. Pseudoroegneria spicata may be a dominant species following replacement fires and a co-dominant after 20 years. Elymus elymoides and Oryzopsis hymenoides are common on more xeric sites. Festuca, Stipa, Poa, and Leymus are among the more common grasses. Percent cover and species richness of understory are determined by site limitations. Pinus monophylla and Juniperus osteosperma may be present, especially in areas protected from fire (Neal 1988, LandFire 2007).


\subsubsection{Distribution}
This widespread system is common to the Basin and Range province. It ranges in elevation from 900 m to 2450+ m (3000 ft - 8000+ ft) and occurs on well-drained soils on foothills, terraces, slopes, and plateaus. It is found on deeper soils (LandFire 2007).

\subsection{Disturbances}


\subsubsection{Wildfire}
Wildfires tend to be high mortality, stand-replacing fires that initiate a process of post-fire forest succession. High mortality fires kill large as well as small shrubs, and may kill many of the forbs and grasses as well, although below-ground organs of at least some individual shrubs and herbs survive and re-sprout. 

Replacement fires generally occur where shrub canopy exceeds 25\% cover, or where grass cover is greater than 15\% and shrub cover is greater than 20\%. Surface fires occur in areas dominated by grasses but are otherwise uncommon (LandFire 2007). A tridentata does not sprout after burning but most of the other shrubs common to the type do (Neal 1988). For the last several decades, post-settlement converstion to Bromus tectorum has become common and results in changes to fire frequency and vegetation dynamics. Extended periods of fire suppression or absence can lead to P. monophylla-J. osteosperma encroachment and subsequent decline of other shrubs and herbaceous plants (LandFire 2007). 

Estimates of fire rotations are available from the LandFire project and a review paper (LandFire 2007, Van de Water and Safford 2011). The LandFire project’s published fire return intervals are based on a series of associated models created using the Vegetation Dynamics Development Tool (VDDT). In VDDT, fires are specified concurrently with the transition that follows them. For example, a replacement fire causes a transition to the early development stage. In the RMLands model, such fires are classified as high mortality. However, in VDDT mixed severity fires may cause a transition to early development, a transition to a more open condition, or no transition at all. In this case, we categorize the first example as a high mortality fire, and the second and third examples as a low mortality fire. Based on this approach, we calculated fire rotations and the probability of high mortality fire for each of the three SAGE condition classes (Table~\ref{tab:sagedesc_fire}). 

\subsubsection{Other Disturbance}
Other disturbances are not currently modeled, but may, depending on the condition affected and mortality levels, reset patches to early development, maintain existing condition classes, or shift/accelerate succession to a more open condition. 

\begin{table}[]
\small
\centering
\caption{Fire rotations (years) and probability of high versus low mortality fires. Values were derived from BpS model 0610800 (LandFire 2007), Van de Water and Safford (2011), and Safford (pers. comm. 2013).}
\label{tab:sagedesc_fire}
\begin{tabular}{@{}lcc@{}}
\toprule
\textbf{Condition}         & \multicolumn{1}{l}{\textbf{Fire Rotation}} & \multicolumn{1}{l}{\textbf{\begin{tabular}[c]{@{}l@{}}Probability of \\ High Mortality\end{tabular}}} \\ \midrule
Target                     & 115      & n/a       \\
Early Development – All    & 200      & 0         \\
Mid Development – Moderate & 125      & 1         \\
Late Development – Closed  & 100      & 0.9       \\ \bottomrule
\end{tabular}
\end{table}



\subsection{Vegetation Seral Stages}
We recognize three separate seral stages for SAGE: Early Development (ED), Mid Development – Moderate Canopy Cover (MDM), and Late Development – Closed Canopy Cover (LDC). Our seral stages are an alternative to “successional” classes that imply a linear progression of states and tend not to incorporate disturbance. The condition classes identified here are derived from a combination of successional processes and anthropogenic and natural disturbance, and are intended to represent a composition and structural condition that can be arrived at from multiple other conditions described for that landcover type. Thus our condition classes incorporate age, size, canopy cover, and vegetation composition as well as relative seral stages. In general, the delineation of stages has originated from the LandFire biophysical setting model descriptive of a given landcover type; however, condition classes are not necessarily identical to the classes identified in those models.

\subsubsection{Early Development (ED)}

\paragraph{Description} A. tridentata does not sprout after burning but most of the other shrubs common to the type do. Consequently, for as long as 20 years after fire the vegetative community may be dominated by Chrysothamnus, Tetradymia, and grasses. A very hot fire in a degraded site may result in a seral community dominated by annual grasses and forbs. Perennial bunchgrasses frequently survive fires and become dominant (Neal 1988). Canopy cover is less than 40\%, but shrub cover may be as little as 10\%. Fuel loading is discontinuous (LandFire 2007).

\paragraph{Succession Transition} In the absence of disturbance, patches in this condition will transition to MDM at 20 years. 

\paragraph{Wildfire Transition} High mortality wildfire is not modeled for this condition class Low mortality wildfire (100\% of fires in this condition) maintains the patch in the ED condition. 

\hrulefill


\subsubsection{Mid Development – Moderate Canopy Cover (MDM)}

\paragraph{Description} A. tridentata usually reaches fairly stable dominance 10 to 20 years after disturbance, with or without an understory of perennial bunchgrass. A. tridentata usually remains dominant indefinitely or until the next disturbance (Neal 1988). Shrub density is sufficient in old stands to carry the fire without fine fuels. Shrubs and herbaceous vegetation can be codominant. Generally, shrub cover averages 30\% (LandFire 2007).

\paragraph{Succession Transition} At 40 years without disturbance, patches in this condition will transition to LDC. 

\paragraph{Wildfire Transition} High mortality wildfire (90\% of fires in this condition) recycles the patch through the ED condition. Low mortality wildfire (10\%) maintains the patch in the MDM condition.

\hrulefill


\subsubsection{Late Development – Closed Canopy Cover (LDC)}

\paragraph{Description} Shrublands with some encroachment from P. monophylla and J. osteosperma possible. Wildfire has not occurred for at least 60 years. Tree species cover is highly variable. In the continued absence of disturbance, shrub cover will decline (LandFire 2007).

\paragraph{Succession Transition} In the absence of disturbance, patches in this condition will maintain. 

\paragraph{Wildfire Transition} High mortality wildfire (100\% of fires in this condition) recycles the patch through the ED condition. Low mortality wildfire is not modeled for this condition class.

\hrulefill

\subsection{Condition Classification}
Because condition classification was done through orthophoto analysis, no polygons are assigned to Late condition, which is actually not an Artemisia-dominated condition. Polygons are assigned to MDO or MDC based on a 20\% break point. Open conditions have less than 20\% cover and closed conditions have greater than 20\% cover. Polygons with a Null value for shrub cover are assigned to ED.

\subsection{Draft Model}
\begin{figure}[htbp]
\centering
\includegraphics[width=0.8\textwidth]{/Users/mmallek/Tahoe/Report3/images/state_trans_model.pdf}
\caption{Generic state and transition model for all non-shrub seral cover types. Boxes show seven condition classes and arrows depict transitions due to vegetation succession and high or low mortality fire.} 
\label{sage_transmodel}
\end{figure}


%\begin{thebibliography}
%\bibitem{LandFire} LandFire. “Biophysical Setting Models.” Biophysical Setting 0610800: Inter-Mountain Basins Big Sagebrush Shrubland. 2007. LANDFIRE Project, U.S. Department of Agriculture, Forest Service; U.S. Department of the Interior. <http://www.landfire.gov/national_veg_models_op2.php>. Accessed 9 November 2012.

%\bibitem{Neal} Neal, Donald L. “Sagebrush (SGB).” A Guide to Wildlife Habitats of California, edited by Kenneth E. Mayer and William F. Laudenslayer. California Deparment of Fish and Game, 1988. <http://www.dfg.ca.gov/biogeodata/cwhr/pdfs/SGB.pdf>. Accessed 4 December 2012.

%\bibitem{Safford} Safford, Hugh. Regional Ecologist, USDA Forest Service. Personal communication, 15 August 2013.

%\bibitem{VandeWater} Van de Water, Kip M. and Hugh D. Safford. “A Summary of Fire Frequency Estimates for California Vegetation Before Euro-American Settlement.” Fire Ecology 7.3 (2011): 26-57. doi: 10.4996/fireecology.0703026.
%\end{thebibliography}
