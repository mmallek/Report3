
\chapter{Analysis by Cover Type}
\label{ch:covtype}
We defined 31 distinct land cover types in the Yuba River watershed and surrounding area for the purposes of \textsc{RMLands} simulations (Table~\ref{covertable}). A few of these were located in the buffer, but not the project area. Several others were treated as \emph{static} in the simulation: they did not undergo vegetation transitions over time or in response to fire. However, four of the \emph{static} types were allowed to experience wildfires: Agriculture, Grassland, Meadow, and Urban. Grasslands may experience fire, but because they are expected to recover from fire in less than five years (the length of one timestep in our simulation), we assume they remain constant in composition and structure. The discussion that follows focus on the two most prevalent cover types within the core project area: Sierran Mixed Conifer - Mesic and Sierran Mixed Conifer - Xeric. We describe the simulated disturbance regime in terms of spatial extent and distribution, frequency, and temporal variability. In addition, we report the vegetation dynamics that result from wildfires and successional processes, and examine each cover type's current departure from the simulated HRV. The same analysis was conducted on the other seven focal types in the project area, and these results are available in Appendix~\ref{app:covtype_analysis}.



%topics
% CHECK disturbed area
% climate? tpi? - Looked at climate for OCFW...basically the same as for landscape. not sure what it adds

%%%%%%%%%%%%%%%%%%%%%%%%%%%%%%%%%%%%%%%%%%%%%%%%%%%%%%%%%%%%%%%%%%%%%%%%%%%%%
%%%%%%%%%%%%%%%%%%%%%%%%%%%%%%%%%%%%%%%%%%%%%%%%%%%%%%%%%%%%%%%%%%%%%%%%%%%%%
%%%%%%%%%%%%%%%%%%%%%%%%%%%%%%%%%%%%%%%%%%%%%%%%%%%%%%%%%%%%%%%%%%%%%%%%%%%%%
%%%%%%%%%%%%%%%%%%%%%%%%%%%%%%%%%%%%%%%%%%%%%%%%%%%%%%%%%%%%%%%%%%%%%%%%%%%%%
%%%%%%%%%%%%%%%%%%%%%%%%%%%%%%%%%%%%%%%%%%%%%%%%%%%%%%%%%%%%%%%%%%%%%%%%%%%%%


\section{Sierran Mixed Conifer - Mesic} 

\begin{figure}[!htbp]
  \centering
  \subfloat[][]{
    \centering
    \includegraphics[width=0.5\textwidth]{/Users/mmallek/Documents/Thesis/Plots/darea/hrv_smcm.png}
    }%
  \subfloat[][]{
    \includegraphics[width=0.5\textwidth]{/Users/mmallek/Documents/Thesis/Plots/darea/hrv_hist_smcm.png}
    }
  \caption{\small (a) Disturbance trajectory for Sierran Mixed Conifer - Mesic. High mortality fire in dark blue; low mortality fire in light blue. (b) Histogram of disturbed hectares with density curve overlaid.} 
  \label{fig:darea_smcm}
\end{figure}

Sierran Mixed Conifer - Mesic (\textsc{smc\_m}) is the dominant cover type within the core project area, encompassing 57,853 ha and comprising roughly 32\% of the project area. The frequency and extent of simulated wildfires in sierran mixed conifer forests varied markedly across the landscape (Figure~\ref{fig:darea_smcm} and Table~\ref{tab:darea_smcm}). %
%
Wildfire was prevalent in this cover type. At least some area burned every five years, and at least 10\% of the cover type burned in 67\% of the simulated timesteps. The median amount of land burned during one timestep was 14\%. Over 25\% burned at a 19 year interval. Wildfires that extended across large extents of mesic mixed conifer forests were infrequent; at least 50\% of the cover type burned only once every 256 years. There was tremendous variability in the amount of area burned, from a minimum of 0.4\% to a maximum of 73\% (over 42,000 ha). Fires were three times more likely to have a low mortality effect on these forests as a high mortality effect. %
%
Under this wildfire regime, the grand mean return interval between fires (of any mortality level) varied widely from 18 years to over 500 years, with a median of 28 years (Figure~\ref{fig:preturn_smcm}). Median return interval and rotation values tend to be shorter in mixed conifer forests than in red fir forests, because their lower elevation corresponds to warmer and drier conditions. Mesic mixed conifer forests had a low mortality fire rotation of 36 years and a high mortality fire rotation of 125 years (Table~\ref{tab:darea_smcm}).  %
%
In general, return intervals and canopy cover varied spatially across the forest and decreased with increasing TPI, reflecting our parameterization, which was based on the theory that higher, more southerly aspects are drier and more susceptible to fires. Canopy cover decreased by about 13\% when comparing minimum to maximum TPI, from an average of 49\% to an average of 43\% (Table~\ref{tab:tpi_cc})..  %
%
Finally, when stands of mesic sierran mixed conifer forests were adjacent to cover types with much shorter or longer return intervals, they also exhibited a directional shift in local return intervals towards that of the adjacent type, reflecting the importance of landscape context on fire regimes.


\begin{table}[!htbp]
\centering
\caption{\small Disturbed area summary statistics for Sierran Mixed Conifer - Mesic. Proportions shown are relative to the total area of Sierran Mixed Conifer - Mesic.}
\label{tab:darea_smcm}
\begin{tabular}{@{}llll@{}}
\toprule
\textbf{\begin{tabular}[c]{@{}l@{}}Summary Statistic \\ (disturbed area/timestep)\end{tabular}} & \textbf{Low Mortality} & \textbf{High Mortality} & \textbf{Any Mortality} \\ \midrule
Minimum       & 0.31  & 0.03  & 0.37  \\
Maximum       & 53.29 & 20.78 & 72.95 \\
Median        & 10.96 & 3.18  & 14.48 \\
Mean          & 14.00 & 3.98  & 18.03 \\
\textbf{Fire Rotation} & 36       & 125       & 28   \\ \bottomrule
\end{tabular}
\end{table}
%%%
The age structure and dynamics of mesic mixed conifer forests illustrates the interaction between disturbance and succession processes. We focus our analysis on the 5$^{\text{th}}$ to 95$^{\text{th}}$ percentile range of variability for our simulation (excluding the equilibration period). %
%
The distribution of area among stand conditions within mesic mixed conifer forests fluctuated over time, as expected (Figure~\ref{fig:covcond_smcm}). For example, the percentage of mesic mixed conifer forests in the Early Development condition varied from 7\%--21\%, reflecting the dynamic nature of this cover type (Table~\ref{tab:covcond2}). This condition is currently within the simulated HRV (56$^{\text{th}}$ percentile). Mid Development - Closed was typically the most extensive condition class (11\%-29\%), but most of the condition classes were common throughout the simulation. The shift towards closed canopies when stands reached the Late Development stage may be due to an increasing resilience to wildfire disturbances by stands of that age: wildfires may burn the understory without significantly affecting overstory canopy cover. %
%
The seral-stage distribution appeared to be in dynamic equilibrium (i.e., the percentage in each stand condition varied about a stable mean). Our calculated current seral-stage distribution was never observed under the simulated HRV (Table~\ref{tab:covcond2}). The most notable departure was an increase in Mid Development - Closed and Late Development - Open extent, and a decrease in Mid Development - Moderate extent during the simulated HRV. These condition classes are currently all outside of the simulated HRV. In fact, Late Development - Open is rare on the current landscape (3.6\%), but present in similar proportions to the other classes during the HRV. The other two Late Development classes are within the HRV, with the closed canopy and moderate canopy conditions currently in the $86^{\text{th}}$ and $54^{\text{th}}$ percentiles, respectively. 

\begin{figure}[!htbp]
  %\centering
  \subfloat[][]{
%    \centering
    \includegraphics[width=0.8\textwidth]{/Users/mmallek/Tahoe/Report2/images/CovcondHRVBarplots/smcm_Early-AllStructures_srvplot_.pdf}
  }\\%
  \subfloat[][]{
%    \centering
    \includegraphics[width=0.8\textwidth]{/Users/mmallek/Tahoe/Report2/images/CovcondHRVBarplots_nolegend/smcm_Mid-Closed_srvplot_.pdf}
  }\\%
    \subfloat[][]{
%    \centering
    \includegraphics[width=0.8\textwidth]{/Users/mmallek/Tahoe/Report2/images/CovcondHRVBarplots_nolegend/smcm_Mid-Moderate_srvplot_.pdf}
  }\\%
    \subfloat[][]{
%    \centering
    \includegraphics[width=0.8\textwidth]{/Users/mmallek/Tahoe/Report2/images/CovcondHRVBarplots_nolegend/smcm_Mid-Open_srvplot_.pdf}
  }\\%
    \subfloat[][]{
%    \centering
    \includegraphics[width=0.8\textwidth]{/Users/mmallek/Tahoe/Report2/images/CovcondHRVBarplots_nolegend/smcm_Late-Closed_srvplot_.pdf}
  }\\%
    \subfloat[][]{
%    \centering
    \includegraphics[width=0.8\textwidth]{/Users/mmallek/Tahoe/Report2/images/CovcondHRVBarplots_nolegend/smcm_Late-Moderate_srvplot_.pdf}
  }\\%
    \subfloat[][]{
%    \centering
    \includegraphics[width=0.8\textwidth]{/Users/mmallek/Tahoe/Report2/images/CovcondHRVBarplots_nolegend/smcm_Late-Open_srvplot_.pdf}
  }\\%
  \caption{Cover-condition barplots for Sierran Mixed Conifer - Mesic dynamics. For each condition class, the color the bar represents the distance from the median value during the simulated HRV. Green represents the 25th-75th percentiles; yellow represents the 5th-25th and 75th to 95th percentiles; red represents the 0th-5th and 95th-100th percentiles. The blue vertical line marks the 50th percentile and the black vertical line indicates the current cover extent. To read the ``Early–All Structures'' barplot, for a given percentage of the cover type extent, the x-axis value indicates an observed proportion, and the color corresponding to that point indicates the percentile range that value falls within. In this example, the current percent of cover extent for this cover type and condition class falls within the 95th-100th percentile range during the simulated HRV.}
  \label{fig:covcondbar_smcm}
\end{figure}

The spatial configuration of stand conditions fluctuated markedly over time as well, although there was considerable variation in the magnitude of variability among configuration metrics (Appendix \ref{sec:full-class-results}). Area-weighted patch and core area, and radius of gyration, exhibited the greatest variability over time. The class-level metrics for mesic mixed conifer forests for our focal metrics fairly consistently show larger patches with more core area, more geometric complexity, and usually greater clumpiness than the current landscape. This result is similar to that for the overall landscape. Late development forests with moderate canopy cover consistently differ (in the opposite direction) from the other condition classes.  Although the metric values do not always fall outside the HRV, we still feel the results indicate that mesic mixed conifer forests today have significantly diverged from the historical pattern. \todo{redid this, definitely changed}

\begin{figure}[!htbp]
  \subfloat[][]{
    \includegraphics[width=0.8\textwidth]{/Users/mmallek/Tahoe/Report2/images/ClassFragPlots_wlegend/AREA_AM-SMC_M_EARLY_ALL-srvplot-.pdf}
  }\\%
  \subfloat[][]{
    \includegraphics[width=0.8\textwidth]{/Users/mmallek/Tahoe/Report2/images/ClassFragPlots_nolegend/AREA_AM-SMC_M_MID_CL-srvplot-.pdf}
  }\\%
    \subfloat[][]{

    \includegraphics[width=0.8\textwidth]{/Users/mmallek/Tahoe/Report2/images/ClassFragPlots_nolegend/AREA_AM-SMC_M_MID_MOD-srvplot-.pdf}
  }\\%
    \subfloat[][]{
    \includegraphics[width=0.8\textwidth]{/Users/mmallek/Tahoe/Report2/images/ClassFragPlots_nolegend/AREA_AM-SMC_M_MID_OP-srvplot-.pdf}
  }\\%
    \subfloat[][]{
    \includegraphics[width=0.8\textwidth]{/Users/mmallek/Tahoe/Report2/images/ClassFragPlots_nolegend/AREA_AM-SMC_M_LATE_CL-srvplot-.pdf}
  }\\%
    \subfloat[][]{
    \includegraphics[width=0.8\textwidth]{/Users/mmallek/Tahoe/Report2/images/ClassFragPlots_nolegend/AREA_AM-SMC_M_LATE_MOD-srvplot-.pdf}
  }\\%
    \subfloat[][]{
%    \centering
    \includegraphics[width=0.8\textwidth]{/Users/mmallek/Tahoe/Report2/images/ClassFragPlots_nolegend/AREA_AM-SMC_M_LATE_OP-srvplot-.pdf}
  }\\%
  \caption{Fragstats class-level results for Sierran Mixed Conifer - Mesic and area-weighted mean patch area. Each bar denotes the metric value for the associated percentile value. A blue bar denotes the median simulated HRV value and a black bar denotes the current value.}
  \label{fig:smcm_areaam}
\end{figure}

\begin{figure}[!htbp]
  \subfloat[][]{
    \includegraphics[width=0.8\textwidth]{/Users/mmallek/Tahoe/Report2/images/ClassFragPlots_wlegend/CORE_AM-SMC_M_EARLY_ALL-srvplot-.pdf}
  }\\%
  \subfloat[][]{
    \includegraphics[width=0.8\textwidth]{/Users/mmallek/Tahoe/Report2/images/ClassFragPlots_nolegend/CORE_AM-SMC_M_MID_CL-srvplot-.pdf}
  }\\%
    \subfloat[][]{

    \includegraphics[width=0.8\textwidth]{/Users/mmallek/Tahoe/Report2/images/ClassFragPlots_nolegend/CORE_AM-SMC_M_MID_MOD-srvplot-.pdf}
  }\\%
    \subfloat[][]{
    \includegraphics[width=0.8\textwidth]{/Users/mmallek/Tahoe/Report2/images/ClassFragPlots_nolegend/CORE_AM-SMC_M_MID_OP-srvplot-.pdf}
  }\\%
    \subfloat[][]{
    \includegraphics[width=0.8\textwidth]{/Users/mmallek/Tahoe/Report2/images/ClassFragPlots_nolegend/CORE_AM-SMC_M_LATE_CL-srvplot-.pdf}
  }\\%
    \subfloat[][]{
    \includegraphics[width=0.8\textwidth]{/Users/mmallek/Tahoe/Report2/images/ClassFragPlots_nolegend/CORE_AM-SMC_M_LATE_MOD-srvplot-.pdf}
  }\\%
    \subfloat[][]{
%    \centering
    \includegraphics[width=0.8\textwidth]{/Users/mmallek/Tahoe/Report2/images/ClassFragPlots_nolegend/CORE_AM-SMC_M_LATE_OP-srvplot-.pdf}
  }\\%
  \caption{Fragstats class-level results for Sierran Mixed Conifer - Mesic and area-weighted mean patch core area. Each bar denotes the metric value for the associated percentile value. A blue bar denotes the median simulated HRV value and a black bar denotes the current value.}
  \label{fig:smcm_coream}
\end{figure}



\begin{figure}[!htbp]
  \subfloat[][]{
    \includegraphics[width=0.8\textwidth]{/Users/mmallek/Tahoe/Report2/images/ClassFragPlots_wlegend/SHAPE_AM-SMC_M_EARLY_ALL-srvplot-.pdf}
  }\\%
  \subfloat[][]{
    \includegraphics[width=0.8\textwidth]{/Users/mmallek/Tahoe/Report2/images/ClassFragPlots_nolegend/SHAPE_AM-SMC_M_MID_CL-srvplot-.pdf}
  }\\%
    \subfloat[][]{

    \includegraphics[width=0.8\textwidth]{/Users/mmallek/Tahoe/Report2/images/ClassFragPlots_nolegend/SHAPE_AM-SMC_M_MID_MOD-srvplot-.pdf}
  }\\%
    \subfloat[][]{
    \includegraphics[width=0.8\textwidth]{/Users/mmallek/Tahoe/Report2/images/ClassFragPlots_nolegend/SHAPE_AM-SMC_M_MID_OP-srvplot-.pdf}
  }\\%
    \subfloat[][]{
    \includegraphics[width=0.8\textwidth]{/Users/mmallek/Tahoe/Report2/images/ClassFragPlots_nolegend/SHAPE_AM-SMC_M_LATE_CL-srvplot-.pdf}
  }\\%
    \subfloat[][]{
    \includegraphics[width=0.8\textwidth]{/Users/mmallek/Tahoe/Report2/images/ClassFragPlots_nolegend/SHAPE_AM-SMC_M_LATE_MOD-srvplot-.pdf}
  }\\%
    \subfloat[][]{
%    \centering
    \includegraphics[width=0.8\textwidth]{/Users/mmallek/Tahoe/Report2/images/ClassFragPlots_nolegend/SHAPE_AM-SMC_M_LATE_OP-srvplot-.pdf}
  }\\%
  \caption{Fragstats class-level results for Sierran Mixed Conifer - Mesic and area-weighted mean patch shape. Each bar denotes the metric value for the associated percentile value. A blue bar denotes the median simulated HRV value and a black bar denotes the current value.}
  \label{fig:smcm_shapeam}
\end{figure}

\begin{figure}[!htbp]
  \subfloat[][]{
    \includegraphics[width=0.8\textwidth]{/Users/mmallek/Tahoe/Report2/images/ClassFragPlots_wlegend/CLUMPY-SMC_M_EARLY_ALL-srvplot-.pdf}
  }\\%
  \subfloat[][]{
    \includegraphics[width=0.8\textwidth]{/Users/mmallek/Tahoe/Report2/images/ClassFragPlots_nolegend/CLUMPY-SMC_M_MID_CL-srvplot-.pdf}
  }\\%
    \subfloat[][]{

    \includegraphics[width=0.8\textwidth]{/Users/mmallek/Tahoe/Report2/images/ClassFragPlots_nolegend/CLUMPY-SMC_M_MID_MOD-srvplot-.pdf}
  }\\%
    \subfloat[][]{
    \includegraphics[width=0.8\textwidth]{/Users/mmallek/Tahoe/Report2/images/ClassFragPlots_nolegend/CLUMPY-SMC_M_MID_OP-srvplot-.pdf}
  }\\%
    \subfloat[][]{
    \includegraphics[width=0.8\textwidth]{/Users/mmallek/Tahoe/Report2/images/ClassFragPlots_nolegend/CLUMPY-SMC_M_LATE_CL-srvplot-.pdf}
  }\\%
    \subfloat[][]{
    \includegraphics[width=0.8\textwidth]{/Users/mmallek/Tahoe/Report2/images/ClassFragPlots_nolegend/CLUMPY-SMC_M_LATE_MOD-srvplot-.pdf}
  }\\%
    \subfloat[][]{
%    \centering
    \includegraphics[width=0.8\textwidth]{/Users/mmallek/Tahoe/Report2/images/ClassFragPlots_nolegend/CLUMPY-SMC_M_LATE_OP-srvplot-.pdf}
  }\\%
  \caption{Fragstats class-level results for Sierran Mixed Conifer - Mesic and clumpiness. Each bar denotes the metric value for the associated percentile value. A blue bar denotes the median simulated HRV value and a black bar denotes the current value.}
  \label{fig:smcm_clumpy}
\end{figure}

%%%%%%%%%%%%%%%%%%%%%%%%%%%%%%%%%%%%%%%%%%%%%%%%%%%%%%%%%%%%%%%%%%%%%%%%%%%%%
%%%%%%%%%%%%%%%%%%%%%%%%%%%%%%%%%%%%%%%%%%%%%%%%%%%%%%%%%%%%%%%%%%%%%%%%%%%%%
%%%%%%%%%%%%%%%%%%%%%%%%%%%%%%%%%%%%%%%%%%%%%%%%%%%%%%%%%%%%%%%%%%%%%%%%%%%%%
%%%%%%%%%%%%%%%%%%%%%%%%%%%%%%%%%%%%%%%%%%%%%%%%%%%%%%%%%%%%%%%%%%%%%%%%%%%%%
%%%%%%%%%%%%%%%%%%%%%%%%%%%%%%%%%%%%%%%%%%%%%%%%%%%%%%%%%%%%%%%%%%%%%%%%%%%%%


\clearpage
\section{Sierran Mixed Conifer - Xeric} 

\begin{figure}[!htbp]
  \centering
  \subfloat[][]{
    \centering
    \includegraphics[width=0.5\textwidth]{/Users/mmallek/Documents/Thesis/Plots/darea/hrv_smcx.png}
    }%
  \subfloat[][]{
    \includegraphics[width=0.5\textwidth]{/Users/mmallek/Documents/Thesis/Plots/darea/hrv_hist_smcx.png}
    }
  \caption{\small (a) Disturbance trajectory for Sierran Mixed Conifer - Xeric. High mortality fire in dark blue; low mortality fire in light blue. (b) Histogram of disturbed hectares with density curve overlaid.} 
  \label{fig:darea_smcx}
\end{figure}

Sierran Mixed Conifer - Xeric (\textsc{smc\_x}) is the second most dominant cover type within the core project area, encompassing 52,198 ha and comprising roughly 29\% of the project area. The frequency and extent of simulated wildfires in xeric sierran mixed conifer forests varied markedly across the landscape (Figure~\ref{fig:darea_smcx} and Table~\ref{tab:darea_smcx}).  %
%
Wildfire is quite prevalent in xeric mixed conifer forests; its overall fire rotation is the lowest of all cover types. At least some area burned during every five-year timestep, and at least 10\% of the cover type burned in over 75\% of the simulated timesteps, or about every 7 years. The median amount of land burned during the simulation was 18\%. Over 25\% of the cover type burned every 17 years. Fires burned over 50\% of the cover type about once every 61 years, making this cover type the most likely to burn extensively. During one five-year interval, 86\% of the xeric mixed conifer forest burned (nearly 45,000 ha). The minimum area burned was 485 hectares, which indicates a remarkable range of variability in disturbance extent. Low mortality fire was about 1.5 times as common as high mortality fire. %
%
Under this wildfire regime, the grand mean return interval between fires (of any mortality level) varied widely from 17 years to over 500 years, with a median of 23 years (Figure~\ref{fig:preturn_smcx}). As expected, median return interval and rotation values are much longer for this cover type as compared to non-ultramafic mixed conifer forests, which occupy similar elevations. Xeric mixed conifer forests had a low mortality fire rotation of 36 years and a high mortality fire rotation of 64 years (Table~\ref{tab:darea_smcx}), which was by far the lowest high mortality rotation period of any cover type. Neither high nor low mortality dominantes the disturbance regime. %
%
In general, return intervals and canopy cover varied spatially across the forest and decreased with increasing TPI, reflecting our parameterization, which was based on the theory that higher, more southerly aspects are drier and more susceptible to fires. Canopy cover decreased by about 25\% when comparing minimum to maximum TPI, from an average of 36\% to an average of 27\% (Table~\ref{tab:tpi_cc}).  %
%
Finally, when stands of xeric mixed conifer forests were adjacent to cover types with much shorter or longer return intervals, they also exhibited a directional shift in local return intervals towards that of the adjacent type, reflecting the importance of landscape context on fire regimes.

\begin{table}[!htbp]
\centering
\caption{\small Disturbed area summary statistics for Sierran Mixed Conifer - Xeric. Proportions shown are relative to the total area of Sierran Mixed Conifer - Xeric.}
\label{tab:darea_smcx}
\begin{tabular}{@{}llll@{}}
\toprule
\textbf{\begin{tabular}[c]{@{}l@{}}Summary Statistic \\ (disturbed area/timestep)\end{tabular}} & \textbf{Low Mortality} & \textbf{High Mortality} & \textbf{Any Mortality} \\ \midrule
Minimum       & 0.65  & 0.07  & 0.93  \\
Maximum       & 54.14 & 36.11 & 86.29 \\
Median        & 10.93 & 5.99  & 17.94 \\
Mean          & 13.78 & 7.84  & 21.71 \\
\textbf{Fire Rotation} & 36       & 64        & 23 \\  \bottomrule
\end{tabular}
\end{table}

%%%
The age structure and dynamics of xeric mixed conifer forests illustrates the interaction between disturbance and succession processes. We focus our analysis on the 5$^{\text{th}}$ to 95$^{\text{th}}$ percentile range of variability for our simulation (excluding the equilibration period). %
%
The distribution of area among stand conditions within xeric mixed conifer forests fluctuated over time, as expected (Figure~\ref{fig:covcond_smcx}). For example, the percentage of xeric mixed conifer forests in the Early Development varied from 24\% to 44\%, reflecting the dynamic nature of this cover type (Table~\ref{tab:covcond3}). During the simulation, Early Development (which includes post-fire chaparral fields) and Mid Development - Open conditions dominated, in contrast to the current distribution, which is somewhat even across classes (although late development, open canopy stands are currently quite rare).  %
%
The seral-stage distribution appeared to be in dynamic equilibrium (i.e., the percentage in each stand condition varied about a stable mean). Our calculated current seral-stage distribution was never observed under the simulated HRV (Table~\ref{tab:covcond3}). In fact, none of the condition classes had a distribution within the simulated HRV. The most dramatic departure was the increase in Early Development and Mid Development - Open during the simulated HRV compared to the current landscape (currently at 19\% and 11\%, respectively). We also observed a much lower proportion of xeric mixed conifer forest in Late Development - Closed during the simulation than in the current landscape (25\%). The decline in the extent of Late Development forests is primarily due to the frequency of high mortality fire, which inhibits stands from succeeding to those stages. As stated above, this cover type experienced the most fire during the simulated HRV. High mortality fire directly led to the increase in Early Development conditions, and the dominance of open canopies within the middle and late development stages is explained by the high frequency of low mortality fire.

\begin{figure}[!htbp]
  %\centering
  \subfloat[][]{
%    \centering
    \includegraphics[width=0.8\textwidth]{/Users/mmallek/Tahoe/Report2/images/CovcondHRVBarplots/smcx_Early-AllStructures_srvplot_.pdf}
  }\\%
  \subfloat[][]{
%    \centering
    \includegraphics[width=0.8\textwidth]{/Users/mmallek/Tahoe/Report2/images/CovcondHRVBarplots_nolegend/smcx_Mid-Closed_srvplot_.pdf}
  }\\%
    \subfloat[][]{
%    \centering
    \includegraphics[width=0.8\textwidth]{/Users/mmallek/Tahoe/Report2/images/CovcondHRVBarplots_nolegend/smcx_Mid-Moderate_srvplot_.pdf}
  }\\%
    \subfloat[][]{
%    \centering
    \includegraphics[width=0.8\textwidth]{/Users/mmallek/Tahoe/Report2/images/CovcondHRVBarplots_nolegend/smcx_Mid-Open_srvplot_.pdf}
  }\\%
    \subfloat[][]{
%    \centering
    \includegraphics[width=0.8\textwidth]{/Users/mmallek/Tahoe/Report2/images/CovcondHRVBarplots_nolegend/smcx_Late-Closed_srvplot_.pdf}
  }\\%
    \subfloat[][]{
%    \centering
    \includegraphics[width=0.8\textwidth]{/Users/mmallek/Tahoe/Report2/images/CovcondHRVBarplots_nolegend/smcx_Late-Moderate_srvplot_.pdf}
  }\\%
    \subfloat[][]{
%    \centering
    \includegraphics[width=0.8\textwidth]{/Users/mmallek/Tahoe/Report2/images/CovcondHRVBarplots_nolegend/smcx_Late-Open_srvplot_.pdf}
  }\\%
  \caption{Cover-condition barplots for Sierran Mixed Conifer - Xeric dynamics. For each condition class, the color the bar represents the distance from the median value during the simulated HRV. Green represents the 25th-75th percentiles; yellow represents the 5th-25th and 75th to 95th percentiles; red represents the 0th-5th and 95th-100th percentiles. The blue vertical line marks the 50th percentile and the black vertical line indicates the current cover extent. To read the ``Early–All Structures'' barplot, for a given percentage of the cover type extent, the x-axis value indicates an observed proportion, and the color corresponding to that point indicates the percentile range that value falls within. In this example, the current percent of cover extent for this cover type and condition class falls within the 95th-100th percentile range during the simulated HRV.}
  \label{fig:covcondbar_smcx}
\end{figure}

The spatial configuration of stand conditions fluctuated markedly over time as well, although there was considerable variation in the magnitude of variability among configuration metrics (Appendix \ref{sec:full-class-results}). Area-weighted patch and core area, as well as edge density, exhibited the greatest variability over time. Four of the seven condition classes consistently fell near or beyond the HRV for the focal metrics. Early development, middle development moderate and open canopy, and late development open canopy cover are currently characterized by smaller, less aggregated, and less geometrically complex patches with less core area than the same condition classes during the simulated HRV. Late development, closed canopy patches trend weakly in the opposite direction, while the remaining conditions typically fall within the HRV.




\begin{figure}[!htbp]
  \subfloat[][]{
    \includegraphics[width=0.8\textwidth]{/Users/mmallek/Tahoe/Report2/images/ClassFragPlots_wlegend/AREA_AM-SMC_X_EARLY_ALL-srvplot-.pdf}
  }\\%
  \subfloat[][]{
    \includegraphics[width=0.8\textwidth]{/Users/mmallek/Tahoe/Report2/images/ClassFragPlots_nolegend/AREA_AM-SMC_X_MID_CL-srvplot-.pdf}
  }\\%
    \subfloat[][]{

    \includegraphics[width=0.8\textwidth]{/Users/mmallek/Tahoe/Report2/images/ClassFragPlots_nolegend/AREA_AM-SMC_X_MID_MOD-srvplot-.pdf}
  }\\%
    \subfloat[][]{
    \includegraphics[width=0.8\textwidth]{/Users/mmallek/Tahoe/Report2/images/ClassFragPlots_nolegend/AREA_AM-SMC_X_MID_OP-srvplot-.pdf}
  }\\%
    \subfloat[][]{
    \includegraphics[width=0.8\textwidth]{/Users/mmallek/Tahoe/Report2/images/ClassFragPlots_nolegend/AREA_AM-SMC_X_LATE_CL-srvplot-.pdf}
  }\\%
    \subfloat[][]{
    \includegraphics[width=0.8\textwidth]{/Users/mmallek/Tahoe/Report2/images/ClassFragPlots_nolegend/AREA_AM-SMC_X_LATE_MOD-srvplot-.pdf}
  }\\%
    \subfloat[][]{
%    \centering
    \includegraphics[width=0.8\textwidth]{/Users/mmallek/Tahoe/Report2/images/ClassFragPlots_nolegend/AREA_AM-SMC_X_LATE_OP-srvplot-.pdf}
  }\\%
  \caption{Fragstats class-level results for Sierran Mixed Conifer - Xeric and area-weighted mean patch area. Each bar denotes the metric value for the associated percentile value. A blue bar denotes the median simulated HRV value and a black bar denotes the current value.}
  \label{fig:smcx_areaam}
\end{figure}

\begin{figure}[!htbp]
  \subfloat[][]{
    \includegraphics[width=0.8\textwidth]{/Users/mmallek/Tahoe/Report2/images/ClassFragPlots_wlegend/CORE_AM-SMC_X_EARLY_ALL-srvplot-.pdf}
  }\\%
  \subfloat[][]{
    \includegraphics[width=0.8\textwidth]{/Users/mmallek/Tahoe/Report2/images/ClassFragPlots_nolegend/CORE_AM-SMC_X_MID_CL-srvplot-.pdf}
  }\\%
    \subfloat[][]{

    \includegraphics[width=0.8\textwidth]{/Users/mmallek/Tahoe/Report2/images/ClassFragPlots_nolegend/CORE_AM-SMC_X_MID_MOD-srvplot-.pdf}
  }\\%
    \subfloat[][]{
    \includegraphics[width=0.8\textwidth]{/Users/mmallek/Tahoe/Report2/images/ClassFragPlots_nolegend/CORE_AM-SMC_X_MID_OP-srvplot-.pdf}
  }\\%
    \subfloat[][]{
    \includegraphics[width=0.8\textwidth]{/Users/mmallek/Tahoe/Report2/images/ClassFragPlots_nolegend/CORE_AM-SMC_X_LATE_CL-srvplot-.pdf}
  }\\%
    \subfloat[][]{
    \includegraphics[width=0.8\textwidth]{/Users/mmallek/Tahoe/Report2/images/ClassFragPlots_nolegend/CORE_AM-SMC_X_LATE_MOD-srvplot-.pdf}
  }\\%
    \subfloat[][]{
%    \centering
    \includegraphics[width=0.8\textwidth]{/Users/mmallek/Tahoe/Report2/images/ClassFragPlots_nolegend/CORE_AM-SMC_X_LATE_OP-srvplot-.pdf}
  }\\%
  \caption{Fragstats class-level results for Sierran Mixed Conifer - Xeric and area-weighted mean patch core area. Each bar denotes the metric value for the associated percentile value. A blue bar denotes the median simulated HRV value and a black bar denotes the current value.}
  \label{fig:smcx_coream}
\end{figure}



\begin{figure}[!htbp]
  \subfloat[][]{
    \includegraphics[width=0.8\textwidth]{/Users/mmallek/Tahoe/Report2/images/ClassFragPlots_wlegend/SHAPE_AM-SMC_X_EARLY_ALL-srvplot-.pdf}
  }\\%
  \subfloat[][]{
    \includegraphics[width=0.8\textwidth]{/Users/mmallek/Tahoe/Report2/images/ClassFragPlots_nolegend/SHAPE_AM-SMC_X_MID_CL-srvplot-.pdf}
  }\\%
    \subfloat[][]{

    \includegraphics[width=0.8\textwidth]{/Users/mmallek/Tahoe/Report2/images/ClassFragPlots_nolegend/SHAPE_AM-SMC_X_MID_MOD-srvplot-.pdf}
  }\\%
    \subfloat[][]{
    \includegraphics[width=0.8\textwidth]{/Users/mmallek/Tahoe/Report2/images/ClassFragPlots_nolegend/SHAPE_AM-SMC_X_MID_OP-srvplot-.pdf}
  }\\%
    \subfloat[][]{
    \includegraphics[width=0.8\textwidth]{/Users/mmallek/Tahoe/Report2/images/ClassFragPlots_nolegend/SHAPE_AM-SMC_X_LATE_CL-srvplot-.pdf}
  }\\%
    \subfloat[][]{
    \includegraphics[width=0.8\textwidth]{/Users/mmallek/Tahoe/Report2/images/ClassFragPlots_nolegend/SHAPE_AM-SMC_X_LATE_MOD-srvplot-.pdf}
  }\\%
    \subfloat[][]{
%    \centering
    \includegraphics[width=0.8\textwidth]{/Users/mmallek/Tahoe/Report2/images/ClassFragPlots_nolegend/SHAPE_AM-SMC_X_LATE_OP-srvplot-.pdf}
  }\\%
  \caption{Fragstats class-level results for Sierran Mixed Conifer - Xeric and area-weighted mean patch shape. Each bar denotes the metric value for the associated percentile value. A blue bar denotes the median simulated HRV value and a black bar denotes the current value.}
  \label{fig:smcx_shapeam}
\end{figure}

\begin{figure}[!htbp]
  \subfloat[][]{
    \includegraphics[width=0.8\textwidth]{/Users/mmallek/Tahoe/Report2/images/ClassFragPlots_wlegend/CLUMPY-SMC_X_EARLY_ALL-srvplot-.pdf}
  }\\%
  \subfloat[][]{
    \includegraphics[width=0.8\textwidth]{/Users/mmallek/Tahoe/Report2/images/ClassFragPlots_nolegend/CLUMPY-SMC_X_MID_CL-srvplot-.pdf}
  }\\%
    \subfloat[][]{

    \includegraphics[width=0.8\textwidth]{/Users/mmallek/Tahoe/Report2/images/ClassFragPlots_nolegend/CLUMPY-SMC_X_MID_MOD-srvplot-.pdf}
  }\\%
    \subfloat[][]{
    \includegraphics[width=0.8\textwidth]{/Users/mmallek/Tahoe/Report2/images/ClassFragPlots_nolegend/CLUMPY-SMC_X_MID_OP-srvplot-.pdf}
  }\\%
    \subfloat[][]{
    \includegraphics[width=0.8\textwidth]{/Users/mmallek/Tahoe/Report2/images/ClassFragPlots_nolegend/CLUMPY-SMC_X_LATE_CL-srvplot-.pdf}
  }\\%
    \subfloat[][]{
    \includegraphics[width=0.8\textwidth]{/Users/mmallek/Tahoe/Report2/images/ClassFragPlots_nolegend/CLUMPY-SMC_X_LATE_MOD-srvplot-.pdf}
  }\\%
    \subfloat[][]{
%    \centering
    \includegraphics[width=0.8\textwidth]{/Users/mmallek/Tahoe/Report2/images/ClassFragPlots_nolegend/CLUMPY-SMC_X_LATE_OP-srvplot-.pdf}
  }\\%
  \caption{Fragstats class-level results for Sierran Mixed Conifer - Xeric and clumpiness. Each bar denotes the metric value for the associated percentile value. A blue bar denotes the median simulated HRV value and a black bar denotes the current value.}
  \label{fig:smcx_clumpy}
\end{figure}


%%%%%%%%%%%%%%%%%%%%%%%%%%%%%%%%%%%%%%%%%%%%%%%%%%%%%%%%%%%%%%%%%%%%%%%%%%%%%
%%%%%%%%%%%%%%%%%%%%%%%%%%%%%%%%%%%%%%%%%%%%%%%%%%%%%%%%%%%%%%%%%%%%%%%%%%%%%
%%%%%%%%%%%%%%%%%%%%%%%%%%%%%%%%%%%%%%%%%%%%%%%%%%%%%%%%%%%%%%%%%%%%%%%%%%%%%
%%%%%%%%%%%%%%%%%%%%%%%%%%%%%%%%%%%%%%%%%%%%%%%%%%%%%%%%%%%%%%%%%%%%%%%%%%%%%
%%%%%%%%%%%%%%%%%%%%%%%%%%%%%%%%%%%%%%%%%%%%%%%%%%%%%%%%%%%%%%%%%%%%%%%%%%%%%



